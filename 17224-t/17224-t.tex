% Project Gutenberg's Chance and Luck, by Richard Proctor
%
% This eBook is for the use of anyone anywhere at no cost and with
% almost no restrictions whatsoever.  You may copy it, give it away or
% re-use it under the terms of the Project Gutenberg License included
% with this eBook or online at www.gutenberg.org
%
% Packages and substitutions:
%
% memoir :  Configurable document class. Required.
% amsmath:  Basic AMS math package. Required.
% amssymb:  Basic AMS symbols
% wrapfig:  text flow around figures.
% tabularx: building tables
%
% Producer's Comments:
%
% The page numbers in the static table of contents are gathered
% with LaTeX page references, hence the file should be compiled
% three times to get them right.
%
% Both pdflatex (to generate pdf) and latex (to generate dvi, and
% from this with appropriate tools other formats) work. The book
% contains no illustrations.
%
% Build instructions:
%
% file should be named chance.tex
% for reference, version used:
%   This is pdfTeX, Version 3.141592653-2.6-1.40.22
%   (TeX Live 2022/dev/Debian)
% Sequence:
%   latex chance
%   latex chance
%   latex chance
%   dvipdfm chance
%   rm chance.aux chance.dvi chance.log chance.toc
% Alternate sequence:
%   pdflatex chance
%   pdflatex chance
%   pdflatex chance
%   rm chance.aux chance.log  chance.toc

\documentclass[letterpaper,12pt,oneside,openany]{memoir}
\listfiles
\usepackage{amsmath}
\usepackage{amssymb}
\usepackage{wrapfig}
\usepackage{tabularx}

\newcommand{\D}{\hspace*{5mm}}
\newcommand{\E}{\hspace*{2mm}---\hspace*{2mm}}

\setstocksize{794pt}{614pt}
\settrimmedsize{\stockheight}{\stockwidth}{*}
\settrims{0pt}{0pt}
\setulmarginsandblock{110pt}{90pt}{*}
\setlrmarginsandblock{90pt}{90pt}{*}
\checkandfixthelayout

\begin{document}
\thispagestyle{empty}
\small
\begin{verbatim}
Project Gutenberg's Chance and Luck, by Richard Proctor

This eBook is for the use of anyone anywhere at no cost and with
almost no restrictions whatsoever.  You may copy it, give it away or
re-use it under the terms of the Project Gutenberg License included
with this eBook or online at www.gutenberg.org


Title: Chance and Luck

Author: Richard Proctor

Release Date: December 4, 2005 [EBook #17224]

Language: English

Character set encoding: TeX

*** START OF THIS PROJECT GUTENBERG EBOOK CHANCE AND LUCK ***


Produced by Joshua Hutchinson, Roger Frank and the Online
Distributed Proofreading Team at https://www.pgdp.net
This file was produced from images from the Cornell
University Library: Historical Mathematics Monographs collection.


\end{verbatim}
\normalsize
\newpage

\chapterstyle{section}
\pagestyle{empty}

\begin{titlingpage}
\begin{center}
\vspace*{1cm}
{\Huge CHANCE AND LUCK:}\\
\bigskip
{\small A DISCUSSION OF}\\
\medskip
{\large THE LAWS OF LUCK, COINCIDENCES,\\
WAGERS, LOTTERIES, AND THE FALLACIES OF GAMBLING;}\\
\bigskip
{\small WITH NOTES ON}\\
\medskip
{\large POKER AND MARTINGALES.}\\
\vspace*{2cm}
{\small BY}\\
\smallskip
{\large RICHARD A. PROCTOR}\\
\bigskip
{\tiny AUTHOR OF `HOW TO PLAY WHIST,' `HOME WHIST,'
`EASY LESSONS IN THE DIFFERENTIAL CALCULUS,'
AND THE ARTICLES ON ASTRONOMY IN THE `ENCYCLOP{\AE}DIA BRITANNICA'
AND THE `AMERICAN CYCLOP{\AE}DIA.'}\\
\vspace*{2cm}
\makebox[1in]{\hrulefill}\\
\smallskip
`Looking before and after.'---\textit{Shakespeare.}\\
\makebox[1in]{\hrulefill}\\
\vspace*{2cm}
\textit{SECOND EDITION}.\\
\bigskip
{\large LONDON:\\
LONGMANS, GREEN, AND CO.\\
1887.}\\
\medskip
{\small \emph{All rights reserved.}}
\end{center}
\end{titlingpage}

\begin{center}
\vspace*{7.0in}
Entered according to Act of Congress, in the year 1887,\\
by Richard Anthony Proctor,\\
in the Office of the Librarian of Congress, at Washington.\\
\end{center}
\clearpage

\begin{center}{\LARGE PREFACE.}\end{center}
\bigskip
The false ideas prevalent among all classes of the
community, cultured as well as uncultured, respecting
chance and luck, illustrate the truth that common
consent (in matters outside the influence of authority)
argues almost of necessity \textit{error}. This, by the way,
might be proved by the method of probabilities. For
if, in any question of difficulty, the chance that an
average mind will miss the correct opinion is but
one-half---and this is much underrating the chance of error---the
probability that the larger proportion of a community
numbering many millions will judge rightly on
any such question is but as one in many millions of
millions of millions. (Those who are too ready to
appeal to the argument from common consent, and on
the strength of it sometimes to denounce or even afflict
their fellow men, should take this fact---for it is fact,
not opinion---very thoughtfully to heart.)

I cannot hope, then, since authority has never been
at the pains to pronounce definitely on such questions
respecting luck and chance as are dealt with here, that
common opinion, which is proclaimed constantly and
loudly in favour of faith in luck, will readily accept the
teachings I have advanced, though they be but the
commonplace of science in regard to the dependence
of what is commonly called \textit{luck}, strictly, and in the
long run, uniformly, on \textit{law}. The gambling fraternity
will continue to proclaim their belief in luck (though
those who have proved successful among them have by
no means trusted to it), and the community on whom
they prey will, for the most part, continue to submit to
the process of plucking, in full belief that they are on
their way to fortune.

If a few shall be taught, by what I have explained
here, to see that in the long run even fair wagering and
gambling must lead to loss, while gambling and wagering
scarcely ever are fair, in the sense of being on even
terms, this book will have served a useful purpose. I
wish I could hope that it would serve the higher purpose
of showing that all forms of gambling and speculation
are essentially immoral, and that, though many
who gamble are not consciously wrong-doers, their very
unconsciousness of evil indicates an uncultured, semi-savage
mind.\\
\medskip
\begin{scshape}
\begin{flushright}
Richard A. Proctor.\\
\end{flushright}
\end{scshape}
Saint Joseph, Mo. 1887.
\clearpage

\setpnumwidth{2.75em}
\setcounter{secnumdepth}{-1}
\setcounter{tocdepth}{1}
\tableofcontents*
\clearpage

% CONTENTS
%
% LAWS OF LUCK.........1
%
% GAMBLER'S FALLACIES........29
%
% FAIR AND UNFAIR WAGERS......79
%
% BETTING ON RACES.........103
%
% LOTTERIES ....... 126
%
% GAMBLING IN SHAKES........162
%
% FALLACIES AND COINCIDENCES......191
%
% NOTES ON POKER ... 226
%
% MARTINGALES  .......     247

\pagestyle{headings}
\title{Chance and Luck}
\thispagestyle{chapter}
\chapter{Laws of Luck}
\pagenumbering{arabic}

To the student of science, accustomed to recognise the
operation of law in all phenomena, even though the
nature of the law and the manner of its operation may
be unknown, there is something strange in the prevalent
belief in luck. In the operations of nature and in the
actions of men, in commercial transactions and in
chance games, the great majority of men recognise the
prevalence of something outside law---the good fortune
or the bad fortune of men or of nations, the luckiness
or unluckiness of special times and seasons---in fine
(though they would hardly admit as much in words),
the influence of something extranatural if not supernatural.
[For to the man of science, in his work as
student of nature, the word `natural' implies the action
of law, and the occurrence of aught depending on what
men mean by luck would be simply the occurrence of
something supernatural.] This is true alike of great
things and of small; of matters having a certain dignity,
real or apparent, and of matters which seem utterly contemptible.
Napoleon announcing that a certain star
(as he supposed) seen in full daylight was \textit{his} star and
indicated at the moment the ascendency of his fortune,
or William the Conqueror proclaiming, as he rose with
hands full of earth from his accidental fall on the Sussex
shore, that he was destined by fate to seize England,
may not seem comparable with a gambler who says that
he shall win because he is in the vein, or with a player
at whist who rejoices that the cards he and his partner
use are of a particular colour, or expects a change from
bad to good luck because he has turned his chair round
thrice; but one and all are alike absurd in the eyes of
the student of science, who sees law, and not luck, in all
things that happen. He knows that Napoleon's imagined
star was the planet Venus, bound to be where
Napoleon and his officers saw it by laws which it had
followed for past millions of years, and will doubtless
follow for millions of years to come. He knows that
William fell (if by accident at all) because of certain
natural conditions affecting him physiologically (probably
he was excited and over anxious) and physically,
not by any influence affecting him extranaturally. But
he sees equally well that the gambler's superstitions
about `the vein,' the `maturity of the chances,' about
luck and about change of luck, relate to matters which
are not only subject to law, but may be dealt with by
processes of calculation. He recognises even in men's
belief in luck the action of law, and in the use which
clever men like Napoleon and William have made of
this false faith of men in luck, a natural result of cerebral
development, of inherited qualities, and of the system
of training which such credulous folk have passed
through.

Let us consider, however, the general idea which
most men have respecting what they call luck. We
shall find that what they regard as affording clear evidence
that there is such a thing as luck is in reality
the result of law. Nay, they adopt such a combination
of ideas about events which seem fortuitous that the
kind of evidence they obtain must have been obtained,
let events fall as they may.

Let us consider the ideas of men about luck in
gambling, as typifying in small the ideas of nearly all
men about luck in life.

In the first place, gamblers recognise some men as
always lucky. I do not mean, of course, that they suppose
some men always win, but that some men never
have spells of bad luck. They are \textit{always} `in the vein,'
to use the phraseology of gamblers like Steinmetz and
others, who imagine that they have reduced their wild
and wandering notions about luck into a science.

Next, gamblers recognise those who start on a
gambling career with singular good luck, retaining that
luck long enough to learn to trust in it confidently, and
then losing it once for all, remaining thereafter constantly
unlucky.

Thirdly, gamblers regard the great bulk of their community
as men of varying luck---sometimes in the
`vein' sometimes not---men who, if they are to be
successful, must, according to the superstitions of the
gambling world, be most careful to watch the progress
of events. These, according to Steinmetz, the great
authority on all such questions (probably because of the
earnestness of his belief in gambling superstitions), may
gamble or not, according as they are ready or not to
obey the dictates of gambling prudence. When they
are in the vein they should gamble steadily on; but so
soon as `the maturity of the chances' brings with it a
change of luck they must withdraw. If they will not
do this they are likely to join the crew of the unlucky.

Fourthly, there are those, according to the ideas of
gamblers, who are pursued by constant ill-luck. They
are never `in the vein.' If they win during the first
half of an evening, they lose more during the latter
half. But usually they lose all the time.

Fifthly, gamblers recognise a class who, having begun
unfortunately, have had a change of luck later, and
have become members of the lucky fraternity. This
change they usually ascribe to some action or event
which, to the less brilliant imaginations of outsiders,
would seem to have nothing whatever to do with the
gambler's luck. For instance, the luck changed when
the man married---his wife being a shrew; or because
he took to wearing white waistcoats; or because so-and-so,
who had been a sort of evil genius to the unlucky
man, had gone abroad or died; or for some equally preposterous
reason.

Then there are special classes of lucky or unlucky
men, or special peculiarities of luck, believed in by
individual gamblers, but not generally recognised.

Thus there are some who believe that they are lucky on
certain days of the week, and unlucky on certain other
days. The skilful whist-player who, under the name
`Pembridge,' deplores the rise of the system of signals
in whist play, believes that he is lucky for a spell of
five years, unlucky for the next five years, and so on
continually. Bulwer Lytton believed that he always
lost at whist when a certain man was at the same table,
or in the same room, or even in the same house. And
there are other cases equally absurd.

Now, at the outset, it is to be remarked that, if any
large number of persons set to work at any form of
gambling---card play, racing, or whatever else it may
be---their fortunes \textit{must} be such, let the individual
members of the company be whom they may, that they
will be divisible into such sets as are indicated above.
If the numbers are only large enough, not one of those
classes, not even the special classes mentioned at the
last, can fail to be represented.

Consider, for instance, the following simple illustrative
case:---

Suppose a large number of persons---say, for instance,
twenty millions---engage in some game depending
wholly on chance, two persons taking part in each
game, so that there are ten million contests. Now, it is
obvious that, whether the chances in each contest are
exactly equal or not, exactly ten millions of the twenty
millions of persons will rise up winners and as many
will rise up losers, the game being understood to be of
such a kind that one player or the other must win. So
far, then, as the results of that first set of contests are
concerned, there will be ten million persons who will
consider themselves to be in luck.

Now, let the same twenty millions of persons engage
a second time in the same two-handed game, the pairs
of players being not the same as at the first encounter,
but distributed as chance may direct. Then there will
be ten millions of winners and ten millions of losers.
Again, if we consider the fortunes of the ten million
winners on the first night, we see that, since the chance
which, each one of these has of being again a winner is
equal to the chance he has of losing, \textit{about} one-half of
the winning ten millions of the first night will be winners
on the second night too. Nor shall we deduce a
wrong general result if, for convenience, we say \textit{exactly}
one-half; so long as we are dealing with very large
numbers we know that this result must be near the
truth, and in chance problems of this sort we require
(and can expect) no more. On this assumption, there
are at the end of the second contest five millions who
have won in both encounters, and five millions who have
won in the first and lost in the second. The other ten
millions, who lost in the first encounter, may similarly
be divided into five millions who lost also in the second,
and as many who won in the second. Thus, at the end
of the second encounter, there are five millions of
players who deem themselves lucky, as they have won
twice and not lost at all; as many who deem themselves
unlucky, having lost in both encounters; while ten
millions, or half the original number, have no reason
to regard themselves as either lucky or unlucky, having
won and lost in equal degree.

Extending our investigation to a third contest, we
find that 2,500,000 will be confirmed in their opinion
that they are very lucky, since they will have won in all
three encounters; while as many will have lost in all
three, and begin to regard themselves, and to be regarded
by their fellow-gamblers, as hopelessly unlucky.
Of the remaining fifteen millions of players, it will be
found that 7,500,000 will have won twice and lost once,
while as many will have lost twice and won once.
(There will be 2,500,000 who won the first two games
and lost the third, as many who lost the first two and
won the third, as many who won the first, lost the
second, and won the third, and so on through the six
possible results for these fifteen millions who had mixed
luck.) Half of the fifteen millions will deem themselves
rather lucky, while the other half will deem
themselves rather unlucky. None, of course, can have
had even luck, since an odd number of games has been
played.

Our 20,000,000 players enter on a fourth series of
encounters. At its close there are found to be 1,250,000
very lucky players, who have won in all four encounters,
and as many unlucky ones who have lost in all four.
Of the 2,500,000 players who had won in three encounters,
one-half lose in the fourth; they had been
deemed lucky, but now their luck has changed. So
with the 2,500,000 who had been thus far unlucky:
one-half of them win on the fourth trial. We have
then 1,250,000 winners of three games out of four, and
1,250,000 losers of three games out of four. Of the
7,500,000 who had won two and lost one, one-half, or
3,750,000, win another game, and must be added to the
1,250,000 just mentioned, making three million winners
of three games out of four. The other half lose the
fourth game, giving us 3,750,000 who have had equal
fortunes thus far, winning two games and losing two.
Of the other 7,500,000, who had lost two and won one,
half win the fourth game, and so give 3,750,000 more
who have lost two games and won two: thus in all we
have 7,500,000 who have had equal fortunes. The
others lose at the fourth trial, and give us 3,500,000 to
be added to the 1,250,000 already counted, who have
lost thrice and won once only.

At the close, then, of the fourth encounter, we find
a million and a quarter of players who have been constantly
lucky, and as many who have been constantly
unlucky. Five millions, having won three games out
of four, consider themselves to have better luck than
the average; while as many, having lost three games
out of four, regard themselves as unlucky. Lastly, we
have seven millions and a half who have won and lost
in equal degree. These, it will be seen, constitute the
largest part of our gambling community, though not
equal to the other classes taken together. They are, in
fact, three-eighths of the entire community.

So we might proceed to consider the twenty millions
of gamblers after a fifth encounter, a sixth, and so
on. Nor is there any difficulty in dealing with the
matter in that way. But a sort of account must be
kept in proceeding from the various classes considered
in dealing with the fourth encounter to those resulting
from the fifth, from these to those resulting from the
sixth, and so on. And although the accounts thus
requiring to be drawn up are easily dealt with, the
little sums (in division by two, and in addition) would
not present an appearance suited to these pages. I
therefore now proceed to consider only the results,
or rather such of the results as bear most upon my
subject.

After the fifth encounter there would be (on the assumption
of results being always exactly balanced, which
is convenient, and quite near enough to the truth for our
present purpose) 625,000 persons who would have won
every game they had played, and as many who had lost
every game. These would represent the persistently
lucky and unlucky men of our gambling community.
There would be 625,000 who, having won four times in
succession, now lost, and as many who, having lost four
times in succession, now won. These would be the
examples of luck---good or bad---continued to a certain
stage, and then changing. The balance of our 20,000,000,
amounting to seventeen millions and a half, would have
had varying degrees of luck, from those who had won
four games (not the first four) and lost one, to those
who had lost four games (not the first four) and won
but a single game. The bulk of the seventeen millions
and a half would include those who would have had no
reason to regard themselves as either specially lucky or
specially unlucky. But 1,250,000 of them would be
regarded as examples of a change of luck, being 625,000
who had won the first three games and lost the remaining
two, and as many who had lost the first three games
and won the last two.

Thus, after the fifth game, there would be only
1,250,000 of those regarded (for the nonce) as persistently
lucky or unlucky (as many of one class as of the
other), while there would be twice as many who would
be regarded by those who knew of their fortunes, and
of course by themselves, as examples of change of luck,
marked good or bad luck at starting, and then bad or
good luck.

So the games would proceed, half of the persistently
lucky up to a given game going out of that class at the
next game to become examples of a change of luck, so
that the number of the persistently lucky would rapidly
diminish as the play continued. So would the number
of the persistently unlucky continually diminish, half
going out at each new encounter to join the ranks of
those who had long been unlucky, but had at last experienced
a change of fortune.

After the twentieth game, if we suppose constant
exact halving to take place as far as possible, and then
to be followed by halving as near as possible, there
would be about a score who had won every game of the
twenty. No amount of reasoning would persuade these
players, or those who had heard of their fortunes, that
they were not exceedingly lucky persons---not in the
sense of being lucky because they \textit{had} won, but of
being \textit{likelier to win} at any time than any of those who
had taken part in the twenty games. They themselves
and their friends---ay, and their enemies too---would
conclude that they `\text{could} not lose.' In like manner,
the score or so who had not won a single game out of
the twenty would be judged to be most unlucky persons,
whom it would be madness to back in any matter of
pure chance.

Yet---to pause for a moment on the case of these
apparently most manifest examples of persistent luck---the
result we have obtained has been to show that
inevitably there must be in a given number of trials
about a score of these cases of persistent luck, good or
bad, and about two score of cases where both good and
bad are counted together. We have shown that, without
imagining any antecedent luckiness, good or bad, there
must be what, to the players themselves, and to all who
heard of or saw what had happened to them, would
seem examples of the most marvellous luck. Supposing,
as we have, that the game is one of pure chance, so that
skill cannot influence it and cheating is wholly prevented,
all betting men would be disposed to say, `These
twenty are persons whose good luck can be depended
on; we must certainly back them for the next game:
and those other twenty are hopelessly unlucky; we may
lay almost any odds against their winning.'

But it should hardly be necessary to say that that
which \textit{must} happen cannot be regarded as due to luck.
There must be \textit{some} set of twenty or so out of our twenty
millions who will win every game of twenty; and the
circumstance that this has befallen such and such persons
no more means that they are lucky, and is no more
a matter to be marvelled at, than the circumstance that
one person has drawn the prize ticket out of twenty
at a lottery is marvellous, or signifies that he would be
always lucky in lottery drawing.

The question whether those twenty persons who
had so far been persistently lucky would be better
worth backing than the rest of the twenty millions,
and especially than the other twenty who had persistently
lost, would in reality be disposed of at the
twenty-first trial in a very decisive way: for of the
former score about half would lose, while of the latter
score about half would win. Among a thousand persons
who had backed the former set at odds there would
be a heavy average of loss; and the like among a thousand
persons who had laid against the latter set at
odds.

It may be said this is assertion only, that experience
shows that some men are lucky and others unlucky at
games or other matters depending purely on chance,
and it must be safer to back the former and to wager
against the latter. The answer is that the matter has
been tested over and over again by experience, with the
result that, as \textit{\`a priori} reasoning had shown, some men
are bound to be fortunate again and again in any
great number of trials, but that these are no more likely
to be fortunate on fresh trials than others, including
those who have been most unfortunate. The success of
the former shows only that they \textit{have been}, not that they
\textit{are} lucky; while the failure of the others shows that
they \textit{have} failed, nothing more.

An objection will---about here---have vaguely presented
itself to believers in luck, viz.\ that, according to
the doctrine of the `maturity of the chances,' which
must apply to the fortunes of individuals as well as to
the turn of events, one would rather expect the twenty
who had been so persistently lucky to lose on the
twenty-first trial, and the twenty who had lost so long
to win at last in that event. Of course, if gambling
superstitions might equally lead men to expect a change
of luck and continuance of luck unchanged, one or
other view might fairly be expected to be confirmed by
events. And on a single trial one or other event---that
is, a win or a loss---\textit{must} come off, greatly to the gratification
of believers in luck. In one case they could say,
`I told you so, such luck as A's was bound to pull him
through again'; in the other, `I told you so, such luck
was bound to change': or if it were the loser of twenty
trials who was in question, then, `I told you so, he was
bound to win at last'; or, `I told you so, such an unlucky
fellow was bound to lose.' But unfortunately,
though the believers in luck thus run with the hare and
hunt with the hounds, though they are prepared to find
any and every event confirming their notions about luck,
yet when a score of trials or so are made, as in our supposed
case of a twenty-first game, the chances are that
they would be contradicted by the event. The twenty
constant winners would not be more lucky than the
twenty constant losers; but neither would they be less
lucky. The chances are that about half would win and
about half would lose. If one who really understands
the laws of probability could be supposed foolish enough
to wager money on either twenty, or on both, he
would unquestionably regard the betting as perfectly
even.

Let us return to the rest of our twenty millions of
players, though we need by no means consider all the
various classes into which they may be divided, for the
number of these classes amounts, in fact, to more than
a million.

The great bulk of the twenty millions would consist
of players who had won about as many games as they
had lost. The number who had won \textit{exactly} as many
games as they had lost would no longer form a large
proportion of the total, though it would form the largest
individual class. There would be nearly 3,700,000 of
these, while there would be about 3,400,000 who had
won eleven and lost nine, and as many who had won
nine and lost eleven; these two classes together would
outnumber the winners of ten games exactly, in the
proportion of 20 to 11 or thereabouts. Speaking generally,
it may be said that about two-thirds of the community
would consider they had had neither good luck nor
bad, though their opinion would depend on temperament
in part. For some men are more sensitive to losses than
to gains, and are ready to speak of themselves as unlucky,
when a careful examination of their varying fortunes
shows that they have neither won nor lost on the whole,
or have won rather more than they have lost. On the
other hand, there are some who are more exhilarated by
success than dashed by failure.

The number of those who, having begun with good
luck, had eventually been so markedly unfortunate,
would be considerable. It might be taken to include
all who had won the first six games and lost all the rest,
or who had won the first seven or the first eight, or any
number up to, say, the first fourteen, losing thence to
the end; and so estimated would amount to about 170,
an equal number being first markedly unfortunate, and
then constantly fortunate. But the number who had
experienced a marked change of luck would be much
greater if it were taken to include all who had
won a large proportion of the first nine or ten games
and lost a large proportion of the remainder, or 
\textit{vice vers\^a}. These two classes of players would 
be well represented.

Thus, then, we see that, setting enough persons
playing at any game of pure chance, and assuming
only that among any large number of players there
will be about as many winners as losers, irrespective of
luck, good or bad, all the five classes which gambling
folk recognise and regard as proving the existence of
luck, must inevitably make their appearance.

Even any special class which some believer in luck,
who was more or less fanciful, imagined he had recognised
among gambling folk, must inevitably appear
among our twenty millions of illustrative players. For
example, there would be about a score of players who
would have won the first game, lost the second, won
the third, and so on alternately to the end; and as
many who had also won and lost alternate games, but
had lost the first game; some forty, therefore, whose
fortune it seemed to be to win only after they had lost
and to lose only after they had won. Again, about
twenty would win the first five games, lose the next
five, win the third five, and lose the last five; and
about twenty more would lose the first five, win the
next, lose the third five, and win the last five: about
forty players, therefore, who seemed bound to win and
lose always five games, and no more, in succession.

Again, if anyone had made a prediction that among
the players of the twenty games there would be one
who would win the first, then lose two, then win three,
then lose four, then win five, and then lose the remaining
five---and yet a sixth if the twenty-first game were
played---that prophet would certainly be justified by
the result. For about a score would be sure to have
just such fortunes as he had indicated up to the
twentieth game, and of these, nine or ten would be
(practically) sure to win the twenty-first game also.

We see, then, that all the different kinds of luck---good,
bad, indifferent, or changing---which believers in
luck recognise, are bound to appear when any considerable
number of trials are made; and all the varied
ideas which men have formed respecting fortune and
her ways are bound to be confirmed.

It may be asked by some whether this is not proving
that there is such a thing as luck instead of over-throwing
the idea of luck. But such a question can
only arise from a confusion of ideas as to what is meant
by luck. If it be merely asserted that such and such
men have been lucky or unlucky, no one need dispute
the proposition; for among the millions of millions of
millions of purely fortuitous events affecting the millions
of persons now living, it could not but chance that
the most remarkable combinations, sequences, alternations,
and so forth, of events, lucky or unlucky, must
have presented themselves in the careers of hundreds.
Our illustrative case, artificial though it may seem, is
in reality not merely an illustration of life and its
chances, but may be regarded as legitimately demonstrating
what must inevitably happen on the wider
arena and amid the infinitely multiplied vicissitudes of
life. But the belief in luck involves much more. The
idea involved in it, if not openly expressed (usually
expressed very freely), is that some men are lucky by
nature, others unlucky, that such and such times and
seasons are lucky or unlucky, that the progress of
events may be modified by the lucky or unlucky influence
of actions in no way relating to them; as, for
instance, that success or failure at cards may be affected
by the choice of a seat, or by turning round thrice in
the seat. This form of belief in luck is not only akin
to superstition, it \textit{is} superstition. Like all superstition,
it is mischievous. It is, indeed, the very essence of the
gambling spirit, a spirit so demoralising that it blinds
men to the innate immorality of gambling. It is this
belief in luck, as something which can be relied on, or
propitiated, or influenced by such and such practices,
which is shown, by reasoning and experience alike, to
be entirely inconsistent not only with facts but with
possibility.

But oddly enough, the believers in luck show by
the form which their belief takes that in reality they
have no faith in luck any more than men really have
faith in superstitions which yet they allow to influence
their conduct. A superstition is an idle dread, or an
equally idle hope, not a real faith; and in like manner
is it with luck. A man will tell you that at cards, for
instance, he always has such and such luck; but if you
say, `Let us have a few games to see whether you will
have your usual luck,' you will usually find him unwilling
to let you apply the test. If you try it, and
the result is unfavourable, he argues that such peculiarities
of luck never do show themselves when submitted
to test. On the other hand, if it so chances
that on that particular occasion he has the kind of
luck which he claims to have \textit{always}, he expects you
to accept the evidence as decisive. Yet the result
means in reality only that certain events, the chances
for and against which were probably pretty equally
divided, have taken place.

So, if a gambler has the notion (which seems to the
student of science to imply something little short of
imbecility of mind) that turning round thrice in his
chair will change the luck, he is by no means corrected
of the superstition by finding the process fail on any
particular occasion. But if the bad luck which has
hitherto pursued him chances (which it is quite as
likely to do as not) to be replaced by good or even by
moderate luck, after the gambler has gone through the
mystic process described, or some other equally absurd
and irrelevant man{\oe}uvre, then the superstition is confirmed.
Yet all the time there is no real faith in it.
Such practices are like the absurd invocation of Indian
`medicine men'; there is a sort of vague hope that
something good may come of them, no real faith in
their efficacy.

The best proof of the utter absence of real faith in
superstitions about luck, even among gambling men,
the most superstitious of mankind, may be found in the
incongruity of their two leading ideas. If there are
two forms of expression more frequently than any
others in the mouth of gambling men, they are those
which relate to being in luck or out of luck on the one
hand, and to the idea that luck must change on the
other. Professional gamblers, like Steinmetz and his
kind, have become so satisfied that these ideas are
sound, whatever else may be unsound, in regard to
luck, that they have invented technical expressions to
present these theories of theirs, failing utterly to notice
that the ideas are inconsistent with each other, and
cannot both be right---though both may be wrong, and
are so.

A player is said to be `in the vein' when he has
for some time been fortunate. He should only go on
playing, if he is wise, at such a time, and at such a
time only should he be backed. Having been lucky he
is likely, according to this notion, to continue lucky.
But, on the other hand, the theory called `the maturity
of the chances' teaches that the luck cannot continue
more than a certain time in one direction; when it has
reached maturity in that direction it must change.
Therefore, when a man has been `in the vein' for a
certain time (unfortunately no Steinmetz can say precisely
how long), it is unsafe to back him, for he must
be on the verge of a change of luck.

Of course the gambler is confirmed in his superstition,
whichever event may befall in such cases. When
he wins he applauds himself for following the luck, or
for duly anticipating a change of luck, as the case may
be; when he loses, he simply regrets his folly in not
seeing that the luck must change, or in not standing
by the winner.

And with regard to the idea that luck must change,
and that in the long run events must run even, it is
noteworthy how few gambling men recognise either, on
the one hand, how inconsistent this idea is with their
belief in luck which may be trusted (or, in their slang,
may be safely backed), or, on the other hand, the real
way in which luck `comes even' after a sufficiently
long run.

A man who has played long with success goes on
because he regards himself as lucky. A man who has
played long without success goes on because he considers
that the luck is bound to change. The latter goes on
with the idea that, if he only plays long enough, he
must at least at some time or other recover his losses.

Now there can be no manner of doubt that if a man,
possessed of sufficient means, goes on playing for a very
long time, his gains and losses will eventually be very
nearly equal; assuming always, of course, that he is not
swindled---which, as we are dealing with gambling
men, is perhaps a sufficiently bold assumption. Yet it
by no means follows that, if he starts with considerable
losses, he will ever recover the sum he has thus had to
part with, or that his losses may not be considerably
increased. This sounds like a paradox; but in reality
the real paradox lies in the opposite view.

This may be readily shown.

The idea to be controverted is this: that if a gambler
plays long enough there must come a time when
his gains and his losses are exactly balanced. Of
course, if this were true, it would be a very strong
argument against gambling; for what but loss of time
can be the result of following a course which must
inevitably lead you, if you go on long enough, to the
place from which you started? But it is not true. If
it were true, of course it involves the inference that, no
matter when you enter on a course of gambling, you
are bound after a certain time to find yourself where
you were at that beginning. It follows that if (which
is certainly possible) you lose considerably in the first
few weeks or months of your gambling career, then, if
you only play long enough you must inevitably find
yourself as great a loser, on the whole, as you were when
you were thus in arrears through gambling losses; for
your play may be quite as properly considered to have
begun when those losses had just been incurred, as to
have begun at any other time. Hence this idea that,
in the long run, the luck must run even, involves the
conclusion that, if you are a loser or a gainer in the
beginning of your play, you must at some time or other
be equally a gainer or loser. This is manifestly inconsistent
with the idea that long-continued play will
inevitably leave you neither a loser nor a gainer. If,
starting from a certain point when you are a thousand
pounds in arrears, you are certain some time or other,
if you only play long enough, to have gained back that
thousand pounds, it is obvious that you are equally
certain some time or other (from that same starting-point)
to be yet another thousand pounds in arrears.
For there is no line of argument to prove you must
regain it, which will not equally prove that some time
or other you must be a loser by that same amount, over
and above what you had already lost when beginning
the games which were to put you right. If, then, you
are to come straight, you must be able certainly to
recover two thousand pounds, and by parity of reasoning
four thousand, and again twice that; and so on 
\textit{ad infinitum}: which is manifestly absurd.

The real fact is, that while the laws of probabilities
do undoubtedly assure the gambler that his losses and
gains will in the long run be nearly equal, the kind of
equality thus approached is not an equality of actual
amount, but of proportion. If two men keep on tossing
for sovereigns, it becomes more and more unlikely,
the longer they toss, that the difference between them
will fall short of any given sum. If they go on till
they have tossed twenty million times, the odds are
heavily in favour of one or the other being a loser of at
least a thousand pounds. But the proportion of the
amount won by one altogether, to the amount won altogether
by the other, is almost certain to be very nearly
a proportion of equality. Suppose, for example, that at
the end of twenty millions of tossings, one player is a
winner of 1,000\textit{l}., then he must have won in all
10,000,500\textit{l}., the other having won in all 9,999,500\textit{l}.
the ratio of these amounts is that of 100005 to 99995,
or 20001 to 19999. This is very nearly the ratio of
10000 to 9999, or is scarcely distinguishable, practically,
from actual equality. Now if these men had only
tossed eight times for sovereigns, it might very well
have happened that one would have won five or six
times, while the other had only won thrice or twice.
Yet with a ratio of 5 to 3, or 3 to 1, against the loser,
he would actually be out of pocket only 2\textit{l}. in one case
and 4\textit{l}. in the other; while in the other case, with a
ratio of almost perfect equality, he would be the loser
of a thousand pounds.

But now it might appear that, after all, this is
proving too much, or, at any rate, proves as much on
one side as on the other; for if one player loses the
other must gain; if a certain set of players lose the
rest gain: and it might seem as though, with the prevalent
ideas of many respecting gambling games, the
chance of winning were a sufficient compensation for
the chance of losing.

Where a man is so foolish that the chance of having
more money than he wants is equivalent in his mind
(or what serves him for a mind) to the risk of being
deprived of the power of getting what is necessary for
himself and for his family, such reasoning may be
regarded as convincing. For those who weigh their
wants and wishes rightly, it has no value whatever.
On the contrary it may be shown that every wager or
gambling transaction, by a man of moderate means,
definitely reduces the actual value of his possessions,
even if the wager or transaction be a fair one. If a
man who has a hundred pounds available to meet his
present wants wagers 50\textit{l}. against 50\textit{l}., or an equal
chance, he is no longer worth 100\textit{l}. He \textit{may}, when
the bet is decided, be worth 150\textit{l}., or he may be worth
only 50\textit{l}. All he can \textit{estimate} his property at is about
87\textit{l}. Supposing the other man to be in the same position,
they are both impoverished as soon as they have
made the bet; and when the wager is decided, the
average value of their possessions in ready money is less
than it was; for the winner gains less by having his
100\textit{l}. raised to 150\textit{l}. (or increased as 2 to 3), than the
loser suffers by having his ready money halved.

Similar remarks apply to participation in lottery
schemes, or the various forms of gambling at places
like San Carlo. Every sum wagered means, at the
moment when it is staked, a depreciation of the gambler's
property; and would mean that, even if the terms
on which the wagering were conducted were strictly fair.
But this is never the case. In all lotteries and in all
established systems of gambling certain odds are always
retained in favour of those who work the lottery or
the gambling system. These odds make gambling in
either form still more injurious to those who take part
in it. Winners of course there are, and in some few
cases winners may retain a large part of their gains, or
at any rate expend them otherwise than in fresh gambling.
Yet it is manifest that, apart from the circumstance
that the \textit{effects} of the gambling gains of one set
of persons never counterbalance the \textit{effects} of the gambling
losses of others, there is always a large deduction
to be made on account of the wild and reckless waste
of money won by gambling. In many cases, indeed,
large gambling gains have brought ruin to the unfortunate
winner: set `on horseback' by lightly acquired
wealth, and unaccustomed to the position, he has ridden
`straightway to the devil.'

But the greed for chance-won wealth is so great
among men of weak minds, and they are so large a
majority of all communities, that the bait may be
dangled for them without care to conceal the hook. In
all lotteries and gambling systems which have yet been
known the hook has been patent, and the evil it must
do if swallowed should have been obvious. Yet it has
been swallowed greedily.

A most remarkable illustration of the folly of those
who trust in luck, and the cool audacity of those who
trust in such folly, with more reason but with more
rascality, is presented by the Louisiana Lottery in
America. This is the only lottery of the kind now
permitted in America. Indeed, it is nominally restricted
to the State of Louisiana; but practically the whole
country takes part in it, tickets being obtainable by
residents in every State of the Union. The peculiarity
of the lottery is \textit{the calm admission, in all advertisements,
that it is a gross and unmitigated swindle}. The
advertisements announce that each month 100,000
tickets will be sold, each at five dollars, shares of one-fifth
being purchasable at one dollar. Two commissioners---Generals
Early and Beauregard---control the
drawings; so that we are told, and may well believe,
the drawings are conducted with fairness and honesty,
and in good faith to all parties. So far all is well. We
see that each month, if all the tickets are sold, the sum
of 500,000 dols.\ will be paid in. From this monthly
payment we must deduct 1,000 dols.\ paid to each, of the
commissioners, and perhaps some 3,000 dols.\ at the outside
for advertising. We may add another sum of
5,000 dols.\ for incidental expenses, machinery, sums
paid to agents as commission on the sale of tickets, and
so forth. This leaves 490,000 dols.\ monthly if all the
tickets are sold. And as the lottery is `incorporated by
the State Legislature of Louisiana for charitable and
educational purposes,' we may suppose that a certain
portion of the sum paid in monthly will be set aside to
represent the proceeds of the concern, and justify the
use of so degrading a method of obtaining money.
Probably it might be supposed that 24 per cent.\ per
annum, or 2 per cent.\ per month, would be a fair return
in this way, the system being entirely free from risk.
This would amount to 9,800 dols., or say 10,000 dols.,
monthly. Those who manage the lottery are not content,
however, with any such sum as this, which would
leave 480,000 dols.\ to be distributed in prizes. They
distribute 215,000 dols.\ less, the total amount given in
prizes amounting to only 265,000 dols. If the 100,000
tickets are all sold---and it is said that few are ever left---the
monthly profit on the transaction is not less than
225,000 dols., or 45 per cent.\ on the total amount
received per month. This would correspond to 540
per cent.\ per annum if it were paid on a capital of
500,000 dols. But in reality it amounts to much more,
as the lottery company runs no risk whatsoever. The
Louisiana Lottery is a gross swindle, besides being disreputable
in the sense in which all lotteries are so.
What would be thought if a man held an open lottery,
to which each of one hundred persons admitted paid
5\textit{l}., and taking the sum of 500\textit{l}. thus collected, were to
say: `The lottery, gentlemen gamblers, will now proceed;
265\textit{l}. of the sum before me I will distribute in
prizes, as follows' (indicating the number of prizes and
their several amounts); `the rest, this sum of 235\textit{l}.,
which I have here separated, I will put into my own
pocket' (suiting the action to the word) `for my trouble
in getting up this lottery'? The Louisiana Lottery is
a transaction of the same rascally type---not rendered
more respectable by being on a very much larger scale.
If the spirit of rash speculation will let men submit
to swindling so gross as this, we can scarcely see any
limit to its operation. Yet hundreds of thousands yield
to the temptation thus offered, to gain suddenly a large
sum, at the expense of a small sum almost certainly lost,
and partly stolen.

It should be known---though, perhaps, even this
knowledge would not keep the moths away from the
destruction to which they seem irresistibly lured---that
gambling carried on long enough is not probable but
certain ruin. There is no sum, however large, which is
not certain to be absorbed at some time in the continuance
of a sufficiently long series of trials, even at fair
risks. Gamblers with moderate fortunes overlook this.
In their idea, mistaken as it is, that luck must run even
at last, they forget that, before that last to which they
look has been reached, their last shilling may have
gone. If they were content even to stay till---possibly---gain
balanced loss, there would be some chance of escape.
But what real gambler ever was content with such an
aim as that? Luck must not only turn till loss has
been recouped, but run on till great gains have been
made. And no gambler was ever yet content to stay
his hand when winning, or to give up when he began
to lose again. The fatal faith in eventual good luck is
the source of all bad luck; it is in itself the worst luck
of all. Every gambler has this faith, and no gambler
who holds to it is likely long to escape ruin.

\chapter{Gamblers' Fallacies}

It might be supposed that those who are most familiar
with the actual results which present themselves in long
series of chance games would form the most correct
views respecting the conditions on which such results
depend---would be, in fact, freest from all superstitious
ideas respecting chance or luck. The gambler who
sees every system---his own infallible system included---foiled
by the run of events, who witnesses the discomfiture
of one gamester after another that for a time
had seemed irresistibly lucky, and who can number by
hundreds those who have been ruined by the love of
play, might be expected to recognise the futility of all
attempts to anticipate the results of chance combinations.
It is, however, but too well known that the
reverse is the case. The more familiar a man becomes
with the multitude of such combinations, the more
confidently he believes in the possibility of foretelling---not,
indeed, any special event, but---the general run
of several approaching events. There has never been a
successful gambler who has not believed that his success
(temporary though such success ever is, where games
of pure chance are concerned) has been the result of
skilful conduct on his own part; and there has never been
a ruined gambler (though ruined gamblers are to be
counted by thousands) who has not believed that when
ruin overtook him he was on the very point of mastering
the secret of success. It is this fatal confidence which
gives to gambling its power of fascinating the lucky as
well as the unlucky. The winner continues to tempt
fortune, believing all the while that he is exerting
some special aptitude for games of chance, until the
inevitable change of luck arrives; and thereafter he
continues to play because he believes that his luck has
only deserted him for a time, and must presently return.
The unlucky gambler, on the contrary, regards his
losses as sacrifices to ensure the ultimate success of his
`system,' and even when he has lost his all, continues
firm in the belief that had he had more money to
sacrifice he could have bound fortune to his side for
ever.

I propose to consider some of the most common
gambling superstitions---noting, at the same time, that
like superstitions prevail respecting chance events (or
what is called fortune) even among those who never
gamble.

Houdin, in his interesting book, \textit{Les Tricheries des
Grecs d{\'e}voil{\'e}es}, has given some amusing instances of
the fruits of long gaming experience. `They are presented,'
says Steinmetz, from whose work, \textit{The Gaming
Table}, I quote them, `as the axioms of a professional
gambler and cheat.' Thus we might expect that,
however unsatisfactory to men of honest mind, they
would at least savour of a certain sort of wisdom. Yet
these axioms, the fruit of long study directed by self-interest,
are all utterly untrustworthy.

`Every game of chance,' says this authority, `presents
two kinds of chances that are very distinct---namely,
those relating to the person interested, that is
the player; and those inherent in the combinations of
the game.' That is, we are to distinguish between the
chances proper to the game, and those depending on the
luck of the player. Proceeding to consider the chances
proper to the game itself, our friendly cheat sums them
all up in two rules. First:---`Though chance can bring
into the game all possible combinations, there are nevertheless
certain limits at which it seems to stop: such,
for instance, as a certain number turning up ten times
in succession at roulette; this is possible, but it has
never happened.' Secondly:---`In a game of chance, the
oftener the same combination has occurred in succession,
the nearer we are to the certainty that it will not
recur at the next cast or turn up. This is the most
elementary of the theories on probabilities; it is termed
`\textit{the maturity of the chances}' (and he might have added
that the belief in this elementary theory had ruined
thousands). `Hence,' he proceeds, `a player must
come to the table not only ``in luck,'' but he must not
risk his money except at the instant prescribed by the
rules of the maturity of the chances.' Then follow the
precepts for personal conduct:---`For gaming prefer
roulette, because it presents several ways of staking
your money---which permits the study of several. A
player should approach the gaming-table perfectly calm
and cool---just as a merchant or tradesman in treaty
about any affair. If he gets into a passion it is all
over with prudence, all over with good luck---for the
demon of bad luck invariably pursues a passionate
player. Every man who finds a pleasure in playing
runs the risk of losing.\footnote{This \textit{na\"ive} admission would appear, as we shall presently see,
to have been the fruit of genuine experience on our gambler's part:
it only requires that, for the words `runs the risk,' we should read
`incurs the certainty,' to be incontrovertible.}
A prudent player, before
undertaking anything, should put himself to the test to
discover if he is `in vein' or in luck. In all doubt
he should abstain. There are several persons who are
constantly pursued by bad luck: to such I say---\textit{never
play}. Stubbornness at play is ruin. Remember that
Fortune does not like people to be overjoyed at her
favours, and that she prepares bitter deceptions for the
imprudent who are intoxicated by success. Lastly,
before risking your money at play, study your `vein,'
and the different probabilities of the game---termed, as
aforesaid, the `maturity of the chances.'

Before proceeding to exhibit the fallacy of the principles
here enunciated---principles which have worked
incalculable mischief---it may be well to sketch the
history of the scamp who enunciated them---so far, at
least, as his gambling successes are concerned. His
first meeting with Houdin took place at a subscription
ball, where he managed to fleece Houdin `and
others to a considerable amount, contriving a dexterous
escape when detected. Houdin afterwards fell in with
him at Spa, where he found the gambler in the greatest
poverty, and lent him a small sum---to practise his
grand theories.' This sum the gambler lost, and
Houdin advised him `to take up a less dangerous occupation.'
It was on this occasion, it would seem, that
the gambler revealed to Houdin the particulars recorded
in his book. `A year afterwards Houdin unexpectedly
fell in with him again; but this time the
fellow was transformed into what is called a ``\textit{demi-millionaire},''
having succeeded to a large fortune on
the death of his brother who died intestate. According
to Houdin, the following was the man's declaration
at the auspicious meeting: ``I have,'' he said, ``completely
renounced gaming; I am rich enough; and
care no longer for fortune. And yet,'' he added proudly,
``if I now cared for the thing, how I could break those
bloated banks in their pride, and what a glorious
vengeance I could take of bad luck and its inflexible
agents! But my heart is too full of my happiness to
allow the smallest place for the desire of vengeance.'''
Three years later he died; and Houdin informs us that
he left the whole of his fortune to various charitable
institutions, his career after his acquisition of wealth
going far to demonstrate the justice of Becky Sharp's
theory that it is easy to be honest on five thousand a
year.

It is remarkable that the principles enunciated above
are not merely erroneous, but self-contradictory. Yet
it is to be noticed that though they are presented as
the outcome of a life of gambling experiences, they
are in reality entertained by all gamblers, however
limited their experience, as well as by many who are
only prevented by the lack of opportunity from entering
the dangerous path which has led so many to
ruin. These contradictory superstitions may be called
severally---the gambler's belief in his own good luck,
and his faith in the turn of luck. When he is considering
his own fortune he does not hesitate to believe
that on the whole the Fates will favour him, though
this belief implies in reality the \textit{persistence} of favourable
conditions. On the contrary, when he is considering
the fortunes of others who are successful in their play
against him, he does not doubt that their good luck
will presently desert them, that is, he believes in the
\textit{non-persistence} of favourable conditions in their case.

Taking in their order the gambling superstitions
which have been presented above, we have, first of all,
to inquire what truth there is in the idea that there
are limits beyond which pure chance has no power of
introducing peculiar combinations. Let us consider
this hypothesis in the light of actual experience. Mr.
Steinmetz tells us that, in 1813, a Mr. Ogden wagered
1,000 guineas to one that `seven' would not be thrown
with a pair of dice ten successive times. The wager
was accepted (though it was egregiously unfair), and
strange to say his opponent threw `seven' \textit{nine times
running}. At this point Mr. Ogden offered 470 guineas
to be off the bet. But his opponent declined (though
the price offered was far beyond the real value of his
chance). He cast yet once more, and threw `nine,' so
that Mr. Ogden won his guinea.

Now here we have an instance of a most remarkable
series of throws, the like of which has never been
recorded before or since. Before those throws had
been made, it might have been asserted that the
throwing of nine successive `sevens' with a pair of
dice was a circumstance which chance could never
bring about, for experience was as much against such
an event as it would seem to be against the turning up
of a certain number ten successive times at \textit{roulette}.
Yet experience now shows that the thing is possible;
and if we are to limit the action of chance, we must
assert that the throwing of `seven' \textit{ten} times in succession
is an event which will never happen. Yet such
a conclusion obviously rests on as unstable a basis as
the former, of which experience has disposed. Observe,
however, how the two gamblers viewed this very eventuality.
Nine successive `sevens' had been thrown;
and if there were any truth in the theory that the
power of chance was limited, it might have been regarded
as all but certain that the next throw would
not be a `seven.' But a run of bad fortune had so
shaken Mr. Ogden's faith in his luck (as well as in the
theory of the `maturity of the chances') that he was
ready to pay 470 guineas (nearly thrice the mathematical
value of his opponent's chance) in order to
save his endangered thousand; and so confident was
his opponent that the run of luck would continue that
he declined this very favourable offer. Experience had
in fact shown both the players, that although `sevens'
could not be thrown for ever, yet there was no saying
when the throw would change. Both reasoned probably
that as an eighth throw had followed seven successive
throws of `seven' (a wonderful chance), and as a ninth
had followed eight successive throws (an unprecedented
event), a tenth might well follow the nine (though
hitherto no such series of throws had ever been heard
of). They were forced as it were by the run of events
to reason justly as to the possibility of a tenth throw
of `seven'---nay, to exaggerate that possibility into
probability; and it appears from the narrative that the
strange series of throws quite checked the betting propensities
of the bystanders, and that not one was led to
lay the wager (which according to ordinary gambling
superstitions would have been a safe one) that the tenth
throw would not give `seven.'

We have spoken of the unfairness of the original
wager. It may interest our readers to know exactly
how much should have been wagered against a single
guinea, that ten `sevens' would not be thrown. With
a pair of dice there are thirty-six possible throws, and
six of these give `seven' as the total. Thus the chance
of throwing `seven' is one sixth, and the chance of
throwing `seven' ten times running is obtained by
multiplying six into itself ten times, and placing the
resulting number under unity, to represent the minute
fractional chance required. It will be found that the
number thus obtained is 60,466,176, and instead
of 1,000 guineas, fairness required that 60,466,175
guineas should have been wagered against one guinea,
so enormous are the chances against the occurrence of
ten successive throws of `seven.' Even against nine
successive throws the fair odds would have been
10,077,595 to one, or about forty thousand guineas to
a farthing. But when the nine throws of `seven' had
been made, the chance of a tenth throw of `seven' was
simply one-sixth as at the first trial. If there were
any truth in the theory of the `maturity of the chances,'
the chance of such a throw would of course be greatly
diminished. But even taking the mathematical value
of the chance, Mr. Ogden need in fairness only have
offered a sixth part of 1,001 guineas (the amount of the
stakes), or 166 guineas 17\textit{s}. 6\textit{d}., to be off his wager.
So that his opponent accepted in the first instance an
utterly unfair offer, and refused in the second instance a
sum exceeding by more than three hundred guineas the
real value of his chance.

Closely connected with the theory about the range
of possibility in the matter of chance combinations, is
the theory of the maturity of the chances---`the most
elementary of the theories on probabilities.' It might
safely be termed the most mischievous of gambling
superstitions.

As an illustration of the application of this theory,
we may cite the case of an Englishman, once well
known at foreign gambling-tables, who had based a
system on a generalisation of this theory. In point of
fact the theory asserts that when there has been a run
in favour of any particular event, the chances in favour
of the event are reduced, and therefore, necessarily,
the chances in favour of other events are increased.
Now our Englishman watched the play at the \textit{roulette}
table for two full hours, carefully noting the numbers
which came up during that time. Then, eschewing
those numbers which had come up oftenest, he staked
his money on those which had come up very seldom or
not at all. Here was an infallible system according
to `the most elementary of the theories of probability.'
The tendency of chance-results to right themselves, so
that events equally likely in the first instance will
occur an equal number of times in the long run, was
called into action to enrich our gambler and to ruin
the unlucky bankers. Be it noted, in passing, that
events do thus right themselves, though this circumstance
does not operate quite as the gambler supposed,
and cannot be trusted to put a penny into any one's
pocket. The system was tried, however, and instead
of reasoning respecting its soundness, we may content
ourselves with recording the result. On the first day
our Englishman won more than seven hundred pounds
in a single hour. `His exultation was boundless. He
thought he had really discovered the ``philosopher's
stone.'' Off he went to his bankers, and transmitted
the greater portion of his winnings to London. The
next day he played and lost fifty pounds; and the
following day he achieved the same result, and had to
write to town for remittances. In fine, in a week he
had lost all the money he won at first, with the
exception of fifty pounds, which he reserved to take
him home; and being thoroughly convinced of the
exceeding fickleness of fortune, he has never staked a
sixpence since, and does all in his power to dissuade
others from playing.'\footnote{From an interesting paper entitled `Le Jeu est fait,' in
\textit{Chambers's Journal}.}

He took a very sound principle of probabilities as
the supposed basis of his system, though in reality he
entirely mistook the nature of the principle. That
principle is, that where the chances for one or another
of two results are equal for each trial, and many trials
are made, the number of events of one kind will bear to
those of the other kind a very nearly equal ratio: the
greater the number of events, the more nearly will the
ratio tend to equality. This is perfectly true; and
nothing could be safer than to wager on this principle.
Let a man toss a coin for an hour, and I would wager
confidently that neither will `heads' exceed `tails,' or
`tails' exceed `heads' in a greater ratio than that of
21 to 20. Let him toss for a day, and I would wager
as confidently that the inequality will not be greater
than that represented by the ratio of 101 to 100. Let
the tossing be repeated day after day for a year, and I
would wager my life that the disproportion will be less
than that represented by the ratio of 1,001 to 1,000.
Yet so little does this principle bear the interpretation
placed upon it by the inventor of the system above
described, that if on any occasion during this long-continued
process of tossings `head' had been tossed (as it
certainly would often be) no less than twenty times in
succession, I would not wager a sixpence on the next
tossing giving `tail,' or trust a sixpence to the chance of
`tail' appearing oftener than `head' in the next five, ten,
or twenty tossings. Not only should reason show the
utter absurdity of supposing that a tossing, or a set of
five, ten, or twenty tossings, can be affected one way or
the other by past tossings, whether proximate or remote;
but the experiment has been tried, and it has appeared
(as might have been known beforehand) that after any
number of cases in which `heads' (say) have appeared
such and such a number of times in succession, the next
tossing has given `heads' as often as it has given `tails.'
Thus, in 124 cases, Buffon, in his famous tossing trial,
tossed `tails' four times running. On the next trial,
in these 124 cases, `head' came 56 times and `tail' 68
times. So most certainly the tossing of `tail' four
times running had not diminished the tendency towards
`tail' being tossed. Among the 68 cases which had
thus given `tail' five times running, 29 failed to give
another `tail,' while the remaining 39 gave another,
that is, a sixth `tail.' Of these 39, 25 failed to give
another `tail,' while 14 gave a seventh `tail'; and here
it might seem we have evidence of the effect of preceding
tosses. The disproportion is considerable, and
even to the mathematician the case is certainly curious;
but in so many trials such curiosities may always be
noticed. That it will not bear the interpretation put
upon it is shown by the next steps. Of the 14 cases,
8 failed to give another `tail,' while the remaining six
gave another, that is, an eighth `tail'; and these
numbers eight and six are more nearly equal than the
preceding numbers 25 and 14; so that the tendency to
change had certainly not increased at this step. However,
the numbers are too small in this part of the experiment
to give results which can be relied upon.
The cases in which the numbers were large prove unmistakably,
what reason ought to have made self-evident,
that past events of pure chance cannot in the
slightest degree affect the result of sequent trials.

To suppose otherwise is, indeed, utterly to ignore
the relation between cause and effect. When anyone
asserts that because such and such things have happened,
therefore such and such other events will happen, he
ought at least to be able to show that the past events
have some direct influence on those which are thus said
to be affected by them. But if I am going to toss a
coin perfectly at random, in what possible way can the
result of the experiment be affected by the circumstance
that during ten or twelve minutes before, I tossed `head'
only or `tail' only?

The system of which I now propose to speak is more
plausible, less readily put to the full test, and consequently
far more dangerous than the one just described.
In it, as in the other, reliance is placed on a `change'
after a `run' of any kind, but not in the same way.

Everyone is familiar with the method of renewing
wagers on the terms `double' or `quits.' It is a very
convenient way of getting rid of money which has been
won on a wager by one who does not care for wagering,
and, not being to the manner born, does not feel comfortable
in pocketing money won in this way. You
have rashly backed some favourite oarsman, let us say,
or your college boat, or the like, for a level sovereign,
not caring to win, but accepting a challenge to so wager
rather than seem to want faith in your friend, college,
or university. You thus find yourself suddenly the
recipient of a coin to which you feel you are about as
much entitled as though you had abstracted it from the
other bettor's pocket. You offer him `double or quits,'
tossing the coin. Perhaps he loses, when you would be
entitled to two sovereigns. You repeat the offer, and if
he again loses (when you are entitled to four sovereigns),
you again repeat it, until at last he wins the toss. Then
you are `quits,' and can be happy again.

The system of winning money corresponds to this
safe system of getting rid of money which has been
uncomfortably won. Observe that if you only go on
long enough with the double-or-quits method, as above,
you are sure to get rid of your sovereign; for your
friend cannot go on losing for ever. He might, indeed,
lose nine or ten times running, when he would owe you
512\textit{l}. or 1,024\textit{l}.; and if he then lost heart, while yet he
regarded his loss, like his first wager, as a debt of honour
from which you could not release him, matters would
be rather awkward. If he lost twenty times he would
owe you a million, which would be more awkward still;
except that, having gone so far, he could not make
matters worse by going a little farther; and in a few
more tossings you would get rid of your millions as
completely as of the sovereign first won. Still, speaking
generally, this double-or-quits method is a sure and
easy way of clearing such scores. But it may be reversed
and become a pretty sure and easy way of making
money.

Suppose a man, whom we will call A, to wager with
another, B, one sovereign on a tossing (say). If he
wins, he gains a sovereign. Suppose, however, he loses
his sovereign. Then let him make a new wager of two
sovereigns. If he wins, he is the gainer of one sovereign
in all: if he loses, he has lost three in all. In the latter
case let him make a new wager, of four sovereigns. If
he wins, he gains one sovereign; if he loses, he has
lost seven in all. In this last case let him wager
eight sovereigns. Then, if he wins, he has gained one
sovereign, and if he loses he has lost fifteen. Wagering
sixteen sovereigns in the latter case, he gains one in all
if he wins, and has lost thirty-one in all if he loses. So
he goes on (supposing him to lose each time) doubling
his wager continually, until at last he wins. Then he
has gained one sovereign. He can now repeat the
process, gaining each time a sovereign whenever he
wins a tossing. And manifestly in this way A can
most surely and safely win every sovereign B has. Yet
every wager has been a perfectly fair one. We seem,
then, to see our way to a safe way of making any
quantity of money. B, of course, would not allow this
sort of wagering to go on very long. But the bankers
of a gambling establishment undertake to accept any
wagers which may be offered, on the system of their
game, whether \textit{rouge-et-noir}, roulette, or what not,
between certain limits of value in the stakes. Say these
limits are from 5\textit{s}. to 100\textit{l}., as I am told is not uncommonly
the case. A man may wager 5\textit{s}. on this plan,
and double eight times before his doublings carry the
stake above 100\textit{l}. Or with more advantage he may let
the successive stakes be such that the eighth doubling
will make the maximum sum, or 100\textit{l}.; so that the
stakes in inverse order will be 100\textit{l}., 50\textit{l}., 25\textit{l}., 12\textit{l}. 10\textit{s}.,
6\textit{l}. 5\textit{s}., 3\textit{l}. 2\textit{s}. 6\textit{d}.,
1\textit{l}. 11\textit{d}. 3\textit{d}., 15\textit{s}. 7\textit{d}. (fractions of a
penny not being allowed, I
suppose\footnote{Possibly pence are not allowed, in which case the successive
stakes would be 7\textit{s}., 14\textit{s}., 1\textit{l}. 8\textit{s}.,
2\textit{l}. 16\textit{s}., 5\textit{l}. 12\textit{s}., 11\textit{l}. 4\textit{s}.,
22\textit{l}. 8\textit{s}.,
44\textit{l}. 16\textit{s}., and lastly, 89\textit{l}. 12\textit{s}.}), and, lastly,
7\textit{s}. 9\textit{d}.; nine stakes, or eight doublings in all. It is so
utterly unlikely, says the believer in this system, that
where the chances are practically equal on two events,
the same event will be repeated nine times running,
that I may safely apply this method, gaining at each
venture (`though really there is no risk at all') 7\textit{s}. 9\textit{d}.,
until at last I shall accumulate in this way a small
fortune, which in time will become a large fortune.

The proprietors of gambling houses naturally encourage
this pleasing delusion. They call this power of
varying the stakes a very important advantage possessed
by the player at such tables. They say, truly enough,
a single player would not wager if the stakes could be
varied in this manner, and he possessed no power of refusing
any offer between such limits. Since a single
player would refuse to allow this arrangement, it is
manifest the arrangement is a privilege. Being a
privilege, it is worth paying for. It is on this account
that we poor bankers, who oblige those possessed of
gambling propensities by allowing them to exercise
their tastes that way, must have a certain small percentage
of odds in our favour. Thus at \textit{rouge-et-noir}
we really must have one of the ``refaits'' allowed us, say
the first, the \textit{trente-et-un}, though any other would suit
us equally well: but even then we do not win what is
on the table; the \textit{refait} may go against us, when the
players save their stakes, and if we win we only win
what has been staked on one colour, and so forth.

Those who like gambling, too, and so like to believe
that the bankers are strictly fair, adopt this argument.
Thus the editor of \textit{The Westminster Paper} says: `The
Table at all games has an extra chance, a chance varying
from one zero at one table to two at another; that is a
chance every player understands when he sits down to
play, \textit{and it is perfectly fair and honest} (!!) That this
advantage over a long series must tell is as certain as
that two and two make four. But~.~.~.~. the bank
does not always win; on the contrary,' we often `hear
of the bank being broken and closed until more cash is
forthcoming. The number of times the bank loses
and nothing is said about it, would amount to a considerable
number of times in the course of a year. A
small percentage on one side or the other, extended over
a long enough series, will tell; but on a single event
the difference in the gambler's eyes' (yes, truly, in his
eyes) `is small. For that percentage the punter is
enabled to vary his stakes from 5\textit{s}. say, to 100\textit{l}. Without
some such advantage, no one would permit his
adversaries thus to vary the stakes. The punter' (poor
moth!) `is willing to pay for this advantage.'

And all the while the truth is that the supposed
advantage is no advantage at all---at least, to the player.
It is of immense advantage to the bankers, because it
encourages so many to play who otherwise might refrain.
But in reality the bankers would make the same winnings
if every stake were of a fixed amount, say 10\textit{l}.,
as when the stakes can be varied---always assuming
that as many players would come to them, and play as
freely, as on the present more attractive system.

Let us consider the actual state of the case, when a
player at a table doubles his stakes till he wins---repeating
the process from the lowest stakes after each
success.

But first---or rather, as a part of this inquiry---let
us consider why our imaginary player B would decline
to allow A to double wagers in the manner described.
In reality, of course, A's power of doubling is limited
by the amount of A's money, or of his available money
for gambling. He cannot go on doubling the stakes
when he has paid away more than half his money.
Suppose, for instance, he has 1,000\textit{l}. in notes and 30\textit{l}.
or so in sovereigns. He can wager successively (if he
loses so often) 1\textit{l}. 2\textit{l}. 4\textit{l}. 8\textit{l}.
16\textit{l}. 32\textit{l}. 64\textit{l}. 128\textit{l}.
256\textit{l}. 512\textit{l}. or ten times. But if he loses his last
wager he will have paid away 1,023\textit{l}., and must stop for
the time, leaving B the gainer of that sum. This is a
very unlikely result for a single trial. It would not be
likely to happen in a hundred or in two hundred trials,
though it might happen at the first trial, or at a very
early one. Even if it happened after five hundred
trials, A would only have won 500\textit{l}. in those, and B
winning 1,023\textit{l}. at the last, would have much the better
of the encounter.

Why, then, would not B be willing to wager on
these terms? For precisely the same reason (if he
actually reasoned the matter out) that he should be
unwilling to pay 1\textit{l}. for one ticket out of 1,024 where
the prize was 1,024\textit{l}. Each ticket would be fairly worth
that sum. And many foolish persons, as we know, are
willing to pay in that way for a ticket in a lottery, even
paying more than the correct value. But no one of any
sense would throw away a sovereign for the chance (even
truly valued at a sovereign) of winning a thousand
pounds. That, really, is what B declines to do. Every
venture he makes with A (supposing A to have about
1,000\textit{l}. at starting, and so to be able to keep on doubling
up to 512\textit{l}.) is a wager on just such terms. B wins
nothing unless he wins 1,024\textit{l}.; he loses at each
failure 1\textit{l}. His chance of winning, too, is the same, at
each venture, as that of drawing a single marked ticket
from a bag containing 1,024 tickets. Each venture,
though it may be decided at the first or second tossing,
is a venture of ten tossings. Now, with ten tossings
there are 1,024 possible results, any one of which is as
likely as any other. One of these, and one only, is
favourable to B, viz.\ the case of ten `heads,' if he is
backing `heads,' or ten `tails,' if he is backing `tails.'
Thus he pays, in effect, one pound for one chance
in 1,024 of winning 1,024\textit{l}., though, in reality, he does
not pay the pound until the venture is decided against
him; so that, if he wins, he receives 1,023\textit{l}., corresponding
(with the 1\textit{l}.) to the total just named.

Now, to wager a pound in this way, for the chance
of winning 1,024\textit{l}., would be very foolish; and though
continually repeating the experiment would in the long
run make the number of successes bear the right proportion
to the number of failures, yet B might be
ruined long before this happened, though quite as probably
A would be ruined. B's ruin, if effected, would
be brought about by steadily continued small losses, A's
by a casual but overwhelming loss. The richer B and
A were, the longer it would be before one or other was
ruined, though the eventual ruin of one or other would
be certain. If one was much richer than the other, his
chance of escaping ruin would be so much the greater,
and so much greater, therefore, the risk of the poorer.
In other words, the odds would be great in favour of the
richer of the two, whether A or B, absorbing the whole
property of the other, if wagering on this plan were
continued steadily for a long time.

Now, if we extend such considerations as these to
the case in which an individual player contends against
a bank, we shall see that, even without any percentage
on the chances, the odds would be largely in favour of
the bank. If the player is persistent in applying his
system, he is practically certain to be ruined. For it is
to be noticed that in such a system the player is exposed
to that which he can least afford, namely, sudden and
great loss; it is by such losses that his ruin will be
brought about if at all. On the other hand, the bank,
which can best afford such losses, has to meet only
a steady slow drain upon its resources, until the inevitable
\emph{coup} comes which restore all that had been thus
drained out, and more along with it. If the player
were even to carry on his system in the manner which
my reasoning has really implied; if, as he made his
small gain at each venture, he set it by to form a reserve
fund---even then his ruin would be inevitable in the
long run. But every one knows that gamblers do
nothing of the sort. `Lightly come, lightly go,' is their
rule, so far as their gains are concerned. [In another
sense, their rule is, lightly come (to the gaming-table)
and heavily go when the last pound has been staked and
lost.] Thus they run a risk which, in their way of
playing, amounts almost to a certainty of ruining themselves,
and they do not even take the precaution which
would alone give them their one small, almost evanescent
chance of escape. On the other hand, the bankers,
who are really playing an almost perfectly safe game,
leave nothing to chance. The bulk of the money gained
by them is reserved to maintain the balance necessary
for safety. Only the actual profits of their system---the
percentage of gain due to their percentage on the chances---is
dealt with as income; that is, as money to be spent.

It is true that in one sense the case between the
bankers and the public resembles that of a player with
a small capital against a player with a large capital;
the bankers have indeed a large capital, but it is small
compared with that of the public at large who frequent
the gaming-tables. But, in the first place, this
does not at all help any single player. It is all but
certain that the public (meaning always the special
gaming public) will not be ruined as a whole, just as it
is all but certain that the whole of an army engaged in
a campaign, even under the most unfavourable circumstances,
will not be destroyed if recruits are always
available at short notice. Now, if the circumstances of
a campaign are such that each individual soldier runs
exceeding risk of being killed, it will not improve the
chances of any single soldier that the army as a whole
will not be destroyed; and in like manner those who
gamble persistently are not helped in their ruin by the
circumstance that, as one is `pushed from the board,
others ever succeed.' Even the chance of the bank
being ruined, however, is not favourable to the gambler
who follows such a system as I am dealing with, but
positively adds to his risks. In the illustrative case of
A playing B, the ruin of B meant that A had gained
all B's money. But in the case of a gambler playing on
the doubling system at a gaming-table, the ruin of the
bank would be one of the chances against him that such
a gambler would have to take into account. It might
happen when he was far on in a long process of doubling,
and would be almost certain to happen when he had to
some degree entered on such a process. He would then
be certainly a loser on that particular venture. If a
winner on the event actually decided when the bank
broke (only one, be it remembered, of the series forming
his venture), he would perhaps receive a share, but
a share only, of the available assets. The rules of the
table may be such that these will always cover the stakes,
and in that case the player, supposing he had won on
the last event decided, would sustain no loss. Should
he have lost on that event, however, which ordinarily
would at least not interfere with the operation of his
system, he is prevented from pursuing the system till he
has recouped his loss. This can never happen in play
between two gamblers on this system. For the very
circumstance that A has lost an event involves of necessity
the possession by B of enough money to continue
the system. B's stake after winning is always double
the last stake, but after winning the amount just staked
of course he must possess double that amount---since he
has his winnings and also a sum at least equal, which
he must have had when he wagered an equal stake.
But when a player at the gaming-tables loses an event
in one of his ventures, it by no means follows equally
that the bank can continue to double (assuming the
highest value allowed to have not been reached). Losses
against other players may compel the bank to close when
the system player has just lost a tolerably heavy \textit{coup}.
His system then is defeated, and he sustains a loss
distinct in character from those which his system normally
involves. In other words, the chances against
him are increased; and, on the other hand, the bankers'
chance of ruin would be small, even if they had no
advantage in the odds, simply because the sum staked
bears a much smaller proportion to their capital than the
wagers of the individual player bear to his property.

Yet the reader must not fall into the mistake of
supposing that because the individual player would have
enormous risks against him, even if the bankers took no
percentage on the chances, the bank would then in the
long run make enormous gains. That would be a paradoxical
result; though at first sight it seems equally
paradoxical to say that while every single player would
be almost certain to be ruined the bank would not
gain in the long run. This, however, is perfectly true.
The fact is, that, among the few who escaped ruin, some
would be enormous gainers. It would be because of
some marvellous runs of luck, and consequent enormous
gains, that they would be saved from ruin; and the
chances would be that some among these would be very
heavy gainers. They would be few; and the action of a
man who gambled heavily on the chance of being one of
these few, would be like that of a man who bought
half a dozen tickets, at a price of 1,000\textit{l}. each (his whole
property being thus expended), among millions of
tickets in a lottery, in which were a few prizes of
1,000,000\textit{l}. each. But though the smallness of the
chance of being one among the few very great gainers at
the gambling-table, makes it absurd for a man to run the
enormous risk of ruin involved in persistent play, yet,
so far as the bankers would be concerned, the great
losses on the few winners would in the long run equalise
the moderate gains on the great majority of their
customers. They would neither gain nor lose a sum
bearing any considerable proportion to their ventures,
and would run some risk, though only a small one,
of being swamped by a long-continued run of bad
luck.

But the bankers do not in this way leave matters to
chance. They take a percentage on the chances. The
percentage they take is often not very large in itself,
though it is nearly always larger than it appears, even
when regarded properly as a percentage on the chances.
But what is usually overlooked by those who deal with
this matter, and especially by those who, being gamblers
themselves, \textit{want} to think that gaming houses give them
very fair chances, is that a very small percentage on the
chances may mean, and necessarily does mean, an
enormous percentage of profits.

Let us take, as illustrating both the seeming smallness
of the percentage on the chances, and the enormous
probable percentage of profits, the game of \textit{rouge-et-noir},
so far as it can be understood from the accounts
given in the books.\footnote{De Morgan remarks on the incomplete and unintelligible way
in which this game is described in the later editions of Hoyle. It
is singular how seldom a complete and clear account of any game
can be found in books, though written by the best card-players. I
have never yet seen a description of cribbage, for example, from
which anyone who knew nothing of the game, and could find no
one to explain it practically to him, could form a correct idea of its
nature. In half a dozen lines from the beginning of a description,
technical terms are used which have not been explained, remarks
are made which imply a knowledge on the reader's part of the
general object of the game of which he should be supposed to know
nothing, and many matters absolutely essential to a right apprehension
of the nature of the game are not touched on from beginning
to end, or are so insufficiently described that they might as well
have been left altogether unnoticed. It is the same with verbal
descriptions. Not one person in a hundred can explain a game of
cards respectably, and not one in a thousand can explain a game
well. A beginner can pick up a game after awhile, by combining
with the imperfect explanations given him the practical illustrations
which the cards themselves afford. But there is no reason in the
nature of things why a written or a verbal description of such a
game as whist or cribbage should not suffice to make an attentive
reader or hearer perfectly understand the nature of the game.
From what I have noticed in this matter, I would assert with some
confidence that anyone who can explain clearly, yet succinctly, a
game at cards, must have the explanatory gift so exceptionally
developed that he could most usefully employ it in the explanation
of such scientific subjects as he might himself be able to master.
I believe, too, that the student of science who desires to explain his
subject to the general public, can find no better exercise, and few
better tests, than the explanation of some simple game---the explanation
to be sufficient for persons knowing nothing of the game.}
I follow De Morgan's rendering
of these confused and imperfect accounts. It seems to
be correct, for his computation of the odds for and
against the player leads to the same result as Poisson
obtained, who knew the game, though he nowhere gives
a description of it.

A number of packs is taken (six, Hoyle says), `and
the cards are well mixed. Bach common card counts
for the number of spots on it, and the court cards are
each reckoned as ten. A table is divided into two compartments,
one called \textit{rouge}, the other \textit{noir}, and a player
stakes his money in which he pleases. The proprietor
of the bank, who risks against all comers, then lays
down cards in one compartment until the number of
spots exceeds thirty; as soon as this has happened, he
proceeds in the same way with the other compartment.'
The number of spots in each compartment is thus
necessarily between 31 and 40, both inclusive. The compartment
in which the total number of spots is least is
the winning one. Thus, if there are 35 spots on the
cards in the \textit{rouge}, and 32 on the cards in the \textit{noir},
\textit{noir} wins, and all players who staked upon \textit{noir} receive
from the bank sums equal to their stakes. The process
is then repeated. So far, it will be observed, the chances
are equal for the players and for the bankers. It will
also be observed that the arrangement is one which
strongly favours the idea (always encouraged by the
proprietors of gaming houses) that the bankers have
little interest in the result. For the bank does not back
either colour. The players have all the backing to
themselves. If they choose to stake more in all on the
red than on the black, it becomes the bank's interest
that black should win; but it was by the players' own
acts that black became for the time the bank's colour.
And not only does this suggest to the players the incorrect
idea, that the bank has little real interest in
the game, but it encourages the correct idea, which it is
the manifest interest of the bankers to put very clearly
before the players, that everything is fairly managed.
If the bank chose a colour, some might think that the
cards, however seemingly shuffled, were in reality arranged,
or else were so manipulated as to make the
bank's colour win oftener than it should do. But since
the players themselves settle which shall be the bank's
colour at each trial, there cannot be suspicion of foul
play of this sort.

We now come to the bank's advantage on the
chances. The number of spots in the black and red
compartments may be equal. In this case (called by
Hoyle a \textit{refait}) the game is drawn; and the players may
either withdraw, increase, or diminish their stakes, as
they please, for a new game, if the number of spots in
each compartment is any except 31. But if the number
in each be 31 (a case called by Hoyle a \textit{refait trente-et-un}),
then the players are not allowed to withdraw their
stakes. And not only must the stakes remain for a new
game, but, whatever happens on this new trial, the
players will receive nothing. Their stakes are for the
moment impounded (or technically, according to Hoyle,
\textit{en prison}). The new game (called an \textit{apr{\'e}s}), unless it
chances to give another \textit{refait}, will end in favour of
either \textit{rouge} or \textit{noir}. Whichever compartment wins, the
players in that compartment save their stakes, but
receive nothing from the bank; the players who have
put their stakes in the other compartment lose them.
De Morgan says here, not quite correctly, `should the
bank win it takes the stakes, should the bank lose the
player recovers his stakes.' This is incorrect, because
it at least suggests the incorrect idea that the bank may
either win or the stakes go clear; whereas in reality,
except in the improbable event of all the players backing
one colour, the bank is sure to win something, viz.,
either the stakes in the red or those in the black compartment,
and the only point to be settled is whether
the larger or the smaller of these probably unequal
sums shall pass to the bank's exchequer. If the \textit{apr{\'e}s}
gives a second \textit{refait}, the stakes still remain impounded,
and another game is played, and no stakes are released
until either \textit{rouge} or \textit{noir} has won. But in the meantime
new stakes may be put down, before the fate of
the impounded stakes has been decided.

Thus, whereas, with regard to games decided at the
first trial, the bank has in the long run no interest one
way or the other, the bank has an exceptional interest
in \textit{refaits}. A \textit{refait trente-et-un} at once gives the
bank a certainty of winning the least sum staked in
the two compartments, and an equal chance of winning
the larger sum instead. Any \textit{refait} gives the bank
the chance that on a new trial a \textit{refait trente-et-un} may
be made; and though this chance (that is, the chance
that there will first be a common \textit{refait} and then a \textit{refait
trente-et-un}) is small, it tells in the long run and
must be added to the advantage obtained from the
chance of a \textit{refait trente-et-un} at once.

Now it may seem as though the bank would gain
very little from so small an advantage. A \textit{refait} may
occur tolerably often in any long series of trials, but a
\textit{refait trente-et-un} only at long intervals. It is only one
out of ten different \textit{refaits}, which to the uninitiated
seem all equally likely to occur; so that he supposes
the chance of a \textit{refait trente-et-un} to be only one-tenth
of the chance (itself small at each trial) that there will
be a \textit{refait} of some sort. But, to begin with, this supposition
is incorrect. Calculation shows that the chance of
a \textit{refait} of some sort occurring is 1,097 in 10,000, or
nearly one in nine. The chance of a \textit{refait trente-et-un}
is not one-tenth of this, or about 110 in 10,000, but
219 in 10,000, or twice as great as the uninitiated
imagine. Thus in very nearly two games in 91,
instead of one game in 91, a \textit{refait trente-et-un} occurs.
It follows from this, combined with the circumstance
that on the average the bank wins half its stakes only
in the case of one of these \textit{refaits} (and account being
also taken of the slight subordinate chance above mentioned),
that the mathematical advantage of the bank is
very nearly one-ninetieth of all the sums deposited. The
actual percentage is $1\frac{1}{10}$ \textit{per deposit},
or 1\textit{l}. 2\textit{s}. per 100\textit{l}.
And in passing it may be noticed as affording good
illustration of the mistakes the uninitiated are apt to
make in such matters, that if instead of the \textit{refait trente-et-un}
the bankers took to themselves the \textit{refait quarante},
then, instead of this percentage per deposit, the percentage
would be only $\frac{3}{20}$, or 3\textit{s}. per 100\textit{l}.

But even an average advantage of 1\textit{l}. 2\textit{s}. per 100\textit{l}. on
each deposit made by the bank is thought by the frequenters
of the table to be very slight. It makes the odds
against the players about 913 to 892 on each trial, and the
difference seems trifling. On considering the probable
results of a year's play, however, we find that the
bankers could obtain tremendous interest for a capital
which would make them far safer against ruin than is
thought necessary in any ordinary mercantile business.
Suppose play went on upon only 100 evenings in each
year; that each evening 100 games were played; and
that on each game the total sum risked on both \textit{rouge}
and \textit{noir} was 50\textit{l}. Then the total sum deposited by
the bank (very much exceeding the total sum \textit{risked},
which on each game is only the difference between the
sums staked on \emph{rouge} and on \emph{noir}) would be 500,000\textit{l}.;
and  $1\frac{1}{10}$ per cent.\ on this sum would be 5,500\textit{l}. I
follow De Morgan in taking these numbers, which are
far below what would generally be deposited in 100
evenings of play. Now, it can be shown that if the
bankers started with such a sum as 5,500\textit{l}., they would
be practically safe from all chance of ruin. So that in
100 playing nights they would probably make cent.\ per cent.\ on
their capital. In places where gambling
is encouraged they could readily in a year make 300
per cent.\ on their capital at the beginning of the year.

De Morgan points out that, though the editor of
Hoyle does not correctly estimate the chances in this
game, underrating the bank's advantage; yet, even
with this erroneous estimate, the gains per annum on a
capital of 5,500\textit{l}. would be 12,000\textit{l}. (instead of 16,500\textit{l}.
as when properly calculated). As he justly says, `the
preceding results, or either of them, being admitted,
it might be supposed hardly necessary to dwell upon
the ruin which must necessarily result to individual
players against a bank which has so strong a chance of
success against its united antagonists.' `But,' he adds,
`so strangely are opinions formed upon this subject,
that it is not uncommon to find persons who think they
are in possession of a specific by which they must infallibly
win.' If both the banker and the player staked
on each game 1-160th part of their respective funds,
and the play was to continue till one or other side was
ruined, the bank would have 49 chances to 1 in its
favour against that one player. But if, as more commonly
is the case, the player's stake formed a far larger
proportion of his property, these odds would be immensely
increased. If a player staked one-tenth of his
money on each game against the same sum, supposed
to be 1-160th of the bank's money, the chances would
be 223 to 1 that he would be ruined if he persisted
long enough. In other words, his chance of escaping
ruin would be the same as that of drawing one single
marked ball out of a bag containing 224.

Other games played at the gaming-tables, however
different in character they may be from \textit{rouge-et-noir},
give no better chances to the players. Indeed, some
games give far inferior chances. There is not one of
them at which any system of play can be safe in the
long run. If the system is such that the risk on each
venture is small, then the gains on each venture will
be correspondingly small. Many ventures, therefore,
must be made in order to secure any considerable
gains; and when once the number of ventures is
largely increased, the small risk on each becomes a
large risk, and if the ventures be very numerous
becomes practically a certainty of loss. On the other
hand there are modes of venturing which, if successful
once only, bring in a large profit; but they involve a
larger immediate risk.

In point of fact, the supposition that any system
can be devised by which success in games of chance
may be made certain, is as utterly unphilosophical as
faith in the invention of perpetual motion. That the
supposition has been entertained by many who have
passed all their lives in gambling proves only---what
might also be safely inferred from the very fact of their
being gamblers---that they know nothing of the laws
of probability. Many men who have passed all their
lives among machinery believe confidently in the possibility
of perpetual motion. They are familiar with
machinery, but utterly ignorant of mechanics. In like
manner, the life-long gambler is familiar with games
of chance, but utterly ignorant of the laws of chance.

It may appear paradoxical to say that chance results
right themselves---nay, that there is an absolute
certainty that in the long run they will occur as often
(in proportion) as their respective chances warrant,
and at the same time to assert that it is utterly
useless for any gambler to trust to this circumstance.
Yet not only is each statement true, but it is of first-rate
importance in the study of our subject that the
truth of each should be clearly recognised.

That the first statement is true, will perhaps not be
questioned. The reasoning on which it is based would
be too abstruse for these pages; but it has been experimentally
verified over and over again. Thus, if a coin
be tossed many thousands of times, and the numbers
of resulting `heads' and `tails' be noted, it is found,
not necessarily that these numbers differ from each
other by a very small quantity, but that their difference
is small compared with either. In mathematical phrase,
the two numbers are nearly in a ratio of equality. Again,
if a dice be tossed, say, six million times, then, although
there will not probably have been exactly a million
throws of each face, yet the number of throws of each
face will differ from a million by a quantity very small
indeed compared with the total number of throws. So
certain is this law, that it has been made the means of
determining the real chances of an event, or of ascertaining
facts which had been before unknown. Thus, De
Morgan relates the following story in illustration of
this law. He received it `from a distinguished naval
officer, who was once employed to bring home a cargo
of dollars.' `At the end of the voyage,' he says, `it
was discovered that one of the boxes which contained
them had been forced; and on making further search
a large bag of dollars was discovered in the possession
of some one on board. The coins in the different
boxes were a mixture of all manner of dates and sovereigns;
and it occurred to the commander, that if the
contents of the boxes were sorted, a comparison of the
proportions of the different sorts in the bag, with those
in the box which had been opened, would afford strong
presumptive evidence one way or the other. This
comparison was accordingly made, and the agreement
between the distribution of the several coins in the bag
and those in the box was such as to leave no doubt as
to the former having formed a part of the latter.' If
the bag of stolen dollars had been a small one the
inference would have been unsafe, but the great number
of the dollars corresponded to a great number of chance
trials; and as in such a large series of trials the several
results would be sure to occur in numbers corresponding
to their individual chances, it followed that the number
of coins of the different kinds in the stolen lot would
be proportional, or very nearly so, to the number of
those respective coins in the forced box. Thus, in this
case the thief increased the strength of the evidence
against him by every dollar he added to his ill-gotten
store.

We may mention, in passing, an even more curious
application of this law, to no less a question than that
much-talked of but little understood problem, the
squaring of the circle. It can be shown by mathematical
reasoning, that, if a straight rod be so tossed
at random into the air as to fall on a grating of equidistant
parallel bars, the chance of the rod falling
through depends on the length and thickness of the
rod, the distance between the parallel bars, \textit{and} the
proportion in which the circumference of a circle exceeds
the diameter. So that when the rod and grating
have been carefully measured, it is only necessary to
know the proportion just mentioned in order to calculate
the chance of the rod falling through. But also,
if we can learn in some other way the chance of the rod
falling through, we can infer the proportion referred to.
Now the law we are considering teaches us that if we
only toss the rod often enough, the chance of its falling
through will be indicated by the number of times it
actually does fall through, compared with the total
number of trials. Hence we can estimate the proportion
in which the circumference of a circle exceeds the
diameter by merely tossing a rod over a grating several
thousand times, and counting how often it falls through.
The experiment has been tried, and Professor De Morgan
tells us that a very excellent evaluation of the celebrated
proportion (the determination of which is equivalent in
reality to squaring the circle) was the result.

And let it be noticed, in passing, that this inexorable
law---for in its effects it is the most inflexible of all the
laws of probability---shows how fatal it must be to
contend long at any game of pure chance, where the
odds are in favour of our opponent. For instance, let
us assume for a moment that the assertion of the foreign
gaming bankers is true, and that the chances are but
from $1\frac{1}{4}$ to $2\frac{1}{2}$ per cent.\ in their favour. Yet in the
long run, this percentage must manifest its effects.
Where a few hundreds have been wagered the bank
may not win $1\frac{1}{4}$ or $2\frac{1}{2}$ on each, or may lose considerably;
but where thousands of hundreds are wagered, the bank
will certainly win about their percentage, and the players
will therefore lose to a corresponding extent. This is
inevitable, so only that the play continue long enough.
Now, it is sometimes forgotten that to ensure such gain
to the bank, it is by no means necessary that the players
should come prepared to stake so many hundreds of
pounds. Those who sit down to play may not have a
tithe of the sum necessary---if only wagered once---to
ensure the success of the bank. But every florin the
players bring with them may be, and commonly is,
wagered over and over again. There is repeated gain
and loss, and loss and gain; insomuch that the player
who finally loses a hundred pounds, may have wagered
in the course of the sitting a thousand or even many
thousand pounds. Those fortunate beings who `break
the bank' from time to time, may even have accomplished
the feat of wagering millions during the process
which ends in the final loss of the few thousands they
may have begun with.

Why is it, then, it will be asked, that this inexorable
law is yet not to be trusted? For this reason, simply,
that the mode of its operation is altogether uncertain.
If in a thousand trials there has been a remarkable
preponderance of any particular class of events, it is
not a whit more probable that the preponderance will
be compensated by a corresponding deficiency in the
next thousand trials than that it will be repeated in
that set also. The most probable result of the second
thousand trials is precisely that result which was most
probable for the first thousand---that is, that there will
be no marked preponderance either way. But there
\emph{may be} such a preponderance; and it may lie either
way. It is the same with the next thousand, and the
next, and for every such set. They are in no way
affected by preceding events. In the nature of things,
how can they be? But, `the whirligig of time brings
in its revenges' in its own way. The balance is restored
just as chance directs. It may be in the next thousand
trials, it may be not before many thousands of trials.
We are utterly unable to guess when or how it will be
brought about.

But it may be urged that this is mere assertion; and
many will be very ready to believe that it is opposed to
experience, or even contrary to common sense. Yet
experience has over and over again confirmed the
matter, and common sense, though it may not avail to
unravel the seeming paradox, yet cannot insist on the
absurdity that coming events of pure chance are affected
by completed events of the same kind. If a person
has tossed `heads' nine times running (we assume fair
and lofty tosses with a well-balanced coin), common
sense teaches him, as he is about to make the tenth
trial, that the chances on that trial are precisely the
same as the chances on the first. It would, indeed,
have been rash for him to predict that he would reach
that trial without once failing to toss `head'; but as
the thing has happened, the odds originally against it
count for nothing. They are disposed of by known
facts. We have said, however, that experience confirms
our theory. It chances that a series of experiments
have been made on coin-tossing. Buffon was
the experimenter, and he tossed thousands of times,
noting always how many times he tossed `head' running
before `tail' appeared. In the course of these
trials he many times tossed `head' nine times running.
Now, if the tossing `head' nine times running rendered
the chance of tossing a tenth head much less
than usual, it would necessarily follow that in considerably
more than one-half of these instances Buffon
would have failed to toss a tenth head. But he did
not. In about half the cases in which he tossed
nine `heads' running, the next trial also gave him
`head'; and about half of these tossings of ten
successive `heads' were followed by the tossing of an
eleventh `head.' In the nature of things this was to
be expected.

And now let us consider the cognate questions suggested
by our sharper's ideas respecting the person who
plays. This person is to consider carefully whether he
is `\textit{in vein},' and not otherwise to play. He is to be
cool and businesslike, for fortune is invariably adverse
to an angry player. Steinmetz, who appears to place
some degree of reliance on the suggestion that a player
should be `in vein,' cites in illustration and confirmation
of the rule the following instance from his own
experience:---`I remember,' he says, `a curious incident
in my childhood which seems very much to the
point of this axiom. A magnificent gold watch and
chain were given towards the building of a church, and
my mother took three chances, which were at a very
high figure, the watch and chain being valued at more
than 100\textit{l}. One of these chances was entered in my
name, one in my brother's, and a third in my mother's.
I had to throw for her as well as myself. My brother
threw an insignificant figure; for myself I did the
same; but, oddly enough, I refused to throw for my
mother on finding that I had lost my chance, saying
that I should wait a little longer---rather a curious
piece of prudence' (read, rather, superstition) `for a
child of thirteen. The raffle was with three dice; the
majority of the chances had been thrown, and thirty-four
was the highest.' (It is to be presumed that the
three dice were thrown twice, yet `thirty-four' is
a remarkable throw with six dice, and `thirty-six'
altogether exceptional.) `I went on throwing the
dice for amusement, and was surprised to find that
every throw was better than the one I had in the raffle.
I thereupon said, ``Now I'll throw for mamma.'' I threw
thirty-six, which won the watch! My mother had been
a large subscriber to the building of the church, and
the priest said that my winning the watch for her was
quite \textit{providential}. According to M. Houdin's authority,
however, it seems that I only got into ``vein''---but
how I came to pause and defer throwing the last
chance has always puzzled me respecting this incident
of childhood, which made too great an impression
ever to be effaced.'

It is probable that most of my readers can recall
some circumstance in their lives, some surprising coincidence,
which has caused a similar impression, and
which they have found it almost impossible to regard
as strictly fortuitous.

In chance games especially, curious coincidences
of the sort occur, and lead to the superstitious notion
that they are not mere coincidences, but in some definite
way associated with the fate or fortune of the player, or
else with some event which has previously taken place---a
change of seats, a new deal, or the like. There is
scarcely a gambler who is not prepared to assert his
faith in certain observances whereby, as he believes, a
change of luck may be brought about. In an old work
on card-games the player is gravely advised, if the luck
has been against him, to turn three times round with
his chair, `for then the luck will infallibly change in
your favour.'

Equally superstitious is the notion that anger brings
bad luck, or, as M. Houdin's authority puts it, that
`the demon of bad luck invariably pursues a passionate
player.' At a game of pure chance good temper makes
the player careless under ill-fortune, but it cannot
secure him against it. In like manner, passion may
excite the attention of others to the player's losses, and
in any case causes himself to suffer more keenly under
them, but it is only in this sense that passion is unlucky
for him. He is as likely to make a lucky hit
when in a rage as in the calmest mood.

It is easy to see how superstitions such as these take
their origin. We can understand that since one who
has been very unlucky in games of pure chance, is not
antecedently likely to continue equally unlucky, a
superstitious observance is not unlikely to be followed
by a seeming change of luck. When this happens the
coincidence is noted and remembered; but failures are
readily forgotten. Again, if the fortunes of a passionate
player be recorded by dispassionate bystanders, he will
not appear to be pursued by worse luck than his neighbours;
but he will be disposed to regard himself as the
victim of unusual ill-fortune. He may perhaps register
a vow to keep his temper in future; and then his luck
may seem to him to improve, even though a careful
record of his gains and losses would show no change
whatever in his fortunes.

But it may not seem quite so easy to explain those
undoubted runs of luck by which players `in the vein'
(as supposed) have broken gaming-banks, and have
enabled those who have followed their fortunes to
achieve temporary success. The history of the notorious
Garcia, and of others who like him have been for
awhile the favourites of fortune, will occur at once to
many of my readers, and will appear to afford convincing
proof of the theory that the luck of such gamesters
has had a real influence on the fortunes of the
game. The following narrative gives an accurate and
graphic picture of the way in which these `bank-breakers'
are followed and believed in, while their
success seems to last.

The scene is laid in one of the most celebrated
German Kursaals.

`What a sudden influx of people into the room!
Now, indeed, we shall see a celebrity. The tall light-haired
young man coming towards us, and attended by
such a retinue, is a young Saxon nobleman who made
his appearance here a short time ago, and commenced
his gambling career by staking very small sums; but,
by the most extraordinary luck, he was able to increase
his capital to such an extent that he now rarely stakes
under the maximum, and almost always wins. They
say that when the croupiers see him place his money
on the table, they immediately prepare to pay him,
without waiting to see which colour has actually won,
and that they have offered him a handsome sum down
to desist from playing while he remains here. Crowds
of people stand outside the Kursaal doors every
morning, awaiting his arrival; and when he comes
following him into the room, and staking as he stakes.
When he ceases playing they accompany him to the
door, and shower on him congratulations and thanks
for the good fortune he has brought them. See how
all the people make way for him at the table, and how
deferential are the subdued greetings of his acquaintances!
He does not bring much money with him,
his luck is too great to require it. He takes some
notes out of a case, and places maximums on \textit{black} and
\textit{couleur}. A crowd of eager hands are immediately outstretched
from all parts of the table, heaping up silver
and gold and notes on the spaces on which he has
staked his money, till there scarcely seems room for
another coin, while the other spaces on the table only
contain a few florins staked by sceptics who refuse to
believe in the count's luck.' He wins; and the narrative
proceeds to describe his continued successes, until
he rises from the table a winner of about one hundred
thousand francs at that sitting.

The success of Garcia was so remarkable at times as
to affect the value of the shares in the \textit{Privilegirte
Bank} ten or twenty per cent. Nor would it be difficult
to cite many instances which seem to supply incontrovertible
evidence that there is something more than
common chance in the temporary successes of these (so-called)
fortunate men.

Indeed, to assert merely that in the nature of things
there can be no such thing as luck that can be depended
on even for a short time, would probably be quite
useless. There is only one way of meeting the infatuation
of those who trust in the fates of lucky gamesters.
We can show that, granted a sufficient number of
trials---and it will be remembered that the number of
those who have risked their fortunes at \textit{roulette} and
\textit{rouge-et-noir} is incalculably great---there must \textit{inevitably}
be a certain number who appear exceptionally
lucky; or, rather, that the odds are overwhelmingly
against the continuance of play on the scale which
prevails at the foreign gambling-tables, without the
occurrence of several instances of persistent runs of luck.

To remove from the question the perplexities resulting
from the nature of the above-named games, let
us suppose that the tossing of a coin is to determine
the success or failure of the player, and that he will
win if he throws `head.' Now if a player tossed `head'
twenty times running on any occasion it would be
regarded as a most remarkable run of luck, and it
would not be easy to persuade those who witnessed the
occurrence that the thrower was not in some special
and definite manner the favourite of Fortune. We may
take such exceptional success as corresponding to the
good fortune of a `bank-breaker.' Yet it is easily
shown that with a number of trials which must fall
enormously short of the number of cases in which
fortune is risked at foreign Kursaals, the throwing of
twenty successive `heads' would be practically insured.
Suppose every adult person in Britain---say 10,000,000
persons in all---were to toss a coin, each tossing until
`tail' was thrown; then it is practically certain that
several among them would toss twenty times before
`tail' was thrown. Thus: It is certain that about five
millions would toss `head' once; of these about one-half,
or some two millions and a half, would toss `head'
on the second trial; about a million and a quarter
would toss `head' on the third trial; about six hundred
thousand on the fourth; some three hundred thousand
on the fifth; and by proceeding in this way---roughly
halving the numbers successively obtained---we find
that some eight or nine of the ten million persons
would be almost certain to toss `head' twenty times
running. It must be remembered that so long as
the numbers continue large the probability that about
half will toss `head' at the next trial amounts almost
to certainty. For example, about 140 toss `head'
sixteen times running: now, it is utterly unlikely that
of these 140, fewer than sixty will toss `head' yet a
seventeenth time. But if the above process failed on
trial to give even one person who tossed `heads'
twenty times running---an utterly improbable
event---yet the trial could be made four or five times, with
practical certainty that not one or two, but thirty or
forty, persons would achieve the seemingly incredible
feat of tossing `head' twenty times running. Nor
would all these thirty or forty persons fail to throw
even three or four more `heads.'

Now, if we consider the immense number of trials
made at gambling-tables, and if we further consider
the gamblers as in a sense typified by our ten millions
of coin-tossers, we shall see that it is not merely
probable but absolutely certain that from time to
time there must be marvellous runs of luck at \textit{roulette},
\textit{rouge-et-noir}, \textit{hazard}, \textit{faro}, and other games of
chance. Suppose that at the public gaming-tables on
the Continent there sit down each night but one
thousand persons in all, that each person makes but
ten ventures each night, and that there are but one
hundred gambling nights in the year---each supposition
falling far below the truth---there are then one
million ventures each year. It cannot be regarded as
wonderful, then, that among the fifty millions of
ventures made (on this supposition) during the last
half century, there should be noted some runs of luck
which on any single trial would seem incredible. On
the contrary, this is so far from being wonderful that
it would be far more wonderful if no such runs of luck
had occurred. It is probable that if the actual number
of ventures, and the circumstances of each, could be
ascertained, and if any mathematician could deal with
the tremendous array of figures in such sort as to
deduce the exact mathematical chance of the occurrence
of bank-breaking runs of luck, it would be found
that the antecedent odds were many millions to one in
favour of the occurrence of a certain number of such
events. In the simpler case of our coin-tossers the
chance of twenty successive `heads' being tossed can
be quite readily calculated. I have made the calculation,
and I find that if the ten million persons had
each two trials the odds would be more than 10,000
to 1 in favour of the occurrence of twenty successive
`heads' once at least; and only a million and a half
need have a single trial each, in order to give an even
chance of such an occurrence.

But we may learn a further lesson from our illustrative
tossers. We have seen that granted only a
sufficient number of trials, runs of luck are practically
certain to occur: but we may also infer that no run of
luck can be \textit{trusted} to continue. The very principle
which has led us to the conclusion that several of our
tossers would throw twenty `heads' successively, leads
also to the conclusion that one who has tossed `heads'
twelve or thirteen times, or any other considerable
number of times in succession, is not more (or less)
likely to toss `head' on the next trial than at the
beginning. \textit{About half}, we said, in discussing the
fortunes of the tossers, would toss `head' at the next
trial: in other words, \textit{about half} would fail to toss
`head.' The chances for and against these lucky
tossers are equal at the next trial, precisely as the
chances for and against the least lucky of the ten
million tossers would be equal at any single tossing.

Yet, it may be urged, experience shows that luck
continues; for many have won by following the lead
of lucky players. Now I might, at the outset, point
out that this belief in the continuance of luck is
suggested by an idea directly contradictory to that on
which is based the theory of the `maturity of the
chances.' If the oftener an event has occurred, the
more unlikely is its occurrence at the next trial---the
common belief---then, contrary to the common
belief, the oftener a player has won (that is, the
longer has been his run of luck), the more unlikely is
he to win at the next venture. We cannot separate
the two theories, and assume that the theory of the
maturity of the chances relates to the play, and the
theory of runs of luck to the player. The success of
the player at any trial is as distinctly an event---a
chance event---as the turning up of ace or deuce at the
cast of a die.

What then are we to say of the experience of those
who have won money by following a lucky player?
Let us revert to our coin-tossers. Let us suppose that
the progress of the venture in a given county is made
known to a set of betting men in that county; and
that when it becomes known that a person has tossed
`head' twelve times running, the betting men hasten
to back the luck of that person. Further, suppose this
to happen in every county in England. Now we have
seen that these persons are no more likely to toss a
thirteenth `head,' than they are to fail. About half
will succeed and about half will fail. Thus about half
their backers will win and about half will lose. But
the successes of the winners will be widely announced;
while the mischances of the losers will be concealed.
This will happen---the like notoriously does happen---for
two reasons. First, gamblers pay little attention
to the misfortunes of their fellows: the professed
gambler is utterly selfish, and moreover he hates the
sight of misfortune because it unpleasantly reminds
him of his own risks. Secondly, losing gamblers do
not like their losses to be noised abroad; they object
to having their luck suspected by others, and they are
even disposed to blind themselves to their own ill-fortune
as far as possible. Thus, the inevitable
success of about one-half of our coin-tossers would be
accompanied inevitably by the success of those who
`backed their luck,' and the successes of such backers
would be bruited abroad and be quoted as examples;
while the failure of those who had backed the other
half (whose luck was about to fail them), would be
comparatively unnoticed. Unquestionably the like
holds in the case of public gambling-tables. If any
doubt this, let them inquire what has been heard of
those who continued to back Garcia and other `bank-breakers.'
We know that Garcia and the rest of these
lucky gamblers have been mined; they had risen too
high and were followed too constantly for their fall to
remain unnoticed. But what has been heard of those
unfortunates who backed Garcia after his last successful
evening, and before the change in his luck had been
made manifest? We hear nothing of them, though a
thousand stories are told of those who made money
while Garcia and the rest were `in luck.'

In passing, we may add to these considerations the
circumstance that it is the interest of gaming-bankers
to conceal the misfortunes of the unlucky, and to
announce and exaggerate the success of the fortunate.

I by no means question, be it understood, the
possibility that money may be gained quite safely by
gambling. Granting, first, odds such as the `banks'
have in their favour; secondly, a sufficient capital to
prevent premature collapse; and thirdly, a sufficient
number of customers, success is absolutely certain in
the long run. The capital of the gambling-public doubtless
exceeds collectively the capital of the gambling-banks;
but it is not used collectively: the fortunes of
the gambling-public are devoured successively, the
sticks which would be irresistible when combined, are
broken one by one. I leave my readers to judge
whether this circumstance should encourage gambling
or the reverse.

I may thus present the position of the gambler
who is not ready to secure Fortune as his ally by
trickery:---If he meets gamblers who are not equally
honest, he is not trying his luck against theirs, but
at the best (as De Morgan puts it) only a part of his
luck against the whole of theirs; if he meets players
as honest as himself, he must nevertheless, as Lord
Holland said to Selwyn, `be in earnest and without
irony---en v\'erit\'e le serviteur tr\`es-humble des \'ev\'enements---in
truth, the very humble servant of events.'

\chapter{Fair and Unfair Wagers}

I gave in my `How to Play Whist' (under the head
`Whist Whittlings') a case in which a certain man of
title used to offer freely 1,000\textit{l}. to 1\textit{l}. against the occurrence
of a whist hand containing no card above a nine
---a most unfair wager. Odds of a thousand pounds
to one are very tempting to the inexperienced. `I risk
my pound,' such a one will say, `but no more, and I
may win a thousand.' That is the chance; and what
is the certainty? The certainty is that in the long
run such bets will involve a loss of 1,828\textit{l}. for each
thousand pounds gained, or a net loss of 828\textit{l}. As
certain to all intents as that two and two make four,
a large number of wagers made on this plan would
mean for the clever layer of the odds a very large
gain. Yet Lord Yarborough would probably have
been indignant to a degree if he had been told that in
taking 1 for each hand on which he wagered which
did not prove to be a `Yarborough,' he was in truth
defrauding the holder of the hand
of 9\textit{s}.\ $0\frac{3}{4}$\textit{d}.,
notwithstanding the preliminary agreement, simply because
the preliminary agreement was an unfair one. As to
his being told that even if he had wagered 1,828\textit{l}.
against 1\textit{l}. the transaction would have been intrinsically
immoral, doubtless he and his opponent would equally
have scouted the idea.

A curious instance of the loss of all sense of honour,
or even honesty, which betting begets, occurred to
me when I was in New Zealand. A bookmaker (`by
profession,' as he said), as genial and good-natured
a man as one would care to meet, and with a strong
sense of right and justice outside betting, had learned
somehow that ten horses can come in (apart from dead
heats) in 3,628,800 different ways. This curious piece
of information seemed to him an admirable way of
gaining money from the inexperienced. So he began
to wager about it, endeavouring---though, as will be
seen, he failed---to win money by wagering on a certainty.
Unfortunately, he came early across a man as
cute as himself and a shade
cuter (\emph{\`a brigand brigand et demi}), who worded the
question on which the wager
turns thus:---`In how many ways can ten horses be
placed?' Of course, this is a very different thing.
Only the first three horses can be placed, and the sets
of three which can be made out of ten horses number
only 10 times 9 times 8, or 720 (there are only 120
actual sets of three, but each set can be placed in six
different ways). My genial, but (whatever he thought
himself) not quite honest friend, submitted the matter
to me. Not noticing, at first, the technical use of
the word `placed,' I told him there were 3,628,800
different arrangements: he rejoiced as though the
money wagered were already in his pocket. When
this was corrected, and I told him his opponent had
certainly won, as the question would be understood by
betting men, he was at first depressed; but presently
recovering, he said, `Ah, well; I shall win more out of
this little trick, now I see through it, than I lose this
time.'

It is well to have some convenient standard of
reference, not only as respects the fairness or unfairness
of betting transactions, but as to the true nature of the
chances involved or supposed to be involved. Many
men bet on horse races without any clear idea of the
chances they are really running. To see that this is so,
it is only necessary to notice the preposterous way in
which many bettors combine their bets. I do not say
that many, even among the idiots who wager on horses
they know nothing about, would lay heavier odds against
the winning of a race by one of two horses than he
would lay against the chance of either horse separately;
but it is quite certain that not one bettor in a hundred
knows either how to combine the odds against two,
three, or more horses, so as to get the odds about the
lot, or how to calculate the chance of double, triple, or
multiple events. Yet these are the very first principles
of betting; and a man who bets without knowing anything
about such matters runs as good a chance of
ultimate success as a man who, without knowing the
country, should take a straight line in the hunting-field.

Now, apart from what may be called roguery in
horse-racing, every bet in a race may be brought into
direct comparison with the simple and easily understood
chance of success in a lottery where there is a single
prize, and therefore only one prize ticket: and the
chance of the winner of a race, where several horses
run, being one particular horse, or one of any two, three,
or more horses, can always be compared with the easily
understood chance of drawing a ball of one colour out
of a vase containing so many balls of that colour and so
many of another. So also can the chance of a double
or triple event be compared with a chance of the second
kind.

Let us first, then, take the case of a simple lottery,
and distinguish between a fair lottery and an unfair
one. Every actual lottery, I remark in passing, is an
unfair one; at least. I have never yet heard of a fair
one, and I can imagine no possible case in which it
would be worth anyone's while to start a fair lottery.

Suppose ten persons each contribute a sovereign to
form a prize of 10\textit{l}.; and that each of the ten is allowed
to draw one ticket from among ten, one marked ticket
giving the drawer the prize. That is a fair lottery;
each person has paid the right price for his chance.
The proof is, that if anyone buys up all the chances at
the price, thus securing the certainty of drawing the
marked ticket, he obtains as a prize precisely the sum
he has expended.

This, I may remark, is the essential condition for a
fair lottery, whatever the number of prizes; though we
have no occasion to consider here any case except the
very simple case of a one-prize lottery. Where there
are several prizes, whether equal or unequal in value,
we have only to add their value together: the price for
all the tickets together must equal the sum we thus
obtain. For instance, if the ten persons in our illustrative
case, instead of marking one ticket were to mark
three, for prizes worth 5\textit{l}., 3\textit{l}., and 2\textit{l}., the lottery would
be equally fair. Anyone, by buying up all the ten
tickets, would be sure of all three prizes, that is, he
would pay ten pounds and get ten pounds---a fair
bargain.

But suppose, reverting to one-prize lotteries, that
the drawer of the marked ticket were to receive only 8\textit{l}.
instead of 10\textit{l}. as a prize. Then clearly the lottery
would be unfair. The test is, that a man must pay 10\textit{l}.
to insure the certainty of winning the prize of 8\textit{l}., and
will then be 2\textit{l}. out of pocket. So of all such cases.
When the prize, if there is but one, or the sum of all
the prizes together, if there are several, falls short of the
price of all the tickets together, the lottery is an unfair
one. The sale of each ticket is a swindle; the total
amount of which the ticket-purchasers are swindled
being the sum by which the value of the prize or prizes
falls short of the price of the tickets.

We see at once that a number of persons in a room
together would never allow an unfair lottery of this sort.
If each of the ten persons put a sovereign into the pool,
each having a ticket, the drawer of the prize ticket
would be clearly entitled to the pool. If one of the ten
started the lottery, and if when the 10\textit{l}., including his
own, has been paid in to the pool, he proposed to take
charge of the pool, and to pay 8\textit{l}. to the drawer of the
marked ticket, it would be rather too obvious that he
was putting 2\textit{l}. in his pocket. But lotteries are not
conducted in this simple way, or so that the swindle
becomes obvious to all engaged. As a matter of fact,
all lotteries are so arranged that the manager or managers
of the lottery put a portion of the proceeds (or pool)
into their pockets. Otherwise it would not be worth
while to start a lottery. Whether a lottery is started
by a nation, or for a cause, or for personal profit, it
always is intended for profit; and profit is always
secured, and indeed can only be secured, by making
the total value of the prizes fall short of the sum received
for the tickets.

I would not be understood to say that I regard all
unfair lotteries as swindles. In the case of lotteries for
a charitable purpose I suppose the object is to add
gambling excitement to the satisfaction derived from
the exercise of charity. The unfairness is understood
and permitted; just as, at a fancy fair, excessive prices
are charged, change is not returned, and other pleasantries
are permitted which would be swindles if practised in
real trading. But in passing I may note that even
lotteries of this kind are objectionable. Those who
arrange them have no wish to gain money for themselves;
and many who buy tickets have no wish to win
prizes, and would probably either return any prize they
might gain or pay its full value. But it is not so with
all who buy tickets; and even a charitable purpose will
not justify the mischief done by the encouragement of
the gambling spirit of such persons. In nearly all cases
the money gained by such lotteries might, with a little
more trouble but at less real cost, be obtained directly
from the charitably minded members of the community.

To return, however, to my subject.

I have supposed the case of ten persons gambling
fairly in such a way that each venture made by the ten
results in a single-prize lottery. But as we know, a
betting transaction is nearly always arranged between
two persons only. I will therefore now suppose only
two persons to arrange such a lottery, in this way:---The
prize is 10\textit{l}., as before, and there are ten tickets;
one of the players, A, puts, say, 3\textit{l}. in the pool, while
the other, B, puts 7\textit{l}.; three tickets are marked as
winning tickets; A then draws at random once only;
if he draws a marked ticket, he wins the pool; if he
draws an unmarked ticket, B takes the pool. This is
clearly fair; in fact it is only a modification of the
preceding case. A takes the chances of three of the
former players, while B takes the chances of the remaining
seven. True, there seems to be a distinction.
If we divided the former ten players into two sets, one
of three, the other of seven, there would not be a single
drawing to determine whether the prize should go to
the three or to the seven; each of the ten would draw
a ticket, all the tickets being thus drawn. Yet in
reality the methods are in principle precisely the same.
When the ten men have drawn their tickets in the
former method, three tickets have been assigned at
random to the three men and seven tickets to the other
seven; and the chance that the three have won is the
chance that one of the three tickets is the marked one.
In the latter method there are ten tickets, of which
three are marked; and the chance that A wins the
prize is the chance that at his single drawing he takes
one of the three marked tickets. But obviously the
chance that a certain marked ticket in ten is one of the
three taken at random must be exactly the same as the
chance that a certain ticket taken at random from
among the ten is one of three marked tickets; for each
of these chances is clearly three times as good as the
chance of drawing, at a single trial, one particular ticket
out of ten.

It will be found that we can now test any wager,
not merely determining whether it is fair or unfair, but
the extent to which it is so, if only the actual chance of
the horse or horses concerned is supposed to be known.
Unfortunately, in the great majority of cases bets are
unfair in another way than that which we are for the
moment considering, the odds not only differing from
those fairly representing the chances of the horse or
horses concerned, but one party to the wager having
better knowledge than the other what those chances
are. Cases of this kind will be considered further on.

Suppose that the just odds against a horse in a race
are 9 to 1. By this I mean that so far as the two
bettors are concerned (that is, from all that they know
about the chances of the horse), it is nine times more
likely that the horse will not win the race than that he
will. Now, it is nine times more likely that a particular
ticket among ten will not be drawn at a single trial
than that it will. So the chance of this horse is correctly
represented by the chance of the prize ticket
being drawn in a lottery where there are ten tickets in
all. If two persons arrange such a lottery, and A pays
in 1\textit{l}. to the pool, while the other, B, pays in 9\textit{l}., making
10\textit{l}. in all, A gets a fair return for his money in a single
drawing, one ticket out of the ten being marked for the
prize. A represents, then, the backer of the horse who
risks 1\textit{l}.; B the layer of the odds who risks 9\textit{l}. The
sum of the stakes is the prize, or 10\textit{l}. If A risks less
than 1\textit{l}., while B risks 9\textit{l}., the total prize is diminished;
or if, while A risks 1\textit{l}., B risks less than 9\textit{l}., the total
is diminished. In either case the wrong done to the
other bettor amounts precisely to the amount by which
the total is diminished. If, for instance, A only wagered
18\textit{s}. against B's 9\textit{l}., the case is exactly the same as
though A and B having severally contributed 1\textit{l}. and
9\textit{l}. to a pool, one ticket out of ten having been marked
and A to have one chance only of drawing it (which we
have just seen would be strictly fair), A abstracted two
shillings from the pool. If B only wagered 7\textit{l}. instead
of 9\textit{l}. against A's 1\textit{l}. the case would be just the same
as though, after the pool had been made up as just described,
B had abstracted 2\textit{l}.

Take another case. The odds are 7 to 3 against a
horse. The chance of its winning is the same as that
of drawing a marked ticket out of a bag containing ten,
when three are marked and seven are unmarked. We
know that in this case two players, A and B, forming
the lottery, must severally contribute 3\textit{l}. and 7\textit{l}. to the
pool, and if on a single drawing one of the three marked
tickets appears, then A wins the pool, or 10\textit{l}., whereas
B takes it if one of the seven unmarked tickets is drawn.
If the backer of the horse, instead of wagering 3\textit{l}.,
wagered only 2\textit{l}. against 7\textit{l}., he would be precisely in
the position of a player A, who, having paid in his 3\textit{l}.
to the pool of 10\textit{l}. in all, should abstract a pound therefrom.
If the layer of the odds wagered only 5\textit{l}. against
3\textit{l}., he would be in the position of a player B, who,
having paid in his 7\textit{l}. to the pool of 10\textit{l}. in all, should
abstract 2\textit{l}. therefrom.

Or, if any difficulty should arise in the reader's
mind from this way of presenting matters, let him put
the case thus:---Suppose the sum of the stakes 10\textit{l}.;
then the odds being 7 to 3 against, the case is as though
three tickets were marked for the prize and seven unmarked;
and the two players ought therefore to contribute
severally 3\textit{l}. and 7\textit{l}. to make up the 10\textit{l}. If the
10\textit{l}. is made up in any other way, there is unfairness;
one player puts in too much, the other puts in too
little. If one puts in 2\textit{l}. 10\textit{s}. instead of 3\textit{l}., the other
puts in 7\textit{l}. 10\textit{s}. instead of 7\textit{l}., and manifestly the former
has wronged the latter to the extent of 1\textit{l}., having
failed to put in 10\textit{s}. which he ought to have put in,
and having got the other to put in 10\textit{s}. which ought
not to have been put in. This seems clearer, I find, to
some than the other way of presenting the matter.
But as, in reality, bets are not made in this way, the
other way, which in principle is the same, is more convenient.
Bettors do not take a certain sum of money
for the total of their stakes, and agree how much each
shall stake towards that sum; but they bet a certain
sum against some other sum. It is easy to take either
of these to find out how much \textit{ought} to be staked against
it, and thus to ascertain to what extent the proper total
of the stakes has been affected either in excess or defect.
And we can get rid of any difficulty arising from the
fact that according to the side we begin from we get
either an excess or a defect, by beginning always from
the side of the one who wagers at least as much as he
should do, at the proper odds, whatever they may be.

As a general rule, indeed, the matter is a good deal
simplified by the circumstance that fraudulent bettors
nearly always lay the odds. It is easy to see why. In
fact, one of the illustrative cases above considered has
already probably suggested the reason to the reader.
I showed that when the odds are 9 to 1 and only 7 to 1
is laid, in pounds, the fraud is the same as removing 2\textit{l}.
from a pool of 10\textit{l}.; whereas with the same odds,
backing the horse by 18\textit{s}. instead of 1\textit{l}., corresponded
to removing two shillings from such a pool. Now, if
a fraudulent gambler had a ready hand in abstracting
coins from a pool, and were playing with some one who
did not count the money handed over to him when he
won, it would clearly be the same thing to him whether
he contributed the larger or smaller sum to the pool, for
he would abstract as many coins as he could, and it
would be so much clear gain. But if he could not get
at the pool, and therefore could only cheat by omitting
to contribute his fair share, it would manifestly be far
better for him to be the buyer of the larger share of the
chances. If he bought nine tickets out of ten, he
might put in 7\textit{l}., pretending to put in 9\textit{l}., and pocket
2\textit{l}.; whereas if he only bought one ticket, he could only
defraud his companion by a few shillings out of the
price of that ticket. Now, this is the hardship under
which the fraudulent bettor labours. He cannot, at
least he cannot generally, get at the stakes themselves:
or, which comes to the same thing, he must pay up
in full when he loses, otherwise he has soon to give up
his profitable trade. Of course he may levant without
paying, but this is only to be adopted as a last resource;
and fraudulent betting is too steadily remunerative to
be given up for the value of a single robbery of the
simpler kind. Thus the bettor naturally prefers laying
the odds. He can keep so much more out of the larger
sum which ought to be laid against a horse than he
could out of the smaller sum with which the horse
should be backed.

Then there is another circumstance which still more
strongly encourages the fraudulent bettor to lay the
odds. It is much easier for him to get his victims to
back a horse than to bet against one. In the first
place, the foolish folk who expect to make a fortune by
betting, take fancies for a particular horse, while they
are not so apt to take fancies against any particular
horse. But secondly, and this is the chief reason of
their mode of betting, they want to make a great and
sudden gain at a small risk. They have not time, for
the most part, to make many wagers on any given race;
and to wager large sums against two or three horses
would involve a great risk for a small profit. This,
then, they do not care to do; preferring to back some
particular horse, or perhaps two or three, by which they
risk a comparatively small sum, and may win a large
one. As Mr. Plyant truly remarks in Hawley Smart's
`Bound to Win,' `The public is dramatic in its fancies;
the public has always a dream of winning a thousand
to ten if it can raise the tenner. The public, Mr.
Laceby, knows nothing about racing, but as a rule is
wonderfully up in the story of Theodore's winning the
Leger, after a hundred pounds to a walking-stick had
been laid against him. The public is always putting
down its walking-stick and taking to crutches in consequence.~.~.~.
What the public will back at the lists
the last few days before the Derby would astonish you:
they've dreams, and tips, and fancies about the fifty to
one lot you couldn't imagine.' Is it to be wondered at
that the public finds its tastes in this respect humoured
by the bookmakers, when we remember that it is from
just such wagers as the public like to make that the
bookmaker can most readily obtain the largest slice of
profit?

But we must not fall into the mistake of supposing
that all the foolish folk who back horses at long odds
necessarily lose. On the contrary, many of them win
money---unfortunately for others, and often for themselves.
It would be a very foolish thing to pay 1\textit{l}. for
one of ten tickets in a lottery where the single prize
was only worth 9\textit{l}. Yet some of the foolish fellows
who did this must win the prize, gaining 8\textit{l}. by the
venture. If many others were encouraged to repeat
such a venture, or if he repeated it himself (inferring
from his success that he was born under a lucky star),
they and he would have reason to repent. He might,
indeed, be lucky yet again; and perhaps more than
once. But the more he won in that way, the more he
would trust in his good luck; and in the long run he
would be sure to lose, if all his ventures were of the
same foolish kind as the first.

We see, however, that the foolish bettor in any
given case is by no means certain to lose. Nor is
the crafty bettor who takes advantage of him at all
sure to win. A man might steal 2\textit{l}. or 3\textit{l}. from the
pool, after making up 9\textit{l}. out of the 10\textit{l}., in the
case I have imagined, and yet lose, because his opponent
might be fortunate enough to draw the single
marked ticket, and so win the 7\textit{l}. or 8\textit{l}. left in the
pool.

In reality, however, though quite possibly some
among the foolish bettors not only win money but
even keep what they win, refraining from trying their
luck afresh, it must not be supposed that the fraudulent
bettor exposes himself to the risk of loss in the long
run. He plays a safe game. Every one of his bets is
a partial swindle; yet in each he runs the risk of loss.
His entire series of bets is a complete swindle, in which
he runs no risk whatever of loss, but insures a certain
gain. Let us see how this is done.

Suppose there are two horses in a race, A and B,
and that the betting is 3 to 1 against B. In other words,
the chance of A winning is as the chance of drawing a
marked ticket out of a bag containing four tickets of
which three are marked, while B's chance of winning is
as that of drawing the single unmarked ticket. In this
case, as the odds are in favour of one horse, our bookmaker
will have to do a little backing, which, preferably,
he would avoid. In fact, a race such as this, that is, a
match between two horses, is not altogether to the
bookmaker's taste; and what he would probably do in
this case would be to obtain special information in some
underhand way about the horses, and bet accordingly.
Supposing, however, that he cannot do this, poor
fellow, let us see how he is to proceed to insure profit.
The first thing is to decide on some amount which shall
be staked over each horse; and the theoretically exact
way---the mathematical manner---of swindling would
be as follows:---Suppose that with some person a wager
were made at the just odds in favour of A, in such sort
that the stakes on both sides amounted, let us say, to
1,200\textit{l}.; the fair wager would be 900\textit{l}. to 300\textit{l}. that A
will win; our swindler, however, having found some
greenhorn X, whom he can persuade to take smaller
odds, takes his book and writes down quickly 800\textit{l}. to
300\textit{l}. in favour of A. He now finds some other greenhorn,
Y, who is very anxious to back A, and having
duly bewailed his misfortune in having no choice but
to lay against a horse who is---so he says---almost
certain to win, he asks and obtains the odds of 900\textit{l}.
to 200\textit{l}. in favour of A; that is to say, he wagers 200\textit{l}.
to 900\textit{l}.\ against A. Let us see how his book stands.
He has wagered---\\[2mm]
\hspace*{1cm}800\textit{l}.\ to 300\textit{l}.\ with X, that A wins;\\
\hspace*{1cm}200\textit{l}.\ to 900\textit{l}.\ with Y, that B wins.\\[2mm]
If A wins, he receives 300\textit{l}.\ from X, and pays 200\textit{l}.\ to
Y, pocketing a balance of 100\textit{l}. If B wins, he pays
800\textit{l}.\ to X and receives 900\textit{l}.\ from Y, pocketing equally
100\textit{l}.

The system by which bookmakers win has great
advantages over the plan formerly adopted at public
gaming-houses, and probably adopted still, though less
publicly. At the gaming-house the bankers did run
some little risk. They were bound to win in the long
run; but they might lose for a night or two, or might
even have a tolerably long run of bad luck. But a
judicious bookmaker can make sure of winning money
on every great race. Of course, if the bookmakers like
a little excitement---and they are men, after all, though
they do make their own providence---they can venture
a little more than the nothing they usually venture.
For instance, instead of laying the odds against all the
horses, they can lay against all but one, and back that
one heavily. Then, if that horse wins, they `skin the
lamb,' in the pleasing language of their tribe. But the
true path to success is that which I have indicated
above, and they know it (or I would assuredly not have
indicated it).

Still, in every depth there is a deeper still. In the
cases hitherto considered I have supposed that the
chances of a horse really are what the public odds
indicate. If they are not, it might be supposed that
only the owner of the horse and a few friends, besides
the trainer, jockey, and one or two other \textit{employ\'es},
would know of this. But, as a matter of fact, the
bookmakers generally find out tolerably soon if anything
is wrong with a horse, or if he has had a very
good trial and has a better chance of winning than had
till then been supposed. Before very long this knowledge
produces its effect in bringing the horse to its true
price, or near it. In the former case the horse is very
diligently `pencilled' by the bookmakers, and recedes
step by step in the betting, till he is either at long odds
or is no longer backed at any price. In the latter, the
horse is as diligently backed, till he has reached short
odds, taking his place among the favourites, or perhaps
as first favourite.

But in either process---that of driving a horse to long
odds, or that of installing him in a position among the
favourites, according to the circumstances---a great deal
of money is made and lost---made by those who know
what has really happened, lost by those who do not.
We may be tolerably sure it is not `the public' which
gains. It is to `the professional,' naturally, that the
information comes first, and he makes a handsome profit
out of it, before the change in the betting shows the
public what has happened.

Now here, unfortunately, we touch on a part of our
subject which affects men who are not, in a proper sense
of the word, `bookmakers.' It is a singular
circumstance---or rather it is not at all singular, but accords
with multiplied experiences, showing how the moral
nature gets warped by gambling transactions---that
men who are regarded by the world, and regard themselves,
as gentlemen, seem to recognise nothing dishonourable
in laying wagers which they \textit{know} not to
accord with the real chances of a horse. A man who
would scorn to note the accidental marks on the backs
of playing cards, and still more to make such marks,
will yet avail himself of knowledge just as unfair in
horse-racing as a knowledge of the backs of certain
cards would be in whist or \'ecart\'e.

I have elsewhere cited as an illustration the use
which Hawley Smart, in one of his novels (`Bound to
Win'), makes of this characteristic of sporting men.
It has been objected, somewhat inconsistently, that in
the first place the novelist's picture is inaccurate, and
in the second the use which the hero of that story makes
of knowledge about his own horses was perfectly legitimate.
As to the first point, I may remark that I do
not need to read Hawley Smart's novels, or any novels,
to be well assured that the picture is perfectly accurate,
and that sporting men do make use of special knowledge
about a horse's chances to make profitable wagers.
As to the second point, I note that it well illustrates
my own position, that gambling has the effect of
darkening men's sense of right and wrong: it shows
that many sporting men regard as legitimate what is
manifestly unfair.

Not to go over ground already trodden, I turn to
another of Hawley Smart's lively tales, the hero of
which is a much more attractive man than Harold
Luxmore in `Bound to Win'---Grenville Rose in `A
Race for a Wife.' He is not, for a wonder, a sporting
hero; in everything but the racing arrangements,
which he allows to be made in his name, he behaves
much as a gentleman should, and manifestly he is
intended to represent an English gentleman. He comes
across information which shows that, by the action of
an old form of tenure called `right of heriot,' a certain
horse which is the leading favourite for the Two Thousand
can be claimed and so prevented from running.
Of the direct use of this information, to free the heroine
from a rascally sporting lawyer, nothing need be said
but `serve the fellow right.' Another use is, however,
made of the knowledge thus obtained, and it is from
this use that the novel derives its name. To a racing
friend of his, a lawyer (like himself and the villain of
the story), the hero communicates the secret. To him
the racing friend addresses this impressive response:---`Look
here, old fellow. Racing is business with me;
if you're not in for a regular mare's nest, there's heaps
of money to be made out of this~.~.~.~.~don't whisper
it to your carpet-bag till you've seen me again. I say
this honestly, (!) with a view to doing my best for you.'
What this best is presently appears. I need not follow
the workings of the plot, nor tell the end of the story.
All that answers my present purpose is to indicate the
nature of the `book' which the gentlemanly Dallison,
Silky Dallison as his friends call him, succeeds in
making for himself and his equally gentlemanly friend
on the strength of the `tip' given by the latter. `We
now stand to win between us 10,170\textit{l}. if Coriander wins
the Two Thousand, and just quits if he loses; not a
bad book, Grenville!' To which Grenville, nothing
loth, responds, `By Jove! no.' Yet every wager by
which this result has been obtained, if rightly considered,
was as certainly a fraud as a wager laid upon a
throw with cogged dice. For, what makes wagers on
such throws unfair, except the knowledge that with such
dice a certain result is more likely than any other? and
what essential difference is there between such knowledge
about dice and special knowledge about a horse's chance
in a race? The doctrine may not be pleasant to sporting
gentlemen who have not considered the matter, but once
duly considered there cannot be a doubt as to its truth:
a wager made with an opponent who does not possess
equally accurate information about the chances involved,
is not a fair wager but a fraud. It is a fraud of the same
kind as that committed by a man who wagers after the
race, knowing what the event of the race has been; and
it only differs from such a fraud in degree in the same
sense that robbing a till differs from robbing a bank.

It may be argued that by the same reasoning good
whist players defraud inferior players who play with
them for equal stakes. But the cases are altogether
different. Good whist players do not conceal their
strength. Their skill is known; and if inferior players
choose to play on equal terms, trusting in good luck to
befriend them, they do it at their own risk. If a
parallel is to be sought from the whist-table, it would
be rather derived from the case of two players who
had privately arranged a system of signalling; for in
such a case there is knowledge on one side which is not
only wanting on the other side, but of the possession of
which the other partners have no suspicion. No one would
hesitate to call that swindling. Now take the case of
one who knows that, as the result of a certain trial, a
horse which is the favourite in a great race will take
part in it, indeed, but will only do so to make running
for a better horse. Until the time when the owner of
the horses declares to win with the latter, such knowledge
enables its possessor to accept safely all wagers in
favour of the horse; and he knows perfectly well, of
course, that not one such wager is offered him except
by persons ignorant of the true state of the case. Even
if such offers are made by bookmakers, whose profession
is swindling, and though we may not have a particle of
sympathy with such men when they lose in this way,
the acceptance of such wagers is in no sense justified.
Two wrongs do not, in this case more than in any other
make a right.

I have said that in every depth there is a deeper
still. In the subject I am dealing with there is a deepest
depth of all. I will not, however, sully these pages
with the consideration of the foulest of the rascalities
to which horse-racing has led. Simply to show those
who bet on horse-races how many risks of loss they expose
themselves to, I mention that some owners of horses
have been known to bring about the defeat of their own
horse, on which the foolish betting public had wagered
large sums, portions of which find their way into the
pockets of the dishonest owners aforementioned. I
may add that, according to an old proverb, there are
more ways of killing a cat than by choking it with
cream. A horse may be most effectually prevented from
winning without any such vulgar devices as pulling,
roping, and so forth. So also a horse, whose owner is
honest, may be `got at' after other fashions than have
been noted yet, either in the police courts or in sporting
novels.

Let us turn, however, from these unsavoury details,
and consider briefly the objections which exist against
gambling, even in the case of cash transactions so conducted
that no unfair advantage is taken on either
side.

The object of all gambling transactions is to win
without the trouble of earning. I apprehend that nearly
every one who wagers money on a horse race has, for
some reason or other, faith in his own good fortune.
It is a somewhat delicate question to determine how far
such faith makes gambling unfair. For if, on the one
hand, we must admit that a really lucky man could not
fairly gamble against others not so lucky, yet, as it is
absolutely certain in the scientific sense that no such
thing as \emph{luck which may be depended upon} exists, it is
difficult to say how far faith in a non-existent quality
can be held to make that fraudulent which would certainly
be fraudulent did the quality exist. Possibly if a
man, A, before laying a wager with another, B, were to
say, `I have won nearly every bet I have made,' B might
decline to encounter A in any wager. In the case of a
man who had been so lucky as A, it is quite probable
that, supposing a wager made with B and won by A, B
would think he had been wronged if A afterwards told
him of former successes. B might say, `You should
have told me that before I wagered with you; it is not
fair to offer wagers where you know you have a better
chance of winning than your opponents.' And though
B would, strictly speaking, be altogether wrong, he
would be reasoning correctly from his incorrect assumption,
and A would be unable to contradict him.

If we were to assume that every man who wagered
because he had faith in his own good luck was guilty of
a moral though not of a logical or legal wrong, we
should have to regard ninety-nine gamblers out of a
hundred as wrong-doers. Let it suffice to point out that,
whether believing in his luck or not, the gambler is
blameworthy, since his desire is to obtain the property
of another without giving an equivalent. The interchange
of property is of advantage to society; because,
if the interchange is a fair one, both parties to the transaction
are gainers. Each exchanges something which
is of less use to him for something which is of more use.
This is equally the case whether there is a direct exchange
of objects of value, or one of the parties to the
exchange gives the other the benefit of his labour or of
his skill acquired by labour. But in gambling, as
where one man robs another, the case is otherwise.
One person has lost what he can perhaps ill spare, while
the other has obtained what he has, strictly speaking,
no right to, and what is almost certainly of less value
to him than to the person who has lost it. Or, as
Herbert Spencer concisely presents the case:---`Benefit
received does not imply effort put forth, and the happiness
of the winner involves the misery of the loser: this
kind of action is therefore essentially anti-social; it sears
the sympathies, cultivates a hard egoism, and so produces
a general deterioration of character and conduct.'

\chapter{Betting on Races}

When I was travelling in Australasia, I saw a good
deal of a class of men with whom, in this country,
only betting men are likely to come much in contact---bookmakers,
or men who make a profession of betting.
What struck me most, perhaps, at first was
that they regarded their business as a distinct profession.
Just as a man would say in England, `I am
a lawyer or a doctor,' so these men would say that they
were bookmakers. Yet, on consideration, I saw that
there was nothing altogether novel in this. Others,
whose business really is to gain money by making
use of the weaknesses of their fellow-men, have not
scrupled to call their employment a trade or a profession.
Madame Rachel might have even raised her
special occupation to the dignity of `a mystery' on
Shakespearean grounds (`Painting, sir, I have heard
say is a mystery, and members of my occupation using
painting, do prove my occupation a mystery'); and if
aught of wrong in his employment could be made out
to the satisfaction of a bookmaker, his answer might be
Shakespearean also, `Other sorts offend as well as we---ay,
and better (qy.\ bettor) too.'

My own views about betting and bookmaking are
regarded by many as unduly harsh, though I have
admitted that the immorality which I find in betting
has no existence with those who have not weighed
the considerations on which a just opinion is based.
I regard betting as essentially immoral so soon as its
true nature is recognised. When a wager is made,
and when after it has been lost and won its conditions
are fulfilled, money has passed from one person to
another without any `work done' by which society is
benefited. The feeling underlying the transaction has
been greed of gain, however disguised as merely strong
advocacy of some opinion---an opinion, perhaps, as to
whether some horse will run a certain distance faster
than another, whether certain dice will show a greater
or less number of points, or the like. If here and there
some few are to be found so strangely constituted
mentally as really to take interest in having correct
opinions on such matters, they are so few that they do
not affect the general conclusion. They may bet to
show they really think in such and such a way, and
not to win money; but the great majority of betting
men, professional (save the mark) or otherwise, want
to win money, which is right enough, and to win
money without working or doing some good for it,
which is essentially immoral. That in a very large
proportion of cases this negative immorality assumes
a positive form---men trying to make unfair wagers
(by betting with unfair knowledge of the real chances)---no
one acquainted with the betting world, no one
who reads a sporting paper, no one even who reads
the sporting columns of the daily papers, can fail to
see. Why, if half the assurances of the various sporting
prophets were trustworthy, betting, assisted by their
instructions, would be as dishonourable as gambling
with marked cards, as dishonest as picking pockets.
Here is my `Vaticinator,'%
\footnote{I hope there is no turf prophet with this \textit{nom-de-plume}. I
know of none, or I would not use the name; but it may have been
hit upon by some sporting man with a taste for polysyllables.}
the betting man might say,
who says that Roguery is almost sure to win the
`Beggar my Neighbour' stakes, but if he does not, that
speedy mare, Rascality, will unquestionably win. Here
are the bookmakers, who seem all quite as ready to lay
the odds against Roguery and Rascality as against any
of the other horses, to say nothing of my friends, Verdant
and Flathead, who will freely back any of these
latter. Now, if I back Roguery and Rascality with the
bookmakers, and lay odds against the certain losers in
the race, I shall certainly win all round. Of course,
`Vaticinator' is not the prophet he claims to be, but
the betting man of our soliloquy supposes that he is;
and so far as the morality of the course the latter follows
is concerned the case is the same as though `Vaticinator's'
prophecies were gospel. There is not a particle
of real distinction between what the bettor wants to do,
and what a gambler, with cogged dice or marked cards,
actually does. The more knowing a betting man claims
to be, the easier it is to see that he wants and expects
to take unfair advantage of other men. Either he
knows more than those he bets with about the real
conditions of the race or contest on which they wager,
or he does not. If he does, he wagers with them unfairly,
and might as well pick their pockets. If he does
not, but fancies he does, he is as dishonest in intention
as he is in the former case in reality. If he does not,
and knows he does not, he simply lies in claiming to
know more than he does. In claiming to be knowing,
he really claims to be dishonest and (which is not
quite the same thing) dishonourable; and probably his
claim is just.

To turn, however, to betting on horse-races as actually
conducted.

There appears every day in the newspapers an
account of the betting on the principal forthcoming
races. The betting on such races as the Two Thousand
Guineas, the Derby, and the Oaks, often begins more
than a year before the races are run; and during the
interval, the odds laid against the different horses
engaged in them vary repeatedly, in accordance with
the reported progress of the animals in their training,
or with what is learned respecting the intentions of
their owners. Many who do not bet themselves find
an interest in watching the varying fortunes of the
horses which are held by the initiated to be leading
favourites, or to fall into the second rank, or merely to
have an outside chance of success. It is amusing to
notice, too, how frequently the final state of the odds is
falsified by the event; how some `rank outsider' will
run into the first place, while the leading favourites are
not even `placed.'

It is in reality a simple matter to understand the
betting on races (or contests of any kind), yet it is
astonishing how seldom those who do not actually bet
upon races have any inkling of the meaning of those
mysterious columns which indicate the opinion of the
betting world respecting the probable results of approaching
contests, equine or otherwise.

Let us take a few simple cases of `odds,' to begin
with; and, having mastered the elements of our subject,
proceed to see how cases of greater complexity are to be
dealt with.

Suppose the newspapers inform us that the betting
is 2 to 1 against a certain horse for such and such a
race, what inference are we to deduce? To learn this,
let us conceive a case in which the \textit{true} odds against a
certain event are as 2 to 1. Suppose there are three
balls in a bag, one being white, the others black. Then,
if we draw a ball at random, it is clear that we are
twice as likely to draw a black as to draw a white ball.
This is technically expressed by saying that the odds
are 2 to 1 \textit{against} drawing a white ball; or 2 to 1 \textit{on}
(that is, in favour of) drawing a black ball. This
being understood, it follows that, when the odds are
said to be 2 to 1 against a certain horse, we are to infer
that, in the opinion of those who have studied the performance
of the horse, and compared it with that of
the other horses engaged in the race, his chance of
winning is equivalent to the chance of drawing one
particular ball out of a bag of three balls.

Observe how this result is obtained: the odds are 2
to 1, and the chance of the horse is as that of drawing
one ball out of a bag of three---three being the sum of
the two numbers 2 and 1. This is the method followed
in all such cases. Thus, if the odds against a horse
are 7 to 1, we infer that the \textit{cognoscenti} consider his
chance equal to that of drawing one particular ball out
of a bag of \textit{eight}.

A similar treatment applies when the odds are not
given as so many to one. Thus, if the odds against a
horse are as 5 to 2, we infer that the horse's chance is
equal to that of drawing a white ball out of a bag
containing five black and two white balls---or seven in
all.

We must notice also that the number of balls may
be increased to any extent, provided the proportion
between the total number and the number of a specified
colour remains unchanged. Thus, if the odds are 5 to 1
against a horse, his chance is assumed to be equivalent
to that of drawing \textit{one} white ball out of a bag containing
six balls, only one of which is white; \textit{or} to that of
drawing a white ball out of a bag containing sixty balls,
of which ten are white---and so on. This is a very
important principle, as we shall now see.

Suppose there are two horses (amongst others)
engaged in a race, and that the odds are 2 to 1 against
one, and 4 to 1 against the other---what are the odds
that one of the two horses will win the race? This
case will doubtless remind my readers of an amusing
sketch by Leech, called---if I remember rightly---`Signs
of the Commission.' Three or four undergraduates are
at a `wine', discussing matters equine. One propounds
to his neighbour the following question:---`I say, Charley,
if the odds are 2 to 1 against \textit{Rataplan}, and 4 to 1
against \textit{Quick March}, what's the betting about the
pair?'---`Don't know, I'm sure,' replies Charley, `but
I'll give you 6 to 1 against them.' The absurdity of
the reply is, of course, very obvious; we see at once
that the odds cannot be heavier against a pair of
horses than against either singly. Still, there are
many who would not find it easy to give a correct reply
to the question. What has been said above, however,
will enable us at once to determine the just odds in
this or any similar case. Thus---the odds against one
horse being 2 to 1, his chance of winning is equal to
that of drawing one white ball out of a bag of \textit{three},
one only of which is white. In like manner, the
chance of the second horse is equal to that of drawing
one white ball out of a bag of \textit{five}, one only of which is
white. Now we have to find a number which is a
multiple of both the numbers three and five. Fifteen
is such a number. The chance of the first horse, modified
according to the principle explained above, is equal
to that of drawing a white ball out of a bag of fifteen
of which \textit{five} are white. In like manner, the chance of
the second is equal to that of drawing a white ball out
of a bag of fifteen of which \textit{three} are white. Therefore
the chance that \textit{one of the two} will win is equal to that
of drawing a white ball out of a bag of fifteen balls, of
which \textit{eight} (\textit{five} added to \textit{three}) are white. There
remain \textit{seven} black balls, and therefore the odds are
8 to 7 \textit{on} the pair.

To impress the method of treating such cases on the
mind of the reader, let us take the betting about three
horses---say 3 to 1, 7 to 2, and 9 to 1 \textit{against} the three
horses respectively. Then their respective chances
are equal to the chance of drawing (1) one white ball
out of \textit{four}, one only of which is white; (2) a white
ball out of \textit{nine}, of which two only are white; and (3)
one white ball out of \textit{ten}, one only of which is white.
The least number which contains four, nine, and ten is
180; and the above chances, modified according to the
principle explained above, become equal to the chance
of drawing a white ball out of a bag containing 180 balls,
when 45, 40, and 18 (respectively) are white. Therefore,
the chance that one of the three will win is equal
to that of drawing a white ball out of a bag containing
180 balls, of which 103 (the sum of 45, 40, and 18)
are white. Therefore, the odds are 103 to 77 \textit{on} the
three.

One does not hear in practice of such odds as 103
to 77. But betting-men (whether or not they apply
just principles of computation to such questions is
unknown to me) manage to run very near the truth.
For instance, in such a case as the above, the odds on
the three would probably be given as 4 to 3---that is,
instead of 103 to 77 (or 412 to 308), the published
odds would be equivalent to 412 to 309.

And here a certain nicety in betting has to be mentioned.
In running the eye down the list of odds, one
will often meet such expressions as 10 to 1 against
such a horse \textit{offered}, or 10 to 1 \textit{wanted}. Now, the
odds of 10 to 1 \textit{taken} may be understood to imply that
the horse's chance is equivalent to that of drawing a
certain ball out of a bag of eleven. But if the odds
are offered and not taken, we cannot infer this. The
offering of the odds implies that the horse's chance is
\textit{not better} than that above mentioned, but the fact that
they are not taken implies that the horse's chance is
\textit{not so good}. If no higher odds are offered against the
horse, we may infer that his chance is \textit{very little worse}
than that mentioned above. Similarly, if the odds of
10 to 1 are \textit{asked for}, we infer that the horse's chance
is \textit{not worse} than that of drawing one ball out of eleven;
if the odds are not obtained, we infer that his chance is
\textit{better}; and if no lower odds are asked for, we infer that
his chance is \textit{very little better}.

Thus, there might be \textit{three} horses (A, B, and C)
against whom the nominal odds were 10 to 1, and yet
these horses might not be equally good favourites,
because the odds might not be taken, or might be
asked for in vain. We might accordingly find three
such horses arranged thus:

\begin{center}
\begin{tabular}{lcl}
&&Odds.\\
A &\ldots& 10 to 1 (wanted).\\
B &\ldots& 10 to 1 (taken).\\
C &\ldots& 10 to 1 (offered).\\
\end{tabular}
\end{center}

Or these different stages might mark the upward or
downward progress of the same horse in the betting.
In fact, there are yet more delicate gradations, marked
by such expressions respecting certain odds, as---\textit{offered freely},
\emph{offered}, \emph{offered and taken} (meaning that
some offers only have been accepted), \textit{taken}, \emph{taken and
wanted}, \emph{wanted}, and so on.

As an illustration of some of the principles I have
been considering, let us take from the day's papers%
\footnote{This was written early in March 1868.}
the state of the odds respecting the `Two Thousand
Guineas.' It is presented in the following form:

\begin{center}
TWO THOUSAND GUINEAS.\\[2mm]
\begin{tabular}{lll}
7 to 2 against &\textit{Rosicrucian} &(off.).\\
6 to 1 against &\textit{Pace} &(off.; 7 to 1 w.).\\
10 to 1 against &\textit{Green Sleeve} &(off.).\\
100 to 7 against &\textit{Blue Gown} &(off.).\\
180 to 80 against &Sir J. Hawley's lot &(t.).
\end{tabular}
\end{center}

This table is interpreted thus: bettors are willing to
lay the same odds against \textit{Rosicrucian} as would be the
true mathematical odds against drawing a white ball
out of a bag containing two white and seven black
balls; but no one is willing to back the horse at this
rate. On the other hand, higher odds are not offered
against him. Hence it is presumable that his chance
is but slightly less than that above indicated. Again,
bettors are willing to lay the same odds against \textit{Pace}
as might fairly be laid against drawing one white ball
out of a bag of seven, one only of which is white; but
backers of the horse consider that they ought to get
the same odds as might be fairly laid against drawing
the white ball when an additional black ball had been
put into the bag. As respects \emph{Green Sleeve} and \emph{Blue
Gown}, bettors are willing to lay the odds which there
would be, respectively, against drawing a white ball
out of a bag containing---(1) eleven balls, one only of
which is white, and (2) one hundred and seven balls,
seven only of which are white. Now, the three horses,
\emph{Rosicrucian}, \emph{Green Sleeve}, and \emph{Blue Gown}, all belong
to Sir Joseph Hawley, so that the odds about the
three are referred to in the last statement of the list
just given. And since none of the offers against the
three horses have been taken, we may expect the odds
actually taken about `Sir Joseph Hawley's lot' to be
more favourable than those obtained by summing
up the three former in the manner we have already
examined. It will be found that the resulting odds
(offered) against Sir J. Hawley's lot---estimated in
this way---should be, as nearly as possible, 132 to 80.
We find, however, that the odds \emph{taken} are 180 to 80.
Hence, we learn that the offers against some or all of
the three horses are considerably short of what backers
require; or else that some person has been induced
to offer far heavier odds against Sir J. Hawley's lot
than are justified by the fair odds against his horses,
severally.

I have heard it asked why a horse is said to be a
favourite, though the odds may be against him. This
is very easily explained. Let us take as an illustration
the case of a race in which four horses are engaged to
run. If all these horses had an equal chance of winning,
it is very clear that the case would correspond to
that of a bag containing four balls of different colours;
since, in this case, we should have an equal chance of
drawing a ball of any assigned colour. Now, the odds
against drawing a particular ball would clearly be 3 to
1. This, then, should be the betting against each of
the three horses. If any one of the horses has less odds
offered against him, he is a \textit{favourite}. There may be
more than one of the four horses thus distinguished;
and, in that case, the horse against which the least
odds are offered is \textit{the first favourite}. Let us suppose
there are two favourites, and that the odds against the
leading favourite are 3 to 2, those against the other 2
to 1, and those against the best non-favourite 4 to 1;
and let us compare the chances of the four horses. I
have not named any odds against the fourth, because, if
the odds against all the horses but one are given, the
just odds against that one are determinable, as we shall
see immediately. The chance of the leading favourite
corresponds to the chance of drawing a ball out of a bag
in which are three black and two white balls, \textit{five} in all;
that of the next to the chance of drawing a ball out of a
bag in which are two black and one white ball, \textit{three} in
all; that of the third, to the chance of drawing a ball
out of a bag in which are four black balls and one white
one, \textit{five} in all. We take, then, the least number containing
both five and three---that is, \textit{fifteen}; and then
the number of white balls corresponding to the chances
of the three horses are respectively six, five, and three,
or fourteen in all; leaving only \textit{one} to represent the
chance of the fourth horse (against which the odds are
therefore 14 to 1). Hence the chances of the four
horses are respectively as the numbers \textit{six}, \textit{five}, \textit{three},
and \textit{one}.

I have spoken above of the published odds. The
statements made in the daily papers commonly refer to
wagers actually made, and therefore the uninitiated
might suppose that everyone who tried would be able
to obtain the same odds. This is not the case. The
wagers which are laid between practised betting-men
afford very little indication of the prices which would be
forced (so to speak) upon an inexperienced bettor.
Bookmakers---that is, men who make a series of bets
upon several or all of the horses engaged in a race---naturally
seek to give less favourable terms than the
known chances of the different horses engaged would
suffice to warrant. As they cannot offer such terms to the
initiated, they offer them---and in general successfully---to
the inexperienced.

It is often said that a man may so lay his wagers
about a race as to make sure of gaining money whichever
horse wins the race. This is not strictly the case.
It is of course possible to make sure of winning if the
bettor can only get persons to lay or take the \textit{odds he
requires to the amount he requires}. But this is precisely
the problem which would remain insoluble if all bettors
were equally experienced.

Suppose, for instance, that there are three horses
engaged in a race with equal chances of success. It is
readily shown that the odds are 2 to 1 against each.
But if a bettor can get a person to take even betting
against the first horse (A), a second person to do the
like about the second horse (B), and a third to do the
like about the third horse (C), and if all these bets are
made to the same amount---say 1,000\textit{l}.---then, inasmuch
as only one horse can win, the bettor loses 1,000\textit{l}.
on that horse (say A), and gains the same sum
on each of the two horses B and C. Thus, on the
whole, he gains 1,000\textit{l}., the sum laid out against each
horse.

If the layer of the odds had laid the true odds to the
same amount on each horse, he would neither have
gained nor lost. Suppose, for instance, that he laid
1,000\textit{l}. to 500\textit{l}. against each horse, and A won; then
he would have to pay 1,000\textit{l}. to the backer of A, and to
receive 500\textit{l}. from each of the backers of B and C.
In like manner, a person who had backed each horse
to the same extent would neither lose nor gain by the
event. Nor would a backer or layer who had wagered
\emph{different} sums \emph{necessarily} gain or lose by the race; he
would gain or lose \emph{according to the event}. This will at
once be seen, on trial.

Let us next take the case of horses with unequal
prospects of success---for instance, take the case of the
four horses considered above, against which the odds
were respectively 3 to 2, 2 to 1, 4 to 1, and 14 to 1.
Here, suppose the same sum laid against each, and for
convenience let this sum be 84\textit{l}. (because 84 contains
the numbers 3, 2, 4, and 14). The layer of the odds
wagers 84\textit{l}. to 56\textit{l}. against the leading favourite, 84\textit{l}. to
42\textit{l}. against the second horse, 84\textit{l}. to 21\textit{l}. against the
third, and 84\textit{l}. to 6\textit{l}. against the fourth. Whichever
horse wins, the layer has to pay 84\textit{l}.; but if the
favourite wins, he receives only 42\textit{l}. on one horse, 21\textit{l}.
on another, and 6\textit{l}. on the third---that is 69\textit{l}. in all, so
that he loses 15\textit{l}.; if the second horse wins, he has to
receive 56\textit{l}., 21\textit{l}., and 6\textit{l}.---or 83\textit{l}. in all, so that he
loses 1\textit{l}.; if the third horse wins, he receives 56\textit{l}., 42\textit{l}.,
and 6\textit{l}.---or 104\textit{l}. in all, and thus gains 20\textit{l}.; and lastly,
if the fourth horse wins, he has to receive 56\textit{l}., 42\textit{l}., and
21\textit{l}.---or 119\textit{l}. in all, so that he gains 35\textit{l}. He clearly
risks much less than he has a chance (however small)
of gaining. It is also clear that in all such cases the
worst event for the layer of the odds is that the
first favourite should win. Accordingly, as professional
bookmakers are nearly always layers of odds, one
often finds the success of a favourite spoken of in the
papers as a `great blow for the bookmakers,' while the
success of a rank outsider will be described as a `misfortune
to backers.'

But there is another circumstance which tends to
make the success of a favourite a blow to layers of the
odds and \emph{vice vers\^a}. In the case we have supposed,
the money actually pending about the four horses
(that is, the sum of the amounts laid \emph{for} and \emph{against}
them) was 140\textit{l}. as respects the favourite, 126\textit{l}. as
respects the second, 105\textit{l}. as respects the third, and
90\textit{l}. as respects the fourth. But, as a matter of fact,
the amounts pending about the favourites bear always
a much greater proportion than the above to the
amounts pending about outsiders. It is easy to see the
effect of this. Suppose, for instance, that instead of
the sums 84\textit{l}. to 56\textit{l}., 84\textit{l}. to 42\textit{l}., 84\textit{l}. to 21\textit{l}., and 84\textit{l}.
to 6\textit{l}., a bookmaker had laid 8,400\textit{l}. to 5,600\textit{l}., 840\textit{l}. to
420\textit{l}., 84\textit{l}. to 21\textit{l}., and 14\textit{l}. to 1\textit{l}., respectively---then
it will easily be seen that he will lose 7,958\textit{l}. by
the success of the favourite; whereas he would gain
4,782\textit{l}. by the success of the second horse, 5,937\textit{l}. by
that of the third, and 6,027\textit{l}. by that of the fourth.
I have taken this as an extreme case; as a general
rule, there is not so great a disparity as has been here
assumed between the sums pending on favourites and
outsiders.

Finally, it may be asked whether, in the case of
horses having unequal chances, it is possible that wagers
can be so proportioned (just odds being given and
taken) that, as in the former case, a person backing or
laying against all the four shall neither gain nor lose.
It is so. All that is necessary is, that the sum actually
pending about each horse shall be the same. Thus, in
the preceding case, if the wagers 9\textit{l}. to 6\textit{l}., 10\textit{l}. to 5\textit{l}.,
12\textit{l}. to 3\textit{l}., and 14\textit{l}. to 1\textit{l}., are either laid or taken by
the same person, he will neither gain nor lose by the
event, whatever it may be. And therefore if unfair
odds are laid or taken about all the horses, in such a
manner that the amounts pending on the several horses
are equal (or nearly so), the unfair bettor must win by
the result. Say, for instance, that instead of the above
odds, he lays 8\textit{l}. to 6\textit{l}., 9\textit{l}. to 5\textit{l}., 11\textit{l}. to 3\textit{l}., and 13\textit{l}. to
1\textit{l}. against the four horses respectively; it will be found
that he \emph{must} win 1\textit{l}. Or if he \emph{takes} the odds 18\textit{l}. to
11\textit{l}., 20\textit{l}. to 9\textit{l}., 24\textit{l}. to 5\textit{l}., and 28\textit{l}. to 1\textit{l}. (the just
odds being 18\textit{l}. to 12\textit{l}., 20\textit{l}. to 10\textit{l}., 24\textit{l}. to 6\textit{l}., and 28\textit{l}.
to 2\textit{l}. respectively), he will win 1\textit{l}. by the race. So
that, by giving or taking such odds to a sufficiently
large amount, a bettor would be certain of pocketing
a considerable sum, whatever the event of a given race
might be.

It is by no means necessary that the system I have
described above should be carried out in a precise and
formal manner. If you have a tolerably large capital,
or if, in case of failure, you have courage (greatly
daring) to run away, you may leave a little to chance
on every race, and then, if chance favours you, your
gains will be proportionately greater.

But for supreme success on the turf, wider measures
must be adopted, which may now be sketched in outline.
The system is exceedingly simple---and it will
be found that when the method of the great bookmakers
is analysed a little, there underlies it the fundamental
idea of the system---yet probably not one
among them knows anything about it in detail, though
he may thoroughly well understand that his method
leaves very little to chance.

Viewing the matter then from the point of view of
those who make a business of betting on horses, and
regard themselves as in the profession, here are the
rules for a success:

First, the bookmaker must always lay odds against
horses, never back them. This is not essential to the
system regarded in its scientific aspect; but in practice,
as will presently appear, it makes it easier to
apply it.

Next, he lays against nearly every horse in a race
as early as possible, when the odds are longest. If he
lays against a few which are certain not to run, so
much the better for him; that is so much clear gain to
start with. He should proportion his wagers so that
the sum of what he lays against a horse, and what he
is backed for, may amount to about the same for each
horse. The precise system requires that it should be
exactly the same, but the bookmaker often improves
upon that by taking advantage, in special cases, of his
own knowledge of a horse's chance and his opponent's
inexperience. In every case he lays odds a point or
two short of the legitimate odds against a horse. Suppose
for a moment that the odds are ten to one against
the horse, then it is always easy to find folk who rather
fancy the horse, and think the odds are not eight to
one, or even six to one, against him; he selects such
persons for his wagers about that horse. He conveys
carefully the idea that he thinks the horse's chance
underrated at eight, or even nine to one; but, as a
favour, he will make the odds nine to one. Of course,
he has no occasion to search about for those who favour
any given horse. Every greenhorn has a fancy for
some horse, and is willing to take something short of
the current odds for the privilege of backing him. The
bookmaker can therefore fill in his book \textit{pro re nat\^a},
until at least he has made up sufficient amounts for
most of the horses engaged, when, of course, he gives
more special attention to those whose leaf in his book
is as yet incomplete.

Now, let us take an illustrative case to see how this
system works:

Suppose there are nine horses in the race, to
wit:---A, B, C, D, E, F, G, H, and K. Let the odds
be---

\medskip
\begin{tabular}{rcl|rcl}
3 to 1 & against & A \qquad & \qquad 11 to 1 & against & F \\
5 to 1 & '' & B \qquad & \qquad 11 to 1 & '' & G \\
7 to 1 & '' & C \qquad & \qquad 19 to 1 & '' & H \\
9 to 1 & '' & D \qquad & \qquad 23 to 1 & '' & K \\
9 to 1 & '' & E \qquad & & &
\end{tabular}
\medskip

(It should be noted that when these odds are reduced
to chances, becoming respectively
\[
\frac{1}{4}, \frac{1}{6}, \frac{1}{8}, \frac{1}{10}, \frac{1}{10},
\frac{1}{12}, \frac{1}{12}, \frac{1}{20}, \frac{1}{24},
\]
their sum should be unity or very near it. It does not
matter at all---except to backers---if the sum is greater
than unity, as it generally is; but if it should be less
than unity, the exact application of the system would
involve loss to the bookmaker and gain to backers,
which is not the bookmaker's object.)

Suppose now the wagers on each horse amount to
1,000\textit{l}. (or for convenience, and to avoid fractions, say
1,200\textit{l}.), if the race is important, and bets much in
request; though the system, in its beautiful adaptability,
may be applied to shillings quite as well as to
pounds.
Apart from the extra points which the bookmaker
allows himself, he may lay, in all, about---

\begin{center}
\begin{tabular}{rcrcr}
\pounds 900   &to &\pounds 300  &against &A\\
\pounds 1,000 &to &\pounds 200  &''  &B\\
\pounds 1,050 &to &\pounds 150  &''  &C\\
\pounds 1,080 &to &\pounds 120  &''  &D\\
\pounds 1,080 &to &\pounds 120  &''  &E\\
\pounds 1,100 &to &\pounds 100  &''  &F\\
\pounds 1,100 &to &\pounds 100  &''  &G\\
\pounds 1,140 &to &\pounds 60   &''  &H\\
\pounds 1,150 &to &\pounds 50   &''  &K
\end{tabular}
\end{center}

But he reasons (with intending backers) that `the race
is a moral certainty for A, and that it is giving away
money' to lay more than (in all) 800\textit{l}. to 300\textit{l}. Again, `B
is a much better horse than people think, so that 900\textit{l}.
to 200\textit{l}. is quite long enough odds against him;' as for C,
`no wonder backers stand by him at the odds;' for his
part the bookmaker `thinks him better than B; and
see what Augur says of him!' and so forth, wherefore
he cannot find it in his conscience to lay more than
950\textit{l}. to 150\textit{l}. (in all) against him. (It gets easier as
the non-favourites are reached to get the odds shortened.)
So he deals with each, cutting off about 100\textit{l}.
(let us say) from the amount he ought to lay against
them severally; but with the horses low in favour, he
can easily cut off more, and the system not only does
not forbid this but encourages it. Say, however, only
100\textit{l}., and then his book is complete.

The bookmaker can now watch the race with
thorough enjoyment. The pleasure of the backers of
the favourites is a good deal impaired by anxiety, and
though backers of non-favourites have less to lose, they
have more to gain, and less chance of gaining it: so
they too are anxious. But the bookmaker can watch
the race with perfect calmness.

For, let the race go as it may, he must clear 100\textit{l}.
If A win, the bookmaker willingly pays A's backers
800\textit{l}., receiving 200\textit{l}. from the backers of B, 150\textit{l}. from
those of C, and so on---in all, 900\textit{l}. If B win, the
bookmaker pays B's backers 900\textit{l}., and receives from the
backers of A, C, D, \&c., 1,000\textit{l}.; and so on, whichever
horse may win. There is not, as a rule, any fear about
being paid; these are debts of honour, and to be paid
before all sordid trade debts---nay, so sacred are these
debts, that many of the bookmaker's clients would deem
it better to break open a till, or to embezzle a round
sum from an employer, than to leave them unpaid. So
he is under no anxiety.

Thus does the bookmaker make a steady income out
of his victims, who go not only complacently to their
fate, but even with a look of wisdom as if they were
rather cleverly taking advantage of the proffered gifts
of fortune.

It is easier to tell how they lose than to show how
the bookmaker gains. They adopt \textit{the other and simpler
part} of the bookmaker's system. \emph{He} always lays the
odds a little short: they always take them so. They
back the favourite boldly, but they do not fail to take
fancies for non-favourites, and to back their fancies
boldly too. It would be absurd to haggle about odds
in the case of a horse which is morally sure to win, or
to insist on ten to one when sure the odds are not seven
to one against a horse. When the simpleton wins he
assures himself he is `in the vein,' and goes on betting;
if he loses, he assures himself `the luck must change,'
and goes on betting. By continuing patiently on this
course, it will be odd if he do not learn before long---how
it is that the bookmakers make so much money.

Of course I have given here but a mild account of
the way in which men who bet on horses make money.
They have been known to go a great deal farther. Some
will willingly take the odds against a horse after they
knew certainly that the horse would not run. Others,
a shade more advanced, have been known to bribe a
jockey to `hold' or `rope' a horse, or a stableman to
poison or even stupefy him. Others, ay, even `noble'
owners, have been known to work the market in ways
fully as flagitious.

Let me, in conclusion, quote two short passages,
one from a letter by Charles Dickens, the other
from a speech by Lord Chief Justice Cockburn.
The first seems to relate to the successful bookmaker:---`I
look at the back of his bad head repeated
in long lines on the racecourse, and in the betting-stand,
and outside the betting-rooms, and I vow to
God I can see nothing in it but cruelty, covetousness,
calculation, insensibility, and low wickedness.~.~.~. If
a boy with any good in him, but with a dawning propensity
to sporting and betting, were but brought here
soon enough, it would cure him.' The other passage
applies to the bookmaker and his victim alike:---`The
pernicious and fatal habit' of betting `is so demoralising
and degrading, that, like some foul leprosy, it will
eat away the conscience until a man comes to think
that it is his duty to himself to ``do his neighbour as
his neighbour would do'' him.'

\chapter{Lotteries}

Long experience has shown that men possessed with
the gambling spirit (ninety out of a hundred if the
truth were known) are not to be deterred from venturing
small sums in order to win large fortunes, even
by the clearest evidence that the price they have to
pay is an unfair one. The Government lotteries in
this country early put this matter to the test. Having
decided on a certain set of money prizes and a certain
number of tickets, the Government did not offer the
tickets to the public for more than they were worth,
but for what they would fetch. They seldom failed
to obtain from contractors at least 16\textit{l}. for a ticket
mathematically worth 10\textit{l}. And the contractors not
only showed by offering these sums their faith in
human credulity, but practically proved the truth of
their faith by disposing of their tickets for 5\textit{l}. or 6\textit{l}.
more than they had paid Government for them.
Thus the Government occupied a very favourable position.
For every million they offered in prizes they
received more than 1,600,000\textit{l}.; yet they asked no
one to pay an unfair price. They left the contractors
to do that, who were not only willing, but anxious to
undertake the task of shearing the public. Nor were
the public less ready to be plundered than the contractors
were to plunder them. Government had to
protect the public, or rather tried to protect them,
from the contractors, not by putting a limit to the
price which contractors might obtain for tickets, but
by endeavouring to prevent men of small means from
buying tickets in shares of less than a certain value.
Of course, the laws made for this purpose were readily
and systematically broken. The smallest sums were
risked, and the only effect of the laws against such
purchases was that higher prices had to be paid to
cover the risk of detection. We learn that `all the
efforts of the police were ineffectual for the suppression
of these illegal proceedings, and for many years a
great and growing repugnance was manifested in
Parliament to this method of raising any part of the
public revenue. At length, in 1823, the last Act that
was sanctioned by Parliament for the sale of lottery-tickets
contained provisions for putting clown all private
lotteries, and for rendering illegal the sale in this
kingdom of all tickets or shares of tickets in any
foreign lottery---which latter provision is to this day
extensively evaded.' This was written forty years ago,
but might have been written to-day.

The simplest, and in many respects the best, form
of lottery is that in which a number of articles are
taken as prizes, their retail prices added together, and
the total divided into some large number of parts, the
same number of tickets being issued at the price thus
indicated. Suppose, for instance, the prizes amount in
value to 200\textit{l}., then a thousand tickets might be sold at
4\textit{s}.~each, or 4,000 at 1\textit{s}.~each, or a larger number at a
correspondingly reduced price. In such a case the
lottery is strictly fair, supposing the prizes in good
saleable condition. The person who arranges the lottery
gains neither more nor less than he would if he sold
the articles separately. There may be a slight expense
in arranging the lottery, but this is fully compensated
by the quickness of the sale. The arrangement, I say,
is fair; but I do not say it is desirable, or even that it
should be permissible. Advantage is taken of the love
of gambling, innate in most men, to make a quick sale
of goods which otherwise might have lain long on hand.
Encouragement is given to a tendency which is inherently
objectionable if not absolutely vicious. And
so far as the convenience is concerned of those who collectively
buy (in fact) the prizes, it manifestly cannot
be so well suited as though those only had bought who
really wanted the articles, each taking the special article
he required. Those who buy tickets want to get
more than their money's worth. Some of them, if not
all, are believers in their own good luck, and expect to
get more than they pay for. They are willing to get,
in this way, something which very likely they do not
want, something therefore which will be worth less to
them in reality than the price for which it is justly
enough valued in the list of prizes.

Unfortunately those who arrange lotteries of this
sort for mere trade purposes (they are not now allowed
in this country, but abroad they are common enough,
and English people are invited to take part in these
foreign swindles) are not careful to estimate the price
of each article justly. They put a fancy price on good
articles, a full price on damaged articles, and throw in
an extra sum for no articles at all. Many of them are
not at all particular, if the sale of tickets is quick, about
throwing in a few hundred more tickets than they had
originally provided for, without in the least considering
it necessary to add correspondingly to the list of prizes.

But this is not all. How much those who arrange such
lotteries really wrong the purchasers of tickets cannot
be known. But we can learn how ready the ticket-buyers
are to be wronged, when we note what they will
allow. It seems absurd enough that they should let the
manager of a lottery act entirely without check or control
as to the number of tickets or the plan according to
which these are drawn. But at least when a day is
appointed for the drawing, and the prizes are publicly
exhibited in the first instance, and as publicly distributed
eventually, the ticket-buyers know that the lottery has
been in some degree \textit{bon\^a fide}. What, however, can we
think of those who will pay for the right of drawing a
ticket from a `wheel of fortune,' without having the
least means of determining what is marked on any of
the tickets, or whether a single ticket is marked for a
prize worth more than the price paid for a chance, or
even worth as much? Yet nothing is more common
where such wheels are allowed, and nothing was more
common when they were allowed here, than for a shopman
to offer for a definite sum, which frequenters of the
shop would readily pay, the chance of drawing a prize-ticket
out of a wheel of fortune, though he merely
assured them, without a particle of proof, that some of
the tickets would give them prizes worth many times
the price they paid. Even when there were such
tickets, again, and someone had secured a prize (though
the chances were that the prize-drawer was connected
with the business), people who had seen this would buy
chances as though the removal of one good prize-ticket
had made no difference whatever in the value of a
chance. They would actually be encouraged to buy
chances by the very circumstance which should have
deterred them. For if a good prize is drawn in such a
case, the chances are that no good prize is left.

Although lotteries of this sort are no longer allowed
by law, yet are they still to some degree countenanced in
connection with charity and the fine arts. Now, setting
aside lotteries connected with the fine arts as singularly
mixed in character---though it must not for a
moment be supposed that I regard a taste for gambling
with a love of the beautiful as forming an agreeable
mixture---I note that in lotteries started for
charitable purposes there is usually no thought of gain
on the part of those who originate the scheme. That
is, they have no wish to gain money for themselves,
though they may be very anxious to gain money for the
special purpose they have in view. This wish may be,
and indeed commonly proves to be, inconsistent with
strict fairness towards the buyers of tickets. But as
these are supposed to be also possessed with the same
desire to advance a charitable purpose which actuates the
promoters of the scheme, it is not thought unfair to sell
them their tickets rather dearly, or to increase the
number of tickets beyond what the true value of the
prizes would in strict justice permit. It is, however, to
be noted that the assumption by which such procedure
is supposed to be justified is far from being always accurate.
It is certain that a large proportion of those who
buy tickets in charitable lotteries take no interest whatever
in the object for which such lotteries are started.
If lotteries were generally allowed, and therefore fairer
lotteries could be formed than the charitable ones---which
are as unfair in reality as the dealings of lady
stall-keepers at fancy bazaars---the sale of tickets at
charitable lotteries would be greatly reduced. It is only
because those who are possessed by the gambling spirit
can join no other lotteries that they join those started
for charitable purposes. The managers of these lotteries
know this very well, though they may not be ready to
admit very publicly that they do. If pressed on the
subject, they speak of spoiling the Egyptians, of the
end justifying the means, and so forth. But, as a
matter of fact, it remains true that these well-intentioned
folk, often most devout and religious persons, do,
in the pursuit of money for charitable purposes, pander
to the selfishness and greed of the true gambler,
encourage the growth of similar evil qualities in
members of their own community, and set an evil
example, moreover, by systematically breaking the law
of the country. It would be harsh, perhaps, to speak
strongly against persons whose intentions are excellent,
and who are in many cases utterly free from selfish aims;
but they cannot be acquitted from a charge of extreme
folly, nor can it be denied that, be their purpose what it
may, their deeds are evil in fact and evil in their consequences.
It might be difficult to determine whether
the good worked by the total sum gained from one of
these charitable lotteries was a fair equivalent for the
mischief wrought in getting it. But this total is not
all gained by choosing an illegal method of getting the
sum required. The actual gain is only some slight
saving of trouble on the part of the promoters of the
charitable scheme, and a further slight gain to the
pockets of the special community in which the charity
is or should be promoted. And it is certain that these
slight gains by no means justify the use of an illegal
and most mischievous way of obtaining money. It
would be difficult to find any justification for the
system, once the immorality of gambling is admitted,
which might not equally well be urged for a scheme by
which the proceeds (say) of one week's run of a common
gaming-table should be devoted to the relief of
the sick poor of some religious community. Nay, if
charitable ends can at all justify immoral means, one
might go further still, and allow money to be obtained
for such purposes by the encouragement of still more
objectionable vices. We might in fact recognise quite
a new meaning in the saying that `Charity covers a
multitude of sins.'


I have said that a lottery in which all the prizes
were goods such as might be sold, retail, at prices
amounting to the total cost of all the tickets sold, would
be strictly fair. I do not know whether a lottery ever
has been undertaken in that way. But certainly it
seems conceivable that such a thing might have happened;
and in that case, despite the objections which,
as we have shown, exist against such an arrangement,
there would have been a perfectly fair lottery.
Adam Smith, in his `Wealth of Nations,' seems to have
omitted the consideration of lotteries of this kind, when
he said that `the world neither ever saw, nor ever will
see, a perfectly fair lottery, or one in which the whole
gain compensated the whole loss; because the undertaker
could gain nothing by it.' Indeed, it has
certainly happened in several cases that there have been
lotteries in which the total price of the tickets fell short
of the total value of the prizes---these being presents
made for a charitable purpose, and the tickets purposely
sold at very low prices. It is well known, too, that in
ancient Rome, where lotteries are said to have been
invented, chances in lotteries were often, if not always,
distributed gratuitously.

But assuredly Adam Smith is justified in his remark
if it be regarded as relating solely to lotteries in which
the prizes have been sums of money, and gain has been
the sole object of the promoters. `In the State lotteries,'
as he justly says, `the tickets are really not worth
the price which is paid by the original subscribers,'
though from his sequent remarks it appears that he had
very imperfect information respecting some of the more
monstrous cases of robbery (no other word meets the
case) by promoters of some of these State swindles.

The first idea in State lotteries seems to have been
to adopt the simple arrangement by which a certain
sum is paid for each of a given number of tickets, the
series of prizes provided being less in total value than
the sum thus obtained.

It was soon found, however, as I have already
pointed out, that people are easily gulled in matters
of chance, so that the State could safely assume a
very disinterested attitude. Having provided prizes
of definite value, and arranged the number of tickets,
it simply offered these for sale to contractors. The
profit to the State consisted in the excess of the sum
which the contractors willingly offered above the just
value (usually 10\textit{l}.) of each ticket. This sum varied
with circumstances, but generally was about 6\textit{l}. or 7\textit{l}.
per ticket beyond the proper price. That is, the contractors
paid about 16\textit{l}. or 17\textit{l}. for tickets really worth
10\textit{l}. They were allowed to divide the tickets into
shares---halves, quarters, eighths, and sixteenths. When
a contractor sold a full ticket he usually got from
21\textit{l}. to 22\textit{l}. for it; but when he sold a ticket in shares
his gain per ticket was considerably greater. The
object in limiting the subdivision to one-sixteenth
was to prevent labouring men from risking their
earnings.

It is hardly necessary to say, however, that the
provision was constantly and easily evaded, or that
the means used for evading the limitation only aggravated
the evil. At illegal offices, commonly known as
`little goes,' any sum, however small, could be risked,
and to cover the chance of detection and punishment
these offices required greater profits than the legal
lottery offices. Precisely as attempts to prevent usury
caused the necessitous borrowers of money to be mulcted
even more severely than they would otherwise have
been, so the attempt to protect the poor from falling
into gambling ways resulted only in driving them to
gamble against more ruinous odds.

The record of national lotteries in England ranges
over two centuries and a half. It forms an interesting,
though little studied, chapter in the history of the
nation, and throws curious light on the follies and
weaknesses of human nature.

The earliest English lottery on record is that of the
year 1569, when 40,000 chances were sold at 10\textit{s}. each,
the prizes being articles of plate, and the profit used in
the repair of certain harbours. The gambling spirit
seems to have developed greatly during the next century;
for, early in the reign of Queen Anne, it was found necessary
to suppress private lotteries `as public nuisances,'
a description far better applicable (in more senses than
one) to public lotteries. `In the early period of the
history of the National Debt,' says a writer (De Morgan,
I believe) in the `Penny Cyclop{\ae}dia,' `it was usual to
pay the prizes in the State lotteries in the form of
terminable annuities. In 1694 a loan of a million was
raised by the sale of lottery-tickets at 10\textit{l}. per ticket,
the prizes in which were funded at the rate of 14 per
cent.\ for sixteen years certain. In 1746 a loan of three
millions was raised on 4 per cent.\ annuities, and a
lottery of 50,000 tickets of 10\textit{l}.\ each; and in the following
year one million was raised by the sale of 100,000
tickets, the prizes in which were funded in perpetual
annuities at the rate of 4 per cent.\ per annum. Probably
the last occasion on which the taste for gambling was
thus made use of occurred in 1780, when every subscriber
of 1,000\textit{l}.\ towards a loan of twelve millions, at 4
per cent., received a bonus of four lottery-tickets, the
intrinsic value of each of which was 10\textit{l}.' About this time
the spirit of gambling had been still more remarkably
developed than in Anne's reign, despite the laws passed
to suppress private lotteries. In 1778 an Act was
passed by which every person keeping a lottery-office
was obliged to take out a yearly license costing 50\textit{l}.
This measure reduced the number of such offices from
400 to 51. In France the demoralisation of the people
resulting from the immorality of the Government in
encouraging by lotteries the gambling spirit, was greater
even than in England.

The fairest system for such lotteries as we have
hitherto considered was that adopted in the Hamburg
lotteries. Here, the whole money for which tickets were
sold was distributed in the form of prizes, except a deduction
of 10 per cent.\ made from the amount of each
prize at the time of payment.

Before pausing to consider the grossly unfair systems
which have been, and still are, adopted in certain foreign
lotteries, it may be well to notice that the immorality of
lotteries was not recognised a century ago so clearly as
it is now; and therefore, in effect, those who arranged
them were not so blameworthy as men are who, in our
own time, arrange lotteries, whether openly or surreptitiously.
Even so late as half a century ago an
American lawyer, of high character, was not ashamed
openly to defend lotteries in these terms. `I am no
friend,' he said, `to lotteries, but I cannot admit that
they are \textit{per se} criminal or immoral when authorised by
law. If they were nuisances, it was in the manner
in which they were managed. In England, if not in
France' (how strange this sounds), `there were lotteries
annually instituted by Government, and it was considered
a fair way to reach the pockets of misers and
persons disposed to dissipate their funds. The American
Congress of 1776 instituted a national lottery, and
perhaps no body of men ever surpassed them in intelligence and virtue.' De~Morgan, remarking on this
expression of opinion, says that it shows what a man of
high character for integrity and knowledge thought of
lotteries twenty years ago (he wrote in 1839). `The
opinions which he expressed were at that time,' continued
De~Morgan, `shared, we venture to say, by a
great number.'

The experience of those who arranged these earlier
State lotteries showed that from men in general, especially
the ignorant (forming the great bulk of the
population who place such reliance on their luck),
almost any price may be asked for the chance of making
a large fortune at one lucky stroke. Albeit, it was
seen that the nature of the fraud practised should
preferably be such that not one man in a thousand
would be able to point out where the wrong really lay.
Again, it was perceived that if the prizes in a lottery
were reduced too greatly in number but increased in
size, the smallness of the chance of winning one of the
few prizes left would become too obvious. A system
was required by which the number of prizes might seem
unlimited and their possible value very great, while also
there should be a possibility of the founders of the
lottery not getting back all they ventured. So long as
it was absolutely certain that, let the event be what it
might, the managers of the lottery would gain, some
might be deterred from risking their money by the
simple statement of this fact. Moreover, under such
conditions, it was always possible that at some time the
wrath of losers (who would form a large part of the
community if lottery operations were successful) might
be roused in a dangerous way, unless it could be shown
that the managers of public lotteries ran some chance,
though it might be only a small chance, of losing,
and even some chance of ruin as absolute as that which
might befall individual gamblers.

It was to meet such difficulties as these that lottery
systems like that sometimes called the Geneva system
were invented. This system I propose now to describe,
as illustrating these more speculative ventures, showing
in particular how the buyers of chances were defrauded
in the favourite methods of venturing.

In the Geneva lottery there are ninety numbers. At
each drawing five are taken. The simplest venture is
made on a single number. A sum is hazarded on a
named number, and if this number is one of the five
drawn, the speculator receives fifteen times the value of
his stake. Such a venture is called a \textit{simple drawing}.
It is easy to see that in the long run the lottery-keeper
must gain by this system. The chance that the number
selected out of ninety will appear among five numbers
drawn, is the same that a selected number out of
eighteen would appear at a single drawing. It is one
chance in eighteen. Now if a person bought a single
ticket out of eighteen, each costing (say) 1\textit{l}., his fair
prize if he drew the winning ticket should be 18\textit{l}. This
is what he would have to pay to buy up all the eighteen
tickets (so making sure of the prize). The position of
the speculator who buys one number at 1\textit{l}. in the Geneva
lottery, is precisely that of a purchaser of such a ticket,
only that, instead of a prize being 18\textit{l}. if he wins, it is
only 15\textit{l}. The lottery-keeper's position on a single
venture is not precisely that of one who should have
sold eighteen tickets at 1\textit{l}. each, for a lottery having one
prize only; for the latter would be certain to gain money
if the prize were any sum short of 18\textit{l}., whereas the
Geneva lottery-keeper will lose on a single venture,
supposing the winning number is drawn, though the
prize is 15\textit{l}. instead of 18\textit{l}. But in the long run the
Geneva lottery-keeper is certain to win at these odds.
He is in the position of a man who continually wagers
odds of 14 to 1 against the occurrence of an event the
real odds against which are 17 to 1. Or his position
may be compared to that of a player who takes seventeen
chances out of eighteen at (say) their just value, 1\textit{l}. each
or 17\textit{l}. in all, his opponent taking the remaining chance
at its value, 1\textit{l}., but instead of the total stakes, 18\textit{l}.,
being left in the pool, the purchaser of the larger
number abstracts 3\textit{l}. from the pool at each venture.

That men can be found to agree to such an arrangement
as this shows that their confidence in their own
good fortune makes them willing to pay, for the chance
of getting fifteen times their stake, what they ought to
pay for the chance of getting eighteen times its value.
The amount of which they are in reality defrauded at
each venture is easily calculated. Suppose the speculator
to venture 1\textit{l}. Now the actual value of one chance
in eighteen of any prize is one-eighteenth of that prize,
which in this case should therefore be 18\textit{l}. If, then,
the prize really played for has but fifteen-eighteenths
of its true value, or is in this case 15\textit{l}., the value of a
single chance amounts only to one-eighteenth of 15\textit{l}., or
to 16\textit{s}. 8\textit{d}. Thus at each venture of 1\textit{l}. the speculator
is cheated out of 3\textit{s}. 4\textit{d}., or one-sixth of his stake.

This, however, is a mere trifle. In the old-fashioned
English system of lotteries, the purchaser of a 10\textit{l}. ticket
often paid more than 20\textit{l}., so that he was defrauded by
more than half his stake; and though less than half the
robbery went into the hands of the contractor who
actually sold the ticket, the rest of the robbery went to
the State.

In other ventures, by the Geneva system, the
old-fashioned English system of robbery was far surpassed.

Instead of naming one number for a drawing (in
which five numbers are taken) the speculator may say
in what position among the five his number is to come.
If he is successful, he receives seventy times his stake.
This is, in effect, exactly the same as though but one
number was drawn. The speculator has only one
chance out of ninety instead of one chance out of five.
He ought then, in strict justice, to receive ninety times
his stake, if he wins. Supposing his venture 1\textit{l}., the
prize for success should be 90\textit{l}. By reducing it to 70\textit{l}.\ the
lottery-keeper reduces the real value of the ticket
from 1\textit{l}. to one-nineteenth part of 70\textit{l}., or
to 15\textit{s}.\ $6\frac{2}{3}$\textit{d}.,
defrauding the speculator of two-ninths of his stake.
Such a venture as this is called a \textit{determinate drawing}.

The next venture allowed in the Geneva system is
called \textit{simple ambe}. Two numbers are chosen. If both
these appear among the five drawn, the prize is 270
times the stake. Now among the 90 numbers the
player can select two, in 8,010 different ways; for he
can first take any one of the 90 numbers, and then he can
take for his second number any one of the 89 numbers
left; that is, he may make 90 different first selections,
each leaving him a choice of 89 different second selections;
so that there are 90 times 89 (or 8,010) possible
selections in all. But in any set of five numbers there
are, treating them in the same way, only 20 (or 5 times
4) different arrangements of two numbers. So that
out of 8,010 possible selections only 20 appear in each
drawing of five numbers. The speculator's chance then
is only 20 in 8,010 or 2 in 801; and he ought, if he
wins, to have for prize his stake increased in the ratio
of 801 to 2, or $400\frac{1}{2}$ times. Instead of this it is increased
only 270 times. At each venture he receives
in return for his stake a chance worth less than his
stake, in the same degree that 270 is less than $400\frac{1}{2}$;
he is, in fact, defrauded of nearly one-third the value of
his stake.

The next venture is called \textit{determinate ambe}. Here
the speculator names the order in which two selected
numbers will appear. Instead of 20 chances at any
drawing of five numbers, he has only one chance---one
chance in 8,010. He ought then to receive 8,010 times
his stake, if he wins. As a matter of fact he receives
only 5,100 times his stake. From this it follows that
he is defrauded of 2,910 out of 8,010 parts of his stake,
or very nearly three-eighths of the stake's value.

But more speculative ventures remain. The speculator
can name three numbers. Now there are 704,880
possible selections of three numbers out of 90. (There
are 8,010 possible selections of two numbers, as already
shown, and with each of these any one of the remaining
88 numbers can be taken to make the third number;
thus we have 88 times 8,010, or 704,880 sets of three
numbers in all.) These can appear among the five
drawn numbers in 60 different ways (5 times 4 times
3). Thus the speculator has 60 chances out of 704,880,
or one chance in 11,748. He ought then to receive
11,748 times his stake, if he wins; but in reality he
receives only 5,500 times his stake in this event. Thus
the lottery-keeper robs him of more than half of his just
winnings, if successful, and of more than half the mathematical
value of his stake at the outset. The venture
in this case is called \textit{simple terne}. \textit{Determinate terne}
is not allowed. If it were, the prize of a successful
guess should be 704,880 times the stake.

\textit{Quaterne} involves the selection of four numbers.
With 90 numbers, 61,334,560 (704,880 times 87)
different selections of four numbers can be made.
Among the five drawn numbers there can only be found
120 arrangements of four numbers. Thus the speculator
has only 120 chances out of 61,334,560, or one
chance out of 511,038. He ought therefore, if he wins,
to receive 511,038 times his stake. The prize is only
75,000 times the stake. The lottery-keeper deducts,
in fact, six-sevenths of the value of the stake at each
venture. \textit{Determinate quaterne} is, of course, not admitted.

Simple \textit{quaterne} is, at present, the most speculative
venture adopted. Formerly \textit{quine} was allowed, the
speculator having five numbers, and, if all five were
drawn, receiving a million times the value of his stake.
He should have received 43,949,268 times its value;
so that, in effect, he was deprived of more than 42 forty-thirds
of the true value of his venture.

The following table shows the amount by which the
terms of the Geneva system reduce the value of the
stake in these different cases, the stake being set at 1\textit{l}.\ for
convenience:

\begin{center}
\begin{tabular}{|l|r@{\ }l|r@{\ }l|}
    \hline

    & \multicolumn{2}{p{1in}}{\centering{Actual Worth\\ of 1\textit{l}.\ Stake}}
    & \multicolumn{2}{|p{1in}|}{\centering{Robbery per\\   1\textit{l}.\ Stake}}      \\ \hline

    & \textit{s} & \textit{d} & \textit{s} & \textit{d} \\

    Simple drawing      & 16 & 8               & 3  & 4              \\
    Determinate drawing & 15 & $6 \frac{3}{4}$ & 4  & $5\frac{1}{4}$ \\
    Simple ambe         & 13 & 6               & 6  & 6              \\
    Determinate ambe    & 12 & 9               & 7  & 3              \\
    Terne               & 9  & $4 \frac{1}{2}$ & 10 & $7\frac{1}{2}$ \\
    Quaterne            & 2  & $11\frac{1}{4}$ & 17 & $0\frac{3}{4}$ \\

    \hline
\end{tabular}
\end{center}

It may be thought, perhaps, that such speculative
ventures as terne and quaterne would very seldom be
made. But the reverse was the case. These were the
favourite ventures; and that they were made very often
is proved to everyone acquainted with the laws of
chance by the circumstance that they not unfrequently
proved successful. For every time such a venture as a
simple quaterne was won, it must have been lost some
half a million times.

It appears that in France the Geneva system was
adopted without any of the limitations we have mentioned,
and with some additional chances for those who
like fanciful ventures. Professor De Morgan, in his
`Budget of Paradoxes' says:---`In the French lottery five
numbers out of ninety were drawn at a time: any
person, in any part of the country, might stake any sum
upon any event he pleased, as that 27 should be drawn;
that 42 and 81 should be drawn; that 42 and 81
should be drawn, and 42 first; and so on up to a \textit{quine
determin\'e}, if he chose, which is betting on five given
numbers in a given order.' The chance of a successful
guess, in this last case, is 1 in 5,274,772,160. Yet if
every grown person in Europe made one guess a day,
venturing a penny on the guess, and receiving the just
prize, or say 4,800,000,000 times his stake, on winning,
it would be practically certain that in less than a year
some one would win 20,000,000\textit{l}.\ for a penny! It
would be equally certain that though this were repeated
dozens of times, the lottery-keepers would gain by the
arrangement, even at the rate above stated. Nay, the
oftener they had to pay 20,000,000\textit{l}.\ for a penny the
greater their gains would be. As the actual prize
in such a case would be 10 million instead of merely
5,275 million times the stake, their real gains, if they
had to pay such prizes often, would be enormous. For,
in the long run, every prize of half a million pounds for
a shilling stake would represent a clear profit of 250
million pounds. The successful ventures would be only
1 in about 5,000 millions of unsuccessful ones, while
paid for only at the rate of 10 million stakes.

No instances are on record of a \textit{quine determin\'e} being
won, but a simple \textit{quine}, the odds against which, be it
remembered, are nearly 44 millions to 1, has been won;
and simple \textit{quaternes}, against which the odds are more
than half a million to 1, have often been won. In
July 1821 a strange circumstance occurred. A gambler
had selected the five numbers 8, 13, 16, 46, and 64,
and for the same drawing another had selected the four
numbers 8, 16, 46, and 64. The numbers actually
drawn were
\[
8 \qquad 46 \qquad 16 \qquad 64 \qquad 13
\]
so that both gamblers won. Their stakes were small,
unfortunately for them and fortunately for the bank,
and their actual winnings were only 131,350 francs and
20,852 francs respectively. If each had ventured 1\textit{l}.\ only,
their respective winnings would have been
l,000,000\textit{l}., and 75,000\textit{l}. The coincidence was so remarkable
(the antecedent probability against two gamblers
winning on a simple drawing or simple quine and a
simple quaterne being about 22 billions to 1), that one
can understand a suspicion arising that a hint had been
given from some one employed at the lottery-office.
M.~Menut insinuates this, and a recent occurrence at
Naples suggests at least the possibility of collusion between
gamblers and the drawers of lottery numbers.
But in the case above cited the smallness of the stakes
warrants the belief that the result was purely accidental.
Certainly the gamblers would have staked more had
they known what was to be the actual result of the
drawing. The larger winner seems to have staked two
sous only, the prize being, I suppose, 1,313,500 times
the stake, instead of 1,000,000 as on a similar venture
in the Geneva lottery. Possibly the stake was a foreign
coin, and hence the actual value of the prize was not a
round number of francs. The smaller winner probably
staked five sous or thereabouts in foreign coin.

Simple \textit{quaternes}, as we have said, occurred frequently
in France. De Morgan remarks that the
enormous number of those who gambled `is proved
to all who have studied chances arithmetically by the
numbers of simple \textit{quaternes} which were gained: in
1822, fourteen; in 1823, six; in 1824, sixteen; in
1825, nine, \&c.' He does not, however, state the
arithmetical proportion involved. If we take the
average number at ten per annum, it would follow that
about five million persons per annum staked money
on this special venture---the simple \textit{quaterne}---alone.
Quetelet states that in the five years 1816--1820, the
total sums hazarded on all forms of venture in the
Paris lottery amounted to 126,944,000 francs---say
5,000,000\textit{l}. The total winnings of the speculators
amounted to 94,750,000 francs---say about 3,790,000\textit{l}.
The total amount returned to the treasury was 32,194,000
francs, or about 1,288,000\textit{l}., a clear average profit of
257,600\textit{l}.\ per annum. Thus the treasury received
rather more than a fourth of the sum hazarded. The
return to the speculators corresponded nearly to that
which would have been received if all the ventures made
had been on a determinate single number.

In all these methods, the greater the number of
speculators the greater the gains of those who keep the
lottery. The most fortunate thing which can happen
to the lottery-keepers is that some remarkably lucky hit
should be made by a speculator, or a series of such hits.
For then they can advertise the great gains made by a
few lucky speculators, saying nothing of the multitudes
who have lost, with the result that millions are tempted
to become speculators. There is this great advantage
in the Geneva system: that the total number of losers
can never be known except to the lottery-keepers. In
the old-fashioned English system the number of losers
was as well known as the number of winners and their
respective gains. But the keepers of the Paris and
Geneva lotteries, as of those which have since been
established on the same system, could publish the lists
of winners without any fear that newspaper writers or
essayists would remind the general public of the actual
number of losers. The student of probabilities might
readily calculate the probable number of losers, and
would be absolutely certain that the real number could
not differ greatly from that calculated; but he could
not definitely assert that so many had lost, or that the
total losses amounted to so much.

It occurred to the Russian Government, which has
at all times been notably ready to take advantage of
scientific discoveries, that a method might be devised
for despoiling the public more effectually than by the
Geneva method. A plan had been invented by those
who wanted the public money, and mathematicians were
simply asked to indicate the just price for tickets, so
that the Government, by asking twice that price, or
more, might make money safely and quickly. The
plan turned out to be wholly impracticable; but the
idea and the result of its investigation are so full of
interest and instruction that I shall venture to give a
full account of them here, noting that the reader who
can catch the true bearing of the problem involved may
consider himself quite safe from any chance of being
taken in by the commoner fallacies belonging to the
subject of probabilities.

The idea was this:---Instead of the drawing of
numbers, the tossing of a coin was to decide the prize
to be paid, and there were to be no blanks. If `head'
was tossed at a first trial the speculator was to receive
a definite sum---2\textit{l}.\ we take for convenience, and also
because this seems to have been nearly the sum originally
suggested in Russian money. If `head' did not
appear till the second trial the speculator was to receive
4\textit{l}.; if `head' did not appear till the third trial, he received
8\textit{l}.; if not till the fourth, he received 16\textit{l}.; if
not till the fifth, 32\textit{l}.; till the sixth, 64\textit{l}.; the seventh
128\textit{l}.; the eighth, 256\textit{l}., and so on; the prize being
doubled for each additional tossing before `head'
appeared. It will be observed that the number of
pounds in the prize is 2 raised to the power corresponding
to the number of that tossing at which `head' first
appears. If it appears first, for instance, at the
tenth trial, then we raise 2 to the 10th power, getting
1,024, and the prize is 1,024\textit{l}.; if `head' appears first
at the twelfth trial, we raise 2 to the 12th power, getting
4,096, and the prize is 4,096\textit{l}.
% original 4048 corrected to 4096

Doubtless the origin of this idea was the observed
circumstance that the more speculative ventures had
a great charm for the common mind. Despite the
enormous deduction made from the just value of the
prize, when \textit{ternes}, \textit{quaternes}, and other such ventures
were made, the public in France, Switzerland, and Italy
bought these ventures by millions, as was shown by the
fact that several times in each year even \textit{quaternes} were
won. Now in the Petersburg plan there was a chance,
however small, of enormous winnings. Head might
not appear till the tenth, twelfth, or even the twentieth
tossing; and then the prize would be 1,024\textit{l}., 4,096\textit{l}.,
% original 4048 corrected to 4096
or 1,048,576\textit{l}., respectively. It was felt that tens of
millions would be tempted by the chance of such enormous
gains; and it was thought that the gains of
Government would be proportionately heavy. All that
was necessary was that the just value of a chance
in this lottery should be ascertained by mathematicians,
and the price properly raised.

Mathematicians very readily solved the problem,
though one or two of the most distinguished (D'Alembert,
for instance) rejected the solution as incomprehensible
and paradoxical. Let the reader who takes
interest enough in such matters pause for a moment
here to inquire what would be a natural and probable
value for a chance in the suggested lottery. Few, we
believe, would give 10\textit{l}.\ for a chance. No one, we are
sure---not even one who thoroughly recognised the
validity of the mathematical solution of the problem---would
offer 100\textit{l}. Yet the just value of a chance is
greater than 10\textit{l}., greater than 100\textit{l}., greater than any
sum which can be named. A Government, indeed,
which would offer to sell these chances at say 50\textit{l}.\ would
most probably gain, even if many accepted the
risk and bought chances---which would be very unlikely,
however. The fewer bought chances the greater would
be the Government's chance of gain, or rather their
chance of escaping loss. But this, of course, is precisely
the contrary to what is required in a lottery system.
What is wanted is that many should be encouraged to
buy chances, and that the more chances are bought the
greater should be the security of those keeping the
lottery. In the Petersburg plan, a high and practically
prohibitory price must first be set on each chance, and
even then the lottery-keepers could only escape loss by
restricting the number of purchases. The scheme was
therefore abandoned.

The result of the mathematical inquiry seems on
the face of it absurd. It seems altogether monstrous,
as De Morgan admits, to say that an infinite amount
of money should in reality be given for each chance, to
cover its true mathematical value. And to all intents
and purposes any very great value would far exceed
the probable average value of any possible number of
ventures. If a million million ventures were made,
first and last, 50\textit{l}.\ per venture would probably bring in
several millions of millions of pounds clear profit to the
lottery-keepers; while 30\textit{l}.\ per venture would as probably
involve them in correspondingly heavy losses: 40\textit{l}.\ per
venture would probably bring them safe, though without
any great percentage of profit. If a thousand
million ventures were made, 30\textit{l}.\ per venture would
probably make the lottery safe, while 35\textit{l}.\ would bring
great gain in all probability, and 25\textit{l}.\ would as probably
involve serious loss. If all the human beings who have
ever lived on this earth, during every day in their lives
had been taking chances in such a lottery, the average
price of all the sums gained would be quite unlikely to
approach 100\textit{l}. Yet still the mathematical proposition
is sound, that if the number of speculators in the
Petersburg lottery were absolutely unlimited, no sum,
however great, would fairly represent the price of a
chance. And while that unpractical result (for the
number of speculators would not be unlimited) is true,
the practical result is easily proved, that the larger the
number of venturers the greater should be the price for
each chance---a relation which absolutely forbids the
employment of this method of keeping lotteries.

Let us see how this can be shown. De Morgan has
given a demonstration, but it is not one to be very
readily understood by those not versed in mathematical
methods of reasoning. I believe, however, that the
following proof will be found easy to understand, while
at the same time satisfactory and convincing.

Suppose that eight ventures only are made, and
that among the eight, four, or exactly half, toss head
the first time; of the remaining four, two half-toss
head at the second trial; of the remaining two, one
tosses head at the third trial; while the other tosses
head at the fourth trial. This may be regarded as
representing what might on the average be expected
from eight trials, though in reality it does not; for of
course, if it did, the average price per chance, inferred
from eight such trials, would be the true average for
eight million trials, or for eight million times eight
million. Still it fairly represents all that could be
hoped for from a single set of eight ventures. Now
we see that the sums paid in prizes, in this case,
would be four times 2\textit{l}.\ for those who tossed `head' at
the first trial; twice 4\textit{l}.\ for those who tossed `head' at
the second trial; 8\textit{l}.\ for him who tossed `head' at the
third trial; and 16\textit{l}.\ for the last and most fortunate of
the eight; or 40\textit{l}.\ in all. This gives an average of 5\textit{l}.\ for
each chance.

Now suppose there are sixteen ventures, and treat
this number in the same way. We get eight who
receive 2\textit{l}.\ each; four who receive 4\textit{l}.\ each; two who
receive 8\textit{l}.\ each; one who receives 16\textit{l}.; and one who
receives 32\textit{l}. The total, then, is 96\textit{l}., giving an average
of 6\textit{l}.\ for each chance.

Next take thirty-two ventures. Sixteen receive 2\textit{l}.\ each;
eight 4\textit{l}.\ each; four 8\textit{l}.\ each; two 16\textit{l}.\ each;
one 32\textit{l}.; and one 64\textit{l}.; a total of 224\textit{l}., giving an
average of 7\textit{l}.\ for each venture.

It will be noticed that the average price per venture
has risen 1\textit{l}.\ at each doubling of the total number of
speculators. Nor is it difficult to perceive that this
increase will proceed systematically. To show this we
take a larger number, 1,024, which is 2 doubled ten
times, or technically 2 raised to the 10th power. Treating
this like our other numbers, we find that 512 speculators
are to receive 2\textit{l}.\ each, making 1,024\textit{l}. in all;
thus we get as many pounds as there are ventures for
this first halving. Next 256 receive 4\textit{l}.\ each, or 1,024
in all; that is, again we get as many pounds as there
are ventures, for this second halving. Next, 128
receive 8\textit{l}., or 1,024\textit{l}.\ in all; or again, we get as many
pounds as there are ventures, for this third halving.
This goes on ten times, the tenth halving giving us one
speculator who receives 1,024\textit{l}., and still leaving one
who has not yet tossed `head.' Since each halving
gives us 1,024\textit{l}., we now have ten times 1,024\textit{l}. The
last speculator tosses `head' at the next trial and wins
2,048\textit{l}.; making a grand total of twelve times 1,024\textit{l}.,
or twelve times as many pounds as there are speculators.
The average, therefore, amounts to 12\textit{l}.\ per chance; and
we see, by the way in which the result has been obtained,
that in every such case the chance will be worth
2\textit{l}.\ more than as many pounds as there are halvings.
Of course the number of halvings is the number representing
the power to which two is raised to give the
number of speculators. The number of speculators
need not necessarily be a power of 2. We have only
supposed it so for simplicity of calculation. But the application
of the method of halving can be almost as readily
made with any number of speculators. It is only when
we get down to small numbers, as 9, 7, 5, or 3, that any
difficulty arises from fractional or half men; but the
result is not materially affected where the original
number is large, by taking 4 or 3 as the next halving
after either 7 or 9 (for example), or 2 as the next
halving after 3. But practically we need not carry out
these halvings, after we have once satisfied ourselves of
the validity of the general rule. Thus, suppose we
require to ascertain a fair value for a million chances.
We find that the nearest power of 2 to the number one
million is the 20th: 22\textit{l}., then, is a fair value.

But of course, the whole train of our reasoning
proves that while probably 22\textit{l}.\ would be a fair value for
a million ventures, it could not be the mathematically
just value. For who is to assure the lottery-keeper
that after the million ventures, another million will not
be taken? Now for two million ventures the probable
value according to our method would be 23\textit{l}., since two
millions is nearly equal to 2 raised to the 21st power.
There might be a million million ventures; and if 22\textit{l}.\ were
really the true price for one million, it would be
the true price for each of the million ventures. But
since a million million are roughly equal to 2 raised to
the 40th power, the price according to our method would
be about 42\textit{l}.\ per chance.

All that can be said is that among any definite number
of trials it is not antecedently probable that there
will be any of those very long runs of `trials' which are
practically certain to occur when, many times that
number of trials (whatever it may be) are made.

The experiment has been actually tried, though it
was not necessary to establish the principle. So far as
the relatively small average value of the chance, when
a few ventures only are made, the reader can readily
try the experiment for himself. Let him make, for
instance, eight trials, each trial ending when he has
tossed head; and according as head comes at the first,
second, or third, \&c.\ tossing in any trial, let him write
down 2\textit{l}., 4\textit{l}., 8\textit{l}., \&c., respectively. The total divided by
eight will give the average value of each trial. Buffon
and each of three correspondents of De Morgan's made
2,048 trials---an experiment which even the most enthusiastic
student of chances will not greatly care to
repeat. Buffon's results, the only set we shall separately
quote, were as follows. In 1,061 trials, `head'
showed at the first tossing; in 494, at the second; in
232, at the third; in 137, at the fourth; in 56, at the
fifth; in 29, at the sixth; in 25, at the seventh; in 8,
at the eighth; in 6, at the ninth. The 2,048 trials,
estimated according to the Petersburg system, would
have given 20,114\textit{l}.\ in all, or nearly 10\textit{l}.\ per game.
According to our method, since 2,048 is the eleventh
power of 2\textit{l}., the average value of each chance would be
13\textit{l}.;\footnote{I note that De Morgan obtains
the value 11\textit{l}.\ instead of 13\textit{l}.
But he strangely omits one of the last pair of trials altogether.
Thus, he says, `in the long run, and on 2,048 trials, we might expect
two sets in which ``heads'' should not appear till the tenth
throw,' which is right, `and one in which no such thing should take
place till the eleventh,' which is also right. But it is because there
will probably be four trials of which two only will probably give
`heads,' that we may expect two to give `tails' yet once more. The
two which gave `heads' are the two first mentioned by De Morgan,
in which `heads' appear at the tenth throw. Of the two remaining
we expect one to give `head,' the other `tail.' The former is the
`one' next mentioned by De Morgan, in which `head' appears at
the eleventh throw. The other in which `tail' may be expected to
appear is the most valuable of all. Even if `head' appears at the
next or twelfth tossing, this trial brings a prize worth twice as many
pounds as the total number of trials---and therefore adds 2\textit{l}.\ to the
average value of each trial. It is quite true that Buffon's experiment
chances to give a result even less than De Morgan's value,
and still further therefore from mine. But as will be seen, the
other experiment gave an average result above his estimate, and
even above mine. It cannot possibly be correct to omit all consideration
of the most profitable trial of all.}
and Buffon's result is quite as near as could be
expected in a single experiment on 2,048 trials.

But when we take the four experiments collectively,
getting in this way the results of 8,192 trials (of which
De Morgan, strangely enough, does not seem to have
thought), we find the average value of each chance
greatly increased, as theory requires---and, as it happens,
increased even beyond the value which theory assigns
as probable for this number of trials. Among them
there was only one in which head appeared after tail
had been tossed 11 times, whereas we might expect that
there would be four such cases; but there was one case
in which head only appeared after tail had been tossed
13 times, and there were two cases in which head only
appeared after tail had been tossed 15 times. Of course
this was purely accidental. We may always be tolerably
sure that in a large number of tossings, about one-half
will be head and about one-half tail. But when
only a few tossings are to be made, this proportion can
no longer be looked for with the same high degree of
probability. When, again, only four or five chances
are left, we may find these all dropping off at once, on
the one hand, or one or two of them may run on with
five or six more successful tossings; and as at each
tossing the prize, already amounting for the last trial
to as many pounds as there were originally chances, is
doubled, we may find the average price of each chance
increased by 1\textit{l}., 2\textit{l}., 4\textit{l}., 8\textit{l}., 16\textit{l}., or more, by the continued
success of the longest lasting trial, or perhaps of
two or three lasting equally long. This happened in
the 8,192 trials whose results are recorded by De
Morgan. I find that the total amount which would
have been due in prizes, according to the Petersburg
plan, would have been 150,830\textit{l}., an average of
18\textit{l}. 8\textit{s}. $2\frac{1}{2}$d\textit{d}. (almost
exactly) per trial; whereas the average for
8,192 trials on my plan would be only 15\textit{l}.
It is manifest that, though in a million trials by
this method some such sum as 30\textit{l}.\ per trial would probably
cover all the prizes gained, it would be unsafe to
put any definite price on each venture, where the
number of venturers would of necessity be unlimited.
And since even a price which would barely cover the
probable expenses would be far more than speculators
would care to give, the plan is utterly unsuited for a
public lottery. It may be well to note how large a
proportion of the speculators would lose by their venture,
even in a case where the total ventured was just
covered by the prizes. Suppose there were rather
more than a million speculators (more exactly, that the
numbers were the 20th power of 2, or 1,048,576), and
that the average result followed, the price per venture
being 22\textit{l}. Then 524,288 persons would receive only
2\textit{l}.\ and lose 20\textit{l}.\ each; 262,144 would receive only 4\textit{l}.\ and
lose 18\textit{l}.\ each; 131,072 would receive 8\textit{l}.\ and lose
14\textit{l}.\ each; 65,536 would receive 16\textit{l}.\ and lose 6\textit{l}.\ each.
All the rest would gain; 32,768 would receive 32\textit{l}.\ and
gain 10\textit{l}.\ each; 16,384 would receive 64\textit{l}.\ and gain 42\textit{l}.\ each;
and so on; 8,192 would receive 128\textit{l}.\ each;
4,096 would receive 256\textit{l}.\ each; 2,048 each 512\textit{l}.;
1,024 each 1,024\textit{l}.; 512 each 2,048\textit{l}.; 256 each 4,096\textit{l}.;
128 each 8,192\textit{l}.; 64 each 16,384\textit{l}.; 32 each 32,768\textit{l}.;
% original "138 each" corrected to "128 each".
16 each 65,536\textit{l}.; 8 each 131,072\textit{l}.; 4 each 262,144\textit{l}.;
2 each 524,288\textit{l}.; 1 would receive 1,048,572\textit{l}.; and
lastly, one would receive 2,097,952\textit{l}. But there would
be only 65,536 out of 1,048,576 speculators who would
gain, or only 1 in 16.
It is singular that whereas it would be almost impossible
to persuade even one person to venture 22\textit{l}.\ in
such a lottery as we have described, almost any number
of persons could be persuaded to join again and again
in a lottery where the prizes and blanks were arranged
as in the way described in the preceding paragraph as
the average outcome of 1,048,576 ventures. In other
words, no one puts so much faith in his luck as to venture
a sum on the chance of gaining a little if he tosses
`tail' four times running (losing if `head' appears
sooner), and of gaining more and more the oftener
`tail' is tossed, until, should he toss tail 20 times running,
he will receive more than two million pounds.
But almost every person who is willing to gamble at all
will be ready to venture the same sum on the practically
equivalent chance of winning in a lottery where there
are rather more than a million tickets, and the same
prizes as in the other case. Whatever advantage there
is, speaking mathematically, is in favour of the tossing
risk; for the purchaser of a trial has not only the chance
of winning such prizes as in a common lottery arranged
to give prizes corresponding to the above-described
average case, but he has a chance, though a small one,
of winning four, eight, sixteen, or more millions of
pounds for his venture of 22\textit{l}. We see then that the
gamblers are very poor judges of chances, rejecting \textit{absolutely}
risks of one kind, while accepting \textit{systematically}
those of another kind, though of equal mathematical
value, or even greater.

In passing, I may note that the possibility of winning
abnormally valuable prizes in the Petersburg lottery
affords another explanation of the apparent paradox
involved in the assertion that no sum, however large,
fairly represents the mathematical value of each trial.
To obtain the just price of a lottery-ticket, we must
multiply each prize by the chance of getting it, and add
the results together; this is the mathematical value of
one chance or ticket. Now in the Petersburg lottery
the possible prizes are 2\textit{l}., 4\textit{l}., 8\textit{l}., 16\textit{l}., and so on,
doubling to infinity; the chances of getting each are,
respectively, one-half, one-fourth, one-eighth, one-sixteenth,
and so on. The value of a chance, then, is the
half of 2\textit{l}., added to the quarter of 4\textit{l}., to the eighth of
8\textit{l}., and so on to infinity, each term of the infinite series
being 1\textit{l}. Hence the mathematical value of a single
chance is infinite. The result appears paradoxical; but
it really means only that the oftener the trial is made,
the greater will be the probable average value of the
prizes obtained. Or, as in fact the solution is that if
the number of trials were infinite the value of each
would be infinite, we only obtain a paradoxical result in
an impossible case. Note also that the two kinds of
infinity involved in the number of trials and in the just
mathematical price of each are different. If the number
of trials were 2 raised to an infinitely high power,
the probable average value of each trial would be the
infinitely high number representing that power. But
2 raised to that power would give an infinitely higher
number. To take very large numbers instead of infinite
numbers, which simply elude us:---Suppose the number
of trials could be 2 raised to the millionth power; then
the probable average value of each would be 1,000,002\textit{l}.,
which is a large number of pounds; but the number is a
mere nothing compared with the number of trials, a number
containing 301,031 \emph{digits}! If the smallest atom, according
to the estimate made by physicists, were divided
into a million millions of parts, the entire volume of a
sphere exceeding a million million times in radius the
distance of the remotest star brought into view by Lord
Rosse's mighty telescope would not contain a million
millionth of that number of these indefinitely minute
subdivisions of the atom. Nay, we might write trillions
or quadrillions where we have just written millions in
the preceding lines, and yet not have a number reaching
a quadrillionth part of the way to the inconceivable
number obtained by raising 2 to the millionth power.
Yet for this tremendous number of trials the average
mathematical value of each would amount but to a
poor million---absolutely nothing by comparison.

\chapter{Gambling in Shares}

If there is any evil quality of human nature which, by
its persistence, its wide-spreading and its mischievous
influence, speaks of the inborn savagery of human
nature, it is the greed for chance-won wealth. In all
ages men have been moved by it. It has seemed so
natural, that men have lost sight of its innate immorality.
`If I take my chance fairly with others and
win,' the gambler argues, `I have done no man wrong,
not even myself or the members of my family. What
I win I can regard as gain, not less legitimate than the
profits on some business transaction. If other men are
ruined, or if I run the risk of ruin myself, this is no more
than happens all the time. Other men may be killed
in various chance ways; I may myself be killed ere the
day is out in some chance manner: why should I not,
since I and others must incur the chances of life, raise
other chance issues by which either gain or loss may
result to others or to myself?'

It may be that false though this reasoning is as a
defence, there is more of excuse in it than those imagine
who use it. Beyond doubt the element of chance which
enters into all lives, has had a most potent influence in
moulding the characters of all men. If we consider the
multitudinous fancies and superstitions of men like
sailors, farmers, and hunters, whose lives depend more
on chance than those of men in other employments, and
recognise this as the natural effect of the influence
which chance has on their fortunes, we need not wonder
if the influence of chance in moulding the minds and
characters of our ancestors during countless generations,
should have produced a very marked effect on human
nature. An immense number of those from whom I
(for instance) inherit descent, must in the old savage
days have depended almost wholly on chance for the
very means of subsistence. When `wild in wood' the
savage (very far, usually, from being noble) ran, he ran
on speculation. He might or might not be lucky
enough to earn his living on any day by a successful
chase, or by finding such fruits of the earth as would
supply him with a satisfactory amount of food. He
might have as much depending on chances which he
could not avoid risking, as the gambler of to-day has
when he `sees red' and stakes his whole fortune on
a throw of the dice or a turn of the cards. We cannot
be doubtful about the effects of such chance influences
on even the individual character. Repeated generation
after generation they must have tended to fill men with
a gambling spirit, only to be corrected by many generations
of steady labour; and unfortunately, even in the
steadiest work the element of chance enters largely
enough to render the corrective influence of such work
on the character of the race (as distinguished from the
individual) much slower than it might otherwise be.
Every man who has to work for a living at all, every
man who has to depend in any way on business for
wealth (which is different from working for a living)
has to trust more or less to chance in many respects.
So that nearly all men have their characters in some
degree modified by this peculiarity of their environment.
The inherited tendency of each one of us towards
gambling, in some one or other of its multitudinous
forms, is undoubtedly strengthened in this way; though
fortunately it may be much more than correspondingly
weakened by training, by thought, and by steady pursuance
of life's proper work.

That gambling is immoral has been recognised by
those who have noticed the effects of established lottery
systems, or of gambling establishments such as formerly
were allowed to flourish in our cities, to the demoralisation
and ruin of thousands---among rich and poor alike.
Governments which once originated lotteries, and
reaped large profit from them, now not only cease to
raise money in so iniquitous a manner, but forbid
lotteries, and, as far as they can, prevent them. That
they remain an attraction for an immense number of
our people is shown by the circumstance that lotteries
permitted on the Continent advertise largely in English
newspapers and periodicals, and that their circulars
reach thousands of Englishmen through the post. I
have myself had experience of the assiduities of Continental
lottery promoters in both forms, having received
dozens of invitations to invest in these demoralising
ventures, and having also had any number of advertisements
offered for `Knowledge.' Yet every
lottery system, when it comes to be examined, proves,
as I have shown in essays on lotteries, to be based on
fraud---in such sort as to bring sure gains to the promoters
of the lottery, sure loss in the long run to the
purchasers of tickets---sure ruin even, if they will but
avail themselves in sufficient degree of the opportunities
for ruin obligingly proffered them.

In England, fortunately, lotteries are illegal. Yet
a method has been devised by which all classes of the
community may court fortune or ruin in the freest
manner, without gambling on card games (which would
attract attention and be unsuitable for those who object
to notoriety) or entering on turf speculations (still more
unpleasantly conspicuous in their method). I know not
that at the worst gambling-hells in the bad old times of
the Georges fortunes (and, what is worse, not fortunes
alone, but competencies and pittances) could be more
readily squandered than by the various forms of speculation
in stocks now made of easy access and convenient
procedure for all classes of our people---for men, for
women, and even for those who are little more than
children.

Speculation on the Stock Exchange has, of course,
been always a recognised method of gambling. In such
speculation as in the system now invitingly offered to
all classes there was often, if not generally, very little
money behind the speculations, compared with the
amount actually supposed to be invested in the various
transactions. (I use the word `supposed' in an entirely
conventional sense, for in Stock Exchange speculations
nothing is supposed to be actually invested, though such
and such amounts of stock are named as bought or
sold.) A speculator need be prepared only to pay the
difference between the value of the stock he is supposed
to have bought or sold at the beginning of the time-bargain
and its diminished or increased value when the
time expires. Thus a man shall nominally buy 10,000\textit{l}.\ in
certain stocks at, we will say, 9,927\textit{l}., which at the
end of the time for which the shares are supposed to
have been bought, shall be worth only 9,811\textit{l}.; in that
case, apart from brokerage or commission, he loses 116\textit{l}.\ on
the transaction. Or, if he had sold stock at 9,927\textit{l}.,
nominally (not really possessing any such amount), and
its value rose to 10,033\textit{l}.\ at the time for which the
bargain was entered on, then he would lose 106\textit{l}. It is
only (as a rule) some such proportion as this of the large
sum bought or sold that he will actually lose if unfortunate,
or gain if he has luck, on a transaction which has
such imposing dimensions.

The system, however, by which gambling in stocks
is now made accessible to all is more inviting than the
system of time-bargains.

By the time-bargain system a man could not tell
how much he was risking, any more than he could
tell how much he might gain. When settling time
came he might have won much or little, or he might
have lost little or much, on any particular speculation.
The probable gains and the probable losses, apart
from special knowledge or supposed knowledge of the
chances of rise or fall in price, were evenly balanced.

Now, though this might do very well for men on
'Change, just as hard gamblers in the good old times
were well content to risk their money on the pulling of
a straw or the toss of a die, risks of this sort have no
attraction for the average gamblers of the ordinary type.
If the history of men who have lost largely on the turf
were known, it would be found that, for one case where
the loss has arisen from wagers on even terms, there will
be a thousand or probably an even larger number in
which men have been ruined by backing horses at odds.
What the average gambler, who is nearly always a
weakling, wants, is a chance of winning a large sum by
risking a small one. If he backs a horse at odds he is
well pleased. But then the horse must also be a
favourite, or at least he must himself have a high
opinion of the horse's chance. Now a horse cannot be
a favourite and also have the odds against him, unless
there is a good field. Hence, the average betting man
of the pigeon type likes to lay his money on one or other
of the favourites in a large race, where the odds are
at least four or five to one against even the chief
favourite. Then if he loses he loses but a small sum
compared with that which he has a chance (and, as he
thinks, almost a certainty) of gaining. The bookmaker,
as we have seen, takes advantage of this delusion. He is
aware that a man who, knowing little about horses, fancies
a particular horse---on the strength, perhaps, of false
information which the bookmaker himself may have
helped to spread---will not be careful to note whether the
precise odds are offered. If the current odds are 12 to 1,
the simpleton will be content with 11 or 10 to 1, or
even less. The bookmaker, then, acts on the contrary
principle. He always, or nearly always, lays the odds
against horses---he seems to risk much to gain little---but,
on the plan he actually follows of always offering
less than the fair odds, his multiplied little gains nearly
always outbalance heavily his occasional heavy losses.
We occasionally hear of a large bookmaker coming to
grief; but not often, not nearly so often, as one could
wish.

Seeing that such are the ways of the gambling
public, it will be seen that the method of gambling
followed by men on 'Change would not be seductive
enough for the general public. Those who live on the
weakness of men for gambling very soon found this out.
Although some among them tried to make the Stock
Exchange system of speculating generally available,
the public, as a whole, were never greatly attracted
by a method of making a fortune which seemed to them
both slow and dangerous.

But a system is in vogue now which is as seductive
as any lottery system, is at present safe (strangely
enough) from check or punishment, and insures a
splendid profit from the foolish folk who take part in it,
even from those who win money by it---as, for a time,
the speculators often do.

This system, which men on 'Change by no means
like for their own transactions, is that called `the cover
system'; as a method of courting ruin it is the perfection
of simplicity.

In the cover system each transaction is closed, not
when a certain time but when a certain money limit is
reached (though in each, at a price, the transaction may
be extended). The speculator---the victim we may call
him, gaily though he trips up the altar steps---pays a
certain sum to a stockbroker of a certain class, as a
`cover' or deposit upon a hundred times that amount in
some stock which he fancies, or thinks he knows, will
rise or fall in price. He may be either a buyer or a
seller (always nominally), either a bull or a bear. It is
not necessary, if he is a buyer, that there should be any
real seller, or, if he is a seller, that there should be any
real buyer. Nothing is necessarily bought or sold---(except
the speculator himself, who is both). The
account having been now opened for that particular
stock, all that has to be done is to wait until the account
can be closed at a profit.

If stock has been nominally bought, the speculator
waits for it to rise, so that when it has risen high
enough he may close the account and gather in his
gains; or, if stock has been sold, he waits in like
manner until it shall fall. When it is rising or falling
to his advantage he is in pleasing doubt whether the
time has arrived to close to the greatest attainable
advantage. If he waits too long and it begins to move
the wrong way, he is apt to wait a little longer for the
motion in that wrong direction to cease---often with
disastrous results; but if he does not wait long enough,
and after the account has been closed the stock still continues
to move in what would have been the right direction
had the account been kept open, then he is made
miserable by the thought that he has thrown away money
which he might have gained. As he very seldom hits
the precise moment when the greatest possible profit is
to be reaped, he nearly always has the discomfort which
arises from the thought that he has closed the account
too soon or too late.

So much when fortune favours the speculator, as it
very often does at the beginning. It is even said, and
doubtless it is the case, that stockbrokers of the class we
are considering, those who lend themselves to the gambling
game which seems so inviting, take care that
beginners who have plenty of money to lose, are led on
by early successes. A poor fellow who cannot afford
to lose more than a paltry ten or twenty pounds, and
may even have had to borrow from his employer's till to
get that, may be cleared out at once; but manifestly it
would not do to dishearten a young fellow who has
thousands to lose. Still, with one or the other, losing
transactions have to be considered, sooner or later.
Here the refined torture arising from anxiety as to the
exact moment when the gain is as great as it is likely to
be is wanting. The speculator scarcely ever troubles
himself even to inquire when his loss, if he closes, is as
great as he can reasonably let it be. So long as the loss
is within the limit of the `cover' he holds on. He may
even, rather than lose the chance of a change of luck,
extend the cover. But whenever his cover, whether left
unchanged or extended as far as he is prepared to go, is
reached by the amount of loss, the account is closed and
his deposit is forfeited.

Let us consider an actual transaction in detail; and
that we may not in any way wrong the persons who
attempt to mislead the more foolish part of the public
in this matter, let us take an account published by one
of themselves:---` ``For instance, then''---says one of the
most notorious of these in an advertisement published
under guise of a story---``having reason to expect a certain
stock (Great Westerns) is likely to go up'' (the
grammar I ``expect, is likely'' to be the stockbroker's
own)---``the present price of which we will suppose is
$132\frac{1}{4}$; a client sends 10\textit{l}.\ 12\textit{s}.\ 6\textit{d}.\ as cover and commission,
with instructions to buy 1,000\textit{l}.\ Great Western
Railway Stock.'' ``If it goes up?'' queried Captain Dayrell,
becoming much interested. ``Paying attention to
the daily quotations, the operator notices that the stock
rises, say, to $133\frac{1}{4} - \frac{1}{2}$, and 10\textit{l}.\ or 10 per cent.\ is realised.
If the stock rises to $134\frac{1}{4} - \frac{1}{2}$, 20\textit{l}.\ or 20 per cent.\ is
realised, and so on in proportion.'' ``Should the reverse
happen?'' ``If, however, contrary to expectation the
stock goes down from $132\frac{1}{4}$ to $131\frac{1}{4} - \frac{1}{2}$, the cover has
run off, and the transaction is closed with the loss of the
10\textit{l}.\ cover only. Beyond this, and the commission
of $\frac{1}{16}$, or 12\textit{s}.\ 6\textit{d}.\ per 1,000\textit{l}.\ stock, there is no further
liability; and the beauty of the thing is, you only lose
what cover you put up.'' ``Suppose I put on more
cover before it is too late?'' ``Then you can keep the
account open,'' replied Roselle. ``It is simple enough,
and very fair.'' ``Yes; it limits the liability of the
operator.'' ``I see; he can choose any stock he pleases
to operate in; and, if his judgment is sound, or the information
good, the profit is certain.'' ``Exactly,'' said
Roselle, with a smile. ``I can see. The profits may
be very large, whilst the loss is always small,'' remarked
the Captain.'

One can see tolerably well, I may remark in passing,
how this account was written. The stockbroker for
whom the series of stories was written (much as poetry
used to be written for Moses \& Son) sent to the writer,
who uses the fine-sounding \textit{nom de plume} of Bracebridge
Hemyng, an example of the way of working the cover
system, and this writer, whose stories fortunately are of
the dull blood-and-thunder type, has simply turned the
account into a dialogue, by breaking it up, and inserting
`the Captain said,' `Roselle replied,' `I see,' \&c.

It will be noticed that in this account, and it is the
same in all such accounts, nothing is said as to whence
the money comes from by which, if the speculator wins,
he gets his winnings. Many of these unfortunate
gamblers have the idea that the stockbroker pays it out
of his own pocket. It never seems to occur even to
those who are not quite so foolish as to imagine this,
that if the method of rapidly making large fortunes
which stockbrokers advertise so freely, were as sure as
they pretend, there would be very little rising stock for
purchase by the outside public, and very little falling
stock for sale to them. The stockbrokers would transact
on their own behalf the business they are so eager to
transact for others for a consideration---the trifling
brokerage of $\frac{1}{16}$ per cent.

If the real nature of the transaction were described,
none but very foolish persons would enter on so transparently
dangerous a course.

The stockbrokers of the particular class we are
considering (for, of course, many stockbrokers are
thoroughly respectable men) say to the moths, `By
risking so much you may gain large sums.' If they
told the truth they would say, `By paying in so much
you enable me to purchase or sell such and such an
amount of stock, at such and such sure profit through
brokerage, without any risk.' The cover system has
been devised to protect the stockbroker, not to profit
the speculator.

Consider the position of the stockbroker in the case
just described, after the sum of 10\textit{l}.\ 12\textit{s}.\ 6\textit{d}.\ has been
paid in. He purchases 1,000\textit{l}.\ Great Western Railway
Stock for his client, and watches the telegraphic tape.
If the stock rises in value his client is able to close the
account at a profit, and in that case will start a new
account, with fresh brokerage, and be profitable to the
stockbroker. Therefore it is better for the broker to have
a lucky client, or even to give occasionally a useful hint
in the beginning of a new client's career. But if the
stock falls in value, the stockbroker, at the moment
when the loss is equal in amount to the cover, closes
the account, without loss to himself, and is the gainer
by the brokerage.

But `the brokerage is only $\frac{1}{16}$ per cent., and that is
a mere nothing.' If the brokerage were $\frac{1}{16}$ per cent.\ on
the money risked by the speculator that might more
reasonably be urged. It is, however, $\frac{1}{16}$ per cent.\ on a
hundred times that amount. That is to say, it is not
\emph{one} but \emph{one hundred} 16ths per cent., or $6\frac{1}{4}$ per cent.\ on
the speculator's money.

If we compare the position of the speculator in such
a transaction as this with that of a man who buys a
ticket in a lottery, we shall be able to see in what position
Stock Exchange speculation stands as compared
with speculation in lotteries, admitted to be a losing
business. In the case of stock gambling above considered,
the speculator pays 12\textit{s}.\ 6\textit{d}.\ and risks 10\textit{l}.\ for
the chance of winning he knows not how much. With
all consideration for his judgment or information, experience
shows that we cannot really regard the stock as
more likely to rise than to fall; and with any but `wild
cat' stock, with which no honest man can safely
meddle,\footnote{As this remark might be misunderstood I explain that no man
can safely speculate in bubble companies unless he is acquainted
with the plans of the promoters---or, in other words, is as great a
rascal as the promoters are.}
it is altogether unlikely that the rise will be such as to
give a profit of 50\textit{l}.\ on the transaction. It is probably
much more than a hundred to one against this. Now
it is a hundred to one against the holder of one ticket
out of a hundred in a lottery drawing the single prize
of 500\textit{l}. To get that chance he ought, strictly speaking,
to pay only 5\textit{l}.; but as the Louisiana lottery, and
most others, are constituted he would probably have to
pay about 10\textit{l}. Here he has risked 10\textit{l}.\ for the same
chance of winning 500\textit{l}.\ that the stock gambler in the
other case has of winning 50\textit{l}. True, the latter has the
chance of winning some smaller sum; but, as a rule,
the gambler in stocks never is content except with a
large profit, of which he may boast as a fine stroke
either of skill or luck.

If we compare the smallness of the amount risked
with the sum which may be gained, all lotteries have a
great and some have an immense advantage over Stock
Exchange gambling. For five dollars or rather more
than a pound, a gambler has in the Louisiana lottery
the chance of winning 200,000 dollars.

Where the stock-gambling system seems to the
dupes to have a great advantage over the lottery system
is in the apparently small percentage of profit reaped
by the person who manages the transaction. Just as
players at \textit{trente-et-un} used to imagine the advantage of
the \textit{refait} held by the bank so small as to leave the terms
of the gambling all but even, and used to rejoice over
the bank's small percentage of advantage on each transaction,
so does the stockbroker's dupe, who would
probably pay ten or twenty per cent., as the lottery
gambler does, rather than not court ruin at all, rejoice
at the nominal $\frac{1}{16}$ per cent. I have shown that in
reality the percentage on the money risked is more than
6 per cent. It may be argued, and justly enough in a
sense, that the risk of the cover-speculator who pays in
10\textit{l}.\ is, in reality, precisely the same as the risk of a full
speculator who actually bought 1,000\textit{l}.'s worth of stock.
The latter could stop short of a loss of 10\textit{l}.\ just as
readily after buying the stock as the cover-speculator
does at the outset. All he would have to do would be
to watch the progress of prices, and sell so soon as the
fall corresponded to a loss of 10\textit{l}. At least he could do
this in the case of far the greater number of stocks---in
fact, this is practically what the stockbroker does for
the cover-speculator. So that the percentage for
brokerage is properly extended to the full amount.
This is perfectly just in the case of a legitimate investment.
But so soon as we consider how the cover-speculator
renews and re-renews his risk on the smaller
amount, we see that the percentage taken by the broker
is very much more than $\frac{1}{16}$, much more even than 6$\frac{1}{3}$
per cent. Like the Homburg bank's advantage on the
\textit{refait}, it is small on individual transactions, but mounts
up to something enormous as a percentage, when considered
with reference to the total amount of probable
gain or loss after steady persistence in gambling.

Here, for example, is a case very favourable indeed
for the speculator (who on the average would have no
such luck):---A man pays 10\textit{l}.\ as cover for 1,000\textit{l}.\ stock,
and 12\textit{s}.\ 6\textit{d}.\ the small percentage for the broker.
He is lucky, and wins 20\textit{l}.\ when the transaction is
closed---say, in a week. This is splendid, for it is earning
money, he thinks (stock gamblers always speak of
earning money, just as racing men speak of bookmaking
as if it were a respectable trade or profession), at
the rate of 1,000\textit{l}.\ a year. He now invests his 20\textit{l}., as
cover on 2,000\textit{l}., in some other stock, either as buyer or
seller, paying 1\textit{l}.\ 5\textit{s}.\ as brokerage. This time he is not
so fortunate; the stock moves the wrong way, and in
the course, say, of another week his 20\textit{l}.\ cover is forfeited.
But, depending on a change of luck, and also
feeling (as every gambler does) that he is essentially a
lucky man, however at times fate may frown on him,
he invests 10\textit{l}.\ as cover, paying 12\textit{s}.\ 6\textit{d}.\ as brokerage
on 1,000\textit{l}., winning in a week 20\textit{l}. He invests---this
wild kind of speculation may be pleasingly called investment---a
sum of 15\textit{l}.\ as cover, which in another week is
forfeited, the brokerage being in this case 18\textit{s}.\ 9\textit{d}. He
invests 16\textit{l}.\ as cover, paying 1\textit{l}.\ brokerage, and wins
16\textit{l}., in a week. He next invests 16\textit{l}.\ again, paying 1\textit{l}.\ again
for brokerage, and forfeits his cover. Here we
leave him, so far as further speculation is concerned,
though it is very unlikely that he would stop his speculative
system here. Let us consider what he has accomplished
in the six weeks (we took always the same
period of time, so that the summing-up might be simplified---about
six weeks for the six transactions would
have served equally well):

He has won 20\textit{l}., 20\textit{l}., and 16\textit{l}.; he has lost 20\textit{l}.,
15\textit{l}., and 16\textit{l}. He has gained thus in all 5\textit{l}. He has
paid in brokerage 12\textit{s}.\ 6\textit{d}., 1\textit{l}.\ 5\textit{s}., 12\textit{s}.\ 6\textit{d}., 18\textit{s}.\ 9\textit{d}.,
1\textit{l}., and 1\textit{l}.---making a total of 5\textit{l}.\ 8\textit{s}. 9\textit{d}. Therefore he
is out of pocket 8\textit{s}.\ 9\textit{d}.; he has lost six weeks' use of
the sum of 10\textit{l}.\ first invested, to say nothing of a loss of
the use of 5\textit{l}.\ more during the third week; and he has
undergone a good deal of worry and anxiety. Yet he
has had better luck than he had a right to expect; for,
on the whole, he has had a balance of gain over loss, so
far as the actual transactions have been concerned. If
he had lost on the whole instead of gained he would
still have lost in addition the sum of 5\textit{l}.\ 8\textit{s}. 9\textit{d}. One
may regard this as what he has paid for the privilege
of investing 10\textit{l}., or thereabouts, during six weeks as
cover on Stock Exchange speculations. He could not
have carried on this preposterously foolish system of
gambling without the kindly proffered assistance of the
advertising class of stockbrokers; and that is what he
has had to pay them for their ministrations. Regarded
as percentage on 10\textit{l}.\ for 6 weeks, it is at the rate of
more than 470 per cent.\ per annum. This is the sort of
percentage which `the utterly insignificant brokerage'
amounts to where the speculative wiseacre persists in
his folly long enough. It is true the broker may not
always reap such a profit in the year on any one
victim---for the victim may be ruined before the year is
out. But that is a misfortune amply repaired for the
broker by the constant influx of fresh victims.

In the series of transactions imagined above the
stockbroker, without risk, secures more than the victim
would have gained if there had been no brokerage.
He would have secured the same percentage had the
investments been all ten times greater, or a hundred
times greater, or a thousand, or ten thousand. Always
he pockets freely and without risk; always, even the
luckiest speculator pays freely, and the unluckiest speculator
has to pay in like manner, besides losing heavily.
Apart from success or failure in the speculations themselves---and
in the long run these are bound to balance
themselves pretty equally unless the speculator gets
`tips,' in which case he is sure to lose heavily in the
long run---the broker always makes a sure and large
gain, the speculator always has a sure and large loss, in
brokerage alone.

Of course the example I have just considered will
not be regarded by the average speculative gambler as
typical. He expects to win very much more than he
loses, or to win always and not lose at all. In reality,
he has no more right to expect a considerable balance
of gain than a farmer has to expect exceptional weather.
Assuming fair bargains, as I have pointed out in
the preceding sections, the gambler in stocks has
no right to expect to gain more than he loses. Of
course he does expect to gain, or he would not speculate.
But if he has a particle of common sense, he
will see that at the best he can only gain on some
transactions rather more than he loses on others.
Hence such a result as I have considered above is
about what might be expected to occur in the case
of a lucky speculator.

Taking a more general view, a speculator would have
reason to regard himself as exceptionally fortunate if his
gains were to his losses in the proportion of nine to
eight. Suppose now that a speculator went on for a
whole year at this rate, gaining on the average 50\textit{l}.\ a
week; and suppose, further, that his gain, when he has
gained, has averaged the amount of cover invested, his
loss, when he has lost, being always the cover paid in.
It will be seen that his full weekly gain has been 450\textit{l}.,
his full weekly loss being 400\textit{l}.; so that the total
amount invested as cover has been 850\textit{l}.\ weekly, the
stock represented being 85,000\textit{l}. The brokerage on
this at $\frac{1}{16}$ per cent.\ amounts to 53\textit{l}.\ 2\textit{s}. 6\textit{d}.; so that in
this case, with a seeming gain of 50\textit{l}.\ weekly, the unfortunate
speculator loses 3\textit{l}.\  2\textit{s}. 6\textit{d}., the broker pocketing all his client's
gains and 3\textit{l}. 2\textit{s}. 6\textit{d}. beyond.

Supposing, for a moment, all the transactions for
one week, having the result just indicated for the winning
speculator, to have been as between him and another,
who has therefore been necessarily a losing speculator,
we find that this other has had to pay \emph{his} broker also
53\textit{l}.\ 2\textit{s}. 6\textit{d}., and has further lost 50\textit{l}. He has lost, then,
in all 103\textit{l}. 2\textit{s}. 6\textit{d}.

Here, then, we have this result, calling the lucky
speculator A, the unlucky speculator B, and two stockbrokers
respectively R and S:

A has won 50\textit{l}., and paid 53\textit{l}.\ 2\textit{s}. 6\textit{d}. in brokerage,
being therefore only \emph{minus} 3\textit{l}.\ 2\textit{s}. 6\textit{d}.

B has lost 50\textit{l}., and paid 53\textit{l}. 2\textit{s}. 6\textit{d}.,
being therefore \emph{minus} 103\textit{l}.\ 2\textit{s}. 6\textit{d}.

A's broker R has gained 53\textit{l}.\ 2\textit{s}. 6\textit{d}.

B's broker S has gained 53\textit{l}.\ 2\textit{s}. 6\textit{d}.

So long as there are many idiotic A's and B's seeking
their own ruin by the cover system, one need not necessarily
assume that R and S stand appropriately for
rascal and swindler. But when stockbrokers choose to
join the ranks of those who advertise for clients of this
sort, who confidently proclaim that speculation of this
kind is a safe and ready way of making a fortune, and
thus ensnare thousands of foolish persons to enter on a
path which leads always to loss and often to ruin and
shame, they must be prepared to find themselves classed
among creatures of prey. They are not the less
wrong-doers that at present the law has not forbidden
them to prey thus on the weak and foolish. The law
should be altered and our gaols enlarged.

The defence is made that, if the speculator has good
judgment or special information, he will win largely.
The same defence has been made for the rascally system
by which bookmakers devour the substance of the young
and silly. Every man who gambles imagines he is
trusting to his judgment, and that he has judgment in
which to trust. From the foolish heir of `noble' or
wealthy family to poor stupid 'Arry, there is not a turf
gambler of the pigeon type who does not think he can
form a tolerably shrewd guess as to the chance of every
favourite in a race, or that he has information which
practically makes him safe to win. Repeated losses may,
after a time, teach the sort of wisdom by which a man
recognises his own inexperience; but even this is unusual.

Now, if the tyro cannot really form any idea as to
the chances of a horse in a race, if the information to
which he trusts is baseless or even misleading, can it
be supposed that any, except the most experienced
business men, can form a sound opinion about the
points on which the ever-changing values of stock
depend? Not one of those who speculate has in reality
any sufficient power of \emph{judging} in such matters at all;
for sound business men never speculate. Nor would
the soundest judgment avail in the case of many kinds
of stock, for the values change as the stock is `pulled'
by hands of whose very existence the ordinary speculator
knows nothing.

If the ordinary speculator even had exceptional
power of discrimination (an idea which is altogether
absurd to those who know how foolish the ordinary
speculator is), and if he always had special information
on which he could rely (which again is absurd), his
position would be altogether unsatisfactory. He would
be less foolish but more knavish than I have been
assuming. He would be much like a player in a card-game
depending properly purely on chance, who should
take advantage of exceptional keenness of sight or of
information conveyed by a confederate to learn the
cards held by the other players. For every pound
one player or speculator gains through such judgment
or information another player loses a pound,
or several other players lose by amounts whose total
is a pound.

It may be said that this is mere exaggeration, that
it would apply to investment as closely as to speculation,
or that it might even be applied to the ordinary
transactions of trade, in which those who show good
judgment and possess good experience succeed, while
the unwise and inexperienced fail. In reality, it might
as reasonably be said that wagering on a tradesman's
chances of success or failure is as legitimate a way of
trying to win money as carrying on trade, or that such
wagering between a man who knows nothing about the
tradesman's chances and one who knows a great deal
about them would be fair and honest.

This last comparison, by the way, is nearer the
truth than probably most persons imagine. It is singular
how little is understood about the real nature of
stocks even by the speculative folk who imagine that
they know all about them. Money invested in stock
is in reality money lent, and usually money lent for
business use. Of course Consols represent money lent
to Government, while various foreign investments represent
money lent to foreign Governments, and these
can hardly be called business loans. But in the main
the stocks dealt with in the business columns of our
papers, the Foreign Market and City Intelligence, are
loans to various companies engaged in commercial
business.

Now, if we ask why these stocks vary as they do in
value, from Consols down to the lowest class of stocks,
we find that theoretically the changes correspond with
the varying degrees of advantage or of security, or both
combined, which the lender recognises in these different
openings for lending his money.

A business company needing money for any particular
purpose, and having good credit, will either
borrow such and such a sum at a definite rate per cent.\ for
interest, to be paid half-yearly or yearly, or else will
nominally borrow a definite sum for a definite time,
really receiving only a certain smaller sum (the difference
being discount), and repaying the full sum at the
expiration of the allotted time. It is manifest that
when the public is to be borrowed from---that is to say,
when a large number of persons are to lend money to
some company---the former arrangement would be inconvenient.
Many might be prepared to lend money for
a time, but not indefinitely; yet it would be most undesirable
that the company should have a large number
of creditors, any of whom might when they chose
demand the return of their money. The plan actually
adopted to avoid inconvenience on both sides is nearer
the discounting arrangement than the other, though
not quite identical with it either. A nominal percentage
is offered to the public in many cases; in others the
prospect of such and such a percentage; in others a
guaranteed percentage, with the possibility of more.
Originally the sum paid as one hundred pounds in the
way of loan to the company (or one hundred pounds in
the company's stock) may be actually one hundred
pounds, or may be a sum greater or less offered (as at
an auction)---greater if the prospects of the company
are regarded as good; less if they are not so highly
esteemed. After the original capital of the company---in
reality the original loan to the company---has been
raised, any part of the capital or loan may pass from
one hand to another, but always at such price for the
nominal possession of each hundred pounds of stock,
or each ten pounds (or whatever may be the unit share),
as the prospects of the company are held to justify.

Thus a man who holds a certain amount of stock in
any company---be it a nation, or a bank, or a railway,
or a trade company---may be considered as for the time
being a man who has nominally lent that sum to the
company, and is to receive interest on it at a fixed rate,
but who has in reality paid perhaps more or perhaps
less than that nominal value because of the higher or
lower degree of prosperity and credit possessed by the
company. For example, he may hold 1,000\textit{l}.\ stock in
a company paying 5 per cent.; but he may have paid,
perhaps, 1,331\textit{l}.\ for that stock, which is as if he had
lent 1,331\textit{l}.\ for 50\textit{l}.\ interest per annum---that is really
for less than 4 per cent.\ per annum. Or he may have
paid, perhaps, only 817\textit{l}.\ for the stock, which is as
though he had lent 817\textit{l}.\ for 50\textit{l}.\ interest per annum,
or really for more than 6 per cent.\ per annum. And in
passing, we note that, considering any single company,
we see at once how definitely a high rate of interest
signifies (as the Duke of Wellington used to tell his
officers) low security; for, just as the prosperity and
credit of a company rises, so does its stock rise in value,
and therefore the rate of interest obtained by purchasers
of such stock diminishes, and \textit{vice vers\^a}.

We note that, according to this method of treating
stock in a company, the interest nominally remains
unchanged; but the amount to be paid for the nominal
sum of 100\textit{l}., on which 3\textit{l}., 4\textit{l}., or 5\textit{l}.\ (or whatever the
nominal rate per cent.\ may be) is to be paid, varies all
the time. It not only varies with actual changes in
the prospects of the company, but it varies also as the
value of money changes, or as, with the changes in the
prospects of other companies, the relative value of the
company alters. If, for instance, owing to certain
changes in the value of money, it becomes as easy to
secure 5 per cent.\ per annum on money lent as it had
been to secure 4 per cent.\ when certain stock was
bought at a certain price, the value of that stock will
evidently be diminished. A buyer, who practically is
one proposing to lend money to the company in place
of the seller who had already done so, can reasonably
expect a better rate of interest when ordinary loans
secure a better rate; he, therefore, reasonably expects
to pay a smaller sum for the same nominal rate per
cent.\ or per share.

Such being the nature of the stock market, it is
obvious that, while investment is a matter which
requires much judgment, and should not be entered on
without good information from business persons as to
the probable stability of the various stocks for sale and
purchase, speculation in stocks is utter folly where it is
not gross rascality. It is seen to be practically not
only \emph{akin} to wagering on the success or failure of a
number of persons engaged in business of the nature of
which we know nothing, but it is actually this very
thing. One might as reasonably go along a street, and,
selecting at random any shop, wager that the owner's
business will improve during the next week, or that it
will fall off, with no surer means of guessing than the
look of the shop, as run the eye down the share lists
and put cover down on the chance that any particular
stock will rise or fall. Nay, wagering on the tradesman's
business would be much the safer, for one would
see the shop and the goods, one could note the shopman
and his ways, and one might form a shrewd idea as to
his probable success or failure. But of the various
companies---nations, banks, railways, trade companies,
and so forth---in the share list, the cover speculator
knows nothing with any certainty, except what is general
knowledge and therefore does not help his chance of
making a lucky hit. For instance, I know that while
Consols are absolutely safe, they will rise or fall as the
relations of Great Britain with other nations improve
or the reverse; but every one else knows as much. I
may know that prospects look favourable or gloomy,
but so also will others. I may form a guess as to
whether the actual change of value in Consols in any
direction will be greater or less than is generally supposed
probable; but so soon as I thus pass beyond what
is common knowledge, I am as likely to be wrong as to
be right. To suppose otherwise is to suppose that
where veteran statesmen who know what is actually
being done, and the strings which are being actually
pulled, can form no sure or trustworthy guess, I can
who have no such knowledge. For a cover-speculator,
necessarily a simpleton, to buy or sell (nominally)
Egyptian, Turkish, or Russian stock, with the idea that
he is likely to form a correct opinion where a Gladstone
or a Salisbury would be certainly as likely to be wrong
as to be right, is preposterous on the face of it. And
so also with the railways, banks, and other business
companies whose names appear in the share lists.
Those who have the best opportunities of knowing the
state of affairs in a company have nothing like the
confidence in their carefully weighed opinion as to the
company's prospects which the cover-speculator has in
his fancy that the company's stock must rise or must
fall. As the tradesman is content with the amount of
chance which enters inevitably into the progress of his
business, without wagering on it, so the persons actually
engaged in a sound mercantile business on the larger
scale are content with the ups and downs which affect
the fortunes of all large companies without incurring
risk by speculating about them. But fools rush in------the
proverb is something musty.

It may be asked, then, whether money has not been
made by speculation, whether it is not a known fact
that there are at this moment men of wealth who have
made their money entirely by Stock Exchange speculation,
never having turned a single honest penny?

Undoubtedly men have become rich in this way, just
as men have become rich on the turf. Where otherwise
could it be supposed that all the money which the foolish
have lost through listening to the wiles of the craftier
sort among stockbrokers, or by betting with bookmakers
on horses, has gone to? Where tens of thousands of
foolish folk are ruined or lose largely, we may be well
assured that hundreds of crafty scoundrels have grown
rich. These `drop off gorged' from the schemes which
leave those `flaccid and drained.' The stockbrokers do
not get all the money lost by the foolish cover-speculators.
In the typical case I cited the stockbrokers
made 106\textit{l}.\ 5\textit{s}.\ between them, and the lucky and the
unlucky speculators lost between them only 56\textit{l}.\ 5\textit{s}.;
but there I was dealing with, the entirely imaginary
case of fair speculation. In actual business cover-speculators
inevitably fall, in many of their transactions, into
the hands of men akin to the bookmakers in turf
gambling, who play with cogged dice. Companies
are started which have no chance of success as business
schemes, but bring money freely into the hands of those
who plan them, or being associates of the gang know
how to utilise their knowledge. The prices are run up
by means familiar to such men, but of which the unfortunate
cover-speculator knows nothing. When the
swindling scheme has done its work, and all the conspirators
have cleared their profits on the rise in the
price of shares, the cover-speculator finds himself moved
to buy stock in the manifestly promising and prospering
concern. To his disgust, but not at first to his alarm,
he finds the price of shares at a standstill or even slightly
falling. He holds on for the renewed rise which he
feels sure---trusting in the judgment he imagines he has
in such matters, or in information which he supposes to
be trustworthy---will assuredly take place. When the
price sinks so as to endanger his deposit, he extends the
cover. Presently the bubble explodes, and he finds
himself one of the large array of those who have been
drained by the rascally promoters.

There are also ways of affecting the price of shares
in thoroughly honest concerns by promulgating false
rumours; and many a poor wretch, who has complained
of fortune frowning when he has seen cover after cover
impounded through the fall of shares when he had
expected a rise, and \textit{vice vers\^a}, has been the victim of
anything but fortune's assaults; his money has been as
deliberately stolen as if his pocket had been picked.

So certain is eventual loss to the cover-speculator
that I would endorse the saying of an esteemed friend
of mine, a merchant in St. Joseph, Missouri, who when
a young man boasted of gaining a large sum by dealing
in `corners in grain' (a system precisely similar to the
cover system, only the varying prices of particular kinds
of grain, instead of the prices of particular stocks, decide
the question of loss or gain), told the lucky gambler that
the very best thing he could do with his winnings was
to fling them into the Missouri.

In fine, no one has any but the minutest chance of
failing to lose largely by cover-speculation---unless he is
prepared to speculate with such knowledge as would
make every transaction a villainy.

\chapter{Fallacies and Coincidences}

Every one is familiar with the occasional occurrence of
coincidences, so strange---considered abstractly---that it
appears difficult to regard them as due to mere casualty.
The mind is dwelling on some person or event, and
suddenly a circumstance happens which is associated
in some altogether unexpected, and as it were improbable,
manner with that person or event. A scheme
has been devised which can only fail if some utterly
unlikely series of events should occur, and precisely
those events take place. Sometimes a coincidence is
utterly trivial, yet attracts attention by the singular
improbability of the observed events. We are thinking
of some circumstance, let us say, in which two or three
persons are concerned, and the first book or paper we
turn to shows, in the very first line we look at, the
names of those very persons, though really relating to
others in no way connected with them; and so on, with
many other kinds of coincidence, equally trivial and
equally singular. Yet again, there are other coincidences
which are rendered striking by their frequent recurrence.
It is to such recurring coincidences that common superstitions
owe their origin, while the special superstitious
thus arising (that is, superstitions entertained by individuals)
are innumerable. It is lucky to do this,
unlucky to do that, say those who believe in common
superstitions; and they can always cite many coincidences
in favour of their opinion. But it is amazing
how common are the private superstitions entertained by
many who smile at the superstitions of the ignorant:
we must suppose that all such superstitions have been
based upon observed coincidences. Again, there are
tricks or habits which have obviously had their origin
in private superstitions. Dr.~Johnson may not have
believed that some misfortune would happen to him if
he failed to place his hand on every post which he passed
along a certain route; he would certainly not have
maintained such an opinion publicly: yet in the first
instance that habit of his must have had its origin in
some observed coincidences; and when once a habit of
the sort is associated with the idea of good luck, even
the strongest minds have been found unready to shake
off the superstition.%
%
\footnote{Here, for instance, is an account given by one keen card-player
of another who was as keen, or keener. `He was very particular
about cutting the cards; he always insisted on the pack being perfectly
square before he would cut, and that they should be placed
in a convenient position. There is an old adage that a slovenly cut
is good for the dealer, but whether there is truth in the statement
we know not. He was superstitious to a degree that was astonishing.'
(It must be a rather startling superstition that would seem
astonishing to a man who could gravely ask whether there is any
truth in the preposterous adage just quoted.) `We are not aware
that any one has ever attempted to solve the problem why so many
great minds' (among card-players, fighting men, and men who have
to work much at odds with fortune) `are superstitious. This is not
the time or place to attempt that solution. We record the fact.
He believed in dress having something to do with luck, and if the
luck followed him, he would wear the same dress, whether it was
adapted to the weather or not. He believed in cards and seats. He
objected to any one making a remark about his luck. He had the
strongest objection to our backing him, because of our bad luck,
and we have often had to refrain from taking odds, because of this
fad. He was distressed beyond measure if any one touched his
counters. His constant system of shuffling the cards was at times
an annoyance.' This was a great card-player!}

It is to be noticed, indeed, that many who reject the
idea that the ordinary superstitions have any real
significance, are nevertheless unwilling to run directly
counter to them. Thus, a man shall be altogether
sceptical as to the evil effects which follow, according
to a common superstition, from passing under a ladder;
he may be perfectly satisfied that the proper reason for
not passing under a ladder is the possibility of its
falling, or of something falling from it: yet he will not
pass under a ladder, even though it is well secured, and
obviously carries nothing which can fall upon him.
So with the old superstition, that a broken mirror
brings seven years of sorrow, which, according to some,
dates from the time when a mirror was so costly as to
represent seven years' savings---there are those who
despise the superstition who would yet be unwilling to
tempt fate (as they put it) by wilfully breaking even
the most worthless old looking-glass. A story is not
unfrequently quoted in defence of such caution.
Every one knows that sailors consider it unlucky for
a ship to sail on a Friday. A person, anxious to
destroy this superstition, had a ship's keel laid on a
Friday, the ship launched on a Friday, her masts taken
in from the sheer-hulk on a Friday, the cargo shipped
on a Friday; he found (heaven knows how, but so the
story runs) a Captain Friday to command her; and
lastly, she sailed on a Friday. But the superstition
was not destroyed, for the ship never returned to port,
nor was the manner of her destruction known. Other
instances of the kind might be cited. Thus a feeling
is entertained by many persons not otherwise superstitious,
that bad luck will follow any wilful attempt to
run counter to a superstition.

It is somewhat singular that attempts to correct
even the more degrading forms of superstition have
often been as unsuccessful as those attempts which may
perhaps not unfairly be called tempting fate. Let me
be understood. To refer to the example already given,
it is a manifest absurdity to suppose that the sailing of
a ship on a Friday is unfortunate; and it would be a
piece of egregious folly to consider such a superstition
when one has occasion to take a journey. But the case
is different when any one undertakes to prove that the
superstition is an absurdity; simply because he must
assume in the first instance that he will succeed, a
result which cannot be certain; and such confidence,
apart from all question of superstition, is a mistake.
In fact, a person so acting errs in the very same way as
those whom he wishes to correct; they refrain from a
certain act because of a blind fear of bad luck, and he
proceeds to the act with an equally blind belief in good
luck.

But one cannot recognise the same objection in the
case of a person who tries to correct some superstition
by actions not involving any tempting of fortune.
Yet it has not unfrequently happened that such actions
have resulted in confirming the superstition. The
following instance may be cited. An old woman came
to Flamsteed, the first Astronomer-Royal, to ask him
whereabouts a certain bundle of linen might be, which
she had lost. Flamsteed determined to show the folly of
that belief in astrology which had led her to Greenwich
Observatory (under some misapprehension as to the
duties of an Astronomer-Royal). He `drew a circle,
put a square into it, and gravely pointed out a ditch,
near her cottage, in which he said it would be found.'
He then waited until she should come back disappointed,
and in a fit frame of mind to receive the
rebuke he intended for her; but `she came back in
great delight, with the bundle in her hand, found in
the very place.'

In connection with this story, though bearing rather
on over-hasty scientific theorising than on ordinary
superstitions, I quote the following story from De
Morgan's `Budget of Paradoxes':---'The late Baron Zach
received a letter from Pons, a successful finder of comets,
complaining that for a certain period he had found no
comets, though he had searched diligently. Zach, a
man of much sly humour, told him that no spots had
been seen on the sun for about the same time---which
was true---and assured him that when the spots came
back the comets would come with them. Some
time after he got a letter from Pons, who informed him
with, great satisfaction that he was quite right; that
very large spots had appeared on the sun, and that he
had found a comet shortly after. I have the story in
Zach's handwriting. It would mend the story exceedingly
if some day a real relation should be established
between comets and solar spots. Of late years
good reason has been shown for advancing a connection
between these spots and the earth's magnetism. If the
two things had been put to Zach he would probably
have chosen the comets. Here is a hint for a paradox:
the solar spots are the dead comets, which have parted
with their light and heat to feed the sun, as was once
suggested. I should not wonder if I were too late,
and the thing had been actually maintained.' De
Morgan was not far wrong. Something very like his
paradox was advocated, before the Royal Astronomical
Society, by Commander Ashe, of Canada, earlier we
believe than the date of De Morgan's remarks. I
happen to have striking evidence in favour of De
Morgan's opinion about the view which Zach would
probably have formed of the theory which connects
sun-spots and the earth's magnetism. When the
theory was as yet quite new, I referred to it in a
company of Cambridge men, mostly high mathematicians,
and it was received at first as an excellent
joke, and welcomed with laughter. It need hardly
be said, however, that when the nature of the evidence
was stated, the matter assumed another aspect.
Yet it may be mentioned, in passing, that there are
those who maintain that, after all, this theory is untrue,
the evidence on which it rests being due only to certain
strange coincidences.

In many instances, indeed, considerable care is required
to determine whether real association or mere
casual coincidence is in question. It is surprising how,
in some cases, an association can be traced between
events seemingly in no way connected. One is reminded
of certain cases of derivation. Ninety-nine persons out
of a hundred, for instance, would laugh at the notion
that the words `hand' and `prize' are connected; yet
the connection is seen clearly enough when `prize' is
traced back to `prehendo,' with the root `hend' obviously
related to `hand,' `hound,' and so on. Equally
absurd at a first view is the old joke that the Goodwin
Sands were due to the building of a certain church; yet
if moneys which had been devoted to the annual removal
of the gathering sand were employed to defray
the cost of the church, mischief, afterwards irreparable,
might very well have been occasioned. Even the explanation
of certain mischances as due to the circumstance
that `there was no weathercock at Kiloe,' may
admit of a not quite unreasonable interpretation. I
leave this as an exercise for the ingenious reader.

But when we have undoubted cases of coincidence,
without the possibility of any real association (setting
the supernatural aside), we have a problem of some
interest to deal with. To explain them as due to some
special miraculous intervention may be satisfactory
to many minds, in certain cases; but in others it is
impossible to conceive that the matter has seemed
worthy of a miracle. Even viewing the question in its
bearing on religious ideas, there are cases where it
seems far more mischievous (as bringing ridicule on
the very conception of the miraculous) to believe in
supernatural intervention, than to reject such an explanation
on the score of antecedent improbability.
Horace's rule, `\textit{Nec deus intersit nisi dignus vindice
nodus},' remains sound when we write `\textit{Deus}' for
`\textit{deus}.'

Now there have been cases so remarkable, yet so
obviously unworthy of supernatural intervention, that
we are perplexed to find any reasonable explanation of
the matter. The following, adduced by De Morgan,
will, I have no doubt, recall corresponding cases in
the experience of readers of these lines:---`In the
summer of 1865,' he says, `I made myself first acquainted
with the tales of Nathaniel Hawthorne, and
the first I read was about the siege of Boston in the
War of Independence. I could not make it out:
everybody seemed to have got into somebody else's
place. I was beginning the second tale when a parcel
arrived: it was a lot of odd pamphlets and other
rubbish, as he called it, sent by a friend who had lately
sold his books, had not thought it worth while to send
these things for sale, but thought I might like to look
at them, and possibly keep some. The first thing I
looked at was a sheet, which, being opened, displayed
``A plan of Boston and its environs, showing the true
situation of his Majesty's army, and also that of the
rebels, drawn by an engineer, at Boston, October 1775.''
Such detailed plans of current sieges being then uncommon,
it is explained that ``The principal part of
this plan was surveyed by Richard Williams, Lieutenant,
at Boston; and sent over by the son of a nobleman
to his father in town, by whose permission it was
published.'' I immediately saw that my confusion
arose from my supposing that the king's troops were
besieging the rebels, when it was just the other way'
(a mistake, by the way, which does not suggest that the
narrative was particularly lucid).

Another instance cited by De Morgan is yet more
remarkable, though it is not nearly so strange as a
circumstance which I shall relate afterwards:---`In
August, 1861,' he says, `M.~Senarmont, of the French
Institute, wrote to me to the effect that Fresnel had
sent to England in, or shortly after, 1824, a paper for
translation and insertion in the ``European Review''
which shortly after expired. The question was what
had become of the paper. I examined the ``Review'' at
the Museum, found no trace of the paper, and wrote
back to that effect, at the Museum, adding that everything
now depended on ascertaining the name of the
editor, and tracing his papers: of this I thought there
was no chance. I posted the letter on my way home,
at a post-office in the Hampstead Road, at the junction
with Edward Street, on the opposite side of which is a
bookstall. Lounging for a moment over the exposed
books, \textit{sicut meus est mos}, I saw within a few moments of
the posting of the letter a little catchpenny book of anecdotes
of Macaulay, which I bought, and ran over for
a minute. My eye was soon caught by this sentence:---``One
of the young fellows immediately wrote to the
Editor (Mr.~Walker) of the `European Review.'\,'' I thus
got the clue by which I ascertained that there was no
chance of recovering Fresnel's papers. Of the mention
of current Reviews not one in a thousand names the
editor.' It will be noticed that there was a double coincidence
in this case. It was sufficiently remarkable
that the first mention of a review, after the difficulty
had been recognised, should relate to the `European,'
and give the name of the editor; but it was even more
remarkable that the occurrence should be timed so
strangely as was actually the case.

But the circumstance I am now to relate seems to
me to surpass in strangeness all the coincidences I have
ever heard of. It relates to a matter of considerable
interest apart from the coincidence.

When Dr.~Thomas Young was endeavouring to
interpret the inscription of the famous Rosetta Stone,
Mr.~Grey (afterwards Sir George Francis Grey) was led
on his return from Egypt to place in Young's hands
some of the most valuable fruits of his researches
among the relics of Egyptian art, including several
fine specimens of writing on papyrus, which he had
purchased from an Arab at Thebes, in 1820. Before
these had reached Young, a man named Casati had
arrived in Paris, bringing with him from Egypt a
parcel of Egyptian manuscripts, among which Champollion
observed one which bore in its preamble some
resemblance to the text of the Rosetta Stone. This
discovery attracted much attention; and Dr. Young
having procured a copy of the papyrus, attempted to
decipher and translate it. He had made some progress
with the work when Mr.~Grey gave him the new
papyri. `These,' says Dr.~Young, `contained several
fine specimens of writing and drawing on papyrus;
they were chiefly in hieroglyphics and of a mythological
nature; but two which he had before described to me,
as particularly deserving attention, and which were
brought, through his judicious precautions, in excellent
preservation, both contained some Greek characters,
written apparently in a pretty legible hand. That
which was most intelligible had appeared at first sight
to contain some words relating to the service of the
Christian Church.' Passing thence to speak of Casati's
papyrus, Dr.~Young remarks that it was the first in
which any intelligible characters of the enchorial form
had been discovered among the many manuscripts and
inscriptions which had been examined, and it `furnished
M.~Champollion with a name which materially
advanced the steps leading him to his very important
extension of the hieroglyphical alphabet. He had
mentioned to me, in conversation, the names of Apollonius,
Antiochus, and Antigonus, as occurring among
the witnesses; and I easily recognised the groups
which he had deciphered; although, instead of \emph{Antiochus},
I read Antimachus; and I did not recollect at
the time that he had omitted the m.'

Now comes the strange part of the story.

`In the evening of the day that Mr.~Grey had
brought me his manuscripts,' proceeds Dr.~Young
(whose English, by the way, is in places slightly
questionable), `I proceeded impatiently to examine
that which was in Greek only; and I could scarcely
believe that I was awake and in my sober senses,
when I observed among the names of the witnesses
\textit{Antimachus Antigenis} (\textit{sic}); and a few lines farther
back, \textit{Portis Apollonii}; although the last word could
not have been very easily deciphered without the
assistance of the conjecture, which immediately occurred
to me, that this manuscript might perhaps be a
translation of the enchorial manuscript of Casati. I
found that its beginning was, ``A copy of an Egyptian
writing''; and I proceeded to ascertain that there were
the same number of names intervening between the
Greek and the Egyptian signatures that I had identified,
and that the same number followed the last of
them. The whole number of witnesses was sixteen
in each.~.~.~. I could not therefore but conclude,'
proceeds Dr.~Young, after dwelling on other points
equally demonstrative of the identity of the Greek and
enchorial inscriptions, `that a most extraordinary
chance had brought into my possession a document
which was not very likely, in the first place, ever to
have existed, still less to have been preserved uninjured,
for my information, through a period of near
two thousand years; but that this very extraordinary
translation should have been brought safely to Europe,
to England, and to me, at the very moment when it
was most of all desirable to me to possess it, as the
illustration of an original which I was then studying,
but without any other reasonable hope of comprehending
it; this combination would, in other times,
have been considered as affording ample evidence of
my having become an Egyptian sorcerer.' The surprising
effect of the coincidence is increased when the
contents of this Egyptian manuscript are described.
`It relates to the sale, not of a house or a field, but of
a portion of the collections and offerings made from
time to time on account or for the benefit of a certain
number of mummies of persons described at length
in very bad Greek, with their children and all their
households.'

The history of astronomy has in quite recent times
afforded a very remarkable instance of repeated coincidences.
I refer to the researches by which the
theory has been established, that meteors and comets
are so far associated that meteor systems travel in the
tracks of comets. It will readily be seen from the
following statements, all of which may be implicitly
relied upon, that the demonstration of this theory must
be regarded as partly due to singular good fortune:

There are two very remarkable meteor systems---the
system which produces the November shooting-stars,
or \textit{Leonides}, and that which produces the August
shooting-stars, or \textit{Perseides}. It chanced that the year
1866 was the time when a great display of November
meteors was expected by astronomers. Hence, in the
years 1865 and 1866 considerable attention was
directed to the whole subject of shooting-stars. Moreover,
so many astronomers watched the display of
1866, that very exact information was for the first time
obtained as to the apparent track of these meteors. It
is necessary to mention that such information was
\emph{essential} to success in the main inquiry. Now it had
chanced that in 1862 a fine comet had been seen,
whose path approached the earth's path very closely
indeed. This led the Italian astronomer Schiaparelli
to inquire whether there might not be some connection
between this comet and the August shooting-stars,
which cross the earth's path at the same place. He
was able, by comparing the path of the comet and the
apparent paths of the meteors, to render this opinion
highly probable. Then came inquiries into the real
paths of the November meteors, these inquiries being
rendered just practicable by several coincidences, as---(1)
the exact observations just mentioned; (2) the
existence of certain old accounts of the meteor shower;
(3) the wonderful mastery obtained by Professor
Adams over all problems of perturbation (for the
whole question depended on the way in which the
November meteors had been perturbed); and (4) the
existence of a half-forgotten treatise by Gauss, supplying
formul{\ae} which reduced Adams' labour by one-half.
The path having been determined (by Adams
alone, I take this opportunity of
insisting),\footnote{Leverrier, Schiaparelli, and others calculated the path on the
assumption that the occurrence of displays three times per century
implies a periodic circulation around the sun in about thirty-three
years and a quarter; but Adams alone proved that this period, and
no other, must be that of the November meteors.}
the
whole question rested on the recognition of a comet
travelling in the same path. If such a comet were
found, Schiaparelli's case was made out. If not, then,
though the evidence might be convincing to mathematicians
well grounded in the theory of probabilities,
yet it was all but certain that Schiaparelli's theory
would presently sink into oblivion. Now there are
probably hundreds of comets which have a period of
thirty-three and a quarter years, but very few are
known---only three certainly---and one of these \textit{had
only just been discovered} when Adams' results were
announced. The odds were enormous against the
required comet being known, and yet greater against
its having been so well watched that its true path had
been ascertained. Yet the comet which had been discovered
in that very year 1866---the comet called
Tempel's, or I.~1866---was the very comet required to
establish Schiaparelli's theory. \textit{There} was the path of
the meteors assigned by Adams, and the path of the
comet had been already calculated by Tempel before
Adams' result had been announced; and these two
paths were found to be to all intents and purposes
(with an accuracy far exceeding indeed the requirements
of the case) \textit{identical}.

To the remarkable coincidences here noted, coincidences
rendered so much the more remarkable by
the fact that the August comet is now known to return
only twice in three centuries, while the November
comet returns only thrice per century, may be added
these:
The comet of 1862 was observed, telescopically, by
Sir John Herschel under remarkably favourable circumstances.
`It passed us closely and swiftly,' says
Herschel, `swelling into importance, and dying away
with unusual rapidity. The phenomena exhibited by
its nucleus and head were on this account peculiarly
interesting and instructive, \emph{it being only on very rare
occasions} that a comet can be closely inspected at the
very crisis of its fate, so that we can witness the actual
effect of the sun's rays on it.' (This was written long
before Schiaparelli's theory had attracted notice.) This
comet was also the last observed and studied by Sir
John Herschel. The November comet, again, was the
\textit{first comet ever analysed with the spectroscope}.

It will be remarked, perhaps, that where coincidences
so remarkable as these are seen to be possible, it may
be questionable whether the theory itself, which is
based on the coincidence of certain paths, can be
accepted as trustworthy. It is to be noticed that,
whether this be so or not, the surprising nature of the
coincidence is in no way affected; it would be as
remarkable (at least) that so many events should
concur to establish a false as to establish a true theory.
This noted, we may admit that in this case, as in many
others, the evidence for a scientific theory amounts in
reality only to extreme probability. However, it is to
be noticed that the probability for the theory belongs
to a higher \textit{order} than the probability against those
observed coincidences which rendered the demonstration
of the theory possible. The odds were thousands
to one, perhaps, against the occurrence of these coincidences:
but they are millions to one against the
coincidence of the paths as well of the November as of
the August meteors with the paths of known comets,
by mere accident.

It may possibly be considered that the circumstances
of the two last cases are not altogether such as to
assure us that special intervention was not in question
in each instance. Indeed, though astronomers have
not recognised anything supernatural in the series of
events which led to the recognition of the association
between meteors and comets, some students of arch{\ae}ology
have been disposed to regard the events narrated
by Dr.~Young as strictly providential dispensations.
`It seems to the reflective mind,' says the author of
the `Ruins of Sacred and Historic Lands,' `that the
appointed time had at length arrived when the secrets
of Egyptian history were at length to be revealed, and
to cast their reflective light on the darker pages of
sacred and profane history.~.~.~. The incident in the
labours of Dr.~Young seems so surprising that it might
be deemed providential, if not miraculous.' The same
will scarcely be thought of such events (and their
name is legion) as De Morgan has recorded; since it
requires a considerable stretch of imagination to conceive
that either the discovery of the name of a certain
editor, or the removal of De Morgan's difficulties
respecting the siege of Boston, was a \textit{nodus} worthy of
miraculous interposition.

For absolute triviality, however, combined with singularity
of coincidence, a circumstance which occurred
to me several years ago appears unsurpassable. I was
raising a tumbler in such a way that at the moment
it was a few inches above my mouth; but whether to
examine its substance against the light, or for what
particular purpose, has escaped my recollection. Be
that as it may, the tumbler slipped from my fingers
and fell so that the edge struck against one of my
lower teeth. The fall was just enough to have broken
the tumbler (at least, against a sharp object like a
tooth), and I expected to have my mouth unpleasantly
filled with glass fragments and perhaps seriously cut.
However, though there was a sharp blow, the glass remained
unbroken. On examining it, I found that a
large drop of wax had fallen on the edge at the very
spot where it had struck my tooth, an indentation being
left by the tooth. Doubtless the softening of the shock
by the interposition of the wax had just saved the glass
from fracture. In any case, however, the surprising
nature of the coincidence is not affected. On considering
the matter it will be seen how enormous were
the antecedent odds against the observed event. It is
not an usual thing for a tumbler to slip in such a
way: it has not at any other time happened to me,
and probably not a single reader of these lines can
recall such an occurrence either in his own experience
or that of others. Then it very seldom happens,
I suppose, that a drop of wax falls on the edge of a
tumbler and there remains unnoticed. That two
events so unusual should be coincident, and that the
very spot where the glass struck the tooth should be
the place where the wax had fallen, certainly seems
most surprising. In fact, it is only the utter triviality
of the whole occurrence which renders it credible; it is
just one of those events which no one would think of
inventing. Whether credible or not, it happened. As
De Morgan says of the coincidences he relates, so
can I say for the above (equally important) circumstance,
`I can solemnly vouch for its literal truth.'
Yet it would be preposterous to say that there was
anything providential in such an occurrence. Swift,
in his `Tale of a Tub,' has indicated in forcible terms
the absurdity of recognising miraculous interventions
in such cases; but should it appear to some of
my readers that, trivial though the event was, I
should have recognised the hand of Providence in it,
I would remark that it requires some degree of self-conceit
to regard oneself as the subject of the special
intervention of Providence, and moreover that Providence
might have contrived the escape in less complicated
sort by simply so arranging matters that the
glass had not fallen at all. So, at least, it appears to
me.

There arises, in certain cases, the question whether
coincidences may not appear so surprising as to justify
the assumption that they are due to a real though
undiscerned association between the coinciding events.
This, of course, is the very basis of the scientific
method; and it is well to notice how far this method
may sometimes be unsafe. If remarkable coincidences
can occur when there is no real connection---as we have
seen to be the case---caution must be required in
recognising coincidence as demonstrative of association.

The rule of science in all such cases is simply to
inquire whether there can possibly be any relation of
cause and effect in such cases. When a housemaid
says, for instance, that putting the poker across a fire
makes the fire burn up, the student of physical laws is
able at once to see that the supposed influence is antecedently
most improbable. Here in a grate are certain
more or less combustible materials, and certain quantities
of matter already burning; combustion is going
on, though indifferently; the air is nourishing this
slowly burning fire, but inefficiently; on the whole, it
seems likely that the fire will go out. In what way
shall I do any good if I stick a rod of iron from the
fender across the top bar? I thus add a certain quantity
of cold metal to the space across which the air has to
come to the fire. Do I increase the draught? On the
contrary, so far as I produce any effect at all on the
draught, I must diminish it. For the draught depends
in the main on the diminished density of the warmed
air in the neighbourhood of the fire, and the cold metal
must to some degree increase the density of this air by
cooling it. The effect may be very slight; but such as
it is, it is unfavourable. But I was once told by a correspondent
that whether theoretically the poker should
make the fire burn up or not, as a matter of fact it does.
Repeatedly he had tried the experiment, and after exhausting
in vain every art he possessed to make the
fire burn up, he found that the poker when put across
the top bar immediately, or almost immediately, produced
the desired result. Science is bound to listen to
evidence of this kind, for science deals with phenomena,
and even, when phenomena seem to point to something
which appears utterly incredible, science has to inquire
into the matter. Well, in this case, what are the facts?
Some one tells us that he has repeatedly tried in vain
to make a fire burn up, but when he put the poker
across it, the fire presently became clear and bright.
Multitudes of contrary cases might no doubt be cited,
but let us suppose that none could. Are we therefore
to infer that in these cases the poker drew the fire up?
A new law of nature would be indicated if this were
so; and a new law of nature is worth learning. But
when due inquiry is made, it appears that there is no
such law---as unfortunately we might have expected.
Our correspondent, who found that when he put the
poker across the fire it drew up, is unquestionably but
an unskilful fireman. He puts on coals, and pokes and
stirs the fire, unconscious of the fact that this is just
the way to put a fire out. When the fire is all but
hopelessly reduced by his unskilful measures, he puts
the poker across the top bar. According to old-fashioned
superstitions, he makes the sign of the cross
across the fire-place, and the fire, in which until now
there seemed to have been some evil spirit (that is
what people mean when they say `the devil's in the
fire'), is purified from the unclean presence and
begins to burn up. That would have been the old-fashioned
interpretation of the change; but science
takes another view of the matter. It sees reason to
believe that the change took place simply because
the disturbance to which the fire had before been
exposed was bad for it. Putting the poker across
the top bar meant letting the fire alone, and giving
it a chance to burn up.

Singularly enough, I had occasion, when the last
sentence was just finished, to leave my study. When
I came back, an hour later, I found that my fire, which
in the meantime must very nearly have gone out, had
been recoaled---and the housemaid, or whoever had
attended to it, had, after the fashion of her tribe, put the
poker across the top bar. The fire was not burning
very brightly---on the contrary, it seemed inclined to
go out. Yet, rashly daring, I put the poker down---from
scientific principles I object to seeing bright
metal smoked and dulled---and went on with my work,
intending, if the fire went out, to call some one in to
light it again. However, it so chanced that after the
poker was put down, the fire began to burn pretty
brightly, and as I write there is every promise of a
good fire. Am I to infer that taking the poker from
across the top bar made the fire burn up? Of course,
the real fact was, that when the fire seemed dull it was
really making steady progress, and whether I had
taken down the poker, or supplemented its salutary
action by putting another poker across the top bar,
would not have made one particle of difference.

That our domestic servants should consider the
poker across the top bar a specific for making a dull
fire burn up is very natural. Their manner of treating
fires is unscientific in the extreme. A Cambridge
Fellow, who knew very little about the fair sex,
except what he might gather from the ways of `bed-makers'
and his recollections, perhaps, of domestic
servants at home, used to define woman as `an inferior
animal, not understanding logic, and poking a fire
from the top.' Most servants do this. They also have
two utterly erroneous ideas about making up a low
fire: first, that the more fuel is put on the better;
secondly, that after putting coal on it is desirable to stir
the fire. As a matter of fact, when a fire is low, the
addition of fuel will often put it out altogether, and the
addition of much fuel is almost certain to do so; and in
every case the time to stir the fire (when low) is before
coals are put on, not after. Generally it is well, when
a fire is low, to stir it deftly, so as to bring together
the well-burning parts, and then to wait a little, till
they begin to glow more brightly; then a few coals
may be put on, and after awhile the fire may again be
stirred and some more coals put on it. When a low fire
has been unwisely treated by being coaled too freely,
and the fresh fuel uselessly stirred, it is generally the
case that the only chance for the fire is leaving it alone.
Susan does this when she puts the poker across the
top bar, and unconsciously she retains the old superstition
that, by thus making the sign of the cross over
the fire, she sends away the evil beings, sprites, or
whatever they may have been, which were extinguishing
it.

That letting the sun shine on a fire puts it out is
not, like the other (in its real origin, at any rate), a
superstition, but simply an illusion. A correspondent
wrote to me that it is believed in by nine persons out of
ten; but in this it is like all other wrong beliefs.
Scientific methods of inquiry and reasoning are followed
by fewer than ten in a hundred; and although nowadays
the views of science are accepted more widely than
in olden times, this is simply because science has shown
its power by material conquests.%
%
\footnote{I do not think that my friend Professor Tomlinson's experiments
on the burning of candles in sunlight and in the dark would be regarded
by all as decisively showing that sunlight does not interfere
with combustion, though, rightly apprehended, they go near to prove
this. But \textit{\`a priori} considerations show conclusively that though by
warming the air around a fire the sun's rays may, in some slight
degree (after a considerable time), affect the progress of combustion,
they cannot possibly put the fire out in the sense in which
they are commonly supposed to do so; in fact, a fire would probably
burn somewhat longer in a room well warmed by a summer
sun than in a room from which the solar rays were excluded. (The
difference would be very slight.)}

Not to take any more scientific instances, of which
perhaps I have already said enough, let us consider
the case of presentiments of death or misfortune.
Here, in the first place, the coincidences which have
been recorded are not so remarkable as might at first
sight appear, simply because such presentiments are
very common indeed. A certain not unusual condition
of health, the pressure of not uncommon difficulties or
dangers, depression arising from atmospheric and other
causes, many circumstances, in fact, may suggest (and
do notoriously suggest) such presentiments. That
some presentiments out of very many thus arising
should be fulfilled is not to be regarded as surprising---on
the contrary, the reverse would be very remarkable.
But again a presentiment may be founded on facts,
known to the person concerned, which may fully justify
the presentiment. `Sometimes,' says De Morgan on
this point, `there is no mystery to those who have
the clue.' He cites instances. `In the ``Gentleman's
Magazine'' (vol.~80, part 2, p.~33) we read, the subject
being presentiment of death, as follows:---``In 1718,
to come nearer the recollection of survivors, at the
taking of Pondicherry, Captain John Fletcher, Captain
De Morgan''\,' (De Morgan's grandfather) `\,``and
Lieutenant Bosanquet each distinctly foretold his own
death on the morning of his fate.'' I have no doubt of
all three; and I knew it of my grandfather long before
I read the above passage. He saw that the battery he
commanded was unduly exposed---I think by the sap
running through the fort when produced.%
%
\footnote{De Morgan writes somewhat inexactly here for a mathematician.
The sap did not run through the fort, but the direction of
the sap so ran.}
%
He represented
this to the engineer officers, and to the
commander-in-chief; the engineers denied the truth
of the statement, the commander believed them, my
grandfather quietly observed that he must make his
will, and the French fulfilled the prediction. His will
bore date the day of his death; and I always thought it
more remarkable than the fulfilment of his prophecy
that a soldier should not consider any danger short of
one like the above sufficient reason to make his will.
I suppose,' proceeds De Morgan, `the other officers
were similarly posted. I am told that military men
very often defer making their wills until just before an
action; but to face the ordinary risks intestate, and to
wait until speedy death must be the all but certain
consequence of a stupid mistake, is carrying the principle
very far.'

As to the fulfilment of dreams and omens, it is to be
noticed that many of the stories bearing on this
subject fail in showing that the dream was fully
described \emph{before} the event occurred which appeared to
fulfil the dream. It is not unlikely that if this had
been done, the fulfilment, in many cases, would not
have appeared quite so remarkable as in the actual
narrative. Without imputing untruth to the dreamer,
we may nevertheless---merely by considering what is
known as to ordinary testimony---believe that the
occurrences of the dream have been somewhat modified
after the event. I do not doubt that if every person
who had a dream leaving a strong impression on the
mind, were at once to record all the circumstances of
the dream, very striking instances of fulfilment would
occur before long; but at present, certainly, nine-tenths
of the remarkable stories about dreams fail in
the point I have referred to.

The great objection, however, to the theory that
certain dreams have been intended to foreshadow real
events, is the circumstance that the instances of fulfilment
are related, while the instances of non-fulfilment
are forgotten. It is known that instances of the latter
sort are very numerous, but what proportion they bear
to instances of the former sort, is unknown; and while
this is the case, it is impossible to form any sound
opinion on the subject, so far as actual evidence is
concerned. It must be remembered that in this case
we are not dealing with a theory which will be disposed
of if one undoubted negative instance be
adduced. It is very difficult to draw the line between
dreams of an impressive nature---such dreams as we
might conceive to be sent by way of warning---and
dreams not specially calculated to attract the dreamer's
attention. A dream which appeared impressive when
it occurred but was not fulfilled by the event, would
be readily regarded, even by the dreamer himself, as
not intended to convey any warning as to the future.
The only way to form a just opinion would be to
record each dream of an impressive nature, immediately
after its occurrence, and to compare the
number of cases in which such dreams are fulfilled
with the number in which there is no fulfilment. Let
us suppose that a certain class of dreams were selected
for this purpose. Thus, let a society be formed, every
member of which undertakes that whenever on the
night preceding a journey he dreams of misfortune on
the route, he will record his dream, with his ideas as
to its impressiveness, before starting on his journey.
A great number of such cases would soon be collected,
and we may be sure that there would be several
striking fulfilments, and probably two or three highly
remarkable cases of the sort; but for my own part, I
strongly entertain the opinion that the percentage of
fulfilments would correspond very closely with the
percentage due to the common risks of travelling, with
or without premonitory dreams. This could readily
be tested, if the members of the society agreed to note
every occasion on which they travelled: it would be
found, I suspect, that the dreamers gained little by
their warnings. Suppose, for instance, that ten thousand
journeys of all sorts were undertaken by the members
of the society in the course of ten years, and that a
hundred of these journeys (one per cent., that is) were
unfortunate; then, if one-tenth of the journeys (a
thousand in all) were preceded by warning dreams, I
conceive that about ten of these warnings (or one per
cent.) would be fulfilled. If more were fulfilled there
would appear, so far as the evidence went, to be a
balance of meaning in the warnings; if fewer, it would
appear that warning dreams were to some slight degree
to be interpreted by the rule of contraries; but if
about the proper average number of ill-omened voyages
turned out unfortunately, it would follow that warning
dreams had no significance or value whatever: and this
is precisely the result I should expect.

Similar reasoning, and perhaps a similar method,
might be applied to cases where the death of a person
has been seemingly communicated to a friend or
relative at a distance, whether in a dream or vision,
or in some other way at the very instant of its
occurrence. It is not, however, by any means so clear
that in such instances we may not have to deal with
phenomena admitting of physical interpretation. This
is suggested, in fact, by the application of considerations
resembling those which lead to the rejection of
the belief that dreams give warning against dangers.
Dreams of death may indeed be sufficiently common,
and but little stress could be laid, therefore, on the
fulfilment of several or even of many such dreams.
But visions of the absent are not common phenomena.
That state of the health which occasions the appearance
of visions is unusual; and if some of the stories
of death-warnings are to be believed, visions of the
absent have appeared to persons in good health. But
setting aside the question of health, visions are unusual
phenomena. Hence, if any considerable proportion of
those narratives be true, which relate how a person has
at the moment of his death appeared in a vision to
some friend at a distance, we must recognise the possibility,
at least, that under certain conditions mind may
act on mind independently of distance. The \textit{\`a priori}
objections to this belief are, indeed, very serious, but
\textit{\`a priori} reasoning does not amount to demonstration.
We do not \emph{know} that even when under ordinary
circumstances we think of an absent friend, his mind
may not respond in some degree to our thoughts, or
else that our thoughts may not be a response to
thoughts in his mind. It is certain that such a law of
thought might exist and remain undetected---it would
indeed be scarcely detectable. At any rate, we know
too little respecting the mind to be certain that no
such law exists. If it exists, then it is quite conceivable
that the action of the mind in the hour of
death might raise a vision in the mind of another.

I shall venture to quote here an old but well-authenticated
story, as given by Mr.~Owen in his
`Debatable Land between this World and the Next,'
leaving to my readers the inquiry whether probabilities
are more in favour of the theory that (1) the story is
untrue, or (2) the event related was only a remarkable
coincidence between a certain event and a certain
cerebral phenomenon, in reality no way associated with
it, or (3) that there was a real association physically
explicable, or (4) that the event was supernatural.
Lord Erskine related to Lady Morgan---herself a
perfect sceptic---(I wish, all the same, that the story
came direct from Erskine) the following personal
narrative:---`On arriving at Edinburgh one morning,
after a considerable absence from Scotland, he met in
the street his father's old butler, looking very pale and
wan. He asked him what brought him to Edinburgh.
The butler replied, ``To meet your honour, and solicit
your interference with my lord to recover a sum due
to me, which the steward at the last settlement did not
pay.'' Lord Erskine then told the butler to step with
him into a bookseller's shop close by, but on turning
round again he was not to be seen. Puzzled at this he
found out the man's wife, who lived in Edinburgh,
when he learnt for the first time that the butler was
dead, and that he had told his wife, on his death-bed,
that the steward had wronged him of some money, and
that when Master Tom returned he would see her
righted. This Lord Erskine promised to do, and
shortly afterwards kept his promise.' Lady Morgan
then says, `Either Lord Erskine did or did not believe
this strange story: if he did, what a strange aberration
of intellect! if he did not, what a stranger aberration
from truth! My opinion is that he \emph{did} believe it.'
Mr.~Owen deals with the hypothesis that aberration of
intellect was in question, and gives several excellent
reasons for rejecting that hypothesis; and he arrives
at the conclusion that the butler's phantom had really
appeared after his death. `The natural inference from
the facts, if they are admitted, is,' he says, `that under
certain circumstances, which as yet we may be unable
to define, those over whom the death-change has
passed, still interested in the concerns of earth, may
for a time at least retain the power of occasional
interference in these concerns; for example, in an
effort to right injustice done.' He thus adopts what,
for want of a better word, may be called the supernatural
interpretation. But it does not appear from
the narrative (assuming it to be true) that the butler
was dead at the moment when Erskine saw the vision
and heard the words. If this moment preceded the
moment of the butler's death, the story falls into the
category of those which seem explicable by the theory
of brain-waves. I express no opinion.

I had intended to pass to the consideration of those
appearances which have been regarded as ghosts
of departed persons, and to the study of some other
matters which either are or may be referred to coincidences
and superstitions. But my space is exhausted.
Perhaps I may hereafter have an opportunity of
returning to the subject---not to dogmatise upon it,
nor to undertake to explain away the difficulties which
surround it, but to indicate the considerations which,
as it appears to me, should be applied to the investigation
of such matters by those who wish to give a reason
for the belief that is in them.

At present I must be content with indicating the
general interpretation of coincidences which appear
very remarkable, but which nevertheless cannot be
reasonably referred to special interpositions of Providence.
The fact really is that occasions are continually
occurring where coincidences of the sort are \emph{possible},
though improbable. Now the improbability in any
particular case would be a reasonable ground for
expecting that in that case no coincidence would
occur. But the matter is reversed when a great multitude
of cases are in question. The probable result
then is that there \textit{will} be coincidences. This may
easily be illustrated by reference to a question of
ordinary probabilities. Suppose there is a lottery
with a thousand tickets and but one prize. Then it is
exceedingly unlikely that any particular ticket-holder
will obtain the prize---the odds are, in fact, 999 to 1
against him. But suppose he had one ticket in each
of a million different lotteries all giving the same
chance of success. Then it would not be surprising
for him to draw a prize; on the contrary, it would be
a most remarkable coincidence if he did not draw one.
The same event---the drawing of a prize---which in
one case must be regarded as highly improbable,
becomes in the other case highly probable. So it is
with coincidences which appear utterly improbable.
It would be a most wonderful thing if such coincidences
did not occur, and occur pretty frequently, in
the experience of every man, since the opportunities
for their occurrence enormously outnumber the chances
against the occurrence of any particular instance.

We may reason in like manner as to superstitions.
Or rather, it is to be noted that the coincidences on
which superstitions are commonly based are in many
instances not even remarkable. Misfortunes are not so
uncommon, for instance, that the occurrence of a disaster
of some sort after the spilling of salt at table can
be regarded as surprising. If three or four persons,
who are discussing the particular superstition relating
to salt-cellars, can cite instances of an apparent connection
between a misfortune and the contact of salt
with a table-cloth, the circumstance is in no sense to
be wondered at; it would be much more remarkable
if the contrary were the case. There is scarcely a
superstition of the commoner sort which is not in like
manner based, \textit{not} on some remarkable coincidence,
but on the occasional occurrence of quite common
coincidences. It may be said, indeed, of the facts on
which nearly all the vulgar superstitions have been
based, that it would have amounted to little less than
a miracle if such facts were not common in the
experience of every person. Any other superstitions
could be just as readily started, and be very quickly
supported by as convincing evidence. If I were to
announce to-morrow in all the papers and on every
wall that misfortune is sure to follow when any person
is ill-advised enough to pare a finger-nail between ten
and eleven o'clock on any Friday morning, that announcement
would be supported within a week by
evidence of the most striking kind. In less than a
month it would be an established superstition. If this
appears absurd and incredible, let the reader consider
merely the absurdity of ordinary superstitions. Take,
for instance, fortune-telling by means of cards. If our
police reports did not assure us that such vaticination is
believed in by many, would it be credible that reasoning
beings could hope to learn anything of the future from
the order in which a few pieces of painted paper
happened to fall when shuffled? Yet it is easy to see
why this or any way of telling fortunes is believed in.
Many persons believe in the predictions of fortune-tellers
for the seemingly excellent reason that such predictions
are repeatedly fulfilled. They do not notice that
(setting apart happy guesses based on known facts)
there would have been as many fulfilments if every prediction
had been precisely reversed. It is the same with
other common superstitions. Reverse them, and they
are as trustworthy as before. Let the superstition be
that to every one spilling salt at dinner some great
piece of good luck will occur before the day is over;
let seven years of good fortune be promised to the
person who breaks a mirror; and so on: these new
superstitions would be before long supported by as
good evidence as those now in existence; and they
would be worth as much---since neither would be
worth anything.

\chapter{Notes on Poker}

The existence and still more the flourishing condition
of such a game as poker, outside mere gambling-dens,
is one of the most portentous phenomena of American
civilisation, though it is not in this aspect that I propose
just now to consider it; for the art which chiefly
avails to help the gambler in playing this game is
nothing more nor less than that art of which the enemy
of man is proverbially said to be the father. Poker has
an advantage over whist in one respect. In whist skill
will do somewhat; but it will not avail to make good
cards yield to bad ones. In poker the case is otherwise.
A man shall have not a point in his hand; yet by sheer
bluffing---in other words, by lying---he shall cause such
an idea to be formed of his hand, that every one else at
the table will throw up his cards, and leave to the liar
full possession of the stakes. Yet, as Lawrence in `Guy
Livingstone,' and Hawley Smart in half a dozen novels,
describe with approval the success of daring swindles,
so the enthusiastic poker-player will tell you with pride
of achievements in bluffing which can only be viewed
in one way by men of honour---to wit, as barefaced
lying.

The game of poker is sufficiently simple, though, as
usual, the explanation given by those who play it is
obscure in the extreme. To every one in the circle five
cards are dealt in the usual way. The eldest hand---\textit{i.e.}\ the
player next the dealer on the left---stakes a sum,
which must be doubled by all who intend to stay in;
the eldest hand doubling his original stake if he decides
to stay in, otherwise forfeiting it. When this is done
all who stay in have staked an equal sum. Each
player may (in his regular turn only) increase his
stake, in which case all who wish to stay must `see'
him---that is, raise their stake in the same degree, or go
better---that is, raise the stake further. When all are
equally in, each of the players can throw out any of his
cards, and draw as many more, to improve his hand.
This done, the real business begins. In due rotation
the players left in raise the stake, or follow in `seeing'
it---that is, in bringing up their stakes to the increased
value. This may go on, and generally does go on, till
each has staked a large sum. If a sum is named which
a player is unwilling to `see,' he lays down his hand.
If all the other players are unwilling to `see' a bet,
they all throw down their hands, and the bettor takes
the pool without showing his hand. But when the bet
goes round to the last player remaining in, and he does
not wish to go better, he may simply `see it' and
`call'; on which all playing must show their hands,
and the best hand wins the pool.

On the rules which determine the value of the
several hands depend whatever qualities the game of
poker has as a game of skill. Just as in \textit{vingt-et-un},
hazard, and like games, there are certain rules of probability
which ought to guide the player (if he must
gamble), so also in poker there are rules, though they
very little affect the play of the average poker-player,
while the really skilled professors of this cheerful game
pay no attention to them whatever.

The points which give a hand value are the presence
of cards of the same denomination (as a \textit{pair}, or
two of the same denomination; \textit{triplets}, or three of a
kind; and \textit{fours}, or four of a kind); a \textit{sequence}---that is,
all the cards in the hand being in sequence, as 9, 10,
knave, queen, king; a \textit{flush}, or all the cards of the same
suit. The lowest kind of hand is one which has none
of these points; such a hand is estimated against others
of the same kind by the highest card in it (the value of
the cards being as in whist). Next in value is a hand
with one pair in it; next a hand with two pairs (different
pairs, of course); next a hand with three cards of
the same denomination, called `threes'; next a sequence
hand; next a flush hand; then a \textit{full} hand---that is, a
hand containing one pair and one triplet; then \textit{fours},
a hand containing four cards of the same denomination;
and, lastly, that is highest and best of all, a \textit{flush sequence}---that
is, a sequence of high cards all of the same
suit. In every case where two hands are of the same
kind, the cards of highest denomination in the pair,
triplet, four, flush, or sequence, wins. Thus a flush
sequence of knave, 10, 9, 8, 7, beats a flush sequence of
9, 8, 7, 6, 5; four aces beat four kings or four queens;
a full of three aces and two deuces beats a full of three
kings and two queens, but a full of three aces and two
threes beats a full of three aces and two deuces; a flush
of king, 7, 5, 3, 2, beats a flush of queen, knave,
10, 9, 7; and so on. In cases of `tie' the stakes are
divided.

It is clear that the game itself is as good as many
which are played in the domestic circle. In such a
game as \textit{vingt-et-un}, for instance, where the players are
all against the dealer, there is about the same element
of chance and about the same room for the exercise of
judgment that there is in a game of poker which is to end
with a call. But the bluffing element, which is what
gives the game its real value to the gambling fraternity,
is independent of any qualities possessed by poker as a
card game. Where there is no `limit' (that is, no
stated sum beyond which no bet must go), one can bluff
as well, and almost as safely, over a bad hand as over a
good one---if one possesses the requisite qualities of
a false face and a steady nerve.

But I wish just now to consider the qualities which
this game possesses as an exercise of the judgment. No
judgment is shown by one who sits down to gamble at
poker; but in the game itself there are points depending
a good deal on judgment, and especially on a knowledge
of the laws of chance. Here, oddly enough,
the professional poker-players have made, for the most
part, little progress. We have before us the reasoning
of one who claims to teach, calling his book `The Complete
Poker Player,' and we find not only much that is
incorrect in theory, but an absolute failure to understand
the real value of the principles of probability to
the poker proficient, and indeed to all who gamble. He
deliberately tells us, in fact, that while theory shows the
odds to be such and such, experience points to other
odds, the real fact being that experience and theory are
in most perfect accord in all matters of probabilities.

In the first place, the problems connected with the
decision, whether to stay in or retire on a given hand,
are very pretty. The case is entirely different from that
to be dealt with in such a game as \textit{vingt-et-un}, where
only the dealer has to be considered, each player being
as it were in contest with him. In poker a player has
to consider, not the chance of having a better hand than
some particular adversary, but the chance that he holds
better cards than \emph{any} of the others. This modifies the
chances in a very interesting manner. Not only are
they different from those existing where each player is
matched against the dealer, but they vary according to
the number of players. Where the players are few a
moderately good hand may be trusted to win against the
company, in the average of a great number of trials;
but where there are many players there is more chance
of a strong hand lying somewhere to beat it, and therefore,
the hand in which the player should decide to
trust must be a better one. For instance, with few
players a pokerist might safely decide that he would not
go in on less than a high pair, as kings or aces, and
adhering to that rule throughout the play would be
likely to come out without heavy loss. But if there
were a large party of players, the average best hand at
each deal would probably be better; and he might,
therefore, deem it well to put low threes, as three fours
or three fives, as the limit below which he would not
back his hand. Apart from `bluffing,' such rules are
not affected by the probability that a `call' may be
made; for the persistence of other players in raising will
depend on the quality of their hand.

But we touch here on a characteristic of this game
of poker, which makes it a really excellent game for
non-gamblers, because calling so largely on the exercise
of judgment, and also depending so much on individual
character. As a parlour game, with counters instead of
coin, it is one of the best and most amusing I know of.
It is strangely contrasted with whist, calling for the
exercise of very different mental faculties, but bringing
out traits of character in quite as marked a degree.

As a result of confidence in luck, either general or
at any particular time, poker-players often trust in
hands of far less value than such as would give a fair
chance of winning. It never seems to occur to them
that the possession of a bad hand should in itself be
regarded, if the theory of luck were sound, as an evidence
that at the moment they were not in the vein;
and that the principle `back your luck' would suggest
that the hand should be thrown up, for backing it means
backing bad luck.

Of course this does not apply to bluffing, which,
however, is not considered good poker-playing, at least
as a system. A player may bluff on almost any hand,
and the bolder his bluff the better his chance of winning;
for his opponent has to pay to see his hand---he
has, indeed, in a sense, not to pay but simply to stake
so much money; but, according to the true doctrine of
chances, staking means payment of a certain sum for a
certain chance. Now, when a poker-player raises the
stakes by a very large amount, he means, if he is not
bluffing, `I have a very good hand;' and it is not wise,
if that is the case, to pay a large sum for the privilege
of seeing how good his hand is, unless your own is so
good as to give you a very good chance of having the
better. Even then it is better to see and go better than
to call. For by so doing you have two chances to one---the
chance that, seeing you so confident, he will not
go on, and the chance that when the call is made you
will be found to have the better hand. Now, a bold
bluff often forces success---\textit{if the player is not given to
bluffing}. If he is, he is soon found out; and thereafter
he bluffs at his proper peril. Probably no bluffing
poker-player has ever been successful for any great
length of time. Even if he is so wealthy that he can
stand a few checks so far as his pocket is concerned, he
begins to lose nerve when a few large bluffs have been
met with a call and his pockets have suffered accordingly.
But the player who nine times out of ten plays the
straight game, may often win largely by an occasional
bluff---if he is ready to overlook the fact that a bluff is
a lie.

But the avoidance of bluffing takes away none of
the good qualities which poker has as a game of skill.
The player may still back his hand with more or less
boldness, according to its quality and his temperament.
He still requires to exercise judgment as to the actual
or relative value of a hand; he still has to note observantly
what is done by other players, what cards they
draw, what their ways are in standing on a hand, in
holding when advances are made by others, and so forth.

In actual play for money the use of a good limit
below which the player makes it a rule to stand out is
sound policy; for in the long run the player whose
lowest hand for backing is a strong one, as two aces,
or low threes at the least in small companies, and high
threes in large companies, must come off well. He will
win more than he loses. But it must be remembered
that constant caution is apt to diminish the profits of
successful ventures. The poker-player wants others to
play high when he has a winning hand, and if it
becomes known that he never backs any but strong
hands, none will `raise' very much against him. To
succeed in pocketing a large share of other people's
money, which is the true poker-player's object, the most
cautious player must indulge in an occasional extravagance.
So also with a very strong hand---one that is
practically sure to win---the judicious poker-player
must play a waiting game. He must reverse the tactics
of the bluffer, who tries to persuade others that his hand
is better than it really is; he must try to persuade the
rest that his hand is but a poor one; so will they see
and raise, see and raise, until there is something in the
pool worth winning, when he can see and raise more
boldly, and finally call or await the call with confidence.
(In fact, lying and lying in wait are the secrets of
success at poker.)

Let us consider briefly what are the chances for each
different kind of hand at poker.

First, the total number of ways in which a set of five
cards can be formed out of a pack containing 52 cards
has to be determined. This is easy enough. You multiply
together 52, 51, 50, 49, and 48, and divide the
product by that obtained from multiplying together 1,
2, 3, 4, and 5. You thus get 2,598,960 as the total
number of poker hands.

It is very easy to determine the number of flushes
and sequences and flush sequences which are possible.

Thus, begin with the flush sequences. We can
have in each suit, Ace, 2, 3, 4, 5; 2, 3, 4, 5, 6; 3, 4,
5, 6, 7; and so on up to 10, Knave, Queen, King, Ace;
or in all there are ten flush sequences in each suit, forty
flush sequences in all.

The number of sequences which are not flush may
be thus determined. The arrangement of numbers may
be any one of the ten just indicated. But taking any
one of these, as 3, 4, 5, 6, 7, the three may be of any
suit out of the four; so that each arrangement may be
obtained in four different ways as respects the first card;
so with the second, third, \&c.; or in all 4 times 4 times
4 times 4 times 4, or 1,024, four of which only will be
flushes. Thus there are 1,020 times 10, or 10,200
sequences which are not flush.

Now as respects flushes their number is very easily
determined. The number of combinations of five cards
which can be formed out of the 13 cards of a suit are
given by multiplying together 13, 12, 11, 10, and 9,
and dividing by the product of 1, 2, 3, 4, 5; this will
be found to be 1,287. Thus there are 4 times 1,287, or
5,148 possible flushes. Of these 5,108 are not sequence
flushes.

The total number of `four' hands may be considered
next. The process for finding it is very simple. There
are of course only 13 fours, each of which can be taken
with any one of the remaining 48 cards; so that there
are 13 times 48, or 624 possible four hands.

Next, to determine the number of `full hands.' This
is not difficult, but requires a little more attention. A
full hand consists of a triplet and a pair. Now manifestly
there are four triplets of each kind---four sets of
three aces, four of three kings, and so forth (for we may
take each ace from the four aces in succession, leaving
in each case a different triplet of aces; and so with the
other denominations). Thus, in all, 4 times 13, or 52
different triplets can be formed out of the pack of 52
cards. When one of these triplets has been formed
there remain 49 cards, out of which the total number of
sets of two which can be formed is obtained by multiplying
49 by 48 and dividing by two; whence we get
1,176 such combinations in all. But the total number
of pairs which can be formed from among these 49 cards
is much smaller. There are four twos, which (as cribbage
teaches us) will give six pairs of twos; so there
are six pairs of threes, six pairs of fours, and so on; or
as there are only twelve possible kinds of pairs (after
our triplet is removed) there are in all 6 times 12, that
is 72, possible pairs which can with the triplet form
a full hand. Hence, as there are 52 possible triplets,
the total number of full hands is 52 times 72, or
3,744.

The number of triplet hands which are not also
fours or fulls (for every four hand contains triplets)
follows at once from the above. There are 52 possible
triplets, each of which can be combined with 1,176
combinations of two cards out of the remaining 49,
giving in all 52 times 1,176, or 61,152 sets of five,
three at least of which are alike. But there are 624
four hands, each of which is not only a triplet hand but
will manifestly make four of the triplet hands our gross
reckoning includes (for from every four you can make
three triplets), and there are 3,744 full hands. These
(to wit 2,496 fours and 3,744 fulls, or 6,240 hands in
all) must be removed from our count, leaving 54,912
triplet hands (proper) in all.

This last result might have been obtained another
way, which (as I shall use it for counting pair hands) I
may as well indicate here. Taking any triplet of the
52 there remain 49 cards, one of which is of the same
denomination as the triplet. Removing this, there are
left 48 cards, out of which the number of sets of two
which can be formed is obtained by multiplying 48 by
47 and dividing by 2; it is therefore 1,128, and among
these 72 are pairs. There remain then 1,056 sets of
two, any one of which can be combined with each of
52 triplets to give a triplet hand pure and simple.
Thus, in all, there are 52 times 1,056 triplet hands, or
54,912, as before.

Next for double and single pairs.

From the whole pack of 52 cards we can form six
times 13 pairs; for 6 aces can be formed, 6 pairs of
deuces, 6 pairs of threes, and so forth. Thus there are
in all 78 different pairs. When we have taken out any
pair, there remain 50 cards. From these we must remove
the two cards of the same denomination, as either or
both of these must not appear in the hand to be formed.
There remain 48 cards, from which we can form 72
other pairs. Each of these can be taken with any one
of the 46 remaining cards, except with those two which
are of the same denomination, or with 44 in all, without
forming a triplet. Each of these combinations can
be taken with each of the 78 pairs, giving a two-pair
hand, only it is obvious that each two-pair hand will be
given twice by this arrangement. Thus the total
number of two-pair hands is half of 78 times 72 times
44; or there are 123,552 such hands in all.

Next, as to simple pairs. We get, as before, 78
different pairs. Each of these can be taken with any
set of three formed out of the 48 cards left when the
other 2 of the same denomination have been removed,
except the 72 times 44 (that is 3,168) pairs indicated
in dealing with the last case, and the 48 triplets which
can be formed out of these same 48 cards, or 3,216 sets
in all. Now the total number of sets of three cards
which can be formed out of 48 is given by multiplying
48 by 47 by 46, and dividing by the product of the
numbers 1, 2, and 3. It is found to be 17,296. We
diminish this by 3,216, getting 14,082, and find that
there are in all 78 times 14,082 or 1,098,240.

The hands which remain are those which are to be
estimated by the highest card in them; and their number
will of course be obtained by subtracting the sum
of the numbers already obtained from the total number
of possible hands. We thus obtain the number
1,302,540.

Thus of the four best classes of hands, there are
the following
numbers:\footnote{It is easy to test the accuracy of the whole series of
calculations by determining independently how many hands there are which
do not belong to the first eight classes. Thus, as all the cards of the
five are of different denominations, we first take the combinations
of the thirteen card names five together. These (as in dealing with
common flushes above) are 1,287 in number. But, as in dealing
with common sequences, we must multiply these by 4 times 4 times
4 times 4 times 4, or by 1,024, getting 1,317,888. Subtracting thence
the flushes and sequences, 15,348 in all, we get 1,302,540 as the total
number of common hands (not containing pairs or the like)---as
above.}

\begin{tabular}{c@{\ }lr}
  Of & flush sequences there may be \qquad\qquad   &        40  \\
  '' & fours                                       &       624  \\
  '' & full hands                                  &     3,744  \\
  '' & common flushes                              &     5,108  \\
  '' & common sequences                            &    10,200  \\
  '' & triplets                                    &    54,912  \\
  '' & two pairs                                   &   123,552  \\
  '' & pairs                                       & 1,098,240  \\
  '' & other hands                                 & 1,302,540  \\
\cline{3-3}
     & Total number of possible hands \qquad\qquad & 2,598,960
\end{tabular}

It will be seen that those who devised the rules for
poker play set the different hands in their proper order.
It is fitting, for instance, that as there are only 40
possible flush sequence hands, out of a total number of
2,598,960 hands, while there are 624 `four' hands, the
flush sequences should come first, and so with the rest.
It is noteworthy, however, that when sequences were
not counted, as was the rule in former times, there was
one hand absolutely unique and unconquerable. The
holder of four aces then wagered on a certainty, for no
one else could hold that hand. At present there is no
absolutely sure winning hand. The holder of ace, king,
queen, knave, ten, flush, \emph{may} (though it is of course exceedingly
unlikely) be met by the holder of the same
cards, flush, in another suit. Or, when we remember
that at whist it \emph{has} happened that the deal divided the
four suits among the four players, to each a complete
suit, we see that four players at poker \emph{might} each receive
a flush sequence headed by the ace. Thus the use
of sequences has saved poker-players from the possible
risk of having either to stand out or wager on a certainty,
which last would of course be very painful to the
feelings of a professional gambler.

We might subdivide the hands above classified into a
much longer array, beginning thus:---4 flush sequences
headed by ace; 4 headed by king, and so on down to 4
headed by five; 48 possible four-aces hands; 48 four-kings
hands; and so on down to 48 four-twos hands;
24 possible `fulls' of 3 aces and 2 kings; as many of 3
aces and 2 queens; and so on down to 24 `fulls' of 3
twos and 2 threes; and so on. Any one who cares to do
this can, by drawing the line at any hand, ascertain at
once the number of hands above and not above that
hand in value: and thus determine the chance that any
hand taken at random is above or below that particular
hand in value. The comparatively simple table above
only shows how many hands there are above or not
above pairs, triplets, and the like. But the more complete
series could be very easily formed.

We note from the above table that more than half the
possible poker hands are below pairs in value. So that
Clay was right enough in wagering on an ace-high
hand, seeing that there are more hands which will not
beat it (supposing the highest next card a king, at any
rate) than there are hands that will; but he was quite
wrong in calling on such a hand, even against a single
opponent.

The effect of increase in the number of hands can
also readily be determined. Many even among gamblers
know so little of the doctrine of chances as not to
be aware of, still less to be able to measure the effect of,
the presence of a great number of other contestants.
Yet it is easy to illustrate the matter.

Thus, suppose a player casts a die single against one
other. If the first has cast four the odds are in favour
of his not being beaten; for there are only two casts
which will beat him and four which will not. The
chance that he will not be beaten by a single opponent
is thus $\frac{4}{6}$ths or $\frac{2}{3}$. If there is another opponent, the
chance that he individually will not cast better than 4,
is also $\frac{2}{3}$. But the chance that neither will throw better
than 4 is obtained by multiplying $\frac{2}{3}$ by $\frac{2}{3}$. It is therefore
$\frac{4}{9}$; or the odds are 5 to 4 in favour of one or other
beating the cast of the first thrower. If there are three
others, in like manner the chance that not one of the
three will throw better than 4 is obtained by multiplying
$\frac{2}{3}$ by $\frac{2}{3}$ by $\frac{2}{3}$. It is therefore $\frac{8}{27}$; or the odds are 19
to 8 in favour of the first thrower's cast of 4 being
beaten. And so with every increase in the number of
other throwers, the chance of the first thrower's cast
being beaten is increased. So that if the first thrower
casts 4, and is offered his share of the stakes before the
next throw is made, the offer is a bad one if there is
but one opponent, a good one if there are two, and a
very good one if there are more than two.

In like manner, the same hand which it would be
safe to stand on (as a rule) at poker against two or
three opponents, may be a very unsafe hand to stand
on against five or six.

Then the player has to consider the pretty chance-problems
involved in drawing.

Suppose, for instance, your original hand contains
a pair---the other three cards being all unlike: should
you stand out? or should you draw? (to purchase
right to which you must stand in); or should you
stand in without drawing? Again, if you draw,
how many of the three loose cards should you throw
out? and what are your chances of improving your
hand?

Here you have to consider first whether you will
stand in, which depends, not on the value of your pair
only, but also on the chance that your hand will be
improved by drawing. Having decided to stand in,
remember that discarding three tells the rest of the
company that in all possibility you are drawing to improve
a pair hand; and at poker, telling anything helps
the enemy. If one of your loose cards is an ace, you
do well to discard only the other two; for this looks
like drawing to a triplet, and you may chance to draw a
pair to your ace. But usually you have so much better
a chance of improving your hand by drawing three that
it is, as a rule, better to do this.

Drawing to a triplet is usually good policy. `Your
mathematical expectation of improvement is slight,'
says `The Complete Poker Player,' `being 1 to 23 of
a fourth card' (it should be \emph{the} fourth card) `of the
same denomination, and 2 to 28 of another pair of
denomination different from the triplet,' a remark suggesting
the comment that to obtain a pair of the same
denomination as the triplet would require play something
like what we hear of in old Mississippi stories,
where a `straight flush' would be met by a very full
pair of hands, to wit, five in one hand and a revolver in
the other! The total expectation of improvement is 1
to 8; but then see what an impression you make by a
draw which means a good hand. Then, too, you may
suggest a yet better hand, without much impairing your
chance of improvement, by drawing one card only.
This gives you one chance in 47 of making fours, and
1 in 16 of picking up one of the three cards of the same
denomination as the odd cards you retain. This is a
chance of 1 in 12.

`Draws to straights and flushes are usually dearly
purchased,' says our oracle; `always so at a small
table. Their value increases directly as the number of
players.' (The word `directly' is here incorrectly used;
the value increases as the number of players, but not
\textit{directly} as the number.) Of course in drawing to a
two-ended straight, that is one which does not begin or
end with an ace, the chance of success is represented
by 8 in 47, for there are 47 cards outside your original
hand of which only eight are good to complete the
straight. For a one-end straight the chance is but 4 in
47: with a small chance, too, of improving your hand,
you are trying for a hand better than you want in any
but a large company. `If you play in a large party,' says
`The Complete Poker Player,' `say seven or eight, and
find occasion to draw for a straight against six players,
do so by all means, even if you split aces.' The advice
is sound. Under the circumstances you need a better
hand than ace-pair to give you your fair sixth share of
the chances.

As to flushes your chances are better, when you
have already four of a suit. You discard one, and out
of the remaining 47 cards any one of nine will make
your flush for you. Your chance then is 1 in $5\frac{2}{9}$. In
dealing with this point our oracle goes altogether wrong,
and adopts a principle so inconsistent with the doctrine
of probabilities as to show that, though he knows much
more than Steinmetz, he still labours under somewhat
similar illusions. `Theoretically,' says he, `the result
just obtained is absolutely true; but I have experimented
with six hands through a succession of 500
deals, and filled only 83 flushes in the 500, equal to one
in six and one-twentieth draws. Of course I am not
prepared to say that this would be the average in many
thousand deals; theoretically it is an untrue result; but
I here suggest a \emph{possible} explanation of what I confess
is to me a mystery.' Then he expounds the very matter
on which we touched above. `In casting dice,' he says,
`\textit{theoretically}, any given throw has no influence upon
the next throw, and is not influenced by the previous
throw. Yet if you throw a die and it turns up six,
while the chances are \textit{theoretically} one to six' (one in
six it should be) `that the next throw will produce a six
because the previous throw of six lies absolutely in the
past, yet you may safely bet something more than the
usual odds against it. Then suppose the second throw
turns up a six, that throw also now lies in the past, and
cannot be proved to have an influence upon throw
number three, which you are preparing to make. If
any \textit{material} influence is suspected you may change the
box and die; and you may now bet twice the usual
odds against the six. Why? Because you know by
experience that it is extremely difficult to throw six
three times in succession, even if you do not know the
precise odds against it. Granted certain odds against
throwing six twice in succession, \&c., yet at any
given moment when the player shakes the box in which
is a six-faced die, he has one chance in six of throwing
a six; and yet if he has just thrown sixes twice, you
may bet twelve to one that he will not throw a six in
that particular cast.' If I did not hold gambling to be
near akin to swindling, and could find but a few
hundred who held this doctrine, how much money
might I not gain by accepting any number of wagers
of this wise sort!

The fact is, the mistake here is just the ridiculous
mistake which Steinmetz called `the maturity of the
chances,' over again. It is a mistake which has misled
to their ruin many thousands of gamblers, who might
have escaped the evil influence of that other equally
foolish mistake about being lucky or unlucky, in the
vein or out of it. Steinmetz puts the matter thus:---`In
a game of chance, the oftener the same combination has
occurred in succession, the nearer are we to the certainty
that it will not recur at the next cast or turn up: this
is the most elementary of the theories on probabilities;
it is termed the maturity of the chances.' The real fact
being that this is not a theory of probabilities at all,
but disproved by the theory of probabilities---and disproved,
whenever it has been put to the test, by facts.

Take the case considered in `The Complete Poker
Player,' and note the evidence on the strength of which
the author of that work rejects the theory in favour of
a practical common-sense notion (as he thinks), which
is, in reality, nonsense. You may expect 9 successful
draws to a flush in 47 hands; therefore, in the 500 deals
he experimented upon, he might have expected 95 or 96;
and he only obtained 83. Now 500 trials are far too
few to test such a matter as this. You can hardly test
even the tossing of a coin properly by fewer than a
thousand trials; and in that case there are but 2 possible
events. Here there are 47, of which 9 are favourable.
It was the failure to recognise this which led the
Astronomer-Royal for Scotland to recognise something
mystical and significant in the preponderance of 3's
and the deficiency of 7's among the digits representing
the proportion of the circumference to the diameter of
a circle. In casting a coin a great number of times, we
do not find that the occurrence of a great number of
successive heads or tails in any way affects the average
proportion of heads or tails coming next after the
series. Thus I have before me the record of a series of
16,317 tossings, in which the number of sequences of
tails (only) were rendered; and I find that after 271
cases in which tails had been tossed 5 times in succession,
the next tossing gave in 132 cases heads, and
in 139 cases tails. Among the 16,317 tossings, two
cases occurred in which tail was tossed 15 times in
succession.

\chapter[Martingales][Martingales]{Martingales; or, Sure(?) Gambling Systems}

In previous pages I have considered, under the head of
`Gamblers' Fallacies,' certain plans by which some
fondly imagine that fortune may be forced. I have
shown how illusory the schemes really are which at
first view appear so promising. There are other plans
the fallacy in which cannot be quite so readily seen,
though in reality unmistakable, when once the conditions
of the problem are duly considered.

Let me in the first place briefly run through the
reasoning relating to one of the simpler methods already
considered at length.

The simplest method for winning constantly at any
such game as \textit{rouge-et-noir} is as follows:---The player
stakes the sum which he desires to win, say 1\textit{l}. Either
he wins or loses. If he wins he again stakes 1\textit{l}.,
having already gained one. If, however, he loses, he
stakes 2\textit{l}. If this time he wins, he gains a balance
of 1\textit{l}., and begins again, staking 1\textit{l}., having already
won 1\textit{l}. If, however, he loses the stake of 2\textit{l}., or 3\textit{l}. in
all (for 1\textit{l}. was lost at the first trial), he stakes 4\textit{l}. If
he wins at this third trial, he is 1\textit{l}. to the good, and
begins again, staking 1\textit{l}. after having already won 1\textit{l}.
If, however, he loses, he stakes 8\textit{l}. It will readily be
seen that by going on in this way the player always
wins 1\textit{l}. when at last the right colour appears. He then,
in every case, puts by the 1\textit{l}. gained and begins again.

It seems then at first as though all the player has
to do is to keep on patiently in this way, starting always
with some small sum which he desires to win at
each trial, doubling the stake after each loss, when he
pockets the amount of his first stake and begins again.
At each trial the same sum seems certainly to be
gained, for he cannot go on losing for ever. So that he
may keep on adding pound to pound, \textit{ad infinitum}, or
until the `bank' tires of the losing game.

The fallacy consists in the assumption that he cannot
always lose. It is true that theoretically a time
must always come when the right colour wins. But
the player has to keep on doubling his stake practically,
not theoretically; and the right colour may not appear
till his pockets are cleared. Theoretically, too, it is
certain that be the sum at his command ever so large,
and the stake the bank allows ever so great, the player
will be ruined at last at this game, if---which is always
the case---the sum at the command of the bank is very
much larger. It would be so even if the bank allowed
itself no advantage in the game, whereas we know that
there is a certain seemingly small, but in reality decisive,
advantage in favour of the bank at every trial.
Apart from this, however, the longest pocket is bound
to win in the long run, at the game of speculation
which I have described. For, though it seems a tolerably
sure game, it is in reality purely speculative. At
every trial there is an enormous probability in favour of
the player winning a certain insignificant sum; but,
\textit{per contra}, there is a certain small probability that he
will lose, not a small sum, or even a large sum, but all
that he possesses---supposing, that is, that he continues
the game with steady courage up to that final doubling
which closes his gambling career, and also supposing
that the bank allows the doubling to continue far
enough; if the bank does not, then the last sum staked
within the bank limit is the amount lost by the player,
and, though he may not be absolutely ruined, he loses
at one fell swoop a sum very much larger than that
insignificant amount which is all he can win at each
trial.

Although this gambling superstition has misled
many, yet after all it is easily shown to be a fallacy.
It is too simple to mislead any reasonable person long.
And indeed, when it has been tried, we find that the
unfortunate victim of the delusion very soon wakes to
the fact that his stakes increase dangerously fast.
When it conies to the fifth or sixth doubling, he is
apt to lose heart, fearing that the luck which has gone
against him five times in succession may go against him
five times more, which would mean that the stake
already multiplied 32 times would be increased, not 32
times, but 32 times 32 times, or 1,024 times, which
would either mean ruin or a sudden foreclosure on the
bank's part and the collapse of the system.
For the benefit of those who too readily see through
a simple scheme such as this, gamblers have invented
other devices for their own or others' destruction, devices
in which the fallacy underlying all such plans is
so carefully hidden that it cannot very readily be detected.

The following is a martingale (as gamblers call
these devices for preventing fortune from rearing against
them) which has misled many:

The gambler\footnote{The account of the system here considered appeared in the
\textit{Cornhill Magazine} under the heading `A San Carlo Superstition,'
and was in that place described as `a pretty little martingale'
recently submitted to me by a correspondent of \textit{Knowledge}.}
first decides on the amount which he
is to win at each venture---if that can be called a venture
which according to his scheme is to be regarded as
an absolute certainty. Let us say that the sum to be
won is 10\textit{l}. He divides this up into any convenient
number of parts, say three; and say that the three
sums making up 10\textit{l}.\ are 3\textit{l}., 3\textit{l}., and 4\textit{l}. Then he prepares
a card on the annexed plan (fig.~1), where \textsc{w}
stands for winnings, \textsc{l} for losses, and \textsc{m}
(for martingale) heads the working
column which guides the gambler in his
successive ventures.

The first part of the play is light and
fanciful: the player---whom we will call
A---stakes any small sums he pleases
until he loses, making no account of any
winnings which may precede his first loss. This first
loss starts his actual operations. Say the first loss
amounts to 2\textit{l}.: A enters this sum in the third column
(see fig.~2) as a loss, and also in the second under the
cross-line. He then stakes the sum of this number, 2,
which is now the lowest in column \textsc{m}, and 3, the uppermost---that
\begin{wrapfigure}[8]{l}{1.0in}
    \begin{tabular}{|c|r|c|}
        \hline
        \textsc{w} & \multicolumn{1}{c|}{\textsc{m}} & \textsc{l} \\
        \hline
        \, & \pounds 3 & \, \\
        \, & 3         & \, \\
        \, & 4         & \, \\
        \cline{2-2}
        \, & \,        & \, \\
        \hline
    \end{tabular}
    \begin{center}\textsc{Fig.~1}\end{center}
\end{wrapfigure}
is, he stakes 5\textit{l}. If he loses, he enters the
lost 5\textit{l}.\ in columns \textsc{m} and \textsc{l}; and next stakes 8\textit{l}., the
sum of the top and bottom figures (3\textit{l}.\ and 5\textit{l}.) in column
\textsc{m}. He goes on thus till he wins, when he enters under
the head \textsc{w} the amount he has won, and scores out in
column \textsc{m} the top and bottom figures---viz., the 3\textit{l}.\ (at
the top), and the last loss (at the bottom). This process
is to be continued, the last stake, if it be lost, being
always scored at the bottom of column \textsc{m}, as well as
in the loss column, the last win being always followed
by the scoring out of the top and bottom remaining
numbers in column \textsc{m}. When this process has continued
until all the numbers in column \textsc{m} are scored out, A will
be found to have won 10\textit{l}.; and whatever the sum he had
set himself to win in the first instance, so long as it lies
well within the tolerably wide limits allowed by the
bank, A will always win just this sum in each operation.

Let us take a few illustrative cases, for in these
matters an abstract description can never be so clear as
the account of some actual case.

\begin{wrapfigure}[15]{r}{1.3in}
    \begin{tabular}{|c|r|c|}
        \hline
        \textsc{w} & \multicolumn{1}{c|}{\textsc{m}} & \textsc{l} \\
        \hline
        \, & \pounds 3 & \, \\
        \, & 3         & \, \\
        \, & 4         & \, \\
        \cline{2-2}
        \, & 2         & \pounds 2 \\
        \, & 5         & 5 \\
        \pounds 8 & 5  & 5 \\
        11 & 8         & 8 \\
         9 & 2         & 2 \\
         4 & \,        & \, \\
        \hline
        \pounds 32 & \, & \pounds 22 \\
        \hline
    \end{tabular}
\begin{center}\textsc{Fig.~2}\end{center}
\end{wrapfigure}
Consider, then, the accompanying account by A of
one of these little operations. The amount which A sets
out to win is, as before, 10\textit{l}. He divides this up into
three parts---3\textit{l}., 3\textit{l}., and 4\textit{l}. He starts with a loss of
2\textit{l}., which he sets in columns \textsc{m} and \textsc{l}. He stakes next
5\textit{l}.\ and loses, setting down 5\textit{l}.\ in columns \textsc{m} and \textsc{l}. He
stakes 8\textit{l}., the sum of the top and bottom numbers in
column \textsc{m}, and wins. He therefore sets 8\textit{l}. under \textsc{w},
and scores out 3\textit{l}. and 5\textit{l}., the top
and bottom numbers in column \textsc{m}.
(The reader should here score out
these numbers in pencil.) The top
and bottom numbers now remaining
are 3\textit{l}. and 2\textit{l}. Therefore A stakes
now 5\textit{l}. Say he loses. He therefore
sets down 5\textit{l}. both in column \textsc{m}
and column \textsc{L}, and stakes 8\textit{l}., the
sum of the top and bottom numbers
under \textsc{m}. Say he loses again. He
therefore puts down 8\textit{l}. under columns \textsc{m} and \textsc{l}, and
stakes 11\textit{l}., the sum of the top and bottom numbers
under \textsc{m}. Say he wins. He puts down 11\textit{l}. under \textsc{w},
and scores out the 3\textit{l}. left at the top and the 8\textit{l}. left at
the bottom of the column under \textsc{m}. (This the reader
should do in pencil.) He then stakes 9\textit{l}., the sum of
the top and bottom numbers (4\textit{l}. and 5\textit{l}. respectively)
left under \textsc{m}. Say he wins again. He then puts down
9\textit{l}. under \textsc{w}, and scores out the 4\textit{l}. left at the top and
the 5\textit{l}. left at the bottom of the column under \textsc{m}. There
now remains only one number under \textsc{m}, namely, 2\textit{l}., and
therefore A stakes 2\textit{l}. Let us suppose that he loses.
He puts down 2\textit{l}. under \textsc{m} and \textsc{l}, and, following the
simple rule, stakes 4\textit{l}. Say he wins. He then puts
down 4\textit{l}. under \textsc{w}, and scores out 2\textit{l}. and 2\textit{l}., the only
two remaining numbers under \textsc{m}. A, therefore, now
closes his little account, finding himself the winner of
8\textit{l}., 11\textit{l}., 9\textit{l}.,
and 4\textit{l}., or 32\textit{l}. in all, and the loser of 2\textit{l}.,
5\textit{l}., 5\textit{l}., 8\textit{l}., and 2\textit{l}., or 22\textit{l}. in all, the balance in his
favour being 10\textit{l}., the sum he set forth to win.

It seems obvious that the repetition of such a process
as this, any convenient number of times at each
sitting, must result in putting into A's pocket a considerable
number of the sums of money dealt with at
each trial. In fact, it seems at a first view that here is
a means of obtaining untold wealth, or at least of ruining
any number of gambling-banks.

Again, at a first view, this method seems in all
respects an immense improvement on the simpler one.
For whereas in the latter only a small sum can be
gained at each trial, while the sum staked increases
after each failure in geometrical progression, in this
second method (though it is equally a gambling superstition)
a large sum may be gained at each trial, and
the stakes only increase in arithmetical progression in
each series of failures.

The comparison between the two plans comes out
best when we take the sum to be won undivided, when
also the system is simpler; and, further, the fallacy
which underlies this, like \emph{every} system for gaining
money with certainty, is more readily detected, when
we consider it thus.

Take, then, the sum of 10\textit{l}., and suppose 5\textit{l}. the
first loss, after which take two losses, one gain, one
loss, and two gains. The table will be drawn up then
as shown---with the balance of 10\textit{l}., according to the
fatal success of this system.

On the other hand, take the other and simpler
method, where we double the original stake after each
failure. Then supposing the losses
and gains to follow in the same
succession as in the case just considered,
note that the first gain
closes the cycle. The table has
the following simple form (counting
three losses to begin with):

\begin{wrapfigure}[12]{l}{1.4in}
\begin{tabular}{|c|r|c|}
    \hline
    \textsc{w} & \multicolumn{1}{c|}{\textsc{m}} & \textsc{l} \\ \hline
    \, & \pounds 10 & \, \\
    \cline{2-2}
    \, & 5          & \pounds 5 \\
    \, & 15         & 15 \\
    \, & 25         & 25 \\
    \, & 20         & 20 \\
\pounds 35 & 20     & 20 \\
    25 & \,         & \, \\
    15 & \,         & \, \\
    \hline
\pounds 75 & \,     & \pounds 65 \\
    \hline
\end{tabular}
\end{wrapfigure}
We see then at once the advantage
in the simpler plan which
counterbalances the chief disadvantage
mentioned above. This disadvantage, the rapid
increase of the sum staked, is undoubtedly serious;
but, on the other hand, there is the important
advantage that at the first success
the sum originally staked is won;
whereas, according to the other plan,
every failure puts a step between the
player and final success. It can readily
be shown that this disadvantage in the
less simple plan just balances the disadvantage in the
simpler plan.

But now let us more particularly consider the probabilities
for and against the player involved in the
plan we are dealing with.

Note in the first place that the player works down
the column under \textsc{m} from the top and bottom, taking
off two figures at each success, and each figure adding
one figure at the bottom after each failure. To get
then the number of figures scored out we must double
the number of successes; to get the number added we
take simply the number of failures, and the total number
of sums under \textsc{m} is therefore the original number
set under \textsc{m}, increased by the number of failures. He
will therefore wipe out, as it were, the whole column,
so soon as twice the number of successes either equals
or exceeds by one the number of failures (including the
first which starts the cycle). Manifestly the former
sum will equal the latter, when the last win removes
two numbers under \textsc{m}, and will exceed the latter by
one when the last win removes only one number
under \textsc{m}.

\begin{wrapfigure}[7]{l}{1.0in}
\begin{tabular}{|c|c|}
  \hline
  \textsc{w} & \textsc{l} \\
  \hline
  \,         & \pounds 10 \\
  \,         & 20         \\
  \pounds 80 & 40         \\
  \hline
  \pounds 80 & \pounds 70 \\
  \hline
\end{tabular}
\end{wrapfigure}
Underlying, then, the belief that this method is
a certain way of increasing the gambler's store, there
is the assumption that in the long run twice the number
of successes will equal the number of failures, together
with the number of sums originally placed under
\textsc{m}, or with this number increased by unity. And this
belief is sound; for according to the doctrine of probabilities,
the number of successes---if the chances are
originally equal---will in the long run differ from the
number of failures by a number which, though it may
perchance be great in itself, will certainly be very small
compared with the total number of trials. So that
twice the number of successes will differ very little
relatively from \emph{twice} the number of failures, when both
numbers are large; and all that is required for our
gambler's success is that twice the number of successes
should equal \emph{once} the number of failures, together with
a \emph{small} number, viz.\ the number of sums originally set
under \textsc{m}, or this number increased by unity. So that
we may say the gambler is practically certain to win in
the long run in any given trial.

In this respect the method we are now considering
resembles the gambling superstition before examined.
In that case also the gambler is sure to win in the long
run, as he requires but a single success to wipe out the
losses resulting from any number of failures. He is in
that case sure to succeed very much sooner (on the
average of a great number of trials) than in the latter.

But we remember that even in that case where
success seems so assured, and where success in the long
run---\textit{granting the long run}---is absolutely certain, the
system steadily followed out means not success but ruin.
No matter what the limit which the bank rules may
assign to the increase of the stakes, so long as there \emph{is}
a limit, and so long as the bank has a practically limitless
control of money as compared with the player, he
must eventually lose all that he possesses.

Hence we cannot assume that, because the method
we are considering insures success in the long run, the
gambler can win to any extent when the long run is
not assured to him. Here lies the fallacy in this, as in
all other methods, of binding fortune to the gambler's
wheel. The player finds that he must win in the long
run, and he never stops to inquire what run is actually
allowed him. It may be a short run, or a fair run, or
even a tolerably long run; but the question for him is,
will it be long enough? And note that it is not only
the limitation which the bank may assign to the
stakes which we have to consider: the gambler's possessions
assign a limit, even though the bank may assign
none.

Let us see, then, what prospect there is that in this,
as in the other case, a run of bad luck may ruin the player---or
rather, let us see whether it be the case that in
this, as in the other system, patient perseverance in
the system may not mean certain ruin, which ruin may
indeed arrive at the very beginning of the confident
gambler's career.

Instead of all but certainty of success in each single
trial which exists in the simpler case, there is in the
case we are considering but a high degree of probability.
It is very much more likely than not that in a
given trial the gambler will clear the stake which he
has set himself to win. (This is why we so often hear
strong expressions of faith in these systems: again and
again we are told with open-mouthed expressions of
wonder that a system of this sort must be infallible,
because, says the narrator, I saw it tried over and over
again, and always with success.) Granted that it is so;
indeed, it would be a poor system which did not give
the gambler an excellent chance of winning a small
stake, in return for the risk, by no means evanescent,
that he may lose a very large one.

Observe, now, how the chances for and against are
balanced between the two systems. Suppose such
a run of ill-luck as in the simpler system would
mean absolute defeat, because of the rapid increase (by
doubling) of the sum staked by the gambler. Say, for
instance, a bank allows no stake to exceed 1,000\textit{l}., so
that ten doublings of a stake of 1\textit{l}., raising the stake to
1,024\textit{l}., would compel the gambler to stop, and leave
him with all his accumulated losses, amounting to
1,023\textit{l}. Now, take the case of a gambler trying the
other system for a gain of 10\textit{l}., divided into three sums,
3\textit{l}., 3\textit{l}., and 4\textit{l}.\ under column \textsc{m}, and suppose that after
winning a number of times he unfortunately starts ten
defeats in succession, his first loss having been 3\textit{l}.; then
his second loss was 6\textit{l}.; the third, 9\textit{l}.; the fourth, 12\textit{l}.,
and so on; the tenth being 30\textit{l}. His total loss up to
this point amounts only to 165\textit{l}., and is, therefore, much
less serious than his loss would have been had he begun
by staking 1\textit{l}., and doubled that sum nine times, losing
ten times in all. Moreover, his next stake, according
to the system, is only 33\textit{l}., which is well within the
supposed limit of the bank. But, on the other hand, to
carry on the system, he now has to go on until he has
cleared off all the thirteen sums in the column under \textsc{m}.
To do this he has to run the risk of several further runs
of ill-luck against him, and it is by no means necessary
that these should be long runs of luck for the score
against him to become very heavy indeed. Be it noticed
that at every win he scores off only a small portion of
the balance against him, while every run of luck against
him adds to that score heavily. And notice, moreover,
that while on this system he does not quickly approach
the limit which the bank may assign to stakes, he much
more quickly encroaches on his own capital---a circumstance
which is quite as seriously opposed to his chance
of eventual success as the finality of the bank limit.
So far as the carrying out of his system is concerned, it
matters little whether he is obliged to stop the play on
the system because his pockets are emptied, or because
the bank will not allow him further to increase his
stake.

Similar remarks apply to the following method,
which has recently been suggested by another correspondent
of `Knowledge' as an improved system:

`My improvements,' he writes, `consisted, first, in
arranging that two players should play in concert, one
staking persistently upon one colour while the other
staked upon the other. A run of ill-luck to one would
then be somewhat counterbalanced by the run of good
luck to the other, while sometimes both would seem to
be winners.

`Second, in staking the \emph{sum} of the extreme figures
in the guide-column only when the number of figures
in it was even; when they were odd, \textit{e.g.},
$\begin{matrix}1\\2\\3\\4\\5\end{matrix}$ only the
highest, 5, is staked. Thus the rise of the stakes is
considerably reduced, while the principle of the play
is still carried out.

\begin{wrapfigure}[17]{r}{2.0in}
    \begin{center}
    \begin{tabular}{r}
        $-1$ \\
        $-2$ \\
        $-3$ \\
        $ 3$ \\
        $-4$ \\
        $-4$ \\
        $ 6$ \\
        $ 6$ \\
        $-9$ \\
        $12$
    \end{tabular}\\
    \textsc{Fig.~3}\\
    \end{center}
    {\small The numerals with a \textit{minus} sign are supposed
    to be struck out.}
\end{wrapfigure}
`Third, in splitting up a game when a run of ill luck
has occurred into two or more games, and winning
these \textit{seriatim}. Suppose, for instance, that the chances
of the game have brought the guide-column into the
form given in the margin. The player has actually lost
30, and must win 36 to gain 6. He might stake 36,
but this would be rash. He should play more cautiously,
and convert the column into 3 new columns,
totalling 12 each, or even into 4, totalling 9,
and play out three or four encounters with
the guidance of these columns. If luck
makes the securing of success in these a
long affair, his partner is meanwhile reaping
the benefit of a run upon his colour.

`I believe that, allowing the bank its small advantage,
the chance of winning 5 events out of 12, 6 out of
15, \&c., is large. But, of course, the possible gain is
small compared with the possible loss; and here, I have
no doubt, the plan breaks down.'

The plan is only safer than the others in the sense
that it prolongs the agony. The introduction of two
partners does not affect the validity of the system one
way or the other; for the chances of each must be considered
separately, though their gains or losses are
afterwards to be divided. The only point to be considered
in that respect is the idea that the bad effects for
one partner of a run on a colour would be corrected by
the good effects for the other. As a matter of fact,
there would be no such compensation. A run on one
colour which would set one of the partners two or three
hundred pounds to the bad, would perhaps gain for
the other forty or fifty pounds at the outside. Then
it must be remembered that we not only have to consider
the actual loss when an unfavourable colour
appears, but its effect on the operation of the system.
During an unfavourable run the stakes are rising and
the distance to be covered before (if ever) safety is
reached is increasing. By the suggested improvements
the rate of increase in the stakes is undoubtedly
diminished, but the rate at which the desired goal is
approached is diminished in equivalent degree.  I
scarcely recommend any one to test any of these systems
experimentally, even though without any idea of putting
them into actual practice. It is easy enough to apply
such a test by tossing a coin or cutting a pack a
sufficient number of times. For, as the essential principle
of all such systems is that they depend on the
improbability of an event whose occurrence---when it
does happen---will involve a heavy loss---a loss more
than cancelling all preceding gains---it is naturally
likely that any moderately long series of trials will
seem to favour the theory, the fatal run not chancing
to show in a series of trials too short to give it a fair
chance of showing.

It has been thus indeed that many foolish folk have
been tempted to trust in a system which has brought
them to their ruin. Consider what an irony underlies
the gambler's faith in such systems. When he starts
with the hope of winning, say, 10\textit{l}., he is perhaps to
some degree doubtful; but he goes on until perhaps he
is at such a stage that if he stopped he would be the
loser of fifty or sixty pounds. Yet such is his confidence
in his system that, although at this stage he is in
a very much worse position than at the beginning, the
mere circumstance that he is working out a system
encourages him to persevere. And so he continues
until the time comes---as with due patience and perseverance
it inevitably must---when either the bank limit
is reached or his pockets are emptied. In one case he
has to begin again with a deficit against him much
\begin{wrapfigure}[9]{l}{1.6in}
    \begin{tabular}{rr}
           1 &           \\
           2 &         1 \\
      3 of 2 & or 4 of 2 \\
           3 &         3 \\
           4 &         3 \\
         --- &       --- \\
    Total 12 &   Total 9
    \end{tabular}
\end{wrapfigure}
larger than any gain he has probably made before; in
the second he has the pleasant satisfaction of noting,
perhaps, that if he had been able to go on a little longer,
fortune would (from his point of view) have changed.
Though as a matter of fact, whether he had had a few
hundreds of pounds more or not only affects his fortunes
in putting off a little longer the inevitable day when
the system fails and he is ruined.

We may compare the trust in a system to such trust
as a bettor on races might put in laying long odds---when
the odds are really long, but not quite so long as
those he offers. Supposing a bettor to lay odds of 30
to 1 in sovereigns systematically, when the true odds
are 25 to 1, he will probably win his sovereign on the
average twenty-five times in twenty-six trials, but the
30\textit{l}.\ he will have to pay in the twenty-sixth case (on the
average) will leave him 5\textit{l}.\ to the bad on that set of
trials, excellent though his chance of success may
appear at each separate trial.

In fine, the moths who seek to gain wealth rapidly
and safely by gambling methods and systems are attracted
almost equally by two equally delusive flames.
They either trust in their own good luck, as in buying
lottery tickets, backing the favourite, or the like,
hoping to win large sums for small sums risked (these
small sums, however, being always in excess of the just
value of the chance); or they trust in the bad luck of
others, as when they try delusive martingales (though
they never see what they are really doing in such cases),
or when they lay long odds (always longer than the just
odds), hoping to win many small sums at small risk of
losing large ones; or they combine both methods.
Inevitably, in the long run, they lose more in many small
sums than they get back in a few large ones; and they
lose more in a few large sums than they get back in
many small ones. They lose all round, yet they delude
themselves all round into the belief that they are wise.
\clearpage

%[Blank Page]
\pagestyle{empty}
\begin{center}
MARCH 1887.\\[3mm]
\makebox[2in]{\hrulefill}\\[2mm]
{\Large GENERAL LISTS OF WORKS}\\
{\tiny PUBLISHED BY}\\
{\large MESSRS\@. LONGMANS, GREEN, \& CO.}\\
{\small 39 PATERNOSTER ROW, LONDON, E.C.}\\
\makebox[2in]{\hrulefill}\\[2mm]
\end{center}
\begin{footnotesize}
\begin{center}
\textbf{HISTORY, POLITICS, HISTORICAL MEMOIRS, \&c.}
\end{center}
Abbey's The English Church and its Bishops, 1700--1800. 2 vols.\ 8vo.\ 24\textit{s}.\\
Abbey and Overton's English Church in the Eighteenth Century. Cr. 8vo.\ 7\textit{s}. 6\textit{d}.\\
Arnold's Lectures on Modern History. 8vo.\ 7\textit{s}. 6\textit{d}.\\
Bagwell's Ireland under the Tudors. Vols. 1 and 2. 2 vols.\ 8vo.\ 32\textit{s}.\\
Ball's The Reformed Church of Ireland, 1537--1886. 8vo.\ 7\textit{s}. 8\textit{d}.\\
Boultbee's History of the Church of England, Pre-Reformation Period. 8vo.\ 15\textit{s}.\\
Buckle's History of Civilisation. 3 vols.\ crown 8vo.\ 24\textit{s}.\\
Cox's (Sir G.W.) General History of Greece. Crown 8vo.\ Maps, 7\textit{s}. 6\textit{d}.\\
Creighton's History of the Papacy during the Reformation. 8vo.\\
\D Vols. 1 and 2, 32\textit{s}.\\
\D Vols. 3 and 4, 24\textit{s}.\\
De Tocqueville's Democracy in America. 2 vols.\ crown 8vo.\ 16\textit{s}.\\
Doyle's English in America:\\
\D Virginia, Maryland, and the Carolinas, 8vo.\ 18\textit{s}.\\
\D The Puritan Colonies, 2 vols.\ 8vo.\ 36\textit{s}.\\
Epochs of Ancient History. Edited by the Rev. Sir G.W. Cox, Bart, and C. Sankey, M.A.\\
\D With Maps. Fcp. 8vo.\ price 2\textit{s}. 6\textit{d}. each.\\
\D Beesly's Gracchi, Marine, and Sulla. \\
\D Capes's Age of the Antonines. \\
\D \E Early Roman Empire.  \\
\D Cox's Athenian Empire. \\
\D \E Greeks and Persians.\\
\D Curteis's Rise of the Macedonian Empire.\\
\D Ihne's Rome to its Capture by the Gauls\\
\D Merivale's Roman Triumvirates\\
\D Sankey's Spartan and Theban Supremacies\\
\D Smith's Rome and Carthage, the Punic Wars.\\
Epochs of Modern History. Edited by C. Colbeck, M.A.
With Maps. Fcp. 8vo.\ price 2\textit{s}. 6\textit{d}. each.\\
\D Church's Beginning of the Middle Ages.\\
\D Cox's Crusades.\\
\D Creighton's Age of Elizabeth.\\
\D Gardiner's Houses of Lancaster and York.\\
\D Gardiner's Puritan Revolution.\\
\D \E Thirty Years War.\\
\D \E (Mrs.) French Revolution, 1789--1795.\\
\D Hale's Fall of the Stuarts.\\
\D Johnson's Normans in Europe.\\
\D Longman's Frederick the Great and the Seven Years' War.\\
\D Ludlow's War of American Independence.\\
\D McCarthy's Epocch of Reform, 1830--1850.\\
\D Moberly's The Early Tudors.\\
\D Morris's Age of Queen Anne.\\
\D \E The Early Hanoverians.\\
\D Seebohm's Protestant Revolution. \\
\D Stubbs's The Early Plantagenets.\\
\D Warburton's Edward III.\\
Epochs of Church History. Edited by the Rev. Mandell Creighton, M.A.\\
\D Fcp. 8vo.\ price 2\textit{s}. 6\textit{d}. each.\\
\D Brodrick's A History of the University of Oxford.\\
\D Overton's The Evangelical Revival in the Eighteenth Century.\\
\D Perry's The reformation in England.\\
\D Plummer's The Church of the Early Fathers.\\
\D Tucker's The English Church in Other Lands.\\
\emph{Other Volumes in preparation.}\\[2mm]
Freeman's Historical Geography of Europe. 2 vols.\ 8vo.\ 31\textit{s}. 6\textit{d}.\\
Froude's English in Ireland in the 18th Century. 3 vols.\ crown 8vo.\ 18\textit{s}.\\
\E History of England. Popular Edition. 12 vols.\ crown 8vo.\ 3\textit{s}. 6\textit{d}. each.\\
Gardiner's History of England from the Accession of James I. to the Outbreak of the Civil War.\\
\D 10 vols.\ crown 8vo.\ 60\textit{s}.\\
\E History of the Great Civil War, 1642--1649 (3 vols.) Vol. 1, 1642--1644, 8vo.\ 21\textit{s}.\\
Greville's Journal of the Reign of Queen Victoria, \\
\D 1837--1852. 3 vols.\ 8vo.\ 36\textit{s}. \\
\D 1852--1860, 2 vols.\ 8vo.\ 24\textit{s}.\\
Historic Towns. Edited by E.A. Freeman. D.C.L. and Rev. William Hunt. M.A.\\
With Maps and Plans. Crown 8vo.\ 3\textit{s}. 6\textit{d}. each.\\
\D London. By W.E. Loftie.\\
\D Exeter. By E.A. Freeman.\\
\D Bristol. By Rev. W. Hunt.\\
\emph{Other Volumes in preparation.}\\[3mm]
Lecky's History of England in the Eighteenth Century. \\
\D Vols. 1 \& 2, 1700--1760, 8vo.\ 36\textit{s}. \\
\D Vols. 3 \& 4, 1760--1784, 8vo.\ 36\textit{s}.\\
\E History of European Morals. 2 vols.\ crown 8vo.\ 16\textit{s}.\\
\E History of Rationalism in Europe. 2 vols.\ crown 8vo.\ 18\textit{s}.\\
Longman's Life and Times of Edward III\@. 2 vols.\ 8vo.\ 28\textit{s}.\\
Macaulay's Complete Works. Library Edition. 8 vols.\ 8vo.\ \pounds5\ 5\textit{s}.\\
\E Complete Works. Cabinet Edition. 16 vols.\ crown 8vo.\ \pounds4\ 16\textit{s}.\\
\E History of England:\\
\D \D Student's Edition. 2 vols.\ cr. 8vo.\ 12\textit{s}.\\
\D \D People's Edition. 4 vols.\ cr. 8vo.\ 16\textit{s}.\\
\D \D Cabinet Edition. 8 vols.\ post 8vo.\ 48\textit{s}.\\
\D \D Library Edition. 5 vols.\ 8vo.\ \pounds4\\
Macaulay's Critical and Historical Essays, with Lays of Ancient Home In One Volume:\\
\D Authorised Edition. Cr. 8vo.\ 2\textit{s}. 6\textit{d}., or 3\textit{s}. 6\textit{d}. gilt edges.\\
\D Popular Edition. Cr. 5vo.\ 2\textit{s}. 6\textit{d}.\\
Macaulay's Critical and Historical Essays:\\
\D Student's Edition. 1 vol.\ cr. 8vo.\ 6\textit{s}.\\
\D People's Edition. 2 vols.\ cr. 8vo.\ 8\textit{s}.\\
\D Cabinet Edition. 4 vols.\ post 8vo.\ 24\textit{s}.\\
\D Library Edition. 3 vols.\ 8vo.\ 36\textit{s}.\\
Macaulay's Speeches corrected by Himself. Crown 8vo.\ 3\textit{s}. 6\textit{d}.\\
Malmesbory's (Earl of) Memoirs of an Ex-Minister. Crown 8vo.\ 7\textit{s}. 6\textit{d}.\\
Maxwell's (Sir W.S.) Don John of Austria. \\
\D Library Edition, with numerous Illustrations. 2 vols.\ royal 8vo.\ 42\textit{s}.\\
May's Constitutional History of England, 1760--1870. 3 vols.\ crown 8vo.\ 18\textit{s}.\\
\E Democracy in Europe. 3 vols.\ 8vo.\ 32\textit{s}.\\
Merivale's Fall of the Roman Republic. 12mo. 7\textit{s}. 6\textit{d}.\\
\E General History of Rome, B.C. 753-A.D. 476. Crown 8vo.\ 7\textit{s}. 6\textit{d}.\\
\E History of the Romans under the Empire. 8 vols.\ post 8vo.\ 48\textit{s}.\\
Nelson's (Lord) Letters and Despatches. Edited by J.K. Langhton. 8vo.\ 16\textit{s}.\\
Outlines of Jewish History from B.C. 586 to C.E. 1885. \\
\E By the author of `About the Jews since Bible Times.' Fcp. 8vo.\ 3\textit{s}. 6\textit{d}.\\
Pears' The Fall of Constantinople. 8vo.\ 16\textit{s}.\\
Seebohm's Oxford Reformers---Colet, Erasmus, \& More. 8vo.\ 14\textit{s}.\\
Short's History of the Church of England. Crown 8vo.\ 7\textit{s}. 6\textit{d}.\\
Smith's Carthage and the Carthaginians. Crown 8vo.\ 10\textit{s}. 6\textit{d}.\\
Taylor's Manual of the History of India. Crown 8vo.\ 7\textit{s}. 6\textit{d}.\\
Walpole's History of England, from 1815. 5 vols.\ 8vo.\ \\
\D Vols. 1 \& 2, 1815--1832, 36\textit{s}.\\
\D Vol. 3, 1832--1841, 18\textit{s}. Vols. 4 \& 5, 1841--1858, 36\textit{s}.\\
Wylie's History of England under Henry IV\@. Vol. 1, crown 8vo.\ 10\textit{s}. 6\textit{d}.
\Needspace{10\baselineskip}
\begin{center}
\textbf{BIOGRAPHICAL WORKS.}
\end{center}
Armstrong's (E.J.) Life and Letters.\\
\E Edited by G.F. Armstrong. Fcp. 8vo.\ 7\textit{s}. 6\textit{d}.\\
Bacon's Life and Letters, by Spedding. 7 vols.\ 8vo.\ \pounds4.\ 4\textit{s}.\\
Bagehot's Biographical Studies. 1 vol.\ 8vo.\ 12\textit{s}.\\
Carlyle's Life, by J.A. Froude. \\
\D Vols. 1 \& 2, 1795--1835, 8vo.\ 32\textit{s}.\\
\D Vols. 3 \& 4, 1834--1881, 8vo.\ 32\textit{s}.\\
\E (Mrs.) Letters and Memorials. 3 vols.\ 8vo.\ 36\textit{s}.\\
Doyle (Sir F.H.) Reminiscences and Opinions. 8vo.\ 16\textit{s}.\\
English Worthies. Edited by Andrew Lang. Crown 8vo.\ 2\textit{s}. 6\textit{d}. each.\\
\D Charles Darwin. By Grant Allen.\\
\D Shaftssbury (The First Earl). By H.D. Traill.\\
\D Admiral Blake. By David Hannay.\\
\D Marlborough. By George Saintsbury.\\
\D Steele. By Austin Dobson.\\
\D Ben Jonson. By J.A. Symonds.\\
\D George Canning. By Frank H. Hill.\\
\emph{Other Volumes in preparation.}\\[3mm]
Fox (Charles James) The Early History of. By Sir G.O. Trevelyan, Bart. Crown 8vo.\ 6\textit{s}.\\
Froude's C{\ae}sar: a Sketch. Crown 8vo.\ 6\textit{s}.\\
Hamilton's (Sir W.R.) Life, by Graves. Vols. 1 and 2, 8vo.\ 15\textit{s}. each.\\
Havelock's Life, by Harslunan. Crown 8vo.\ 3\textit{s}. 6\textit{d}.\\
Hobart Pacha's Sketches from my Life. Crown 8vo.\ 7\textit{s}. 6\textit{d}.\\
Macaulay's (Lord) Life and Letters. By his Nephew, Sir G.O. Trevelyan, Bart. \\
\D Popular Edition, 1 vol.\ crown 8vo.\ 6\textit{s}. Cabinet Edition, a vols.\ post 8vo.\ 12\textit{s}. \\
\D Library Edition, 2 vols.\ 8vo.\ 36\textit{s}.\\
Mendelssohn's Letters. Translated by Lady Wallace. 2 vols.\ cr. 8vo.\ 5\textit{s}. each.\\
Mill (James) Biography of, by Prof. Bain. Crown 8vo.\ 5\textit{s}.\\
Mill (John Stuart) Recollections of, by Prof. Bain. Crown 8vo.\ 2\textit{s}. 6\textit{d}.\\
\E Autobiography. 8vo.\ 7\textit{s}. 6\textit{d}.\\
M\"uller's (Max) Biographical Essays. Crown 8vo.\ 7\textit{s}. 6\textit{d}.\\
Newman's Apologia pro Vit{\^a} Su{\^a}. Crown 8vo.\ 6\textit{s}.\\
Pasteur (Louis) His Life and Labours. Crown 8vo.\ 7\textit{s}. 6\textit{d}.\\
Shakespeare's Life (Outlines of), by Halliwell-Phillipps. 2 vols.\ royal 8vo.\ 10\textit{s}. 6\textit{d}.\\
Southey's Correspondence with Caroline Bowles. 8vo.\ 14\textit{s}.\\
Stephen's Essays in Ecclesiastical Biography. Crown 8vo.\ 7\textit{s}. 6\textit{d}.\\
Wellington's Life, by Gleig. Crown 8vo.\ 6\textit{s}.
\Needspace{10\baselineskip}
\begin{center}
\textbf{MENTAL AND POLITICAL PHILOSOPHY, FINANCE, \&c.}
\end{center}
Amos's View of the Science of Jurisprudence. 8vo.\ 18\textit{s}.\\
\E Primer of the English Constitution. Crown 8vo.\ 6\textit{s}.\\
Bacon's Essays, with Annotations by Whately. 8vo.\ 10\textit{s}. 6\textit{d}.\\
\E Works, edited by Spedding. 7 vols.\ 8vo.\ 73\textit{s}. 6\textit{d}.\\
Bagehot's Economic Studies, edited by Hutton. 8vo.\ 10\textit{s}. 6\textit{d}.\\
\E The Postulates of English Political Economy. Crown 8vo.\ 2\textit{s}. 6\textit{d}.\\
Bain's Logic, Deductive and Inductive. Crown 8vo.\ 10\textit{s}. 6\textit{d}.\\
\D \textsc{Part I.} Deduction, 4\textit{s}.\\
\D \textsc{Part II.} Induction, 6\textit{s}. 6\textit{d}.\\
\E Mental and Moral Science. Crown 8vo.\ 10\textit{s}. 6\textit{d}.\\
\E The Senses and the Intellect. 8vo.\ 15\textit{s}.\\
\E The Emotions and the Will. 8vo.\ 15\textit{s}.\\
\E Practical Essays. Crown 8vo.\ 4\textit{s}. 6\textit{d}.\\
Buckle's (H. T.) Miscellaneous and Posthumous Works. 2 vols.\ crown 8vo.\ 21\textit{s}.\\
Crozier's Civilization and Progress. 8vo.\ 14\textit{s}.\\
Crump's A Short Enquiry into the Formation of English Political Opinion. 8vo.\ 7\textit{s}., 6\textit{d}.\\
Dowell's A History of Taxation and Taxes in England. 4 vols.\ 8vo.\ 48\textit{s}.\\
Green's (Thomas Hill) Works. (3 vols.) Vols. 1 \& 2, Philosophical Works. 8vo.\ 16\textit{s}. each.\\
Hume's Essays, edited by Green \& Grose. 2 vols.\ 8vo.\ 28\textit{s}.\\
\D Treatise of Human Nature, edited by Green \& Grose. 2 vols.\ 8vo.\ 28\textit{s}.\\
Lang's Custom and Myth: Studies of Early Usage and Belief. Crown 8vo.\ 7\textit{s}. 6\textit{d}.\\
Leslie's Essays in Political and Moral Philosophy. 8vo.\ 10\textit{s}. 6\textit{d}.\\
Lewes's History of Philosophy. 2 vols.\ 8vo.\ 32\textit{s}.\\
Lubbock's Origin of Civilisation. 8vo.\ 18\textit{s}.\\
Macleod's Principles of Economical Philosophy. In 2 vols.\ Vol 1, 8vo.\ 15\textit{s}. Vol. 2, Part I. 12\textit{s}.\\
\E The Elements of Economics. (2 vols.)\\
\D\D Vol. 1, cr. 8vo.\ 7\textit{s}. 6\textit{d}. Vol. 2, Part I. cr. 8vo.\ 7\textit{s}. 6\textit{d}.\\
\E The Elements of Banking. Crown 8vo.\ 5\textit{s}.\\
\E The Theory and Practice of Banking.\\
\D\D Vol. 1, 8vo.\ 12\textit{s}. Vol. 2, 14\textit{s}.\\
\E Economics for Beginners. 8vo.\ 2\textit{s}. 6\textit{d}.\\
\E Lectures on Credit and Banking. 8vo.\ 5\textit{s}.\\
Mill's (James) Analysis of the Phenomena of the Human Mind. 2 vols.\ 8vo.\ 28\textit{s}.\\
Mill (John Stuart) on Representative Government. Crown 8vo.\ 2\textit{s}.\\
\E \E on Liberty. Crown 8vo.\ 1\textit{s}. 4\textit{d}.\\
\E \E Examination of Hamilton's Philosophy. 8vo.\ 16\textit{s}.\\
\E \E Logic. Crown 8vo.\ 5\textit{s}.\\
\E \E Principles of Political Economy. 2 vols.\ 8vo.\ 30\textit{s}. \\
\E \E People's Edition, 1 vol.\ crown 8vo.\ 5\textit{s}.\\
\E \E Subjection of Women. Crown 8vo.\ 6\textit{s}.\\
\E \E Utilitarianism. 8vo.\ 5\textit{s}.\\
\E \E Three Essays on Religion, \&c. 8vo.\ 5\textit{s}.\\
Mulhall's History of Prices since 1850. Crown 8vo.\ 6\textit{s}.\\
Sandars's Institutes of Justinian, with English Notes. 8vo.\ 18\textit{s}.\\
Seebohm's English Village Community. 8vo.\ 16\textit{s}.\\
Sully's Outlines of Psychology. 8vo.\ 12\textit{s}. 6\textit{d}.\\
\D Teacher's Handbook of Psychology. Crown 8vo.\ 6\textit{s}. 6\textit{d}.\\
Swinburne's Picture Logic. Post 8vo.\ 5\textit{s}.\\
Thompson's A System of Psychology. 2 vols.\ 8vo.\ 36\textit{s}.\\
Thomson's Outline of Necessary Laws of Thought. Crown 8vo.\ 6\textit{s}.\\
Twiss's Law of Nations in Time of War. 8vo.\ 21\textit{s}.\\
\D in Time of Peace. 8vo.\ 15\textit{s}.\\
Webb's The Veil of Isis. 8vo.\ 10\textit{s}. 6\textit{d}.\\
Whately's Elements of Logic. Crown 8vo.\ 4\textit{s}. 6\textit{d}.\\
\D Rhetoric. Crown 8vo, 4\textit{s}. 6\textit{d}.\\
Wylie's Labour, Leisure, and Luxury. Crown 8vo.\ 6\textit{s}.\\
Zeller's History of Eclecticism in Greek Philosophy. Crown 8vo.\ 10\textit{s}. 6\textit{d}.\\
\E Plato and the Older Academy. Crown 8vo.\ 18\textit{s}.\\
\E Pre-Socratic Schools. 2 vols.\ crown 8vo.\ 30\textit{s}.\\
\E Socrates and the Socratic Schools. Crown 8vo.\ 10\textit{s}. 6\textit{d}.\\
\E Stoics, Epicureans, and Sceptics. Crown 8vo.\ 15\textit{s}.\\
\E Outlines of the History of Greek Philosophy. Crown 8vo.\ 10\textit{s}. 6\textit{d}.
\Needspace{10\baselineskip}
\begin{center}
\textbf{MISCELLANEOUS WORKS.}
\end{center}
A. K. H. B., The Essays and Contributions of. Crown 8 Vo.\\
\D Autumn Holidays of a Country Parson. 3\textit{s}. 6\textit{d}.\\
\D Changed Aspects of Unchanged Truths. 3\textit{s}. 6\textit{d}.\\
\D Common-Place Philosopher in Town and Country. 3\textit{s}. 6\textit{d}.\\
\D Critical Essays of a Country Parson. 3\textit{s}. 6\textit{d}.\\
\D Counsel and Comfort spoken from a City Pulpit. 3\textit{s}. 6\textit{d}.\\
\D Graver Thoughts of a Country Parson. Three Series. 3\textit{s}. 6\textit{d}. each.\\
\D Landscapes, Churches, and Moralities. 3\textit{s}. 6\textit{d}.\\
\D Leisure Hours in Town. 3\textit{s}. 6\textit{d}. Lessons of Middle Age. 3\textit{s}. 6\textit{d}.\\
\D Our Homely Comedy; and Tragedy. 3\textit{s}. 6\textit{d}.\\
\D Our Little Life. Essays Consolatory and Domestic. Two Series. 3\textit{s}. 6\textit{d}.\\
\D Present-day Thoughts. 3\textit{s}. 5\textit{d}. each.\\
\D Recreations of a Country Parson. Three Series. 3\textit{s}. 6\textit{d}. each.\\
\D Seaside Musings on Sundays and Week-Days. 3\textit{s}. 6\textit{d}.\\
\D Sunday Afternoons in the Parish Church of a University City. 3\textit{s}. 6\textit{d}.\\
Armstrong's (Ed. J.) Essays and Sketches. Fcp. 8vo.\ 5\textit{s}.\\
Arnold's (Dr. Thomas) Miscellaneous Works. 8vo.\ 7\textit{s}. 6\textit{d}.\\
Bagehot's Literary Studies, edited by Hutton. 2 vols.\ 8vo.\ 28\textit{s}.\\
Beaconsfield (Lord), The Wit and Wisdom of. Crown 8vo.\ 1\textit{s}. boards; 1\textit{s}. 6\textit{d}. cl.\\
Evans's Bronze Implements of Great Britain. 8vo.\ 25\textit{s}.\\
Farrar's Language and Languages. Crown 8vo.\ 6\textit{s}.\\
Fronde's Short Studies on Great Subjects. 4 vols.\ crown 8vo.\ 24\textit{s}.\\
Lang's Letters to Dead Authors. Fcp. 8vo.\ 6\textit{s}. 6\textit{d}.\\
\E Books and Bookmen. Crown 8vo.\ 6\textit{s}. 6\textit{d}.\\
Macaulay's Miscellaneous Writings. 2 vols.\ 8vo.\ 21\textit{s}. 1 vol.\ crown 8vo.\ 4\textit{s}. 6\textit{d}.\\
\E Miscellaneous Writings and Speeches. Crown 8vo.\ 6\textit{s}.\\
\E Miscellaneous Writings, Speeches, Lays of Ancient Rome, \&c. \\
\E Cabinet Edition. 4 vols.\ crown 8vo.\ 24\textit{s}.\\
\E Writings, Selections from. Crown 8vo.\ 6\textit{s}.\\
M\"uller's (Max) Lectures on the Science of Language. 2 vols.\ crown 8vo.\ 16\textit{s}.\\
\E \E Lectures on India. 8vo.\ 12\textit{s}. 6\textit{d}.\\
Proctor's Chance and Luck. Crown 8vo.\ 5\textit{s}.\\
Smith (Sydney) The Wit and Wisdom of. Crown 8vo.\ 1\textit{s}. boards; 1\textit{s}. 6\textit{d}. cloth.
\Needspace{10\baselineskip}
\begin{center}
\textbf{ASTRONOMY.}
\end{center}
Herschel's Outlines of Astronomy. Square crown 8vo.\ 12\textit{s}.\\
Proctor's Larger Star Atlas. Folio, 15\textit{s}. or Maps only, 12\textit{s}. 6\textit{d}.\\
\E New Star Atlas. Crown 8vo.\ 5\textit{s}.\\
\E Light Science for Leisure Hours. 3 Series. Crown 8vo.\ 5\textit{s}. each.\\
\E The Moon. Crown 8vo.\ 6\textit{s}.\\
\E Other Worlds than Ours. Crown 8vo.\ 5\textit{s}.\\
\E The Sun. Crown 8vo.\ 14\textit{s}.\\
\E Studies of Venus-Transits. 8vo.\ 5\textit{s}.\\
\E Orbs Around Us. Crown 8vo.\ 5\textit{s}.\\
\E Universe of Stars. 8vo.\ 10\textit{s}. 6\textit{d}.\\
Webb's Celestial Objects for Common Telescopes. Crown 8vo.\ 9\textit{s}.
\Needspace{10\baselineskip}
\begin{center}
\textbf{THE `KNOWLEDGE' LIBRARY.}\\
{\tiny Edited by \textsc{Richard A. Proctor.}}\\
\end{center}
How to Play Whist. Crown 8vo.\ 5\textit{s}.\\
Home Whist. 16mo. 1\textit{s}.\\
The Borderland of Science. Cr. 8vo.\ 6\textit{s}.\\
Nature Studies. Crown 8vo.\ 6\textit{s}.\\
Leisure Readings. Crown 8vo.\ 6\textit{s}.\\
The Stars in their Seasons. Imp. 8vo.\ 5\textit{s}.\\
Myths and Marvels of Astronomy. Crown 8vo.\ 6\textit{s}.\\
Pleasant Ways in Science. Cr. 8vo.\ 6\textit{s}.\\
Star Primer. Crown 4to. 2\textit{s}. 6\textit{d}.\\
The Seasons Pictured. Demy 4to. 5\textit{s}.\\
Strength and Happiness. Cr. 8vo.\ 5\textit{s}.\\
Rough Ways made Smooth. Cr. 8vo.\ 6\textit{s}.\\
The Expanse of Heaven. Cr. 8vo.\ 5\textit{s}.\\
Our Place among Infinities. Cr. 8vo.\ 5\textit{s}.
\Needspace{10\baselineskip}
\begin{center}
\textbf{CLASSICAL LANGUAGES AND LITERATURE.}
\end{center}
{\AE}chylus, The Eumenides of Text, with Metrical English Translation, by J.F. Davies. 8vo.\ 7\textit{s}.\\
Aristophanes' The Acharnians, translated by R.Y. Tyrrell. Crown 8vo.\ 2\textit{s}.\ 6\textit{d}.\\
Aristotle's The Ethics, Text and Notes, by Sir Alex.\ Grant, Bart. 2 vols.\ 8vo.\ 32\textit{s}.\\
Aristotle's The Niomachean Ethics, translated by Williams, crown 8vo.\ 7\textit{s}.\ 6\textit{d}.\\
Aristotle's The Politics, Books I\@. III\@. IV\@. (VII.) with Translation, \&c.\ by Bolland and Lang.\\
\D Crown 8vo.\ 7\textit{s}.\ 6\textit{d}.\\
Becker's \emph{Charicles and Gallus}, by Metcalfe. Post 8vo.\ 7\textit{s}.\ 6\textit{d}.\ each.\\
Cicero's Correspondence, Text and Notes, by R.Y. Tyrrell. Vols.\ 1 \& 2, 8vo.\ 12\textit{s}.\ each.\\
Homer's Iliad, Homometrically translated by Cayley. 8vo.\ 12\textit{s}.\ 6\textit{d}.\\
\E Greek Text, with Verse Translation, by W. C. Green. Vol. 1, Books I.-XII\@. Crown 8vo.\ 6\textit{s}.\\
Mahaffy's Classical Greek Literature. Crown 8vo.\ Vol.\ 1, The Poets, 7\textit{s}.\ 6\textit{d}. \\
\D Vol.\ 2, The Prose Writers, 7\textit{s}.\ 6\textit{d}.\\
Plato's Parmenides, with Notes, \&c.\ by J. Maguire. 8vo.\ 7\textit{s}.\ 6\textit{d}.\\
Virgil's Works, Latin Text, with Commentary, by Kennedy. Crown 8vo.\ 10\textit{s}.\ 6\textit{d}.\\
\E {\AE}neid, translated into English Verse, by Conington. Crown 8vo.\ 9\textit{s}.\\
\E {\AE}neid, translated into English Verse, by W. J. Thornhill. Cr.\ 8vo.\ 7\textit{s}.\ 6\textit{d}.\\
\E Poems, translated into English Verse, by Conington. Crown 8vo.\ 9\textit{s}.\\
Witt's Myths of Hellas, translated by F. M. Younghusband. Crown 8vo.\ 3\textit{s}.\ 6\textit{d}.\\
\E The Trojan War, translated by F. M. Younghusband. Fcp.\ 8vo.\ 2\textit{s}.\\
\E The Wanderings of Ulysses, translated by F. M. Younghusband. Crown 8vo.\ 3\textit{s}.\ 6\textit{d}.
\Needspace{10\baselineskip}
\begin{center}
\textbf{NATURAL HISTORY, BOTANY, \& GARDENING.}
\end{center}
Allen's Flowers and their Pedigrees. Crown 8vo.\ Woodcuts, 5\textit{s}.\\
Decaisne and Le Maout's General System of Botany. Imperial 8vo.\ 31\textit{s}.\ 6\textit{d}.\\
Dixon's Rural Bird Life. Crown 8vo.\ Illustrations, 5\textit{s}.\\
Hartwig's Aerial World, 8vo.\ 10\textit{s}.\ 6\textit{d}.\\
\E  Polar World, 8vo.\ 10\textit{s}.\ 6\textit{d}.\\
\E  Sea and its Living Wonders. 8vo.\ 10\textit{s}.\ 6\textit{d}.\\
\E  Subterranean World, 8vo.\ 10\textit{s}.\ 6\textit{d}.\\
\E  Tropical World, 8vo.\ 10\textit{s}.\ 6\textit{d}.\\
Lindley's Treasury of Botany. 2 vols.\ fcp.\ 8vo.\ 12\textit{s}.\\
Loudon's Encyclop{\ae}dia of Gardening. 8vo.\ 21\textit{s}.\\
\E Encyclop{\ae}dia Plants. 8vo.\ 42\textit{s}.\\
Rivers's Orchard House. Crown 8vo.\ 5\textit{s}.\\
\E Miniature Fruit Garden. Fcp.\ 8vo.\ 42\textit{s}.\\
Stanley's Familiar History of British Birds. Crown 8vo.\ 6\textit{s}.\\
Wood's Bible Animals. With 112 Vignettes. 8vo.\ 10\textit{s}.\ 6\textit{d}.\\
\E Common British Insects. Crown 8vo.\ 3\textit{s}.\ 6\textit{d}.\\
\E Homes Without Hands, 8vo.\ 10\textit{s}.\ 6\textit{d}.\\
\E Insects Abroad, 8vo.\ 10\textit{s}.\ 6\textit{d}.\\
\E Horse and Man.  8vo.\ 14\textit{s}.\\
\E Insects at Home. With 700 Illustrations. 8vo.\ 10\textit{s}.\ 6\textit{d}.\\
\E Out of Doors. Crown 8vo.\ 5\textit{s}.\\
\E Petland Revisited. Crown 8vo.\ 7\textit{s}.\ 6\textit{d}.\\
\E Strange Dwellings. Crown 8vo.\ 5\textit{s}. Popular Edition, 4to.\ 6\textit{d}.
\Needspace{10\baselineskip}
\begin{center}
\textbf{THE FINE ARTS AND ILLUSTRATED EDITIONS.}
\end{center}
Eastlake's Household Taste in Furniture, \&c. Square crown 8vo.\ 14\textit{s}.\\
Jameson's Sacred and Legendary Art. 6 vols.\ square 8vo.\\
\D Legends of the Madonna. 1 vol.\ 21\textit{s}.\\
\D \E Monastic Orders 1 vol.\ 21\textit{s}.\\
\D \E Saints and Martyrs. 2 vols.\ 31\textit{s}. 6\textit{d}.\\
\D \E Saviour. Completed by Lady Eastlake. 2 vols.\ 42\textit{s}.\\
Macaulay's Lays of Ancient Rome, illustrated by Scharf. Fcp. 4to. 10\textit{s}. 6\textit{d}.\\
The same, with \textit{Ivry} and the \textit{Armada}, illustrated by Weguelin. Crown 8vo.\ 3\textit{s}. 6\textit{d}.\\
New Testament (The) illustrated with Woodcuts after Paintings by the Early Masters. 4to. 21\textit{s}.
\Needspace{10\baselineskip}
\begin{center}
\textbf{CHEMISTRY ENGINEERING, \& GENERAL SCIENCE.}
\end{center}
Arnott's Elements of Physics or Natural Philosophy. Crown 8vo.\ 12\textit{s}. 6\textit{d}.\\
Barrett's English Glees and Part-Songs: their Historical Development. Crown 8vo.\ 7\textit{s}. 6\textit{d}.\\
Bourne's Catechism of the Steam Engine. Crown 8vo.\ 7\textit{s}. 6\textit{d}.\\
\E Examples of Steam, Air, and Gas Engines. 4to. 70\textit{s}.\\
\E Handbook of the Steam Engine. Fcp. 8vo.\ 9\textit{s}.\\
\E Recent Improvements in the Steam Engine. Fcp. 8vo.\ 6\textit{s}.\\
\E Treatise on the Steam Engine. 4to. 42\textit{s}.\\
Bruckton's Our Dwellings, Healthy and Unhealthy. Crown 8vo.\ 3\textit{s}. 6\textit{d}.\\
Clerk's The Gas Engine. With Illustrations. Crown 8vo.\ 7\textit{s}. 6\textit{d}.\\
Crookes's Select Methods in Chemical Analysis. 8vo.\ 24\textit{s}.\\
Culler's Handbook of Practical Telegraphy. 8vo.\ 16\textit{s}.\\
Fairbairn's Useful Information for Engineers. 3 vols, crown 8vo.\ 31\textit{s}. 6\textit{d}.\\
\E Mills and Millwork. 1 vol.\ 8vo.\ 25\textit{s}.\\
Ganot's Elementary Treatise on Physics, by Atkinson. Large crown 8vo.\ 15\textit{s}.\\
\E Natural Philosophy, by Atkinson. Crown 8vo.\ 7\textit{s}. 6\textit{d}.\\
Grove's Correlation of Physical Forces. 8vo.\ 15\textit{s}.\\
Haughton's Six Lectures on Physical Geography. 8vo.\ 19\textit{s}.\\
Helmholtz on the Sensations of Tone. Royal 8vo.\ 28\textit{s}.\\
Helmholtz's Lectures on Scientific Subjects. 2 vols.\ crown 8vo.\ 7\textit{s}. 6\textit{d}. each.\\
Hudson and Gosse's The Rotifera or `Wheel Animalcules.' With 30 Coloured Plates.\\
\D 6 parts. 4to. 10\textit{s}. 6\textit{d}. each. Complete, 2 vols.\ 4to. \pounds3.\ 10\textit{s}.\\
Hullah's Lectures on the History of Modern Music. 8vo.\ 8\textit{s}. 6\textit{d}.\\
\E Transition Period of Musical History. 8vo.\ 10\textit{s}. 6\textit{d}.\\
Jackson's Aid to Engineering Solution. Royal 8vo.\ 21\textit{s}.\\
Jago's Inorganic Chemistry, Theoretical and Practical. Fcp. 8vo.\ 2\textit{s}.\\
Jeans' Railway Problems. 8vo.\ 12\textit{s}. 6\textit{d}.\\
Kolbe's Short Text-Book of Inorganic Chemistry. Crown 8vo.\ 7\textit{s}. 6\textit{d}.\\
Lloyd's Treatise on Magnetism. 8vo.\ 10\textit{s}. 6\textit{d}.\\
Macalister's Zoology and Morphology of Vertebrate Animals. 8vo.\ 10\textit{s}. 6\textit{d}.\\
Macfarren's Lectures on Harmony. 8vo.\ 12\textit{s}.\\
Miller's Elements of Chemistry, Theoretical and Practical. 3 vols.\ 8vo.\ \\
\D Part I\@. Chemical Physics, 16\textit{s}.\\
\D Part II\@. Inorganic Chemistry, 24\textit{s}.\\
\D Part III\@. Organic Chemistry, price 31\textit{s}. 6\textit{d}.\\
Mitchell's Manual of Practical Assaying. 8vo.\ 31\textit{s}. 6\textit{d}.\\
Noble's Hours with a Three-inch Telescope. Crown 8vo.\ 4\textit{s}. 6\textit{d}.\\
Northcott's Lathes and Turning. 8vo.\ 18\textit{s}.\\
Owen's Comparative Anatomy and Physiology of the Vertebrate Animals.\\
\D 3 vols.\ 8vo.\ 73\textit{s}. 6\textit{d}.\\
Piesse's Art at Perfumery. Square crown 8vo.\ 21\textit{s}.\\
Reynolds's Experimental Chemistry. Fcp. 8vo.\ Part I. 1\textit{s}. 6\textit{d}. Part II. 2\textit{s}. 6\textit{d}. Part III. 3\textit{s}. 6\textit{d}.\\
Schellen's Spectrum Analysis. 8vo.\ 31\textit{s}. 6\textit{d}.\\
Sennett's Treatise on the Marine Steam Engine. 8vo.\ 21\textit{s}.\\
Smith's Air and Rain. 8vo.\ 24\textit{s}.\\
Stoney's The Theory of the Stresses on Girders, \&c. Royal 8vo.\ 36\textit{s}.\\
Tilden's Practical Chemistry. Fcp. 8vo.\ 1\textit{s}. 6\textit{d}.\\
Tyndall's Faraday as a Discoverer. Crown 8vo.\ 3\textit{s}. 6\textit{d}.\\
\E Floating Matter of the Air. Crown 8vo.\ 7\textit{s}. 6\textit{d}.\\
\E Fragments of Science. 2 vols.\ post 8vo.\ 16\textit{s}.\\
\E Heat a Mode of Motion. Crown 8vo.\ 12\textit{s}.\\
\E Lectures on Light delivered in America. Crown 8vo.\ 5\textit{s}.\\
\E Lessons on Electricity. Crown 8vo.\ 2\textit{s}. 6\textit{d}.\\
\E Notes on Electrical Phenomena. Crown 8vo.\ 1\textit{s}. sewed, 1\textit{s}. 6\textit{d}. cloth.\\
\E Notes of Lectures on Light. Crown 8vo.\ 1\textit{s}. sewed, 1\textit{s}. 6\textit{d}. cloth.\\
\E Sound, with Frontispiece and 203 Woodcuts. Crown 8vo.\ 10\textit{s}. 6\textit{d}.\\
Watts's Dictionary of Chemistry. 9 vols.\ medium 8vo.\ \pounds15.\ 2\textit{s}. 6\textit{d}.\\
Wilson's Manual of Health-Science. Crown 8vo.\ 2\textit{s}. 6\textit{d}.
\Needspace{10\baselineskip}
\begin{center}
\textbf{THEOLOGICAL AND RELIGIOUS WORKS.}
\end{center}
Arnold's (Rev. Dr. Thomas) Sermons. 6 vols.\ crown 8vo.\ 5\textit{s}. each.\\
Boultbee's Commentary on the 39 Articles. Grown 8vo.\ 6\textit{s}.\\
Browne's (Bishop) Exposition of the 39 Articles. 8vo.\ 16\textit{s}.\\
Bullinger's Critical Lexicon and Concordance to the English and Greek New Testament.\\
\D Royal 8vo.\ 15\textit{s}.\\
Colenso on the Pentateuch and Book of Joshua. Crown 8vo.\ 6\textit{s}.\\
Condor's Handbook of the Bible. Post 8vo.\ 7\textit{s}. 6\textit{d}.\\
Conybeare \& Howson's Life and Letters of St. Paul:---\\
\D Library Edition, with Maps, Plates, and Woodcuts. 2 vols.\ square crown 8vo.\ 21\textit{s}.\\
\D Student's Edition, revised and condensed, with 46 Illustrations and Maps.\\
\D\D 1 vol.\ crown 8vo.\ 7\textit{s}. 6\textit{d}.\\
Cox's (Homersham) The First Century of Christianity. 8vo.\ 12\textit{s}.\\
Davidson's Introduction to the Study of the New Testament. 2 vols.\ 8vo.\ 30\textit{s}.\\
Edersheim's Life and Times of Jesus the Messiah. 2 vols.\ 8vo.\ 24\textit{s}.\\
\E Prophecy and History in relation to the Messiah. 8vo.\ 12\textit{s}.\\
Ellicott's (Bishop) Commentary on St. Paul's Epistles. 8vo.\ Galatians, 8\textit{s}. 6\textit{d}.\\
\D Ephesians, 8\textit{s}. 6\textit{d}. Pastoral Epistles, 10\textit{s}. 6\textit{d}. \\
\D Philippians, Colossians and Philemon, 10\textit{s}. Set. Thessalonians, 7\textit{s}. 6\textit{d}.\\
\D\E Lectures on the Life of our Lord. 8vo.\ 12\textit{s}.\\
Ewald's Antiquities of Israel, translated by Solly. 8vo.\ 12\textit{s}. 6\textit{d}.\\
\E History of Israel, translated by Carpenter \& Smith. 8 vols.\ 8vo.\ Vols 1 \& 2, 24\textit{s}.\\
\D\D Vols. 3 is 4, 21\textit{s}. Vol. 5, 18\textit{s}. Vol. 6, 16\textit{s}. Vol. 7, 21\textit{s}.\\
\D\D Vol. 8, 18\textit{s}.\\
Hobart's Medical Language of St. Luke. 8vo.\ 16\textit{s}.\\
Hopkins's Christ the Consoler. Fcp. 8vo.\ 2\textit{s}. 6\textit{d}.\\
Jukes's New Man and the Eternal Life. Crown 8vo.\ 6\textit{s}.\\
\D Second Death and the Restitution of all Things. Crown 8vo.\ 3\textit{s}. 6\textit{d}.\\
\D Types of Genesis. Crown 8vo.\ 7\textit{s}. 6\textit{d}.\\
\D The Mystery of the Kingdom. Crown 8vo.\ 3\textit{s}. 6\textit{d}.\\
Lenormant's New Translation of the Book of Genesis. Translated into English. 8vo.\ 10\textit{s}. 6\textit{d}.\\
Lyra Germanica: Hymns translated by Miss Winkworth. Fcp. 8vo.\ 5\textit{s}.\\
Macdonald's (G.) Unspoken Sermons. Two Series, Crown 8vo.\ 3\textit{s}. 6\textit{d}. each.\\
\D The Miracles of our Lord. Crown 8vo.\ 3\textit{s}. 6\textit{d}.\\
Manning's Temporal Mission of the Holy Ghost. Crown 8vo.\ 8\textit{s}. 6\textit{d}.\\
Martineau's Endeavours after the Christian Life. Crown 8vo.\ 7\textit{s}. 6\textit{d}.\\
\E Hymns of Praise and Prayer. Crown 8vo.\ 4\textit{s}. 6\textit{d}. 32mo. 1\textit{s}. 6\textit{d}.\\
\E Sermons, Hours of Thought on Sacred Things. 2 vols.\ 7\textit{s}. 6\textit{d}. each.\\
Monsell's Spiritual Songs for Sundays and Holidays. Fcp. 8vo.\ 5\textit{s}. 18mo. 2\textit{s}.\\
M\"uller's (Max) Origin and Growth of Religion. Crown 8vo.\ 7\textit{s}. 8\textit{d}.\\
\E Science of Religion. Crown 8vo.\ 7\textit{s}. 6\textit{d}.\\
Newman's Apologia pro Vit\^a Su\^a. Crown 8vo.\ 6\textit{s}.\\
\E The Idea of a University Defined and Illustrated. Crown 8vo.\ 7\textit{s}.\\
\E Historical Sketches. 3 vols.\ crown 8vo.\ 6\textit{s}. each.\\
\E Discussions and Arguments on Various Subjects. Crown 8vo.\ 6\textit{s}.\\
\E An Essay on the Development of Christian Doctrine. Crown 8vo.\ 6\textit{s}.\\
\E Certain Difficulties Felt by Anglicans In Catholic Teaching Considered.\\
\D\D Vol. I, crown 8vo.\ 7\textit{s}. 6\textit{d}. Vol. 2, crown 8vo.\ 5\textit{s}. 6\textit{d}.\\
\E The Via Media of the Anglican Church, Illustrated in Lectures, \&c. 2 vols.\ crown 8vo.\ 6\textit{s}. each\\
\E Essays, Critical and Historical. 2 vols.\ crown 8vo.\ 12\textit{s}.\\
\E Essays on Biblical and on Ecclesiastical Miracles. Crown 8vo.\ 6\textit{s}.\\
\E An Essay in Aid of a Grammar of Assent. 7\textit{s}. 6\textit{d}.\\
Overton's Life in the English Church (1660--1714). 8vo.\ 14\textit{s}.\\
Supernatural Religion. Complete Edition. 3 vols.\ 8vo.\ 36\textit{s}.\\
Younghusband's The Story of Our Lord told in Simple Language for Children. Illustrated.\\
\D Crown 8vo.\ 2\textit{s}. 6\textit{d}. \\
\D cloth plain; 3\textit{s}. 6\textit{d}. \\
\D cloth extra, gilt edges.
\Needspace{10\baselineskip}
\begin{center}
\textbf{TRAVELS, ADVENTURES, \&c.}
\end{center}
Alpine Club (The) Map of Switzerland. In Four Sheets. 42\textit{s}.\\
Baker's Eight Yeats in Ceylon. Grown 8vo.\ 5\textit{s}.\\
\D Rifle and Hound in Ceylon. Crown 8vo.\ 5\textit{s}.\\
Ball's Alpine Guide. 3 vols.\ post 8vo.\ with Maps and Illustrations:---\\
\D I\@. Western Alps, 6\textit{s}. 6\textit{d}. \\
\D II\@. Central Alps, 7\textit{s}. 6\textit{d}. \\
\D III\@. Eastern Alps, 10\textit{s}. 6\textit{d}.\\
Ball on Alpine Travelling, and on the Geology of the Alps, 1\textit{s}.\\
Brassey's Sunshine and Storm in the East. Library Edition, 8vo.\ 21\textit{s}. \\
\D \D Cabinet Edition, crown 8vo.\ 7\textit{s}. 6\textit{d}. \\
\D \D Popular Edition, 4to. 6\textit{d}.\\
\E Voyage in the Yacht `Sunbeam.' Library Edition, 8vo.\ 21\textit{s}. \\
\D \D Cabinet Edition, crown 8vo.\ 7\textit{s}. 6\textit{d}. \\
\D \D School Edition, top. 8vo.\ 2\textit{s}. \\
\D \D Popular Edition, 4to. 6\textit{d}.\\
\E In the Trades, the Tropics, and the `Roaring Forties.' \\
\D \D Library Edition, 8vo.\ 21\textit{s}. \\
\D \D Cabinet Edition, crown 8vo.\ 17\textit{s}. 6\textit{d}. \\
\D \D Popular Edition, 4to. 6\textit{d}.\\
Fronde's Oceana; or, England and her Colonies. Crown 8vo.\ 2\textit{s}. boards; 2\textit{s}. 6\textit{d}. cloth.\\
Howitt's Visits to Remarkable Places. Crown 8vo.\ 7\textit{s}. 6\textit{d}.\\
Three in Norway. By Two of Them. Crown 8vo.\ Illustrations, 6\textit{s}.
\Needspace{10\baselineskip}
\begin{center}
\textbf{WORKS OF FICTION.}
\end{center}
Beaconsfield's (The Earl of) Novels and Tales. \\
\D Hughenden Edition, with 2 Portraits on Steel and 11 Vignettes on Wood. \\
\D\D 11 vols.\ crown 8vo.\ \pounds2\ 2\textit{s}.\\
\D Cheap Edition, 11 vols.\ crown 8vo.\ 1\textit{s}. each, boards; 1\textit{s}. 6\textit{d}. each, cloth.\\
\D Lothair.\\
\D Sybil.\\
\D Coningsby.\\
\D Tancred.\\
\D Venetia.\\
\D Henrietta Temple.\\
\D Contarini Fleming.\\
\D Alroy, Ixion, \&c.\\
\D The Young Duke, \&c.\\
\D Vivian Grey.\\
\D Emdymion.\\
Black Poodle (The) and other Tales. By the Author of `Vice Vers{\^a}' Cr. 8vo.\ 6\textit{s}.\\
Brabourne's (Lord) Friends and Foes from Fairyland. Crown 8vo.\ 6\textit{s}.\\
Caddy's (Mrs.) Through the Fields with Linn{\'e}s: a Chapter in Swedish History.\\
\D 2 vols.\ crown 8vo.\ 16\textit{s}.\\
Haggard's (H. Rider) She: a History of Adventure. Crown 8vo.\ 6\textit{s}.\\
Harte (Bret) On the Frontier. Three Stories. 16mo. 1\textit{s}.\\
\E By Shore and Sedge. Three Stories. 16mo. 1\textit{s}.\\
\E In the Carquinez Woods. Crown 8vo.\ 2\textit{s}. boards; 2\textit{s}. 6\textit{d}. cloth.\\
Melville's (Whyte) Novels. 8 vols.\ fcp. 8vo.\ 1\textit{s}. each, boards; 1\textit{s}. 6\textit{d}. each, cloth.\\
\D Digby Grand.\\
\D General Bounce.\\
\D Kate Coventry.\\
\D The Gladiators.\\
\D Good for Nothing.\\
\D Holmby House.\\
\D The Interpreter.\\
\D The Queen's Maries.\\
Molesworth's (Mrs.) Marrying and Giving in Marriage. Crown 8vo.\ 7\textit{s}. 6\textit{d}.\\
Novels by the Author of `The Atelier du Lys':\\
\D The Atelier du Lys; or, An Art Student In the Reign of Terror. Crown 8vo.\ 2\textit{s}. 6\textit{d}.\\
\D Mademoiselle Mori: a Tale of Modern Rome. Crown 8vo.\ 2\textit{s}. 6\textit{d}.\\
\D In the Olden Time: a Tale of the Peasant War in Germany. Crown 8vo.\ 2\textit{s}. 6\textit{d}.\\
\D Hester's Venture. Crown 8vo.\ 6\textit{s}.\\
Oliphant's (Mrs.) Madam. Crown 8vo.\ 3\textit{s}. 6\textit{d}.\\
\E In Trust: the Story of a Lady and her Lover. Crown 8vo.\ 2\textit{s}. boards; 2\textit{s}. 6\textit{d}. cloth.\\
Payn's (James) The Luck of the Darrells. Crown 8vo.\ 3\textit{s}. 6\textit{d}.\\
\E Thicker than Water. Crown 8vo.\ 2\textit{s}. boards; 3\textit{s}. 6\textit{d}. cloth.\\
Reader's Fairy Prince Follow-my-Lead. Crown 8vo.\ 5\textit{s}.\\
Reader's The Ghost of Brankinshaw; and other Tales. Fcp. 8vo.\ 2\textit{s}. 6\textit{d}.\\
Ross's (Percy) A Comedy without Laughter. Crown 8vo.\ 6\textit{s}.\\
Sewell's (Miss) Stories and Tales. Crown 8vo.\ 1\textit{s}. each, boards; \\
\D 1\textit{s}. 6\textit{d}. cloth; 2\textit{s}. 6\textit{d}. cloth extra, gilt edges.\\
\D Amy Herbert.\\
\D Cleve Hall.\\
\D The Earl's Daughter.\\
\D Experience of Life.\\
\D Gertrude.\\
\D Ivors.\\
\D A Glimpse of the World.\\
\D Katharine Ashton.\\
\D Laneton Parsonage.\\
\D Margaret Percival.\\
\D Ursula.\\
Stevenson's (R.L.) The Dynamiter. Fcp. 8vo.\ 1\textit{s}. sewed; 1\textit{s}. 6\textit{d}. cloth.\\
Stevenson's (R.L.) Strange Case of Dr. Jekyll and Mr. Hyde. Fcp. 8vo.\ 1\textit{s}. sewed; 1\textit{s}. 6\textit{d}. cloth.\\
Trollope's (Anthony) Novels. Fcp. 8vo.\ 1\textit{s}. each, boards; 1\textit{s}. 6\textit{d}. cloth.\\
\D The Warden\\
\D Barchester Towers.
\Needspace{10\baselineskip}
\begin{center}
\textbf{POETRY AND THE DRAMA.}
\end{center}
Armstrong's (Ed. J.) Poetical Works. Fcp. 8vo.\ 5\textit{s}.\\
Armstrong's (G.F.) Poetical Works:---\\
\D Poems, Lyrical and Dramatic. Fcp. 8vo.\ 6\textit{s}.\\
\D Ugone: a Tragedy. Fcp. 8vo.\ 6\textit{s}.\\
\D A Garland from Greece. Fcp. 8vo.\ 9\textit{s}.\\
\D King Saul. Fcp. 8vo.\ 5\textit{s}.\\
\D King David. Fcp. 8vo.\ 6\textit{s}.\\
\D King Solomon. Fcp. 8vo.\ 6\textit{s}.\\
\D Stories of Wicklow. Fcp. 8vo.\ 9\textit{s}.\\
Bowen's Harrow Songs and other Verses. Fcp. 8vo.\ 2\textit{s}. 6\textit{d}.; or printed on hand-made paper, 5\textit{s}.\\
Bowdler's Family Shakespeare. Medium 8vo.\ 14\textit{s}. 6 vols.\ Fcp. 8vo.\ 21\textit{s}.\\
Dante's Divine Comedy, translated by James Innes Minchin. Crown 8vo.\ 15\textit{s}.\\
Goethe's Faust, translated by Birds. Large crown 8vo.\ 12\textit{s}. 6\textit{d}.\\
\E \E translated by Webb. 8vo.\ 12\textit{s}. 6\textit{d}.\\
\E \E edited by Selss. Crown 8vo.\ 5\textit{s}.\\
Ingelow's Poems. Vols. 1 and 2, fcp. 8vo.\ 12\textit{s}. Vol. 3 fcp. 8vo.\ 5\textit{s}.\\
\E Lyrical and other Poems. Fcp. 8vo.\ 2\textit{s}. 6\textit{d}. cloth, plain; 3\textit{s}. cloth, gilt edges.\\
Macaulay's Lays of Ancient Rome, with Ivry and the Armada.\\
\D Illustrated by Weguelin. Crown 8vo.\ 3\textit{s}. 6\textit{d}. gilt edges.\\
\D The same, Popular Edition. Illustrated by Scharf. Fcp. 4to. 6\textit{d}. swd., 1\textit{s}. cloth.\\
Nesbit's Lays and Legends. Crown 8vo.\ 5\textit{s}.\\
Reader's Voices from Flowerland, a Birthday Book, 2\textit{s}. 6\textit{d}. cloth, 3\textit{s}. 6\textit{d}. roan.\\
Southey's Poetical Works. Medium 8vo.\ 14\textit{s}.\\
Stevenson's A Child's Garden of Verses. Fcp. 8vo.\ 5\textit{s}.\\
Virgil's {\AE}neid, translated by Conington. Crown 8vo.\ 9\textit{s}.\\
\E Poems, translated into English Prose. Crown 8vo.\ 9\textit{s}.
\Needspace{10\baselineskip}
\begin{center}
\textbf{AGRICULTURE, HORSES, DOGS, AND CATTLE.}
\end{center}
Dunster's How to Make the Land Pay. Crown 8vo.\ 5\textit{s}.\\
Fitzwygram's Horses and Stables. 8vo.\ 5\textit{s}.\\
Lloyd's The Science of Agriculture. 8vo.\ 12\textit{s}.\\
London's Encyclop{\ae}dia of Agriculture. 21\textit{s}.\\
Miles's Horse's Foot, and How to Keep it Sound. Imperial 8vo.\ 12\textit{s}. 6\textit{d}.\\
\E Plain Treatise on Horse-Shoeing. Post 8vo.\ 2\textit{s}. 6\textit{d}.\\
\E Remarks on Horses' Teeth. Post 8vo.\ 1\textit{s}. 6\textit{d}.\\
\E Stables and Stable-Fittings. Imperial 8vo.\ 15\textit{s}.\\
Nevile's Farms and Farming. Crown 8vo.\ 6\textit{s}.\\
\E Horses and Riding. Crown 8vo.\ 6\textit{s}.\\
Steel's Diseases of the Ox, a Manual of Bovine Pathology. 8vo.\ 15\textit{s}.\\
Stonehenge's Dog in Health and Disease. Square crown 8vo.\ 7\textit{s}. 6\textit{d}.\\
\E Greyhound. Square crown 8vo.\ 15\textit{s}.\\
Taylor's Agricultural Note Book. Fcp. 8vo.\ 2\textit{s}. 6\textit{d}.\\
Ville on Artificial Manures, by Crookes. 8vo.\ 21\textit{s}.\\
Youatt's Work on the Dog. 8vo.\ 6\textit{s}.\\
\E Work on the Horse. 8vo.\ 7\textit{s}. 6\textit{d}.
\Needspace{10\baselineskip}
\begin{center}
\textbf{SPORTS AND PASTIMES.}
\end{center}
The Badminton Library of Sports and Pastimes. Edited by the Duke of Beaufort\\
\D and A.E.T. Watson. With numerous Illustrations. Crown 8vo.\ 10\textit{s}. 6\textit{d}. each.\\
\D Hunting, by the Duke of Beaufort, \&c.\\
\D Fishing, by H. Cholmondeley-Pennell, \&c. 2 vols.\\
\D Racing, by the Earl of Suffolk, \&c.\\
\D Shooting, by Lord Walsingham, \&c. 2 vols.\\
\D Cycling. By Viscount Bury.\\
\emph{Other Volumes in preparation.}\\[3mm]
Campbell-Walker's Correct Card, or How to Play at Whist Fcp. 8vo.\ 2\textit{s}. 9\textit{d}.\\
Dead Shot (The) by Marksman. Crown 8vo.\ 10\textit{s}. 6\textit{d}.\\
Francis's Treatise on Fishing in all its Branches. Post 8vo.\ 15\textit{s}.\\
Longman's Chess Openings. Fcp. 8vo.\ 2\textit{s}. 6\textit{d}.\\
Pease's The Cleveland Hounds as a Trencher-Fed Pack. Royal 8vo.\ 18\textit{s}.\\
Pole's Theory of the Modern Scientific Game of Whist. Fcp. 8vo.\ 2\textit{s}. 6\textit{d}.\\
Proctor's How to Play Whist. Crown 8vo.\ 5\textit{s}.\\
Ronalds's Fly-Fisher's Entomology. 8vo.\ 14\textit{s}.\\
Verney's Chess Eccentricities. Crown 8vo.\ 10\textit{s}. 6\textit{d}.\\
Wilcocks's Sea-Fisherman. Post 8vo.\ 6\textit{s}.
\Needspace{10\baselineskip}
\begin{center}
\textbf{ENCYCLOP{\AE}DIAS, DICTIONARIES, AND BOOKS OF REFERENCE.}
\end{center}
\raggedbottom
Acton's Modern Cookery for Private Families. Fcp. 8vo.\ 4\textit{s}. 6\textit{d}.\\
Ayre's Treasury of Bible Knowledge. Fcp. 8vo.\ 6\textit{s}.\\
Brande's Dictionary of Science, Literature, and Art. 3 vols.\ medium 8vo.\ 63\textit{s}.\\
Cabinet Lawyer (The), a Popular Digest of the Laws of England. Fcp. 8vo.\ 9\textit{s}.\\
Cates's Dictionary of General Biography. Medium 8vo.\ 28\textit{s}.\\
Doyle's The Official Baronage of England. Vols. I.--III. 3 vols.\ 4to. \pounds5\ 5\textit{s}.\\
Gwilt's Encyclop{\ae}dia of Architecture. 8vo.\ 52\textit{s}. 6\textit{d}.\\
Keith Johnston's Dictionary of Geography, or General Gazetteer. 8vo.\ 42\textit{s}.\\
M'Culloch's Dictionary of Commerce and Commercial Navigation. 8vo.\ 63\textit{s}.\\
Maunder's Biographical Treasury. Fcp. 8vo.\ 6\textit{s}.\\
\E Historical Treasury. Fcp. 8vo.\ 6\textit{s}.\\
\E Scientific and Literary Treasury. Fcp. 8vo.\ 6\textit{s}.\\
\E Treasury of Bible Knowledge, edited by Ayre. Fcp. 8vo.\ 6\textit{s}.\\
\E Treasury of Botany, edited by Lindley \& Moore. Two Parts, 12\textit{s}.\\
\E Treasury of Geography. Fcp. 8vo.\ 6\textit{s}.\\
\E Treasury of Knowledge and Library of Reference. Fcp. 8vo.\ 6\textit{s}.\\
\E Treasury of Natural History. Fcp. 8vo.\ 6\textit{s}.\\
Quain's Dictionary of Medicine. Medium 8vo.\ 31\textit{s}. 6\textit{d}., or in 2 vols.\ 34\textit{s}.\\
Reeve's Cookery and Housekeeping. Crown 8vo.\ 7\textit{s}. 6\textit{d}.\\
Rich's Dictionary of Roman and Greek Antiquities. Crown 8vo.\ 7\textit{s}. 6\textit{d}.\\
Roget's Thesaurus of English Words and Phrases. Crown 8vo.\ 10\textit{s}. 6\textit{d}.\\
Ure's Dictionary of Arts, Manufactures, and Mines. 4 vols.\ medium 8vo, \pounds7 7\textit{s}.\\
Willich's Popular Tables, by Marriott. Crown 8vo.\ 10\textit{s}. 6\textit{d}.
\end{footnotesize}
\clearpage
\begin{center}
{\large A SELECTION}\\
{\small OF}\\
{\large EDUCATIONAL WORKS.}\\
\makebox[2in]{\hrulefill}\\
\bigskip
{\footnotesize\textbf{TEXT-BOOKS OF SCIENCE}}\\
{\tiny FULLY ILLUSTRATED.}
\end{center}
\begin{footnotesize}
Abney's Treatise on Photography. Fcp. 8vo.\ 3\textit{s}. 6\textit{d}.\\
Anderson's Strength of Materials. 3\textit{s}. 6\textit{d}.\\
Armstrong's Organic Chemistry. 3\textit{s}. 6\textit{d}.\\
Ball's Elements of Astronomy. 6\textit{s}.                  \\
Barry's Railway Appliances. 3\textit{s}. 6\textit{d}.                 \\
Bauerman's Systematic Mineralogy. 6\textit{s}.\\
\E Descriptive Mineralogy. 6\textit{s}.\\
Bloxam and Huntington's Metals. 5\textit{s}.\\
Glazebrook's Physical Optics. 6\textit{s}.\\
Glazebrook and Shaw's Practical Physics. 6\textit{s}.\\
Gore's Art of Electro-Metallurgy. 6\textit{s}.\\
Griffin's Algebra and Trigonometry. 3\textit{s}. 6\textit{d}. Notes and Solutions, 3\textit{s}. 6\textit{d}.\\
Holmes's The Steam Engine. 6\textit{s}.\\
Jenkin's Electricity and Magnetism. 3\textit{s}. 6\textit{d}.\\
Maxwell's Theory of Heat. 3\textit{s}. 6\textit{d}.\\
Merrifield's Technical Arithmetic and Mensuration. 3\textit{s}. 6\textit{d}. Key, 3\textit{s}. 6\textit{d}.\\
Miller's Inorganic Chemistry. 3\textit{s}. 6\textit{d}.\\
Preece and Sivewright's Telegraphy. 5\textit{s}.\\
Rutley's Study of Rocks, a Text-Book of Petrology. 4\textit{s}. 6\textit{d}.\\
Shelley's Workshop Appliances. 4\textit{s}. 6\textit{d}.\\
Thom{\'e}'s Structural and Physiological Botany. 6\textit{s}.\\
Thorpe's Quantitative Chemical Analysis. 4\textit{s}. 6\textit{d}.\\
Thorpe and Muir's Qualitative Analysis. 3\textit{s}. 6\textit{d}.\\
Tilden's Chemical Philosophy. 3\textit{s}. 6\textit{d}. With Answers to Problems. 4\textit{s}. 6\textit{d}.\\
Unwin's Elements of Machine Design. 6\textit{s}.\\
Watson's Plane and Solid Geometry. 3\textit{s}. 6\textit{d}.
\Needspace{10\baselineskip}
\begin{center}
\textbf{THE GREEK LANGUAGE.}
\end{center}
Bloomfield's College and School Greek Testament. Fcp. 8vo.\ 5\textit{s}.\\
Bolland \& Lang's Politics of Aristotle. Post 8vo.\ 7\textit{s}. 6\textit{d}.\\
Collis's Chief Tenses of the Greek Irregular Verbs. 8vo.\ 1\textit{s}.\\
\E Pontes Gr{\ae}ci, Stepping-Stone to Greek Grammar. 12mo. 3\textit{s}. 6\textit{d}.\\
\E Praxis Gr{\ae}ca, Etymology. 12mo. 2\textit{s}. 6\textit{d}.\\
\E Greek Verse-Book, Praxis Iambica. 12mo. 4\textit{s}. 6\textit{d}.\\
Farrar's Brief Greek Syntax and Accidence. 12mo. 4\textit{s}. 6\textit{d}.\\
\E Greek Grammar Rules for Harrow School, 12mo. 1\textit{s}. 6\textit{d}.\\
Geare's Notes on Thucydides. Book I. Fcp. 8vo.\ 2\textit{s}. 6\textit{d}.\\
Hewitt's Greek Examination-Papers. 12mo. 1\textit{s}. 6\textit{d}.\\
Isbister's Xenophon's Anabasis, Books I. to III. with Notes. 12mo. 3\textit{s}. 6\textit{d}.\\
Jerram's Graec{\`e} Reddenda, Crown 8vo.\ 1\textit{s}. 6\textit{d}.\\
Kennedy's Greek Grammar. 12mo. 4\textit{s}. 6\textit{d}.\\
Liddell \& Scott's English-Greek Lexicon. 4to. 36\textit{s}.; Square 12mo. 7\textit{s}. 6\textit{d}.\\
Mahaffy's Classical Greek Literature. Crown 8vo.\ Poets, 7\textit{s}. 6\textit{d}. Prose Writers, 7\textit{s}. 6\textit{d}.\\
Morris's Greek Lessons. Square 18mo. Part I. 2\textit{s}. 6\textit{d}.; Part II. 1\textit{s}.\\
Parry's Elementary Greek Grammar. 12mo. 3\textit{s}. 6\textit{d}.\\
Plato's Republic, Book I. Greek Text, English Notes by Hardy. Crown 8vo.\ 3\textit{s}.\\
Sheppard and Evans's Notes on Thucydides. Crown 8vo.\ 7\textit{s}. 6\textit{d}.\\
Thucydides, Book IV. with Notes by Barton and Chavasse. Crown 8vo.\ 5\textit{s}.\\
Valpy's Greek Delectus, improved by White. 12mo. 2\textit{s}. 6\textit{d}. Key, 2\textit{s}. 6\textit{d}.\\
White's Xenophon's Expedition of Cyrus, with English Notes. 12mo. 7\textit{s}. 6\textit{d}.\\
Wilkins's Manual of Greek Prose Composition. Crown 8vo.\ 5\textit{s}. Key, 5\textit{s}.\\
\E Exercises in Greek Prose Composition. Crown 8vo.\ 4\textit{s}. 6\textit{d}. Key, 2\textit{s}. 6\textit{d}.\\
\E New Greek Delectus. Crown 8vo.\ 3\textit{s}. 6\textit{d}. Key, 2\textit{s}. 6\textit{d}.\\
\E Progressive Greek Delectus. 12mo. 4\textit{s}. Key, 2\textit{s}. 6\textit{d}.\\
\E Progressive Greek Anthology. 12mo. 5\textit{s}.\\
\E Scriptores Attici, Excerpts with English Notes. Crown 8vo.\ 7\textit{s}. 6\textit{d}.\\
\E Speeches from Thucydides translated. Post 8vo.\ 6\textit{s}.\\
Yonge's English-Greek Lexicon. 4to. 21\textit{s}.; Square 12mo. 8\textit{s}. 6\textit{d}.
\Needspace{10\baselineskip}
\begin{center}
\textbf{THE LATIN LANGUAGE.}
\end{center}
Bradley's Latin Prose Exercises. 12mo. 3\textit{s}. 6\textit{d}. Key, 5\textit{s}.\\
\E Continuous Lessons in Latin Prose. 12mo. 5\textit{s}. Key, 5\textit{s}. 6\textit{d}.\\
\E Cornelius Nepos, improved by White. 12mo. 3\textit{s}. 6\textit{d}.\\
\E Eutropius, improved by White. 12mo. 2\textit{s}. 6\textit{d}.\\
\E Ovid's Metamorphoses, improved by White. 12mo. 4\textit{s}. 6\textit{d}.\\
\E Select Fables of Ph{\'e}rus, improved by White. 12mo. 2\textit{s}. 6\textit{d}.\\
Collis's Chief Tenses of Latin Irregular Verbs. 8vo.\ 1\textit{s}.\\
Collis's Pontes Latini, Stepping-Stone to Latin Grammar. 12mo. 3\textit{s}. 6\textit{d}.\\
Hewitt's Latin Examination-Papers. 12mo. 1\textit{s}. 6\textit{d}.\\
Isbister's C{\ae}sar, Books I.-VII. 12mo. 4\textit{s}.; or with Reading Lessons, 4\textit{s}. 6\textit{d}.\\
Isbister's C{\ae}sar's Commentaries, Books I.-V. 12 mo. 3\textit{s}. 6\textit{d}.\\
Isbister's First Book of C{\ae}sar's Gallic War. 12mo. 1\textit{s}. 6\textit{d}.\\
Jerram's Latine Reddenda. Crown 8vo.\ 1\textit{s}. 6\textit{d}.\\
Kennedy's Child's Latin Primer, or First Latin Lessons. 12mo. 2\textit{s}.\\
\E Child's Latin Accidence. 12mo. 1\textit{s}.\\
\E Elementary Latin Grammar. 12mo. 3\textit{s}. 6\textit{d}.\\
\E Elementary Latin Reading Book, or Tirocinium Latinum. 12mo. 2\textit{s}.\\
\E Latin Prose, Pal{\'e}tra Stili Latini. 12mo. 6\textit{s}.\\
\E Latin Vocabulary. 12mo. 2\textit{s}. 6\textit{d}.\\
\E Subsidia Primaria, Exercise Books to the Public School Latin Primer.\\
\D\D I\@. Accidence and Simple Construction, 2\textit{s}. 6\textit{d}. II\@. Syntax, 3\textit{s}. 6\textit{d}.\\
\E Key to the Exercises in Subsidia Primaria, Parts I. and II.\ price 5\textit{s}.\\
\E Subsidia Primaria, III. the Latin Compound Sentence. 12mo. 1\textit{s}.\\
\E Curriculum Stili Latini. 12mo. 4\textit{s}. 6\textit{d}. Key, 7\textit{s}. 6\textit{d}.\\
\E Pal{\'e}tra Latina, or Second Latin Reading Book. 12mo. 5\textit{s}.\\
Millington's Latin Prose Composition. Crown 8vo.\ 3\textit{s}. 6\textit{d}.\\
\E  Selections from Latin Prose. Crown 8vo.\ 2\textit{s}. 6\textit{d}.\\
Moody's Eton Latin Grammar. 12mo. 2\textit{s}. 6\textit{d}. The Accidence separately, 1\textit{s}.\\
Morris's Elementa Latina. Fcp. 8vo.\ 1\textit{s}. 6\textit{d}. Key, 2\textit{s}. 6\textit{d}.\\
Parry's Origines Roman{\ae}, from Livy, with English Notes. Crown 8vo.\ 4\textit{s}.\\
The Public School Latin Primer. 12mo. 2\textit{s}. 6\textit{d}.\\
\E \E \E \E \E Grammar, by Rev. Dr. Kennedy. Post 8vo.\ 7\textit{s}. 6\textit{d}.\\
Prendergast's Mastery Series, Manual of Latin. 12mo. 2\textit{s}. 6\textit{d}.\\
Rapier's Introduction to Composition of Latin Verse. 12mo. 3\textit{s}. 6\textit{d}. Key, 2\textit{s}. 6\textit{d}.\\
Sheppard and Turner's Aids to Classical Study. 12mo. 5\textit{s}. Key, 6\textit{s}.\\
Valpy's Latin Delectus, improved by White. 12mo. 2\textit{s}. 6\textit{d}. Key, 3\textit{s}. 6\textit{d}.\\
Virgil's {\AE}neid, translated into English Verse by Conington. Crown 8vo.\ 9\textit{s}.\\
\E  Works, edited by Kennedy. Crown 8vo.\ 10\textit{s}. 6\textit{d}.\\
\E \E \D translated into English Prose by Conington. Crown 8vo.\ 9\textit{s}.\\
Walford's Progressive Exercises in Latin Elegiac Verse. 12mo. 2\textit{s}. 6\textit{d}. Key, 5\textit{s}.\\
White and Riddle's Large Latin-English Dictionary. I. vol.\ 4to. 21\textit{s}.\\
\E  Concise Latin-Eng. Dictionary for University Students. Royal 8vo.\ 12\textit{s}.\\
\E  Junior Students' Eng.-Lat. \& Lat.-Eng. Dictionary. Square 12mo. 5\textit{s}. Separately\\
\D \D The Latin-English Dictionary, price 3\textit{s}.\\
\D \D The English-Latin Dictionary, price 3\textit{s}.\\
Yonge's Latin Gradus. Post 8vo.\ 9\textit{s}.; or with Appendix, 12\textit{s}.
\Needspace{10\baselineskip}
\begin{center}
\textbf{WHITE'S GRAMMAR-SCHOOL GREEK TEXTS.}
\end{center}
{\AE}sop (Fables) \& Pal{\ae}phatus (Myths). 32mo. 1\textit{s}.\\
Euripides, Hecuba. 2\textit{s}.\\
Homer, Iliad, Book I. 1\textit{s}.\\
\E Odyssey, Book I. 1\textit{s}.\\
Lucian, Select Dialogues. 1\textit{s}.\\
Xenophon, Anabasis, Books I\@. III\@. IV\@. V\@. \& VI. 1\textit{s}.
6\textit{d}. each; Book II. 1\textit{s}.; Book VII. 2\textit{s}.\\
Xenophon, Book I. without Vocabulary.   3\textit{d}.\\
St. Matthew's and St. Luke's Gospels. 2\textit{s}. 6\textit{d}. each.\\
St. Mark's and St. John's Gospels. 1\textit{s}. 6\textit{d}. each.\\
The Acts of the Apostles. 2\textit{s}. 6\textit{d}.\\
St. Paul's Epistle to the Romans. 1\textit{s}. 6\textit{d}.\\
The Four Gospels in Greek, with Greek-English Lexicon. \\
\D Edited by John T. White, D.D. Oxon. Square 32mo.\ price 5\textit{s}.
\Needspace{10\baselineskip}
\begin{center}
\textbf{WHITE'S GRAMMAR-SCHOOL LATIN TEXTS.}
\end{center}
C{\ae}sar, Gallic War, Books I\@. \& II\@. V\@. \& VI\@. 1\textit{s}. each.
Book I. without Vocabulary, 3\textit{d}.\\
C{\ae}sar, Gallic War, Books III. \& IV. 9\textit{d}. each.\\
C{\ae}sar, Gallic War, Book VII. 1\textit{s}. 6\textit{d}.\\
Cicero, Cato Major (Old Age). 1\textit{s}. 6\textit{d}.\\
Cicero, L{\ae}lius (Friendship). 1\textit{s}. 8\textit{d}.\\
Eutropius, Roman History, Books I. \& II. 1\textit{s}. Books III. \& IV. 1\textit{s}.\\
Horace, Odes, Books I. II. \& IV. 1\textit{s}. each.\\
Horace, Odes, Book III. 1\textit{s}. 6\textit{d}.\\
Horace, Epodes and Carmen Seculare. 1\textit{s}.\\
Nepos, Miltiades, Simon, Pausanias, Aristides. 9\textit{d}.\\
Ovid. Selections from Epistles and Fasti, 1\textit{s}.\\
Ovid, Select Myths from Metamorphoses. 9\textit{d}.\\
Ph{\ae}drus, Select Easy Fables.\\
Ph{\ae}drus, Fables, Books I. \& II. 1\textit{s}.\\
Sallust, Bellum Catilinarium. 1\textit{s}. 6\textit{d}.\\
Virgil, Georgics, Book IV. 1\textit{s}.\\
Virgil, {\AE}neid, Books I. to VI. 1\textit{s}. each. Book I. without Vocabulary, 3\textit{d}.\\
Virgil, {\AE}neid, Books VII\@. VIII\@. X\@. XI\@. XII\@. 1\textit{s}. 6\textit{d}. each.
\Needspace{10\baselineskip}
\begin{center}
\textbf{THE FRENCH LANGUAGE.}
\end{center}
Albit\'es's How to Speak French. Fcp. 8vo.\ 5\textit{s}. 6\textit{d}.\\
\E Instantaneous French Exercises. Fcp. 2\textit{s}. Key, 2\textit{s}.\\
Cassal's French Genders. Crown 8vo.\ 3\textit{s}. 6\textit{d}.\\
Cassal \& Karcher's Graduated French Translation Book.  Part I. 3\textit{s}. 6\textit{d}.\\
Part II. 5\textit{s}. Key to Part I. by Professor Cassal, price 5\textit{s}.\\
Contanseau's Practical French and English Dictionary. Post 8vo.\ 3\textit{s}. 6\textit{d}.\\
\E Pocket French and English Dictionary. Square 18mo. 1\textit{s}. 6\textit{d}.\\
\E Premi{\'e}res Lectures. 12mo. 2\textit{s}. 6\textit{d}.\\
\E First Step in French. 12mo. 2\textit{s}. 6\textit{d}. Key, 3\textit{s}.\\
\E French Accidence. 12mo. 2\textit{s}. 6\textit{d}.\\
\E French Grammar. 12mo. 4\textit{s}. Key, 6\textit{s}.\\
Contanseau's Middle-Class French Course. Fcp. 8vo.:---\\
\D Accidence, 8\textit{d}.\\
\D Syntax, 8\textit{d}.\\
\D French Conversation-Book, 8\textit{d}.\\
\D First French Exercise-Book, 8\textit{d}.\\
\D Second French Exercise-Book, 8\textit{d}.\\
\D French Translation-Book, 8\textit{d}.\\
\D Easy French Delectus, 8\textit{d}.\\
\D First French Reader, 8\textit{d}.\\
\D Second French Reader, 8\textit{d}.\\
\D French and English Dialogues, 8\textit{d}.\\
Contanseau's Guide to French Translation. 12mo. 3\textit{s}. 6\textit{d}. Key 3\textit{s}. 6\textit{d}.\\
\E Prosateurs et Po{\'e}tes Fran{\c c}ais. 12mo. 5\textit{s}.\\
\E Pr{\'e}cis de la Litt{\'e}rature Fran{\c c}aise. 12mo. 3\textit{s}. 6\textit{d}.\\
\E Abr{\'e}g{\'e} de l'Histoire de France. 12mo. 2\textit{s}. 6\textit{d}.\\
F{\'e}val's Chonans et Bleus, with Notes by C. Sankey, M.A., Fcp. 8vo.\ 2\textit{s}. 6\textit{d}.\\
Jerram's Sentences for Translation into French. Cr. 8vo.\ 1\textit{s}. Key, 2\textit{s}. 6\textit{d}.\\
Prendergast's Mastery Series, French. 12mo. 2\textit{s}. 6\textit{d}.\\
Souvestre's Philosophe sous les Toits, by Sti{\'e}venard. Square 18 mo. 1\textit{s}. 6\textit{d}.\\
Stepping-Stone to French Pronunciation. 18 mo. 1\textit{s}.\\
Sti{\'e}venard's Lectures Fran{\c c}aises from Modern Authors. 12mo. 4\textit{s}. 6\textit{d}.\\
\E Rules and Exercises on the French Language. 12mo. 3\textit{s}. 6\textit{d}.\\
Tarver's Eton French Grammar. 12mo. 6\textit{s}. 6\textit{d}.
\Needspace{10\baselineskip}
\begin{center}
\textbf{THE GERMAN LANGUAGE.}
\end{center}
\raggedbottom
Blackley's Practical German and English Dictionary. Post 8vo.\ 3\textit{s}. 6\textit{d}.\\
Buchheim's German Poetry, for Repetition. 18mo. 1\textit{s}. 6\textit{d}.\\
Collis's Card of German Irregular Verbs. 8vo.\ 2\textit{s}.\\
Fischer-Fischart's Elementary German Grammar. Fcp. 8vo.\ 2\textit{s}. 6\textit{d}.\\
Just's German Grammar. 12mo. 1\textit{s}. 6\textit{d}.\\
\E German Reading Book. 12mo. 3\textit{s}. 6\textit{d}.\\
Longman's Pocket German and English Dictionary. Square 18mo. 2\textit{s}. 6\textit{d}.\\
Naftel's Elementary German Course for Public Schools. Fcp. 8vo.\\
\D German Accidence. 9\textit{d}.\\
\D German Syntax. 9\textit{d}.\\
\D First German Exercise-Book. 9\textit{d}.\\
\D Second German Exercise-Book. 9\textit{d}.\\
\D German Prose Composition Book. 9\textit{d}.\\
\D First German Reader. 9\textit{d}.\\
\D Second German Reader. 9\textit{d}.\\
Prendergast's Mastery Series, German. 12mo. 2\textit{s}. 6\textit{d}.\\
Quick's Essentials of German. Crown 8vo.\ 3\textit{s}. 6\textit{d}.\\
Selss's School Edition of Goethe's Faust. Crown 8vo.\ 5\textit{s}.\\
Selss's Outline of German Literature. Crown 8vo.\ 4\textit{s}. 6\textit{d}.\\
Wirth's German Chit-Chat. Crown 8vo.\ 2\textit{s}. 6\textit{d}.
\end{footnotesize}
\clearpage
% {Spottiswoode \& Co. Printers, New-street Square, London.}
\noindent
Production Note\\[5mm]
Cornell University Library produced this volume to replace
the irreparably deteriorated original. It was scanned using
Xerox software and equipment at 600 dots per inch resolution
and compressed prior to storage using CCITT Group 4
compression. The digital data were used to create Cornell's
replacement volume on paper that meets the ANSI Standard
Z39.48-1984. The production of this volume was supported in
part by the Commission on Preservation and Access and the
Xerox Corporation. 1991.
\newpage

\small
\pagenumbering{gobble}
\begin{verbatim}

End of Project Gutenberg's Chance and Luck, by Richard Proctor

*** END OF THIS PROJECT GUTENBERG EBOOK CHANCE AND LUCK ***

*** This file should be named 17224-t.tex or 17224-t.zip ***
*** or                    17224-pdf.pdf or 17224-pdf.pdf ***
This and all associated files of various formats will be found in:
        https://www.gutenberg.org/1/7/2/2/17224/

Produced by Joshua Hutchinson, Roger Frank and the Online
Distributed Proofreading Team at https://www.pgdp.net
This file was produced from images from the Cornell
University Library: Historical Mathematics Monographs collection.


Updated editions will replace the previous one--the old editions
will be renamed.

Creating the works from public domain print editions means that no
one owns a United States copyright in these works, so the Foundation
(and you!) can copy and distribute it in the United States without
permission and without paying copyright royalties.  Special rules,
set forth in the General Terms of Use part of this license, apply to
copying and distributing Project Gutenberg-tm electronic works to
protect the PROJECT GUTENBERG-tm concept and trademark.  Project
Gutenberg is a registered trademark, and may not be used if you
charge for the eBooks, unless you receive specific permission.  If
you do not charge anything for copies of this eBook, complying with
the rules is very easy.  You may use this eBook for nearly any
purpose such as creation of derivative works, reports, performances
and research.  They may be modified and printed and given away--you
may do practically ANYTHING with public domain eBooks.
Redistribution is subject to the trademark license, especially
commercial redistribution.



*** START: FULL LICENSE ***

THE FULL PROJECT GUTENBERG LICENSE PLEASE READ THIS BEFORE YOU
DISTRIBUTE OR USE THIS WORK

To protect the Project Gutenberg-tm mission of promoting the free
distribution of electronic works, by using or distributing this work
(or any other work associated in any way with the phrase "Project
Gutenberg"), you agree to comply with all the terms of the Full
Project Gutenberg-tm License (available with this file or online at
https://gutenberg.org/license).


Section 1.  General Terms of Use and Redistributing Project
Gutenberg-tm electronic works

1.A.  By reading or using any part of this Project Gutenberg-tm
electronic work, you indicate that you have read, understand, agree
to and accept all the terms of this license and intellectual
property (trademark/copyright) agreement.  If you do not agree to
abide by all the terms of this agreement, you must cease using and
return or destroy all copies of Project Gutenberg-tm electronic
works in your possession. If you paid a fee for obtaining a copy of
or access to a Project Gutenberg-tm electronic work and you do not
agree to be bound by the terms of this agreement, you may obtain a
refund from the person or entity to whom you paid the fee as set
forth in paragraph 1.E.8.

1.B.  "Project Gutenberg" is a registered trademark.  It may only be
used on or associated in any way with an electronic work by people
who agree to be bound by the terms of this agreement.  There are a
few things that you can do with most Project Gutenberg-tm electronic
works even without complying with the full terms of this agreement.
See paragraph 1.C below.  There are a lot of things you can do with
Project Gutenberg-tm electronic works if you follow the terms of
this agreement and help preserve free future access to Project
Gutenberg-tm electronic works.  See paragraph 1.E below.

1.C.  The Project Gutenberg Literary Archive Foundation ("the
Foundation" or PGLAF), owns a compilation copyright in the
collection of Project Gutenberg-tm electronic works.  Nearly all the
individual works in the collection are in the public domain in the
United States.  If an individual work is in the public domain in the
United States and you are located in the United States, we do not
claim a right to prevent you from copying, distributing, performing,
displaying or creating derivative works based on the work as long as
all references to Project Gutenberg are removed.  Of course, we hope
that you will support the Project Gutenberg-tm mission of promoting
free access to electronic works by freely sharing Project
Gutenberg-tm works in compliance with the terms of this agreement
for keeping the Project Gutenberg-tm name associated with the work.
You can easily comply with the terms of this agreement by keeping
this work in the same format with its attached full Project
Gutenberg-tm License when you share it without charge with others.

1.D.  The copyright laws of the place where you are located also
govern what you can do with this work.  Copyright laws in most
countries are in a constant state of change.  If you are outside the
United States, check the laws of your country in addition to the
terms of this agreement before downloading, copying, displaying,
performing, distributing or creating derivative works based on this
work or any other Project Gutenberg-tm work.  The Foundation makes
no representations concerning the copyright status of any work in
any country outside the United States.

1.E.  Unless you have removed all references to Project Gutenberg:

1.E.1.  The following sentence, with active links to, or other
immediate access to, the full Project Gutenberg-tm License must
appear prominently whenever any copy of a Project Gutenberg-tm work
(any work on which the phrase "Project Gutenberg" appears, or with
which the phrase "Project Gutenberg" is associated) is accessed,
displayed, performed, viewed, copied or distributed:

This eBook is for the use of anyone anywhere at no cost and with
almost no restrictions whatsoever.  You may copy it, give it away or
re-use it under the terms of the Project Gutenberg License included
with this eBook or online at www.gutenberg.org

1.E.2.  If an individual Project Gutenberg-tm electronic work is
derived from the public domain (does not contain a notice indicating
that it is posted with permission of the copyright holder), the work
can be copied and distributed to anyone in the United States without
paying any fees or charges.  If you are redistributing or providing
access to a work with the phrase "Project Gutenberg" associated with
or appearing on the work, you must comply either with the
requirements of paragraphs 1.E.1 through 1.E.7 or obtain permission
for the use of the work and the Project Gutenberg-tm trademark as
set forth in paragraphs 1.E.8 or 1.E.9.

1.E.3.  If an individual Project Gutenberg-tm electronic work is
posted with the permission of the copyright holder, your use and
distribution must comply with both paragraphs 1.E.1 through 1.E.7
and any additional terms imposed by the copyright holder.
Additional terms will be linked to the Project Gutenberg-tm License
for all works posted with the permission of the copyright holder
found at the beginning of this work.

1.E.4.  Do not unlink or detach or remove the full Project
Gutenberg-tm License terms from this work, or any files containing a
part of this work or any other work associated with Project
Gutenberg-tm.

1.E.5.  Do not copy, display, perform, distribute or redistribute
this electronic work, or any part of this electronic work, without
prominently displaying the sentence set forth in paragraph 1.E.1
with active links or immediate access to the full terms of the
Project Gutenberg-tm License.

1.E.6.  You may convert to and distribute this work in any binary,
compressed, marked up, nonproprietary or proprietary form, including
any word processing or hypertext form.  However, if you provide
access to or distribute copies of a Project Gutenberg-tm work in a
format other than "Plain Vanilla ASCII" or other format used in the
official version posted on the official Project Gutenberg-tm web
site (www.gutenberg.org), you must, at no additional cost, fee or
expense to the user, provide a copy, a means of exporting a copy, or
a means of obtaining a copy upon request, of the work in its
original "Plain Vanilla ASCII" or other form.  Any alternate format
must include the full Project Gutenberg-tm License as specified in
paragraph 1.E.1.

1.E.7.  Do not charge a fee for access to, viewing, displaying,
performing, copying or distributing any Project Gutenberg-tm works
unless you comply with paragraph 1.E.8 or 1.E.9.

1.E.8.  You may charge a reasonable fee for copies of or providing
access to or distributing Project Gutenberg-tm electronic works
provided that

- You pay a royalty fee of 20% of the gross profits you derive from
   the use of Project Gutenberg-tm works calculated using the method
   you already use to calculate your applicable taxes.  The fee is
   owed to the owner of the Project Gutenberg-tm trademark, but he
   has agreed to donate royalties under this paragraph to the
   Project Gutenberg Literary Archive Foundation.  Royalty payments
   must be paid within 60 days following each date on which you
   prepare (or are legally required to prepare) your periodic tax
   returns.  Royalty payments should be clearly marked as such and
   sent to the Project Gutenberg Literary Archive Foundation at the
   address specified in Section 4, "Information about donations to
   the Project Gutenberg Literary Archive Foundation."

- You provide a full refund of any money paid by a user who notifies
   you in writing (or by e-mail) within 30 days of receipt that s/he
   does not agree to the terms of the full Project Gutenberg-tm
   License.  You must require such a user to return or
   destroy all copies of the works possessed in a physical medium
   and discontinue all use of and all access to other copies of
   Project Gutenberg-tm works.

- You provide, in accordance with paragraph 1.F.3, a full refund of
   any money paid for a work or a replacement copy, if a defect in
   the electronic work is discovered and reported to you within 90
   days of receipt of the work.

- You comply with all other terms of this agreement for free
   distribution of Project Gutenberg-tm works.

1.E.9.  If you wish to charge a fee or distribute a Project
Gutenberg-tm electronic work or group of works on different terms
than are set forth in this agreement, you must obtain permission in
writing from both the Project Gutenberg Literary Archive Foundation
and Michael Hart, the owner of the Project Gutenberg-tm trademark.
Contact the Foundation as set forth in Section 3 below.

1.F.

1.F.1.  Project Gutenberg volunteers and employees expend
considerable effort to identify, do copyright research on,
transcribe and proofread public domain works in creating the Project
Gutenberg-tm collection.  Despite these efforts, Project
Gutenberg-tm electronic works, and the medium on which they may be
stored, may contain "Defects," such as, but not limited to,
incomplete, inaccurate or corrupt data, transcription errors, a
copyright or other intellectual property infringement, a defective
or damaged disk or other medium, a computer virus, or computer codes
that damage or cannot be read by your equipment.

1.F.2.  LIMITED WARRANTY, DISCLAIMER OF DAMAGES - Except for the
"Right of Replacement or Refund" described in paragraph 1.F.3, the
Project Gutenberg Literary Archive Foundation, the owner of the
Project Gutenberg-tm trademark, and any other party distributing a
Project Gutenberg-tm electronic work under this agreement, disclaim
all liability to you for damages, costs and expenses, including
legal fees.  YOU AGREE THAT YOU HAVE NO REMEDIES FOR NEGLIGENCE,
STRICT LIABILITY, BREACH OF WARRANTY OR BREACH OF CONTRACT EXCEPT
THOSE PROVIDED IN PARAGRAPH F3.  YOU AGREE THAT THE FOUNDATION, THE
TRADEMARK OWNER, AND ANY DISTRIBUTOR UNDER THIS AGREEMENT WILL NOT
BE LIABLE TO YOU FOR ACTUAL, DIRECT, INDIRECT, CONSEQUENTIAL,
PUNITIVE OR INCIDENTAL DAMAGES EVEN IF YOU GIVE NOTICE OF THE
POSSIBILITY OF SUCH DAMAGE.

1.F.3.  LIMITED RIGHT OF REPLACEMENT OR REFUND - If you discover a
defect in this electronic work within 90 days of receiving it, you
can receive a refund of the money (if any) you paid for it by
sending a written explanation to the person you received the work
from.  If you received the work on a physical medium, you must
return the medium with your written explanation.  The person or
entity that provided you with the defective work may elect to
provide a replacement copy in lieu of a refund.  If you received the
work electronically, the person or entity providing it to you may
choose to give you a second opportunity to receive the work
electronically in lieu of a refund.  If the second copy is also
defective, you may demand a refund in writing without further
opportunities to fix the problem.

1.F.4.  Except for the limited right of replacement or refund set
forth in paragraph 1.F.3, this work is provided to you 'AS-IS', WITH
NO OTHER WARRANTIES OF ANY KIND, EXPRESS OR IMPLIED, INCLUDING BUT
NOT LIMITED TO WARRANTIES OF MERCHANTIBILITY OR FITNESS FOR ANY
PURPOSE.

1.F.5.  Some states do not allow disclaimers of certain implied
warranties or the exclusion or limitation of certain types of
damages. If any disclaimer or limitation set forth in this agreement
violates the law of the state applicable to this agreement, the
agreement shall be interpreted to make the maximum disclaimer or
limitation permitted by the applicable state law.  The invalidity or
unenforceability of any provision of this agreement shall not void
the remaining provisions.

1.F.6.  INDEMNITY - You agree to indemnify and hold the Foundation,
the trademark owner, any agent or employee of the Foundation, anyone
providing copies of Project Gutenberg-tm electronic works in
accordance with this agreement, and any volunteers associated with
the production, promotion and distribution of Project Gutenberg-tm
electronic works, harmless from all liability, costs and expenses,
including legal fees, that arise directly or indirectly from any of
the following which you do or cause to occur: (a) distribution of
this or any Project Gutenberg-tm work, (b) alteration, modification,
or additions or deletions to any Project Gutenberg-tm work, and (c)
any Defect you cause.


Section  2.  Information about the Mission of Project Gutenberg-tm

Project Gutenberg-tm is synonymous with the free distribution of
electronic works in formats readable by the widest variety of
computers including obsolete, old, middle-aged and new computers.
It exists because of the efforts of hundreds of volunteers and
donations from people in all walks of life.

Volunteers and financial support to provide volunteers with the
assistance they need, is critical to reaching Project Gutenberg-tm's
goals and ensuring that the Project Gutenberg-tm collection will
remain freely available for generations to come.  In 2001, the
Project Gutenberg Literary Archive Foundation was created to provide
a secure and permanent future for Project Gutenberg-tm and future
generations. To learn more about the Project Gutenberg Literary
Archive Foundation and how your efforts and donations can help, see
Sections 3 and 4 and the Foundation web page at
https://www.pglaf.org.


Section 3.  Information about the Project Gutenberg Literary Archive
Foundation

The Project Gutenberg Literary Archive Foundation is a non profit
501(c)(3) educational corporation organized under the laws of the
state of Mississippi and granted tax exempt status by the Internal
Revenue Service.  The Foundation's EIN or federal tax identification
number is 64-6221541.  Its 501(c)(3) letter is posted at
https://pglaf.org/fundraising.  Contributions to the Project
Gutenberg Literary Archive Foundation are tax deductible to the full
extent permitted by U.S. federal laws and your state's laws.

The Foundation's principal office is located at 4557 Melan Dr. S.
Fairbanks, AK, 99712., but its volunteers and employees are
scattered throughout numerous locations.  Its business office is
located at 809 North 1500 West, Salt Lake City, UT 84116, (801)
596-1887, email business@pglaf.org.  Email contact links and up to
date contact information can be found at the Foundation's web site
and official page at https://pglaf.org

For additional contact information:
     Dr. Gregory B. Newby
     Chief Executive and Director
     gbnewby@pglaf.org

Section 4.  Information about Donations to the Project Gutenberg
Literary Archive Foundation

Project Gutenberg-tm depends upon and cannot survive without wide
spread public support and donations to carry out its mission of
increasing the number of public domain and licensed works that can
be freely distributed in machine readable form accessible by the
widest array of equipment including outdated equipment.  Many small
donations ($1 to $5,000) are particularly important to maintaining
tax exempt status with the IRS.

The Foundation is committed to complying with the laws regulating
charities and charitable donations in all 50 states of the United
States.  Compliance requirements are not uniform and it takes a
considerable effort, much paperwork and many fees to meet and keep
up with these requirements.  We do not solicit donations in
locations where we have not received written confirmation of
compliance.  To SEND DONATIONS or determine the status of compliance
for any particular state visit https://pglaf.org

While we cannot and do not solicit contributions from states where
we have not met the solicitation requirements, we know of no
prohibition against accepting unsolicited donations from donors in
such states who approach us with offers to donate.

International donations are gratefully accepted, but we cannot make
any statements concerning tax treatment of donations received from
outside the United States.  U.S. laws alone swamp our small staff.

Please check the Project Gutenberg Web pages for current donation
methods and addresses.  Donations are accepted in a number of other
ways including including checks, online payments and credit card
donations.  To donate, please visit: https://pglaf.org/donate


Section 5.  General Information About Project Gutenberg-tm
electronic works.

Professor Michael S. Hart was the originator of the Project
Gutenberg-tm concept of a library of electronic works that could be
freely shared with anyone.  For thirty years, he produced and
distributed Project Gutenberg-tm eBooks with only a loose network of
volunteer support.

Project Gutenberg-tm eBooks are often created from several printed
editions, all of which are confirmed as Public Domain in the U.S.
unless a copyright notice is included.  Thus, we do not necessarily
keep eBooks in compliance with any particular paper edition.

Most people start at our Web site which has the main PG search
facility:

     https://www.gutenberg.org

This Web site includes information about Project Gutenberg-tm,
including how to make donations to the Project Gutenberg Literary
Archive Foundation, how to help produce our new eBooks, and how to
subscribe to our email newsletter to hear about new eBooks.

*** END: FULL LICENSE ***
\end{verbatim}
\end{document}
---------------------------------------------------------
Below is appended the log from the most recent compile.
You may use it to compare against a log from a new
compile to help spot differences.
---------------------------------------------------------
This is e-TeX, Version 3.141592-2.2 (MiKTeX 2.4) (preloaded format=latex 2005.4.11)  4 DEC 2005 15:42
entering extended mode
**17224-t.tex
(17224-t.tex
LaTeX2e <2003/12/01>
Babel <v3.8a> and hyphenation patterns for english, french, german, ngerman, du
mylang, nohyphenation, loaded.
(C:\texmf\tex\latex\memoir\memoir.cls
Document Class: memoir 2004/04/05 v1.61 configurable document class
\onelineskip=\skip41
\lxvchars=\skip42
\xlvchars=\skip43
\@memcnta=\count79
\stockheight=\skip44
\stockwidth=\skip45
\trimtop=\skip46
\trimedge=\skip47
(C:\texmf\tex\latex\memoir\mem12.clo
File: mem12.clo 2004/03/12 v0.3 memoir class 12pt size option
)
\spinemargin=\skip48
\foremargin=\skip49
\uppermargin=\skip50
\lowermargin=\skip51
\headdrop=\skip52
\normalrulethickness=\skip53
\headwidth=\skip54
\c@storedpagenumber=\count80
\thanksmarkwidth=\skip55
\thanksmarksep=\skip56
\droptitle=\skip57
\abstitleskip=\skip58
\absleftindent=\skip59
\absrightindent=\skip60
\absparindent=\skip61
\absparsep=\skip62
\c@part=\count81
\c@chapter=\count82
\c@section=\count83
\c@subsection=\count84
\c@subsubsection=\count85
\c@paragraph=\count86
\c@subparagraph=\count87
\beforechapskip=\skip63
\midchapskip=\skip64
\afterchapskip=\skip65
\chapindent=\skip66
\bottomsectionskip=\skip67
\secindent=\skip68
\beforesecskip=\skip69
\aftersecskip=\skip70
\subsecindent=\skip71
\beforesubsecskip=\skip72
\aftersubsecskip=\skip73
\subsubsecindent=\skip74
\beforesubsubsecskip=\skip75
\aftersubsubsecskip=\skip76
\paraindent=\skip77
\beforeparaskip=\skip78
\afterparaskip=\skip79
\subparaindent=\skip80
\beforesubparaskip=\skip81
\aftersubparaskip=\skip82
\pfbreakskip=\skip83
\c@@ppsavesec=\count88
\c@@ppsaveapp=\count89
\ragrparindent=\dimen102
\parsepi=\skip84
\topsepi=\skip85
\itemsepi=\skip86
\parsepii=\skip87
\topsepii=\skip88
\topsepiii=\skip89
\m@msavetopsep=\skip90
\m@msavepartopsep=\skip91
\@enLab=\toks14
\c@vslineno=\count90
\c@poemline=\count91
\c@modulo@vs=\count92
\vleftskip=\skip92
\vrightskip=\skip93
\stanzaskip=\skip94
\versewidth=\skip95
\vgap=\skip96
\vindent=\skip97
\c@verse=\count93
\c@chrsinstr=\count94
\beforepoemtitleskip=\skip98
\afterpoemtitleskip=\skip99
\col@sep=\dimen103
\extrarowheight=\dimen104
\NC@list=\toks15
\extratabsurround=\skip100
\backup@length=\skip101
\TX@col@width=\dimen105
\TX@old@table=\dimen106
\TX@old@col=\dimen107
\TX@target=\dimen108
\TX@delta=\dimen109
\TX@cols=\count95
\TX@ftn=\toks16
\heavyrulewidth=\dimen110
\lightrulewidth=\dimen111
\cmidrulewidth=\dimen112
\belowrulesep=\dimen113
\belowbottomsep=\dimen114
\aboverulesep=\dimen115
\abovetopsep=\dimen116
\cmidrulesep=\dimen117
\cmidrulekern=\dimen118
\defaultaddspace=\dimen119
\@cmidla=\count96
\@cmidlb=\count97
\@aboverulesep=\dimen120
\@belowrulesep=\dimen121
\@thisruleclass=\count98
\@lastruleclass=\count99
\@thisrulewidth=\dimen122
\ctableftskip=\skip102
\ctabrightskip=\skip103
\abovecolumnspenalty=\count100
\@linestogo=\count101
\@cellstogo=\count102
\@cellsincolumn=\count103
\crtok=\toks17
\@mincolumnwidth=\dimen123
\c@newflo@tctr=\count104
\@contcwidth=\skip104
\@contindw=\skip105
\abovecaptionskip=\skip106
\belowcaptionskip=\skip107
\subfloattopskip=\skip108
\subfloatcapskip=\skip109
\subfloatcaptopadj=\skip110
\subfloatbottomskip=\skip111
\subfloatlabelskip=\skip112
\subfloatcapmargin=\dimen124
\c@@contsubnum=\count105
\beforeepigraphskip=\skip113
\afterepigraphskip=\skip114
\epigraphwidth=\skip115
\epigraphrule=\skip116
LaTeX Info: Redefining \emph on input line 4419.
LaTeX Info: Redefining \em on input line 4420.
\tocentryskip=\skip117
\tocbaseline=\skip118
\cftparskip=\skip119
\cftbeforepartskip=\skip120
\cftpartindent=\skip121
\cftpartnumwidth=\skip122
\cftbeforechapterskip=\skip123
\cftchapterindent=\skip124
\cftchapternumwidth=\skip125
\cftbeforesectionskip=\skip126
\cftsectionindent=\skip127
\cftsectionnumwidth=\skip128
\cftbeforesubsectionskip=\skip129
\cftsubsectionindent=\skip130
\cftsubsectionnumwidth=\skip131
\cftbeforesubsubsectionskip=\skip132
\cftsubsubsectionindent=\skip133
\cftsubsubsectionnumwidth=\skip134
\cftbeforeparagraphskip=\skip135
\cftparagraphindent=\skip136
\cftparagraphnumwidth=\skip137
\cftbeforesubparagraphskip=\skip138
\cftsubparagraphindent=\skip139
\cftsubparagraphnumwidth=\skip140
\c@maxsecnumdepth=\count106
\bibindent=\dimen125
\bibitemsep=\skip141
\indexcolsep=\skip142
\indexrule=\skip143
\indexmarkstyle=\toks18
\@indexbox=\insert233
\sideparvshift=\skip144
\sideins=\insert232
\sidebarhsep=\skip145
\sidebarvsep=\skip146
\sidebarwidth=\skip147
\footmarkwidth=\skip148
\footmarksep=\skip149
\footparindent=\skip150
\footinsdim=\skip151
\footinsv@r=\insert231
\@mpfootinsv@r=\insert230
\m@m@k=\count107
\m@m@h=\dimen126
\m@mipn@skip=\skip152
\c@sheetsequence=\count108
\c@lastsheet=\count109
\c@lastpage=\count110
\every@verbatim=\toks19
\afterevery@verbatim=\toks20
\verbatim@line=\toks21
\tab@position=\count111
\verbatim@in@stream=\read1
\verbatimindent=\skip153
\verbatim@out=\write3
\bvboxsep=\skip154
\c@bvlinectr=\count112
\bvnumlength=\skip155
\FrameRule=\dimen127
\FrameSep=\dimen128
\c@cp@cntr=\count113
LaTeX Info: Redefining \: on input line 7541.
LaTeX Info: Redefining \! on input line 7543.
\c@ism@mctr=\count114
\c@xsm@mctr=\count115
\c@csm@mctr=\count116
\c@ksm@mctr=\count117
\c@xksm@mctr=\count118
\c@cksm@mctr=\count119
\c@msm@mctr=\count120
\c@xmsm@mctr=\count121
\c@cmsm@mctr=\count122
\c@bsm@mctr=\count123
\c@workm@mctr=\count124
\c@figure=\count125
\c@lofdepth=\count126
\c@lofdepth=\count126
\cftbeforefigureskip=\skip156
\cftfigureindent=\skip157
\cftfigurenumwidth=\skip158
\c@table=\count126
\c@lotdepth=\count127
\c@lotdepth=\count127
\cftbeforetableskip=\skip159
\cfttableindent=\skip160
\cfttablenumwidth=\skip161
 (C:\texmf\tex\latex\memoir\mempatch.sty
File: mempatch.sty 2005/02/07 v3.6 Patches for memoir class v1.61
\abs@leftindent=\dimen129
LaTeX Info: Redefining \em on input line 142.
LaTeX Info: Redefining \em on input line 384.
LaTeX Info: Redefining \emph on input line 392.
)) (C:\texmf\tex\latex\amsmath\amsmath.sty
Package: amsmath 2000/07/18 v2.13 AMS math features
\@mathmargin=\skip162

For additional information on amsmath, use the `?' option.
(C:\texmf\tex\latex\amsmath\amstext.sty
Package: amstext 2000/06/29 v2.01
 (C:\texmf\tex\latex\amsmath\amsgen.sty
File: amsgen.sty 1999/11/30 v2.0
\@emptytoks=\toks22
\ex@=\dimen130
)) (C:\texmf\tex\latex\amsmath\amsbsy.sty
Package: amsbsy 1999/11/29 v1.2d
\pmbraise@=\dimen131
)
(C:\texmf\tex\latex\amsmath\amsopn.sty
Package: amsopn 1999/12/14 v2.01 operator names
)
\inf@bad=\count127
LaTeX Info: Redefining \frac on input line 211.
\uproot@=\count128
\leftroot@=\count129
LaTeX Info: Redefining \overline on input line 307.
\classnum@=\count130
\DOTSCASE@=\count131
LaTeX Info: Redefining \ldots on input line 379.
LaTeX Info: Redefining \dots on input line 382.
LaTeX Info: Redefining \cdots on input line 467.
\Mathstrutbox@=\box26
\strutbox@=\box27
\big@size=\dimen132
LaTeX Font Info:    Redeclaring font encoding OML on input line 567.
LaTeX Font Info:    Redeclaring font encoding OMS on input line 568.
\macc@depth=\count132
\c@MaxMatrixCols=\count133
\dotsspace@=\muskip10
\c@parentequation=\count134
\dspbrk@lvl=\count135
\tag@help=\toks23
\row@=\count136
\column@=\count137
\maxfields@=\count138
\andhelp@=\toks24
\eqnshift@=\dimen133
\alignsep@=\dimen134
\tagshift@=\dimen135
\tagwidth@=\dimen136
\totwidth@=\dimen137
\lineht@=\dimen138
\@envbody=\toks25
\multlinegap=\skip163
\multlinetaggap=\skip164
\mathdisplay@stack=\toks26
LaTeX Info: Redefining \[ on input line 2666.
LaTeX Info: Redefining \] on input line 2667.
)
(C:\texmf\tex\latex\amsfonts\amssymb.sty
Package: amssymb 2002/01/22 v2.2d

(C:\texmf\tex\latex\amsfonts\amsfonts.sty
Package: amsfonts 2001/10/25 v2.2f
\symAMSa=\mathgroup4
\symAMSb=\mathgroup5
LaTeX Font Info:    Overwriting math alphabet `\mathfrak' in version `bold'
(Font)                  U/euf/m/n --> U/euf/b/n on input line 132.
))
(C:\texmf\tex\latex\wrapfig\wrapfig.sty
\wrapoverhang=\dimen139
\WF@size=\dimen140
\c@WF@wrappedlines=\count139
\WF@box=\box28
\WF@everypar=\toks27
Package: wrapfig 2003/01/31  v 3.6
)

******************************************************
Stock height and width: 794.0pt by 614.0pt
Top and edge trims: 0.0pt and 0.0pt
Page height and width: 794.0pt by 614.0pt
Text height and width: 592.0pt by 434.0pt
Spine and edge margins: 90.0pt and 90.0pt
Upper and lower margins: 110.0pt and 92.0pt
Headheight and headsep: 14.0pt and 19.8738pt
Footskip: 30.0pt
Columnsep and columnseprule: 10.0pt and 0.0pt
Marginparsep and marginparwidth: 7.0pt and 50.0pt
******************************************************

(17224-t.aux)
LaTeX Font Info:    Checking defaults for OML/cmm/m/it on input line 71.
LaTeX Font Info:    ... okay on input line 71.
LaTeX Font Info:    Checking defaults for T1/cmr/m/n on input line 71.
LaTeX Font Info:    ... okay on input line 71.
LaTeX Font Info:    Checking defaults for OT1/cmr/m/n on input line 71.
LaTeX Font Info:    ... okay on input line 71.
LaTeX Font Info:    Checking defaults for OMS/cmsy/m/n on input line 71.
LaTeX Font Info:    ... okay on input line 71.
LaTeX Font Info:    Checking defaults for OMX/cmex/m/n on input line 71.
LaTeX Font Info:    ... okay on input line 71.
LaTeX Font Info:    Checking defaults for U/cmr/m/n on input line 71.
LaTeX Font Info:    ... okay on input line 71.
\c@lofdepth=\count140
\c@lotdepth=\count141
 [1

] [1] [1

] [2

] (17224-t.toc
LaTeX Font Info:    Try loading font information for U+msa on input line 1.
 (C:\texmf\tex\latex\amsfonts\umsa.fd
File: umsa.fd 2002/01/19 v2.2g AMS font definitions
)
LaTeX Font Info:    Try loading font information for U+msb on input line 1.
 (C:\texmf\tex\latex\amsfonts\umsb.fd
File: umsb.fd 2002/01/19 v2.2g AMS font definitions
)) [3

] [1


] [2] [3] [4] [5] [6] [7]
[8] [9] [10] [11] [12] [13] [14] [15

]
Overfull \hbox (5.60655pt too wide) in paragraph at lines 1228--1263
[]\OT1/cmr/m/n/12 Before pro-ceed-ing to ex-hibit the fal-lacy of the prin-ci-p
les here enunciated---principles
 []

[16] [17] [18] [19] [20] [21] [22] [23] [24] [25] [26] [27] [28] [29] [30]
[31] [32] [33] [34] [35] [36] [37] [38] [39

] [40] [41] [42] [43] [44] [45]
[46] [47] [48] [49] [50] [51

] [52] [53] [54] [55] [56] [57] [58] [59] [60]
[61] [62

] [63] [64] [65] [66] [67] [68] [69] [70] [71] [72] [73] [74] [75]
[76] [77] [78] [79] [80

] [81] [82] [83] [84] [85] [86] [87] [88] [89] [90]
[91] [92] [93] [94

] [95] [96] [97] [98] [99] [100] [101] [102] [103] [104]
[105] [106] [107] [108] [109] [110] [111

] [112] [113] [114] [115] [116]
[117] [118] [119] [120] [121] [122

] [123] [124] [125] [126] [127] [128]
[129] [130] [131

] [132] [133] [134] [135] [136] [137] [138] [139] [140]
[141] [142] [143] [144] [145

] [146] [147] [148] [149] [150

] [1] [2] [3]
[4] [5] [6] [7] [8] [9]
\tf@toc=\write4
 (17224-t.aux)

 *File List*
  memoir.cls    2004/04/05 v1.61 configurable document class
   mem12.clo    2004/03/12 v0.3 memoir class 12pt size option
mempatch.sty    2005/02/07 v3.6 Patches for memoir class v1.61
 amsmath.sty    2000/07/18 v2.13 AMS math features
 amstext.sty    2000/06/29 v2.01
  amsgen.sty    1999/11/30 v2.0
  amsbsy.sty    1999/11/29 v1.2d
  amsopn.sty    1999/12/14 v2.01 operator names
 amssymb.sty    2002/01/22 v2.2d
amsfonts.sty    2001/10/25 v2.2f
 wrapfig.sty    2003/01/31  v 3.6
    umsa.fd    2002/01/19 v2.2g AMS font definitions
    umsb.fd    2002/01/19 v2.2g AMS font definitions
 ***********

 ) 
Here is how much of TeX's memory you used:
 3278 strings out of 95898
 38170 string characters out of 1195288
 119895 words of memory out of 1114456
 6286 multiletter control sequences out of 60000
 15462 words of font info for 58 fonts, out of 500000 for 1000
 14 hyphenation exceptions out of 607
 27i,11n,24p,223b,340s stack positions out of 1500i,500n,5000p,200000b,32768s

Output written on 17224-t.dvi (164 pages, 667412 bytes).

